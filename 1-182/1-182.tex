\documentclass[a5paper,10pt]{article}
 
% \usepackage{extsizes}
\usepackage{cmap}
\usepackage[T2A]{fontenc}
\usepackage[utf8x]{inputenc}
\usepackage[english, russian]{babel}

\usepackage{misccorr}

%%%%%%%%%%%%%%%%%%%%%%%%%%%%%%%%%%%%%%%%%%%%%%%%%%%%%%%%%%%%%%%%%%%%%%%%%%%%%%%%%%  
\usepackage{graphicx} % для вставки картинок
\graphicspath{{img/}}
\usepackage{amssymb,amsfonts,amsmath,amsthm} % математические дополнения от АМС

% \usepackage{fontspec}
% \usepackage{unicode-math}
% \usepackage[outline]{contour}
\usepackage{indentfirst} % отделять первую строку раздела абзацным отступом тоже
\usepackage[usenames,dvipsnames]{color} % названия цветов
\usepackage{makecell}
\usepackage{multirow} % улучшенное форматирование таблиц
\usepackage{ulem} % подчеркивания
\linespread{1.3} % полуторный интервал
% \renewcommand{\rmdefault}{ftm} % Times New Roman (не работает)
\frenchspacing
\usepackage{geometry}
\geometry{left=1cm,right=1cm,top=2cm,bottom=1cm,bindingoffset=0cm}
\usepackage{titlesec}
\usepackage{float}
% \definecolor{black}{rgb}{0,0,0}
% \usepackage[colorlinks, unicode, pagecolor=black]{hyperref}
% \usepackage[unicode]{hyperref} %ссылки
\usepackage{fancyhdr} %загрузим пакет
\pagestyle{fancy} %применим колонтитул
\fancyhead{} %очистим хидер на всякий случай
\fancyhead[R]{Сарафанов Ф.Г.} %номер страницы слева сверху на четных и справа на нечетных
\fancyhead[C]{Механика}
\fancyhead[L]{Иродов -- №1.182} 
\fancyfoot{} %футер будет пустой
% \fancyfoot[CO,CE]{\thepage}
\renewcommand{\labelenumii}{\theenumii)}


\usepackage{tikz}
\usetikzlibrary{scopes}
\usetikzlibrary{%
    decorations.pathreplacing,%
    decorations.pathmorphing,%
    patterns,%
    calc,%
    scopes,%
    arrows,%
    quotes,%
    angles,%
    babel
    % arrows.spaced,%
}

\begin{document}

\begin{figure}[H]
    \centering
\begin{tikzpicture}[
    force/.style={>=latex,draw=blue,fill=blue},
    % axis/.style={densely dashed,gray,font=\small},
    axis/.style={densely dashed,black!60,font=\small},
    interface/.style={
        pattern = north east lines,
        draw    = none,
        pattern color=gray!60,          
    },
    cargo/.style={
        rectangle,
        fill=magenta!40,
        draw=black!50,
        inner sep=2.5mm,
    },
    spring/.style={
        decoration={
            aspect=0.3, 
            segment length=.8mm, 
            amplitude=2mm,
            coil},
        decorate,
        draw=magenta!70
    }
]
\matrix[column sep=2cm] {
    \begin{scope}[rotate=14]    
        \draw[force,->] (0,0) coordinate (a) -- 
            node[midway,fill=white!20, opacity=0.9]  
                {$2\vec{v}_c$} 
            (4,0);

        \draw[force,->] (0,0)  -- 
            node[midway,fill=white!20, opacity=0.9]  
                {$\vec{p}_1$} 
            (1,2) coordinate (b);

        \draw[force,->] (1,2)  -- 
            node[midway,fill=white!20, opacity=0.9]  
                {$\vec{p}_2$} 
            (4,0) coordinate (c);       

        \draw pic["$\Theta_1$",draw=magenta,->,angle eccentricity=1.5,angle radius=0.5cm] {angle=a--b--c};                 
    \end{scope}
    &
    \begin{scope}[rotate=14]    
        \draw[force,->] (0,0) coordinate (a) -- 
            node[midway,fill=white!20, opacity=0.9]  
                {$2\vec{v}_c$} 
            (4,0);

        \draw[force,->] (0,0)  -- 
            node[midway,fill=white!20, opacity=0.9]  
                {$\vec{p}_{1\text{н}}$} 
            (3,2) coordinate (b);

        \draw[force,->] (b)  -- 
            node[midway,fill=white!20, opacity=0.9]  
                {$\vec{p}_{2\text{н}}$} 
            (4,0) coordinate (c);       

        \draw pic["$\Theta_2$",draw=magenta,->,angle eccentricity=1.5,angle radius=0.5cm] {angle=a--b--c};                 
    \end{scope}    
    \\
};
\end{tikzpicture}
\end{figure}

По определению,
\begin{equation}
    \vec{v}_c=\frac{m\vec{v}_1+m\vec{v}_2}{m+m}
\end{equation}
После очень простых преобразований:
\begin{equation}
    2m\vec{v}_c=\vec{p}_1+\vec{p}_2=const
\end{equation}
При ударе сохраняется скорость центра масс, отсюда
\begin{equation}
    \vec{p}_1+\vec{p}_2=
    \vec{p}_{1\text{н}}+\vec{p}_{2\text{н}}=
    2m\vec{v}_c
\end{equation}
Запишем ЗСЭ в виде $W=\frac{p^2}{2m}$:
\begin{equation}
    \frac{p_1^2}{2m}+\frac{p_2^2}{2m}=
    \frac{p_{1\text{н}}^{2}}{2m}+\frac{p_{2\text{н}}^{2}}{2m}
\end{equation}
Откуда
\begin{equation}
    {p_1^2}+{p_2^2}=
    {p_{1\text{н}}^{2}}+{p_{2\text{н}}^{2}}
\end{equation}
По теореме косинусов
\begin{equation}
    4mv_c^2=p_1^2+p_2^2-2p_1p_2\cos\Theta_1
\end{equation}
и
\begin{equation}
    4mv_c^2=p_{1\text{н}}^2+p_{2\text{н}}^2-2p_{1\text{н}}p_{2\text{н}}\cos\Theta_2
\end{equation}
Вычитая последние два уравнения, получим
\begin{equation}
    p_1p_2\cos\Theta_1=p_{1\text{н}}p_{2\text{н}}\cos\Theta_2
\end{equation}
Откуда получим окончательный ответ
\begin{equation}
    \cos\Theta_2=\cos\Theta_1\frac{v_1 \cdot v_2}{v_{1\text{н}} \cdot v_{2\text{н}}}
\end{equation}
\end{document}