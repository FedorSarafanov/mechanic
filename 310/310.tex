\documentclass[a5paper,10pt]{article}
 
% \usepackage{extsizes}
\usepackage{cmap}
\usepackage[T2A]{fontenc}
\usepackage[utf8x]{inputenc}
\usepackage[english, russian]{babel}

\usepackage{misccorr}

%%%%%%%%%%%%%%%%%%%%%%%%%%%%%%%%%%%%%%%%%%%%%%%%%%%%%%%%%%%%%%%%%%%%%%%%%%%%%%%%%%  
\usepackage{graphicx} % для вставки картинок
\graphicspath{{img/}}
\usepackage{amssymb,amsfonts,amsmath,amsthm} % математические дополнения от АМС

% \usepackage{fontspec}
% \usepackage{unicode-math}

\usepackage{indentfirst} % отделять первую строку раздела абзацным отступом тоже
\usepackage[usenames,dvipsnames]{color} % названия цветов
\usepackage{makecell}
\usepackage{multirow} % улучшенное форматирование таблиц
\usepackage{ulem} % подчеркивания
\linespread{1.3} % полуторный интервал
% \renewcommand{\rmdefault}{ftm} % Times New Roman (не работает)
\frenchspacing
\usepackage[portrait]{geometry}
\geometry{left=1cm,right=1cm,top=2cm,bottom=1cm,bindingoffset=0cm}
\usepackage{titlesec}
\usepackage{float}
% \definecolor{black}{rgb}{0,0,0}
% \usepackage[colorlinks, unicode, pagecolor=black]{hyperref}
% \usepackage[unicode]{hyperref} %ссылки
\usepackage{fancyhdr} %загрузим пакет
\pagestyle{fancy} %применим колонтитул
\fancyhead{} %очистим хидер на всякий случай
\fancyhead[LE,RO]{Сарафанов Ф.Г.} %номер страницы слева сверху на четных и справа на нечетных
\fancyhead[CO, CE]{Механика}
\fancyhead[LO,RE]{№310 -- Яковлев} 
\fancyfoot{} %футер будет пустой
% \fancyfoot[CO,CE]{\thepage}
\renewcommand{\labelenumii}{\theenumii)}


\usepackage{tikz}
\usepackage{etoolbox}
\usetikzlibrary{scopes}
\usetikzlibrary{%
     decorations.pathreplacing,%
     decorations.pathmorphing,%
    patterns,%
    calc,%
    scopes,%
    arrows,%
    % arrows.spaced,%
}
\makeatletter
\newif\if@gather@prefix 
\preto\place@tag@gather{% 
  \if@gather@prefix\iftagsleft@ 
    \kern-\gdisplaywidth@ 
    \rlap{\gather@prefix}% 
    \kern\gdisplaywidth@ 
  \fi\fi 
} 
\appto\place@tag@gather{% 
  \if@gather@prefix\iftagsleft@\else 
    \kern-\displaywidth 
    \rlap{\gather@prefix}% 
    \kern\displaywidth 
  \fi\fi 
  \global\@gather@prefixfalse 
} 
\preto\place@tag{% 
  \if@gather@prefix\iftagsleft@ 
    \kern-\gdisplaywidth@ 
    \rlap{\gather@prefix}% 
    \kern\displaywidth@ 
  \fi\fi 
} 
\appto\place@tag{% 
  \if@gather@prefix\iftagsleft@\else 
    \kern-\displaywidth 
    \rlap{\gather@prefix}% 
    \kern\displaywidth 
  \fi\fi 
  \global\@gather@prefixfalse 
} 
\newcommand*{\beforetext}[1]{% 
  \ifmeasuring@\else
  \gdef\gather@prefix{#1}% 
  \global\@gather@prefixtrue 
  \fi
} 
\makeatother
\begin{document}
\def\iangle{60} % Angle of the inclined plane

\def\down{-90}
\def\arcr{0.5cm} % Radius of the arc used to indicate angles
\def\FIN{\vec{F}_{in}^\text{пост}}
\hspace{-2em}
\begin{tikzpicture}[
    axis/.style={densely dashed,black!60,font=\small},
    force/.style={>=latex,draw=blue,fill=blue},
    % m/.style={rectangle,draw,fill=lightgray,minimum size=0.5cm,thin},
    m/.style={draw=black!30, rectangle,draw,thin, fill=blue!2, minimum size=0.5cm},
    m/.style={draw=black!30, rectangle,draw,thin, fill=blue!2, minimum size=0.5cm},
    interface/.style={draw=gray!60,
        postaction={draw=gray!60,decorate,decoration={border,angle=-135,
                    amplitude=0.3cm,segment length=2mm}}},
    plane/.style={draw=black!30, very thick, fill=blue!5},
    string/.style={draw=black, thick},
    pulley/.style={thick},
]

\matrix[column sep=0.5cm] {
    \begin{scope}[]
        \xdef\r{3cm}
         \xdef\rp{5pt}
        \draw[thick] (0,0) circle(\r);
        \foreach \i in {0,40,...,360}{
            \draw[thick] (0,0) -- (\i:\r);
        }
        \draw[fill=black!40] (\r,0) circle (4pt) node [above right] {$m$};
        \draw[force, very thick, ->] (\r,0) -- (\r/2,0) node[above] {$\vec{v_0}$};
              \draw [axis, ->]  (0,0) -- (1.5*\r,0)  node[anchor=south] {$+x$};
     {[force]
          \draw (0,0)  circle (\rp) node[left, xshift=-1em] {$\vec{\omega}$};
        \draw [fill=blue,thick] (0,0) circle (\rp/2);
     }
    \end{scope}
&
\begin{scope}[rotate=0]
    \draw[fill=black!40] (\r,0) circle (4pt) node [above right] {$m$};
        \draw[force, very thick, ->] (\r,0) -- (\r+\r/2,0) node[above] {$\vec{F_{in}^{post}}$};
        \draw[force, very thick, ->] (\r,0) -- (\r,\r/2) node[above] {$\vec{F_{in}^{kor}}$};
        \draw [axis, ->]  (2,0) -- (2*\r,0)  node[anchor=south] {$+x$};

\end{scope}
&
\begin{scope}[]   
 
\end{scope}
\\
};
\end{tikzpicture}

Решим задачу, используя теорему о изменении кинетической энергии.
\begin{gather}
    % \beforetext{З.:}%
    \frac{m\cdot0^2}{2}-\frac{mv_0^2}{2}=A_f=\int_R^0F_xdx\\
    \frac{mv_0^2}{2}=\int_0^R[m\omega^2x]dx\\
    \frac{mv_0^2}{2}=m\omega^2\frac{R^2}{2}\\
    v_0=\omega{R}
    % m\vec{a}=m\vec{g}+\vec{F}^\text{кор}_\text{in}+\vec{F}^\text{цб}_\text{in}\\
    % \beforetext{Т.к. $\vec{F}^\text{цб}_\text{in}$ мало:}%
    % m\vec{a}=m\vec{g}+\vec{F}^\text{кор}_\text{in}\\
    % \beforetext{Проекция на y:} ma_y=-mg\\
    % \beforetext{Опуская интегрирование:} v'_y(t)=v'_{0y}-gt\\
    % \beforetext{Тогда условие остановки:} t_{stop}=\frac{v'_{0y}}{g}\Longrightarrow t_\text{движ}=\frac{2v'_{0y}}{g}\\
    % {\omega_\perp}={\omega}\cdot\cos{60}\\
    % \beforetext{Проекция на z:} ma^\text{кор}_z=-2m|[\vec{\omega_\perp}\times\vec{v}']|\\
    % \beforetext{Тогда} a^\text{кор}_z\approx-2\omega_\perp(v'_{0y}-gt)]\\
%
\end{gather}

Ясно, что решению будет удовлетворять любая бОльшая скорость, так как при ней 
тело точно достигнет оси:
\begin{equation}
    v\geq \omega{R}
\end{equation}


\end{document}