\documentclass[a5paper,10pt]{article}
\def\source{/home/lab/tex/templates}

\usepackage{cmap}
\usepackage[T2A]{fontenc}
\usepackage[utf8x]{inputenc}
\usepackage[english, russian]{babel}

\usepackage
	{
		amssymb,
		% misccorr,
		amsfonts,
		amsmath,
		amsthm,
		wrapfig,
		makecell,
		multirow,
		indentfirst,
		ulem,
		graphicx,
		geometry,
		fancyhdr,
		subcaption,
		float,
		tikz,
		csvsimple,
		color,
	}  

\usepackage[outline]{contour}
\usepackage[mode=buildnew]{standalone}


\geometry
	{
		left=1cm,
		right=1cm,
		top=2cm,
		bottom=1cm,
		bindingoffset=0cm,
	}

\linespread{1.3} 
\frenchspacing 


\usetikzlibrary{scopes}
\usetikzlibrary
	{
		decorations.pathreplacing,
		decorations.pathmorphing,
		patterns,
		calc,
		scopes,
		arrows,
		through,
		shapes.misc,
		arrows.meta,
	}


\tikzset{
	force/.style=	{
		>=latex,
		draw=blue,
		fill=blue,
				 	}, 
	%				 	
	axis/.style=	{
		densely dashed,
		gray,
		font=\small,
					},
	%
	acceleration/.style={
		>=open triangle 60,
		draw=blue,
		fill=blue,
					},
	%
	inforce/.style=	{
		force,
		double equal sign distance=2pt,
					},
	%
	interface/.style={
		pattern = north east lines, 
		draw    = none, 
		pattern color=gray!60,
					},
	cross/.style=	{
		cross out, 
		draw=black, 
		minimum size=2*(#1-\pgflinewidth), 
		inner sep=0pt, outer sep=0pt,
					},
	%
	cargo/.style=	{
		rectangle, 
		fill=black!70, 
		inner sep=2.5mm,
					},
	%
	}

\pagestyle{fancy} %применим колонтитул
\fancyhead{} %очистим хидер на всякий случай
\fancyhead[R]{Сарафанов Ф.Г.} %номер страницы слева сверху на четных и справа на нечетных
\fancyhead[C]{Механика}
% \fancyhead[L]{Задача под запись - <<АУУ-2>>} 
\fancyfoot{} %футер будет пустой

\newcommand{\irodov}[1]{\fancyhead[L]{Иродов -- №#1}}
\newcommand{\yakovlev}[1]{\fancyhead[L]{Яковлев -- №#1}}
\newcommand{\wrote}[1]{\fancyhead[L]{Под запись -- <<#1>>}}

\newenvironment{tikzpict}
    {
	    \begin{figure}[htbp]
		\centering
		\begin{tikzpicture}
    }
    { 
		\end{tikzpicture}
		% \caption{caption}
		% \label{fig:label}
		\end{figure}
    }

\newcommand{\vbLabel}[3]{\draw ($(#1,#2)+(0,5pt)$) -- ($(#1,#2)-(0,5pt)$) node[below]{#3}}
\newcommand{\vaLabel}[3]{\draw ($(#1,#2)+(0,5pt)$) node[above]{#3} -- ($(#1,#2)-(0,5pt)$) }

\newcommand{\hrLabel}[3]{\draw ($(#1,#2)+(5pt,0)$) -- ($(#1,#2)-(5pt,0)$) node[right, xshift=1em]{#3}}
\newcommand{\hlLabel}[3]{\draw ($(#1,#2)+(5pt,0)$) node[left, xshift=-1em]{#3} -- ($(#1,#2)-(5pt,0)$) }

% Draw line annotation
% Input:
%   #1 Line offset (optional)
%   #2 Line angle
%   #3 Line length
%   #5 Line label
% Example:
%   \lineann[1]{30}{2}{$L_1$}
\newcommand{\lineann}[4][0.5]{%
    \begin{scope}[rotate=#2, blue,inner sep=2pt, ]
        \draw[dashed, blue!40] (0,0) -- +(0,#1)
            node [coordinate, near end] (a) {};
        \draw[dashed, blue!40] (#3,0) -- +(0,#1)
            node [coordinate, near end] (b) {};
        \draw[|<->|] (a) -- node[fill=white, scale=0.8] {#4} (b);
    \end{scope}
}

\irodov{1.310}

\begin{document}

    %     \draw pic["$\Theta_1$",draw=magenta,->,angle eccentricity=1.5,angle radius=0.5cm] {angle=a--b--c};                 
    % \end{scope}
    % &
    % \begin{scope}[rotate=14]    
    %     \draw[force,->] (0,0) coordinate (a) -- 
    %         node[midway,fill=white!20, opacity=0.9]  
    %             {$2\vec{v}_c$} 
    %         (4,0);
\begin{tikzpict}
\matrix[column sep=2cm] {

	\draw[interface] (-0.5,1) rectangle (-0.1,1.5) (0.1,1.5) rectangle (0.5,1);

	\draw[interface] (-0.5,-1) rectangle (-0.1,-1.5) (0.1,-1.5) rectangle (0.5,-1);

	\draw[axis,magenta] (0,-1.5) -- (0,1.5);
	\draw[magenta,->] 
		(0,1.75) -- ++(0,0.5) 
		node[above] {$\vec\omega$};
	\draw[magenta,->]	(0.5,1.75) -- ++ (0,0.5)
		node[above] {$\vec{N}$} 
		;

	\draw[fill=magenta, magenta] (-0.5,-0.15) rectangle (0.5,0.15);

	\fill[magenta] (0,-1.25) coordinate (1) circle (2pt) node[below, yshift=-0.5em] {$O$};

	\fill[magenta] (0,1.25) coordinate (2) circle (2pt);

	\draw[->, thick, blue] (1) -- node[caption] {$\vec{l}$} (2);
	\draw[->, thick, magenta] (2)++(-1,0) -- node[caption] {$\vec{M}$} ++(-2,0);  

	\draw[fill=white, draw=blue] (0,1.25) circle (4.3pt) node {$\bigotimes$} node[right, xshift=0.5em] {$\vec{F}$};
    \draw[axis] (-0.6,-3) arc (-20:20:10);

    % &

    &    
    \begin{scope}
	\draw[magenta,->]	(0,0) coordinate (o) -- ++ (0,1.5) coordinate (i)
		node[above, black] {$\vec{N}(t)$};

	\draw[magenta,->]	(0,0) -- ++ (-2.25,1.5) coordinate (ii)
		node[above, black] {$\vec{N}(t+dt)$};		

	\draw[magenta,->] (i) --node[caption] {$d\vec{N}$} (ii);	

	\draw[magenta,->] (i)++(0,1) --node[caption] {$\vec{M}$} ++(-2,0);	

	\draw pic["$d\phi$",draw=magenta,->,angle eccentricity=1.5,angle radius=0.5cm] {angle=i--o--ii};    	

	\draw (-1,-1) node {$d\vec{N}=\vec{N}d\phi$};
    \end{scope}    
    \\
};
	
		
\end{tikzpict}
Из простых соображений неразрывности конструкции угол поворота корабля $d\phi$ совпадает с углом поворота его маховика, отсюда кинематически
\begin{equation}
	\Omega=\frac{v}{R}
\end{equation}
Где $\Omega$ -- угловая скорость поворота оси вращения, а значит, и момента импульса маховика.
С одной стороны, 
\begin{equation}
	I\omega_z={N}_z
\end{equation}
С другой,
\begin{equation}
	{M}=\frac{d{N_z}}{dt}=\frac{1}{dt}{N_z}\cdot d\phi ={N_z}\Omega
\end{equation}
Отсюда
\begin{equation}
	M={N_z\Omega(t)}={I\omega\Omega}
\end{equation}
Круговая частота связана с частотой
\begin{equation}
	\omega=2\pi n
\end{equation}
Тогда момент силы в подшипниках будет
\begin{equation}
	M=\frac{2\pi n\cdot I\cdot v}{R}
\end{equation}
\end{document}