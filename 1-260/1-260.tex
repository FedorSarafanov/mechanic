\documentclass[a5paper,10pt]{article}


\usepackage{cmap}
\usepackage[T2A]{fontenc}
\usepackage[utf8x]{inputenc}
\usepackage[english, russian]{babel}

\usepackage
	{
		amssymb,
		% misccorr,
		amsfonts,
		amsmath,
		amsthm,
		wrapfig,
		makecell,
		multirow,
		indentfirst,
		ulem,
		graphicx,
		geometry,
		fancyhdr,
		subcaption,
		float,
		tikz,
		csvsimple,
		color,
	}  

\usepackage[outline]{contour}
\usepackage[mode=buildnew]{standalone}


\geometry
	{
		left=1cm,
		right=1cm,
		top=2cm,
		bottom=1cm,
		bindingoffset=0cm,
	}

\linespread{1.3} 
\frenchspacing 


\usetikzlibrary{scopes}
\usetikzlibrary
	{
		decorations.pathreplacing,
		decorations.pathmorphing,
		patterns,
		calc,
		scopes,
		arrows,
		through,
		shapes.misc,
		arrows.meta,
	}


\tikzset{
	force/.style=	{
		>=latex,
		draw=blue,
		fill=blue,
				 	}, 
	%				 	
	axis/.style=	{
		densely dashed,
		gray,
		font=\small,
					},
	%
	acceleration/.style={
		>=open triangle 60,
		draw=blue,
		fill=blue,
					},
	%
	inforce/.style=	{
		force,
		double equal sign distance=2pt,
					},
	%
	interface/.style={
		pattern = north east lines, 
		draw    = none, 
		pattern color=gray!60,
					},
	cross/.style=	{
		cross out, 
		draw=black, 
		minimum size=2*(#1-\pgflinewidth), 
		inner sep=0pt, outer sep=0pt,
					},
	%
	cargo/.style=	{
		rectangle, 
		fill=black!70, 
		inner sep=2.5mm,
					},
	%
	}

\pagestyle{fancy} %применим колонтитул
\fancyhead{} %очистим хидер на всякий случай
\fancyhead[R]{Сарафанов Ф.Г.} %номер страницы слева сверху на четных и справа на нечетных
\fancyhead[C]{Механика}
% \fancyhead[L]{Задача под запись - <<АУУ-2>>} 
\fancyfoot{} %футер будет пустой

\newcommand{\irodov}[1]{\fancyhead[L]{Иродов -- №#1}}
\newcommand{\yakovlev}[1]{\fancyhead[L]{Яковлев -- №#1}}
\newcommand{\wrote}[1]{\fancyhead[L]{Под запись -- <<#1>>}}

\newenvironment{tikzpict}
    {
	    \begin{figure}[htbp]
		\centering
		\begin{tikzpicture}
    }
    { 
		\end{tikzpicture}
		% \caption{caption}
		% \label{fig:label}
		\end{figure}
    }

\newcommand{\vbLabel}[3]{\draw ($(#1,#2)+(0,5pt)$) -- ($(#1,#2)-(0,5pt)$) node[below]{#3}}
\newcommand{\vaLabel}[3]{\draw ($(#1,#2)+(0,5pt)$) node[above]{#3} -- ($(#1,#2)-(0,5pt)$) }

\newcommand{\hrLabel}[3]{\draw ($(#1,#2)+(5pt,0)$) -- ($(#1,#2)-(5pt,0)$) node[right, xshift=1em]{#3}}
\newcommand{\hlLabel}[3]{\draw ($(#1,#2)+(5pt,0)$) node[left, xshift=-1em]{#3} -- ($(#1,#2)-(5pt,0)$) }

% Draw line annotation
% Input:
%   #1 Line offset (optional)
%   #2 Line angle
%   #3 Line length
%   #5 Line label
% Example:
%   \lineann[1]{30}{2}{$L_1$}
\newcommand{\lineann}[4][0.5]{%
    \begin{scope}[rotate=#2, blue,inner sep=2pt, ]
        \draw[dashed, blue!40] (0,0) -- +(0,#1)
            node [coordinate, near end] (a) {};
        \draw[dashed, blue!40] (#3,0) -- +(0,#1)
            node [coordinate, near end] (b) {};
        \draw[|<->|] (a) -- node[fill=white, scale=0.8] {#4} (b);
    \end{scope}
}

\irodov{1.260}

\begin{document}

\begin{tikzpict}
	\draw[magenta, fill=magenta!10] (0,0) circle (2cm);
	\draw[magenta, fill=magenta!0] (-1,0) circle (1cm);

	\draw (0,0) node[blue]{$\times$};
	\draw (-1,0) node[blue]{$\times$};
	\draw (1/6,0) node[magenta]{$\times$};

	\begin{scope}[yshift=0cm, xshift=0cm]
		\lineann[3cm]{0}{2}{$R$};
	\end{scope}	

	\draw[axis] (-3,0) -- (3,0) node [right] {$+x$};

	% \draw[magenta, fill=magenta!1] (0,0) circle (2cm);

	% \draw[magenta, fill=magenta!10] (-1,1.5) -- ++(0, 0.1) -- ++ (2,0)  coordinate (I) -- ++ (0,-0.1)  coordinate (II) -- cycle;

	% \draw[axis] (-3,-1) -- ++(0,5);
	% \draw[axis,->] (0,-1) -- ++(0,5) node[above] {$+x$};

	% \hlLabel{-3}{0}{$0$};
	% \hlLabel{-3}{3}{$h$};

	% \draw[blue,<-] (2,1.6) -- ++ (0,0.3);
	% \draw[blue,<-] (2,1.5) -- ++ (0,-0.3);
	% \draw[blue] ($(2,1.5)!0.5!(2,1.6)$) node[right, xshift=0.5em] {$dx$};

	% \draw[axis] (I) -- ++(1,0);
	% \draw[axis] (II) -- ++(1,0);
	% \begin{scope}[yshift=1.5cm, xshift=0cm]
	% 	\lineann{0}{1}{$r(x)$};
	% \end{scope}

\end{tikzpict}

Найдем центр масс диска (масса -- $\tilde{m}$) с вырезом, формально представив вырез  как наложенный (в месте выреза) на основной диск с массой $M$ малый диск с отрицательной массой <<$-m$>>.

Массы таких дисков найдем из условия:
\begin{equation}
	\rho(\pi R^2-\pi\frac{R^2}{4})=\tilde{m}
	\Rightarrow
	\tilde{m}=\rho\frac{3}{4}\pi R^2
\end{equation}
Тогда
\begin{gather}
	m=\rho \frac{1}{4} \pi R^2=\frac{1}{3}\tilde{m}\\
	M=\rho \pi R^2=\frac{4}{3}\tilde{m}
\end{gather}
Относительно оси $x$ диск симметричен $\Rightarrow$ $y_c=0$.
Тогда
\begin{equation}
	x_c=\frac{M\cdot0-m\cdot(-\frac{R}{2})}{\tilde{m}}=
	\frac{1}{6}R
\end{equation}
Момент инерции диска относительно геометрического центра найдем как разность моментов целого диска и вырезанного, где момент вырезанного относительно $O$ выразим через теорему Гюйгенса-Штейнера:
\begin{equation}
	I_o=\frac{MR^2}{2}-\left(\frac{mR^2}{8}+m\left(\frac{R}{2}\right)^2\right)=
	% \frac{4}{3}\tilde{m}\frac{R^2}{2}+
	\frac{4}{3}\tilde{m}\frac{R^2}{2}-
	\frac{3}{8}\cdot\frac{1}{3}\tilde{m}\frac{R^2}{2}=\frac{13}{24}\tilde{m}R^2
\end{equation}
По той же теореме найдем момент инерции относительно ц.м.:
\begin{equation}
	I_c=I_o-\tilde{m}\cdot\left(\frac{1}{6}R\right)^2=
	\frac{13}{24}\tilde{m}R^2-\frac{1}{36}\tilde{m}R^2=
	\frac{37}{72}\tilde{m}R^2
\end{equation}
\end{document}