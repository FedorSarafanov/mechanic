\documentclass[a5paper,10pt]{article}
\def\source{/home/lab/tex/templates}

\usepackage{cmap}
\usepackage[T2A]{fontenc}
\usepackage[utf8x]{inputenc}
\usepackage[english, russian]{babel}

\usepackage
	{
		amssymb,
		% misccorr,
		amsfonts,
		amsmath,
		amsthm,
		wrapfig,
		makecell,
		multirow,
		indentfirst,
		ulem,
		graphicx,
		geometry,
		fancyhdr,
		subcaption,
		float,
		tikz,
		csvsimple,
		color,
	}  

\usepackage[outline]{contour}
\usepackage[mode=buildnew]{standalone}


\geometry
	{
		left=1cm,
		right=1cm,
		top=2cm,
		bottom=1cm,
		bindingoffset=0cm,
	}

\linespread{1.3} 
\frenchspacing 


\usetikzlibrary{scopes}
\usetikzlibrary
	{
		decorations.pathreplacing,
		decorations.pathmorphing,
		patterns,
		calc,
		scopes,
		arrows,
		through,
		shapes.misc,
		arrows.meta,
	}


\tikzset{
	force/.style=	{
		>=latex,
		draw=blue,
		fill=blue,
				 	}, 
	%				 	
	axis/.style=	{
		densely dashed,
		gray,
		font=\small,
					},
	%
	acceleration/.style={
		>=open triangle 60,
		draw=blue,
		fill=blue,
					},
	%
	inforce/.style=	{
		force,
		double equal sign distance=2pt,
					},
	%
	interface/.style={
		pattern = north east lines, 
		draw    = none, 
		pattern color=gray!60,
					},
	cross/.style=	{
		cross out, 
		draw=black, 
		minimum size=2*(#1-\pgflinewidth), 
		inner sep=0pt, outer sep=0pt,
					},
	%
	cargo/.style=	{
		rectangle, 
		fill=black!70, 
		inner sep=2.5mm,
					},
	%
	}

\pagestyle{fancy} %применим колонтитул
\fancyhead{} %очистим хидер на всякий случай
\fancyhead[R]{Сарафанов Ф.Г.} %номер страницы слева сверху на четных и справа на нечетных
\fancyhead[C]{Механика}
% \fancyhead[L]{Задача под запись - <<АУУ-2>>} 
\fancyfoot{} %футер будет пустой

\newcommand{\irodov}[1]{\fancyhead[L]{Иродов -- №#1}}
\newcommand{\yakovlev}[1]{\fancyhead[L]{Яковлев -- №#1}}
\newcommand{\wrote}[1]{\fancyhead[L]{Под запись -- <<#1>>}}

\newenvironment{tikzpict}
    {
	    \begin{figure}[htbp]
		\centering
		\begin{tikzpicture}
    }
    { 
		\end{tikzpicture}
		% \caption{caption}
		% \label{fig:label}
		\end{figure}
    }

\newcommand{\vbLabel}[3]{\draw ($(#1,#2)+(0,5pt)$) -- ($(#1,#2)-(0,5pt)$) node[below]{#3}}
\newcommand{\vaLabel}[3]{\draw ($(#1,#2)+(0,5pt)$) node[above]{#3} -- ($(#1,#2)-(0,5pt)$) }

\newcommand{\hrLabel}[3]{\draw ($(#1,#2)+(5pt,0)$) -- ($(#1,#2)-(5pt,0)$) node[right, xshift=1em]{#3}}
\newcommand{\hlLabel}[3]{\draw ($(#1,#2)+(5pt,0)$) node[left, xshift=-1em]{#3} -- ($(#1,#2)-(5pt,0)$) }

% Draw line annotation
% Input:
%   #1 Line offset (optional)
%   #2 Line angle
%   #3 Line length
%   #5 Line label
% Example:
%   \lineann[1]{30}{2}{$L_1$}
\newcommand{\lineann}[4][0.5]{%
    \begin{scope}[rotate=#2, blue,inner sep=2pt, ]
        \draw[dashed, blue!40] (0,0) -- +(0,#1)
            node [coordinate, near end] (a) {};
        \draw[dashed, blue!40] (#3,0) -- +(0,#1)
            node [coordinate, near end] (b) {};
        \draw[|<->|] (a) -- node[fill=white, scale=0.8] {#4} (b);
    \end{scope}
}


\irodov{1.339}

\begin{document}

\begin{tikzpict}

	\fill[magenta] (2,0)++(0,-0.5) rectangle ++ (0.25,-1.5);
	\fill[magenta] (2.75,0)++(0,-1.5) rectangle ++ (0.25,-0.5);
	\fill[magenta] (2,0)++(0,-2) rectangle ++ (1,0.25);

	\draw[black!30, decoration={random steps, amplitude=1.5mm}, decorate] (0,0) -- ++(0,2);
	\draw[black!30, decoration={random steps, amplitude=1.5mm}, decorate] (4,0) -- ++(0,2);
	\draw (0,2) -- ++(4,0); 
	\draw (0,0) -- ++(2,0) -- 
						++ 	(0,-2) 	--
						++ 	(1,0) 	--
						++ 	(0,2) 	--
						++	(1,0); 
	\draw (2.25,0) -- ++ (0,-1.75) -- ++ (0.5,0) -- ++ (0,2.75) -- ++(-0.5,0) coordinate (c);

	\draw (c) ++ (0,0.25) -- ++ (0.75,0) -- ++(0,-1.25);
	\draw (2.25,0) -- ++ (1,0);

	\draw[magenta] (1.5,1.12) circle (2pt) node[above, yshift=0.5em] {$A$};
	\draw[magenta] (2.4,1.12) circle (2pt) node[above, yshift=0.5em] {$B$};

	\draw[magenta,->] (0.25,1) -- ++(0.5,0);
	\draw[magenta,->] (0.25,0.7) -- ++(0.5,0);
	\draw[magenta,->] (0.25,1.3) -- ++(0.5,0);
	% Draw line annotation
	% Input:
	%   #1 Line offset (optional)
	%   #2 Line angle
	%   #3 Line length
	%   #5 Line label
	\begin{scope}[xshift=3cm, yshift=-1.5cm]
	  \lineann[2]{90}{1}{$\Delta h$}
	\end{scope}
\end{tikzpict}
Запишем уравнение Бернулли для для линии тока газа в точках $A$ и $B$:
\begin{equation}
	p_1+\frac{\rho v^2}{2}=p_2
\end{equation}
Здесь $p_1$ и $p_2$ -- давления на границах трубки Пито (и на соответствующих границах жидкости).
Запишем уравнение Бернулли для границ жидкости:
\begin{equation}
	p_1+\rho_0g\Delta h=p_2
\end{equation}
Из этих двух уравнений
\begin{equation}
	\rho_0g\Delta h=\frac{\rho v^2}{2}
\end{equation}
Откуда скорость потока в трубе
\begin{equation}
	v=\sqrt{2g\Delta h\frac{\rho_0}{\rho}}
\end{equation}
По определению, объемный расход
\begin{equation}
	Q=\frac{dV}{dt}=S\frac{dx}{dt}=Sv
\end{equation}
Тогда получаем окончательный ответ
\begin{equation}
	Q=S\sqrt{2g\Delta h\frac{\rho_0}{\rho}}
\end{equation}
\end{document}