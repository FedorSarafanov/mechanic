\documentclass[a5paper,10pt]{article}
 
% \usepackage{extsizes}
\usepackage{cmap}
\usepackage[T2A]{fontenc}
\usepackage[utf8x]{inputenc}
\usepackage[english, russian]{babel}

\usepackage{misccorr}

%%%%%%%%%%%%%%%%%%%%%%%%%%%%%%%%%%%%%%%%%%%%%%%%%%%%%%%%%%%%%%%%%%%%%%%%%%%%%%%%%%  
\usepackage{graphicx} % для вставки картинок
\graphicspath{{img/}}
\usepackage{amssymb,amsfonts,amsmath,amsthm} % математические дополнения от АМС

% \usepackage{fontspec}
% \usepackage{unicode-math}

\usepackage{indentfirst} % отделять первую строку раздела абзацным отступом тоже
\usepackage[usenames,dvipsnames]{color} % названия цветов
\usepackage{makecell}
\usepackage{multirow} % улучшенное форматирование таблиц
\usepackage{ulem} % подчеркивания
\linespread{1.3} % полуторный интервал
% \renewcommand{\rmdefault}{ftm} % Times New Roman (не работает)
\frenchspacing
\usepackage{geometry}
\geometry{left=1cm,right=1cm,top=2cm,bottom=1cm,bindingoffset=0cm}
\usepackage{titlesec}
\usepackage{float}
% \definecolor{black}{rgb}{0,0,0}
% \usepackage[colorlinks, unicode, pagecolor=black]{hyperref}
% \usepackage[unicode]{hyperref} %ссылки
\usepackage{fancyhdr} %загрузим пакет
\pagestyle{fancy} %применим колонтитул
\fancyhead{} %очистим хидер на всякий случай
\fancyhead[R]{Сарафанов Ф.Г.} %номер страницы слева сверху на четных и справа на нечетных
\fancyhead[C]{Механика}
\fancyhead[L]{Иродов -- №1.206} 
\fancyfoot{} %футер будет пустой
% \fancyfoot[CO,CE]{\thepage}
\renewcommand{\labelenumii}{\theenumii)}


\usepackage{tikz}
\usepackage{tikz-3dplot}
\usetikzlibrary{scopes}
\usetikzlibrary{%
     decorations.pathreplacing,%
     decorations.pathmorphing,%
    patterns,%
    calc,%
    scopes,%
    arrows,%
    arrows.meta,%
    % arrows.spaced,%
}

\tikzset{
    % MyPersp/.style={scale=1.8,x={(1.1cm,-0cm)},y={(0.5cm,1cm)}, z={(0cm,0.8cm)}},
 % MyPersp/.style={scale=1.5,x={(0cm,0cm)},y={(1cm,0cm)}, z={(0cm,1cm)}}, 
 % MyPersp/.style={scale=1.5,x={(1cm,0cm)},y={(0cm,1cm)}, z={(0cm,0cm)}}, 
    % MyPoints/.style={fill=black,draw=black},
    force/.style={>=latex,draw=blue,fill=blue},
    % angular/.style={-{Stealth[open, angle=30:5pt,line width=1pt]}, draw=magenta},
    angular/.style={-{Latex[length=3mm, line width=0.4pt,open,magenta, fill=white]}, draw=magenta},
    % axis/.style={densely dashed,gray,font=\small},
    axis/.style={densely dashed,black!60,font=\small},
    interface/.style={
        pattern = north east lines,
        draw    = none,
        pattern color=gray!60,          
    },
    plank/.style={
        fill=black!60, 
        draw=black,
        minimum width=3cm,
        inner ysep=0.1cm,
        outer sep=0pt,
        yshift=0.75cm,
        pattern = north east lines,
        pattern color=gray!60, 
    },
    cargo/.style={
        rectangle,
        fill=black!70,              
        inner sep=2.5mm,
    }
}
\begin{document}

\begin{figure}[H]
    \centering
    \vspace{-2em}
\begin{tikzpicture}[scale=2]
    \xdef\r{2}
    \draw (0,3) ellipse (2 and 0.4);
    \draw (0,0) -- ({\r},3) (0,0) -- (-\r,3);
    \draw[axis] (-2,3) -- (2,3);

    \draw[axis, rotate=-10] (0,0) ++(-0.7,1.02) arc (-110:-220:1.05cm and 0.7cm);
    \draw[axis, ->] (0,0) -- ++(0,3.45) node[above,black] {$+z$};

    \def\h{1.11}
    \def\hh{2.29}
    \draw[axis] (0,\hh) -- ++(-3,0);
    \draw[axis] (0,\h) -- ++(-2,0) node[left] {$W_p=0$};
    \draw[axis] (0,0) -- ++(-3,0);

    \draw[axis,<->] (-2,0) -- node[left, black] {$h_1$} ++(0,\h);
    \draw[axis,<->] (-3,0) -- node[left, black] {$h_2$} ++(0,\hh);

    \draw[fill=black] (-0.49,\h) coordinate (I)  circle (1pt);
    \draw[fill=black] (-0.75,\hh)  coordinate (II) circle (1pt);

    \draw[force,black!25, <-] (I)  -- node[below, black, pos=0.5, right] {$\vec{r}_1$} (0,0);%++(0.49,0.3); 
    \draw[force,black!25, <-] (II)  -- node[below, black, pos=0.5, right] {$\vec{r}_2$} (0,0);%++(0.75,-0.3); 

    \draw[force,black!20, <-] (I)  -- ++(0.49,0.3); 
    \draw[force,black!20, <-] (II)  -- ++(0.75,-0.3); 

    \draw[force,->] (I)  -- node[left, pos=1] {$\vec{v}_1$} ++(-1,0.3); 
    \draw[force,->] (II)  -- node[left, pos=1.7, yshift=-6pt] {$\vec{v}_1$} ++(0.5,0.2); 
 
    \draw[force,->] (I)  --  ++(0,-0.9) node[below, yshift=0.4em] {$m\vec{g}$}; 
    \draw[force,->] (II)  --  ++(0,-0.9) node[below, yshift=0.4em] {$m\vec{g}$}; 
    
    \draw[angular] (I)  -- node[right, yshift=-0.5em] {$\vec{N}_1$} ++(0,-0.5); 
    \draw[angular] (II)  -- node[right, yshift=-1.4em] {$\vec{N}_2$} ++(0,-0.5); 

    \draw[force,->] (I)  -- node[anchor=south west, yshift=1em] {$\vec{F}_{N1}$} ++(0.39,0.4); 
    \draw[force,->] (II)  -- node[anchor=south west, yshift=-1.5em] {$\vec{F}_{N2}$} ++(0.6,0.1); 

\end{tikzpicture}
\end{figure}
Так как в каждой точке конуса радиус-вектор от оси конуса, векторы силы тяжести и силы реакции опоры лежат в одной плоскости, то момент силы $\vec{M}=[\vec{r}\times(m\vec{g}+\vec{F}_N)]$ будет ей перпендикулярен и направлен горизонтально по касательной к конусу. 

Тогда проекция момента на ось конуса $z$:  $M_z=0$.
\begin{equation*}
    \begin{aligned}[c]
        \frac{dN_z}{dt}=M_z=0
    \end{aligned}
        \qquad\Longrightarrow\qquad
    \begin{aligned}[c]
        N_z=const
    \end{aligned}
\end{equation*}
Тогда
\begin{equation*}
\left\{
    \begin{aligned}[c]
        N_z={mv_1}r_1\sin\phi\\
        N_z={mv_2}r_2\sin\phi\\
    \end{aligned}\right.
        \qquad\Longrightarrow\qquad
    \begin{aligned}[c]
        v_2=v_1\frac{r_1}{r_2}=v_1\frac{h_1}{h_2}
    \end{aligned}
\end{equation*}
Можно записать З.С.М.Э., выбрав ноль потенциальной энергии
\begin{gather*}
\xdef\mv{\frac{mv_1^2}{2}}%
\xdef\mvv{\frac{mv_2^2}{2}}%
	\begin{aligned}[c]
        \mvv+mg(h_2-h_1)=\mv\\
        \mv-\mvv=mg(h_2-h_1)\\
	\end{aligned}
		\qquad\Longrightarrow\qquad
	\begin{aligned}[c]
        \frac{v_1^2}{2}(1-\frac{h_1^2}{h_2^2})=g(h_2-h_1)\\
        \frac{v_1^2}{2}\cdot\frac{h_2+h_1}{h_2^2}=g
	\end{aligned}\\
% \end{gather*}
% \begin{gather*}
        \frac{2g}{v_1^2}{h_2^2}-h_2-h_1=0\\
        h_2\text{(имеющий физический смысл)}=v_1^2\frac{1+\sqrt{1+\frac{8gh_1}{v_1^2}}}{4g}
\end{gather*}
\end{document}