\documentclass[a5paper,10pt]{article}
\def\source{/home/lab/tex/templates}

\usepackage{cmap}
\usepackage[T2A]{fontenc}
\usepackage[utf8x]{inputenc}
\usepackage[english, russian]{babel}

\usepackage
	{
		amssymb,
		% misccorr,
		amsfonts,
		amsmath,
		amsthm,
		wrapfig,
		makecell,
		multirow,
		indentfirst,
		ulem,
		graphicx,
		geometry,
		fancyhdr,
		subcaption,
		float,
		tikz,
		csvsimple,
		color,
	}  

\usepackage[outline]{contour}
\usepackage[mode=buildnew]{standalone}


\geometry
	{
		left=1cm,
		right=1cm,
		top=2cm,
		bottom=1cm,
		bindingoffset=0cm,
	}

\linespread{1.3} 
\frenchspacing 


\usetikzlibrary{scopes}
\usetikzlibrary
	{
		decorations.pathreplacing,
		decorations.pathmorphing,
		patterns,
		calc,
		scopes,
		arrows,
		through,
		shapes.misc,
		arrows.meta,
	}


\tikzset{
	force/.style=	{
		>=latex,
		draw=blue,
		fill=blue,
				 	}, 
	%				 	
	axis/.style=	{
		densely dashed,
		gray,
		font=\small,
					},
	%
	acceleration/.style={
		>=open triangle 60,
		draw=blue,
		fill=blue,
					},
	%
	inforce/.style=	{
		force,
		double equal sign distance=2pt,
					},
	%
	interface/.style={
		pattern = north east lines, 
		draw    = none, 
		pattern color=gray!60,
					},
	cross/.style=	{
		cross out, 
		draw=black, 
		minimum size=2*(#1-\pgflinewidth), 
		inner sep=0pt, outer sep=0pt,
					},
	%
	cargo/.style=	{
		rectangle, 
		fill=black!70, 
		inner sep=2.5mm,
					},
	%
	}

\pagestyle{fancy} %применим колонтитул
\fancyhead{} %очистим хидер на всякий случай
\fancyhead[R]{Сарафанов Ф.Г.} %номер страницы слева сверху на четных и справа на нечетных
\fancyhead[C]{Механика}
% \fancyhead[L]{Задача под запись - <<АУУ-2>>} 
\fancyfoot{} %футер будет пустой

\newcommand{\irodov}[1]{\fancyhead[L]{Иродов -- №#1}}
\newcommand{\yakovlev}[1]{\fancyhead[L]{Яковлев -- №#1}}
\newcommand{\wrote}[1]{\fancyhead[L]{Под запись -- <<#1>>}}

\newenvironment{tikzpict}
    {
	    \begin{figure}[htbp]
		\centering
		\begin{tikzpicture}
    }
    { 
		\end{tikzpicture}
		% \caption{caption}
		% \label{fig:label}
		\end{figure}
    }

\newcommand{\vbLabel}[3]{\draw ($(#1,#2)+(0,5pt)$) -- ($(#1,#2)-(0,5pt)$) node[below]{#3}}
\newcommand{\vaLabel}[3]{\draw ($(#1,#2)+(0,5pt)$) node[above]{#3} -- ($(#1,#2)-(0,5pt)$) }

\newcommand{\hrLabel}[3]{\draw ($(#1,#2)+(5pt,0)$) -- ($(#1,#2)-(5pt,0)$) node[right, xshift=1em]{#3}}
\newcommand{\hlLabel}[3]{\draw ($(#1,#2)+(5pt,0)$) node[left, xshift=-1em]{#3} -- ($(#1,#2)-(5pt,0)$) }

% Draw line annotation
% Input:
%   #1 Line offset (optional)
%   #2 Line angle
%   #3 Line length
%   #5 Line label
% Example:
%   \lineann[1]{30}{2}{$L_1$}
\newcommand{\lineann}[4][0.5]{%
    \begin{scope}[rotate=#2, blue,inner sep=2pt, ]
        \draw[dashed, blue!40] (0,0) -- +(0,#1)
            node [coordinate, near end] (a) {};
        \draw[dashed, blue!40] (#3,0) -- +(0,#1)
            node [coordinate, near end] (b) {};
        \draw[|<->|] (a) -- node[fill=white, scale=0.8] {#4} (b);
    \end{scope}
}


\irodov{1.268}

\begin{document}

\begin{tikzpict}
	\draw[interface] (-2,0) rectangle ++ (5,-0.5);
	\draw[] (-2,0) -- ++(5,0);

	\draw[interface] (3,-0.5) rectangle ++ (0.5,5);
	\draw[] (3,0) -- ++(0,4.5);

	\draw[] (1,2) coordinate (o) circle (2) node[magenta, scale=1.5] {$\bigotimes$};
	\draw (o) pic[<-, magenta,]{carc=100:180:1cm};

	\coordinate (1) at (3,2);
	\coordinate (2) at (1,0);

	\draw[force, thick, ->] (1) -- ++ (0,1.5) node[left] {$\vec{f}_1$};
	\draw[force, thick, ->] (2) -- ++ (1.5,0) node[above] {$\vec{f}_2$};

	\draw[force, thick, ->] (o) -- ++ (0,-0.65) node[left] {$m\vec{g}$};
	\draw[force, thick, ->] (2) -- ++ (0,0.65) node[left] {$\vec{N}_2$};
	\draw[force, thick, ->] (1) -- ++ (-0.65,0) node[left] {$\vec{N}_1$};

	\draw[axis,->] (-2,0) -- ++ (6,0) node[right] {$+x$};
	\draw[axis,->] (3,0) -- ++ (0,5) node[right] {$+y$};





	\draw[fill=white, draw=none] (1,2) coordinate (o) circle (2.7mm);
	\draw[] (1,2) coordinate (o) circle (2) node[magenta, scale=1.5] {$\bigotimes$};	

	\draw[] (4,2) node[magenta, scale=1.5] {$\bigodot$} node[right, xshift=0.5em] {$\vec{M}_1, \vec{M}_2$};	

\end{tikzpict}

Запишем второй закон Ньютона в проеции на оси $x$ и $y$:
\begin{equation}
	N_2-mg+f_1=0
\end{equation}
\begin{equation}
	f_2-N_1=0
\end{equation}
Т.к. $f_1=kN_1$, а $f_2=kN_2$, то решая эти два уравнения, найдём
\begin{equation}
	N_1=\frac{kmg}{k^2+1}, \quad
	N_2=\frac{mg}{k^2+1}
\end{equation}
\begin{equation}
	\vec{M}=\vec{M}_1+\vec{M}_2=[\vec{R}_1,\vec{f}_1]+[\vec{R}_2,\vec{f}_2]
\end{equation}
\begin{equation}
	M_z=-Rf_1+(-Rf_2)=-R(f_1+f_2)
\end{equation}
Тогда можем записать уравнение моментов (ось $z$ -- от нас):
\begin{equation}
	I\gamma_z=-kmgR	\left(
						\frac{1}{k^2+1}+
						\frac{k}{k^2+1}
					\right)
\end{equation}
Так как момент инерции нам известен ($I=\frac{mR^2}{2}$) то
\begin{equation}
	\gamma_z=-\frac{2kg}{R}
					\left(
						\frac{1}{k^2+1}+
						\frac{k}{k^2+1}
					\right)
\end{equation}
Интегрируя, найдем
\begin{equation}
	\int\limits_{\omega_0}^{\omega_z(t)}{d\omega_z}=
	\int\limits_0^t \gamma_z dt
\end{equation}
\begin{equation}
	\omega_z(t)=\omega_0+\gamma_z\cdot t
\end{equation}
Отсюда условие остановки:
\begin{equation}
	\omega_z(t_{stop})=0
\end{equation}
\begin{equation}
	t_{stop}=\frac{\omega_0}{\gamma_z}
\end{equation}
Так как
\begin{equation}
	\omega_z(t)=\frac{d\phi}{dt}
\end{equation}
\begin{equation}
	\int\limits_{0}^{\phi(t)}{d\phi}=
	\int\limits_0^t \omega_z(t) dt
\end{equation}
После второго интегрирования получим
\begin{equation}
	\phi(t)=\omega_0\cdot t+\gamma_z\cdot \frac{t^2}{2}
\end{equation}
Тогда количество оборотов выразится как
\begin{equation}
	n=\frac{|\phi(t=t_{stop})|}{2\pi}
\end{equation}
\begin{equation}
	n=\frac{\omega_0}{2\pi}\cdot\frac{\omega_0}{\gamma}
		-
		\gamma\cdot \frac{\omega^2_0}{4\pi\gamma^2}=
		\frac{\omega_0^2}{4\pi\gamma}
\end{equation}
\begin{equation}
	n=\frac{\omega_0^2}{4\pi\cdot
	\frac{2kg}{R}
					\left(
						\frac{1}{k^2+1}+
						\frac{k}{k^2+1}
					\right)
	}
\end{equation}
\begin{equation}
	n=\frac {\omega_0^2 R (k^2+1)}
			{8\pi kg(1+k)}
\end{equation}
\end{document}