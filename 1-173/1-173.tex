\documentclass[a5paper,10pt]{article}
 
\usepackage{extsizes}
\usepackage{cmap}
\usepackage[T2A]{fontenc}
\usepackage[utf8x]{inputenc}
\usepackage[english, russian]{babel}

\usepackage{misccorr}

%%%%%%%%%%%%%%%%%%%%%%%%%%%%%%%%%%%%%%%%%%%%%%%%%%%%%%%%%%%%%%%%%%%%%%%%%%%%%%%%%%  
\usepackage{graphicx} % для вставки картинок
\graphicspath{{img/}}
\usepackage{amssymb,amsfonts,amsmath,amsthm} % математические дополнения от АМС

% \usepackage{fontspec}
% \usepackage{unicode-math}

\usepackage{indentfirst} % отделять первую строку раздела абзацным отступом тоже
\usepackage[usenames,dvipsnames]{color} % названия цветов
\usepackage{makecell}
\usepackage{multirow} % улучшенное форматирование таблиц
\usepackage{ulem} % подчеркивания
\linespread{1.3} % полуторный интервал
% \renewcommand{\rmdefault}{ftm} % Times New Roman (не работает)
\frenchspacing
\usepackage{geometry}
\geometry{left=1cm,right=1cm,top=2cm,bottom=1cm,bindingoffset=0cm}
\usepackage{titlesec}
\usepackage{float}
% \definecolor{black}{rgb}{0,0,0}
% \usepackage[colorlinks, unicode, pagecolor=black]{hyperref}
% \usepackage[unicode]{hyperref} %ссылки
\usepackage{fancyhdr} %загрузим пакет
\pagestyle{fancy} %применим колонтитул
\fancyhead{} %очистим хидер на всякий случай
\fancyhead[LE,RO]{Сарафанов Ф.Г.} %номер страницы слева сверху на четных и справа на нечетных
\fancyhead[CO, CE]{Механика}
\fancyhead[LO,RE]{Иродов -- №1.173} 
\fancyfoot{} %футер будет пустой
% \fancyfoot[CO,CE]{\thepage}
\renewcommand{\labelenumii}{\theenumii)}


\usepackage{tikz}
\usetikzlibrary{scopes}
\usetikzlibrary{%
     decorations.pathreplacing,%
     decorations.pathmorphing,%
    patterns,%
    calc,%
    scopes,%
    arrows,%
    % arrows.spaced,%
}

\begin{document}

\begin{figure}[H]
    \centering
\begin{tikzpicture}[
    force/.style={>=latex,draw=blue,fill=blue},
    % axis/.style={densely dashed,gray,font=\small},
    axis/.style={densely dashed,black!60,font=\small},
    interface/.style={
        pattern = north east lines,
        draw    = none,
        pattern color=gray!60,          
    },
    cargo/.style={
        rectangle,
        fill=magenta!40,
        draw=black!50,
        inner sep=2.5mm,
    },
    spring/.style={
        decoration={
            aspect=0.3, 
            segment length=.8mm, 
            amplitude=2mm,
            coil},
        decorate,
        draw=magenta!25
    }
]

    \begin{scope}[yshift=0cm]
        \node[cargo] (b) at (0,0) {};
        \node[above, yshift=1em] at (b) {$m_1$};

        \node[cargo] (c) at (2.25,0) {};
        \node[above, yshift=1em] at (c) {$m_2$};

        % \draw[interface] (0,2.5) rectangle ++(-0.25,-5);
        \draw[spring, decoration={segment length=1mm}] (b) -- node[above, yshift=1em, xshift=-0.25em, black] {$\varkappa$} (c); 

        \draw[interface] (-0.25,-0.25) rectangle ++(-0.25,0.7);
        \draw[interface] (-0.5,-0.25) rectangle ++(5,-0.25);
        \draw[thick] (-0.25, -0.25) coordinate (left) -- ++(0,0.7) (left) -- ++(4.75,0);

        % \draw[axis,->] (-0.25,-0.7) -- ++(5,0) node[right] {$+x$};
        \draw[axis, <-] (-0.25,-0.7) node[left] {$+x$} -- ++(5,0) ;

        \draw (2.25,-0.8) -- ++ (0,0.2) node [below, yshift=-0.5em] {$x$};

        \draw (4,-0.8) -- ++ (0,0.2) node [below, yshift=-0.5em] {$0$};

        \draw[force,->] (b.west)++(0,-0.05) -- ++ (0.75,0) node[right, yshift=0.1em, xshift=-3pt] {$\vec{N}$};
        \draw[force,->] (b.east) -- ++(-0.75,0) node[left] {$\vec{f}'_e$};
        \draw[force,->] (c.west) -- ++(0.75,0) node[right] {$\vec{f}_e$};
    \end{scope}

    \begin{scope}[yshift=-2.25cm]
         \node[cargo] (b) at (0,0) {};
        \node[above, yshift=1em] at (b) {$m_1$};

        \node[cargo] (c) at (4,0) {};
        \node[above, yshift=1em] at (c) {$m_2$};

        % \draw[interface] (0,2.5) rectangle ++(-0.25,-5);
        \draw[spring, decoration={segment length=1.7mm}] (b) -- node[above, yshift=1em, xshift=-0.25em, black] {$\varkappa$} (c); 

        \draw[interface] (-0.25,-0.25) rectangle ++(-0.25,0.7);
        \draw[interface] (-0.5,-0.25) rectangle ++(5,-0.25);
        \draw[thick] (-0.25, -0.25) coordinate (left) -- ++(0,0.7) (left) -- ++(4.75,0);

        \draw[axis, <-] (-0.25,-0.7) node[left] {$+x$} -- ++(5,0) ;

        \draw (2.25,-0.8) -- ++ (0,0.2) node [below, yshift=-0.5em] {$x$};

        \draw (4,-0.8) -- ++ (0,0.2) node [below, yshift=-0.5em] {$0$};       
        \draw[force,->] (c.east) -- ++(0.75,0) node[right] {$\vec{v}_{20}$};

    \end{scope}
\end{tikzpicture}
% \vspace{-2em}
\end{figure}

Изначально второй груз удерживается, и пружина сжата на $x$.
На первый груз действует сила упругости $\vec{f}'_e$, прижимая его к стенке. Пока сила упругости не станет равна нулю, сдвинуться $m_1$ с места не сможет.

Все силы консервативные. По закону сохранения механической энергии:
\begin{equation}
\label{v20}
\frac{m_2v_{20}^2}{2}=\frac{\varkappa x^2}{2}
\end{equation}

Когда пружина распрямится, скорость $m_2$ будет $v_{20}$, а $m_1$ -- равна нулю.

Тогда запишем:
\begin{equation}
    \label{vc}
    \vec{v}_c=\frac{m_1\cdot0+m_2\vec{v}_{20}}{m_1+m_2}
\end{equation}
Выразим из (\ref{v20}) $v_{20}$ и подставим в (\ref{vc}), перейдя к модулям векторов (можно обойтись без проекций, так как в уравнении (\ref{vc}) связаны два вектора сугубо положительным множителем):
\begin{equation}
    v_c=\sqrt\frac{\varkappa x^2}{m_2}\cdot \frac{m_2}{m_1+m_2}
\end{equation}
Тогда можем окончательно записать ответ:
\begin{equation}
    v_c=\frac{x\sqrt{\varkappa \cdot m_2}}{m_1+m_2}
\end{equation}

\end{document}