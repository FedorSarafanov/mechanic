\documentclass[a5paper,10pt]{article}
 
\usepackage{extsizes}
\usepackage{cmap}
\usepackage[T2A]{fontenc}
\usepackage[utf8x]{inputenc}
\usepackage[english, russian]{babel}

\usepackage{misccorr}

%%%%%%%%%%%%%%%%%%%%%%%%%%%%%%%%%%%%%%%%%%%%%%%%%%%%%%%%%%%%%%%%%%%%%%%%%%%%%%%%%%  
\usepackage{graphicx} % для вставки картинок
\graphicspath{{img/}}
\usepackage{amssymb,amsfonts,amsmath,amsthm} % математические дополнения от АМС

% \usepackage{fontspec}
% \usepackage{unicode-math}

\usepackage{indentfirst} % отделять первую строку раздела абзацным отступом тоже
\usepackage[usenames,dvipsnames]{color} % названия цветов
\usepackage{makecell}
\usepackage{multirow} % улучшенное форматирование таблиц
\usepackage{ulem} % подчеркивания
\linespread{1.3} % полуторный интервал
% \renewcommand{\rmdefault}{ftm} % Times New Roman (не работает)
\frenchspacing
\usepackage{geometry}
\geometry{left=1cm,right=1cm,top=2cm,bottom=1cm,bindingoffset=0cm}
\usepackage{titlesec}
\usepackage{float}
% \definecolor{black}{rgb}{0,0,0}
% \usepackage[colorlinks, unicode, pagecolor=black]{hyperref}
% \usepackage[unicode]{hyperref} %ссылки
\usepackage{fancyhdr} %загрузим пакет
\pagestyle{fancy} %применим колонтитул
\fancyhead{} %очистим хидер на всякий случай
\fancyhead[R]{Сарафанов Ф.Г.} %номер страницы слева сверху на четных и справа на нечетных
\fancyhead[C]{Механика}
\fancyhead[L]{Иродов 1.88} 
\fancyfoot{} %футер будет пустой
% \fancyfoot[CO,CE]{\thepage}
\renewcommand{\labelenumii}{\theenumii)}


\usepackage{tikz}
\usetikzlibrary{scopes}
\usetikzlibrary{%
	 decorations.pathreplacing,%
	 decorations.pathmorphing,%
	patterns,%
	calc,%
	scopes,%
	arrows,%
	through,%
	% arrows.spaced,%
}
\newcommand{\vangle}{\mathop{\mathstrut^\wedge}\nolimits}
\newcommand{\average}[1]{\langle{#1}\rangle}
\usepackage{wrapfig}
\newcommand{\RN}[1]{%
  \textup{\tiny\uppercase\expandafter{\romannumeral#1}}%
}
\begin{document}

\begin{figure}[H]
	\centering
\begin{tikzpicture}[
	force/.style={>=latex,draw=blue,fill=blue},
	acceleration/.style={>=open triangle 60,draw=blue,fill=blue},
	% axis/.style={densely dashed,gray,font=\small},
	axis/.style={densely dashed,black!60,font=\small},
	M/.style={rectangle,draw,fill=lightgray,minimum size=0.5cm,thin},
	m2/.style={draw=black!30, rectangle,draw,thin, fill=blue!2, minimum width=0.7cm,minimum height=0.7cm},
	m1/.style={draw=black!30, rectangle,draw,thin, fill=blue!2, minimum width=0.7cm,minimum height=0.7cm},
	plane/.style={draw=black!30, very thick, fill=blue!5, line width=1pt},
	% base/.style={draw=black!70, very thick, fill=blue!4, line width=2pt},
	string/.style={draw=black, thick},
	pulley/.style={thick},
	interface1/.style={draw=gray!60,
		% The border decoration is a path replacing decorator. 
		% For the interface style we want to draw the original path.
		% The postaction option is therefore used to ensure that the
		% border decoration is drawn *after* the original path.
		postaction={draw=gray!60,decorate,decoration={border,angle=-135,
					amplitude=0.3cm,segment length=2mm}}},
	interface/.style={
		pattern = north east lines,
		draw    = none,
		pattern color=gray!60,          
	},
	plank/.style={
		fill=black!60, 
		draw=black,
		minimum width=3cm,
		inner ysep=0.1cm,
		outer sep=0pt,
		yshift=0.75cm,
		pattern = north east lines,
		pattern color=gray!60, 
	},
	cargo/.style={
		rectangle,
		fill=black!70,              
		inner sep=2.5mm,
	}
]
	\draw (0,0) arc(-90:90:3cm);
	\draw[dotted](0,0) arc(-90:-270:3cm);
	\draw (0,0) -- +(-5,0);
	\draw[dotted] (0,0) -- +(5,0);

	\coordinate (0) at (0,0);
	\coordinate (I) at (3,3);
	\coordinate (II) at (0,6);
	\coordinate (c) at (0,3);

	\draw[fill=black] (c) circle (1.25pt) (I) circle (1.25pt) (II) circle (1.25pt) (0) circle (1.25pt);
	\draw[axis] (c) -- (I) (c) -- (II) (c) -- (0);

	\draw[force,->] (0) -- ++(1,0) node[below] {$\vec{v}$};
	\draw[force,->] (I) -- ++(0,1) node[right] {$\vec{v}$};
	\draw[force,->] (II) -- ++(-1,0) node[above] {$\vec{v}$};

	\draw[force,->] (0) -- ++(0,-0.5) node[right] {$m\vec{g}$};
	\draw[force,->] (I) -- ++(0,-0.5) node[left] {$m\vec{g}$};
	\draw[force,->] (II) -- ++(0,-0.5) node[right] {$m\vec{g}$};

	\draw[force,->] (0) -- ++(0,1) node[right] {$\vec{N}_1$};
	\draw[force,axis,->] (I) -- ++(-1,0) node[below] {$\vec{N}_{2n}$};
	\draw[force,axis,->] (I) -- ++(0,0.5) node[right] {$\vec{N}_{2\tau}$};
	\draw[force,axis,->] (I) -- ++(-1,0.5) node[above] {$\vec{N}_{2}$};
	\draw[force,->] (II) -- ++(0,-1) node[right] {$\vec{N}_3$};

	% \draw[force, axis, ->] (0) -- ++(0,0.5) node[left] {$\vec{Q}_1$};
	% \draw[force, axis, ->] (I) -- ++(-1,-0.5) node[below] {$\vec{Q}_2$};
	% \draw[force, axis, ->] (II) -- ++(0,-1.5) node[right] {$\vec{Q}_3$};

	\draw[force, axis, ->] (0) -- ++(0,-1.5) node[left] {$\vec{P}_1$};
	\draw[force, axis, ->] (I) -- ++(1,-0.5) node[above] {$\vec{P}_2$};
	\draw[force, axis, ->] (II) -- ++(0,0.5) node[right] {$\vec{P}_3$};


\end{tikzpicture}
\end{figure}
Введем мгновенные нормальную ось $n$, направленную к центру окружности, и тангенциальную ось $\tau$.
\begin{equation*}
	v=const \Longrightarrow a=a_\tau=\frac{v^2}{R}
\end{equation*}
\begin{equation*}
	\begin{aligned}[c]
		\textbf{I.  }
		m\vec{a}=\vec{N}_3+m\vec{g}\\
		\text{n: }ma_n=N_3+mg\\
		P_3=N_3\\
		P_3=ma_n-mg=\\
		=m(\frac{v^2}{R}-g)=700\text{\,H}
	\end{aligned}
		% \qquad\Longrightarrow\qquad
		\qquad\qquad
	\begin{aligned}[c]
	\textbf{II.  }
	m\vec{a}=\vec{N}_2+m\vec{g}\\
	\text{n: }ma_n=N_{2n}\\
	\text{$\tau$: }mg=N_{2\tau}\\
	P_2=N_2=\sqrt{N_{2n}^2+N_{2\tau}^2}=\\
	m\sqrt{\frac{v^4}{R^2}+g^2}=1565\text{\,H}
	\end{aligned}
\end{equation*}
\begin{gather*}
	\textbf{III.  }
	m\vec{a}=\vec{N}_1+m\vec{g}\\
	\text{n: }ma_n=N_{1n}-mg\\
	P_1=|N_{1n}|=ma_n+mg=m(\frac{v^2}{R}+g)=2100\text{\,H}
\end{gather*}

\end{document}