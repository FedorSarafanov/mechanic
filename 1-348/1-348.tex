\documentclass[a5paper,10pt]{article}
\def\source{/home/lab/tex/templates}

\usepackage{cmap}
\usepackage[T2A]{fontenc}
\usepackage[utf8x]{inputenc}
\usepackage[english, russian]{babel}

\usepackage
	{
		amssymb,
		% misccorr,
		amsfonts,
		amsmath,
		amsthm,
		wrapfig,
		makecell,
		multirow,
		indentfirst,
		ulem,
		graphicx,
		geometry,
		fancyhdr,
		subcaption,
		float,
		tikz,
		csvsimple,
		color,
	}  

\usepackage[outline]{contour}
\usepackage[mode=buildnew]{standalone}


\geometry
	{
		left=1cm,
		right=1cm,
		top=2cm,
		bottom=1cm,
		bindingoffset=0cm,
	}

\linespread{1.3} 
\frenchspacing 


\usetikzlibrary{scopes}
\usetikzlibrary
	{
		decorations.pathreplacing,
		decorations.pathmorphing,
		patterns,
		calc,
		scopes,
		arrows,
		through,
		shapes.misc,
		arrows.meta,
	}


\tikzset{
	force/.style=	{
		>=latex,
		draw=blue,
		fill=blue,
				 	}, 
	%				 	
	axis/.style=	{
		densely dashed,
		gray,
		font=\small,
					},
	%
	acceleration/.style={
		>=open triangle 60,
		draw=blue,
		fill=blue,
					},
	%
	inforce/.style=	{
		force,
		double equal sign distance=2pt,
					},
	%
	interface/.style={
		pattern = north east lines, 
		draw    = none, 
		pattern color=gray!60,
					},
	cross/.style=	{
		cross out, 
		draw=black, 
		minimum size=2*(#1-\pgflinewidth), 
		inner sep=0pt, outer sep=0pt,
					},
	%
	cargo/.style=	{
		rectangle, 
		fill=black!70, 
		inner sep=2.5mm,
					},
	%
	}

\pagestyle{fancy} %применим колонтитул
\fancyhead{} %очистим хидер на всякий случай
\fancyhead[R]{Сарафанов Ф.Г.} %номер страницы слева сверху на четных и справа на нечетных
\fancyhead[C]{Механика}
% \fancyhead[L]{Задача под запись - <<АУУ-2>>} 
\fancyfoot{} %футер будет пустой

\newcommand{\irodov}[1]{\fancyhead[L]{Иродов -- №#1}}
\newcommand{\yakovlev}[1]{\fancyhead[L]{Яковлев -- №#1}}
\newcommand{\wrote}[1]{\fancyhead[L]{Под запись -- <<#1>>}}

\newenvironment{tikzpict}
    {
	    \begin{figure}[htbp]
		\centering
		\begin{tikzpicture}
    }
    { 
		\end{tikzpicture}
		% \caption{caption}
		% \label{fig:label}
		\end{figure}
    }

\newcommand{\vbLabel}[3]{\draw ($(#1,#2)+(0,5pt)$) -- ($(#1,#2)-(0,5pt)$) node[below]{#3}}
\newcommand{\vaLabel}[3]{\draw ($(#1,#2)+(0,5pt)$) node[above]{#3} -- ($(#1,#2)-(0,5pt)$) }

\newcommand{\hrLabel}[3]{\draw ($(#1,#2)+(5pt,0)$) -- ($(#1,#2)-(5pt,0)$) node[right, xshift=1em]{#3}}
\newcommand{\hlLabel}[3]{\draw ($(#1,#2)+(5pt,0)$) node[left, xshift=-1em]{#3} -- ($(#1,#2)-(5pt,0)$) }

% Draw line annotation
% Input:
%   #1 Line offset (optional)
%   #2 Line angle
%   #3 Line length
%   #5 Line label
% Example:
%   \lineann[1]{30}{2}{$L_1$}
\newcommand{\lineann}[4][0.5]{%
    \begin{scope}[rotate=#2, blue,inner sep=2pt, ]
        \draw[dashed, blue!40] (0,0) -- +(0,#1)
            node [coordinate, near end] (a) {};
        \draw[dashed, blue!40] (#3,0) -- +(0,#1)
            node [coordinate, near end] (b) {};
        \draw[|<->|] (a) -- node[fill=white, scale=0.8] {#4} (b);
    \end{scope}
}


\irodov{1.348}

\begin{document}

\begin{tikzpict}
	\draw[interface] (-1,0) rectangle ++(4,-0.25);
	\draw[thick] (-1,0) -- ++(4,0);

	\fill[magenta!10, draw=none] (0,0) rectangle ++(2,2.75);
	\draw (0,3.2)--(0,0)--(2,0)--(2,3.2);

	\draw[white] (2,0.5) -- ++(0,0.4);
	\draw[white] (0,1.5) -- ++(0,0.4);

% Draw line annotation
% Input:
%   #1 Line offset (optional)
%   #2 Line angle
%   #3 Line length
%   #5 Line label
% Example:
\begin{scope}[yshift=0.7cm]
	\lineann[-2]{90}{1}{$\Delta h$}
\end{scope}

	\fill[magenta!10] (-0.4,1.5) rectangle ++(-0.4,0.4);
	\fill[magenta!10] (2.8,0.5) rectangle ++(-0.4,0.4);


	\draw[force,->] (-0.6,1.7) -- ++ (-1,0) node[left] {$\vec{v}_1$};
	\draw[force,->] (2.6,0.7) -- ++ (1,0) node[right] {$\vec{v}_2$};

	\draw[force,->] (0,1.7) -- ++ (1,0) node[right, above] {$\vec{F}_{r1}$};
	\draw[force,->] (2,0.7) -- ++ (-1,0) node[left, below] {$\vec{F}_{r2}$};

	\draw[axis] (-1,-0.5) -- ++(4,0) node[right] {$+x$};
\end{tikzpict}
Согласно формуле Торричелли, скорости истечения из малых отверстий будут:
\begin{equation}
	v_1=\sqrt{2gh_1},\qquad v_2=\sqrt{2gh_2} %\quad (h_1-h_2=\Delta h)
\end{equation}
Рассмотрим нижнее отверстие (высота от верхнего уровня жидкости $h_2$). За время $dt$ из него выливается объем $dV_2$ со скоростью $v_2$. Его импульс
\begin{equation}
	dp_2=dm_2\cdot v_2
\end{equation}
Но 
\begin{equation}
	dm_2=\rho dV_2 = \rho Sv_2\cdot dt
\end{equation}
Тогда
\begin{equation}
	dp_2=\rho Sv^2_2\cdot dt
\end{equation}
Возникает сила реакции отделяющейся жидкости (реактивная тяга). Точка приложения -- центр истечения жидкости. Из уравнения Мещерского очевидно, что направление силы противоположно направлению истечения жидкости.

Найдем силу для нижнего отверстия:
\begin{equation}
	F_{r2}=\frac{dp_2}{dt}=\rho Sv^2_2
\end{equation}
Аналогично
\begin{equation}
	F_{r1}=\frac{dp_1}{dt}=\rho Sv^2_1
\end{equation}
Тогда суммарная сила будет:
\begin{equation}
	F_x=F_{r1}-F_{r2}=\rho S(v_2^2-v_1^2)=
	2\rho gS\Delta h 
\end{equation}
\end{document}