\documentclass[a5paper,10pt]{article}
 
\usepackage{extsizes}
\usepackage{cmap}
\usepackage[T2A]{fontenc}
\usepackage[utf8x]{inputenc}
\usepackage[english, russian]{babel}

\usepackage{misccorr}

%%%%%%%%%%%%%%%%%%%%%%%%%%%%%%%%%%%%%%%%%%%%%%%%%%%%%%%%%%%%%%%%%%%%%%%%%%%%%%%%%%  
\usepackage{graphicx} % для вставки картинок
\graphicspath{{img/}}
\usepackage{amssymb,amsfonts,amsmath,amsthm} % математические дополнения от АМС

% \usepackage{fontspec}
% \usepackage{unicode-math}

\usepackage{indentfirst} % отделять первую строку раздела абзацным отступом тоже
\usepackage[usenames,dvipsnames]{color} % названия цветов
\usepackage{makecell}
\usepackage{multirow} % улучшенное форматирование таблиц
\usepackage{ulem} % подчеркивания
\linespread{1.3} % полуторный интервал
% \renewcommand{\rmdefault}{ftm} % Times New Roman (не работает)
\frenchspacing
\usepackage{geometry}
\geometry{left=1cm,right=1cm,top=2cm,bottom=1cm,bindingoffset=0cm}
\usepackage{titlesec}
\usepackage{float}
% \makeatletter
\newif\if@gather@prefix 
\preto\place@tag@gather{% 
  \if@gather@prefix\iftagsleft@ 
    \kern-\gdisplaywidth@ 
    \rlap{\gather@prefix}% 
    \kern\gdisplaywidth@ 
  \fi\fi 
} 
\appto\place@tag@gather{% 
  \if@gather@prefix\iftagsleft@\else 
    \kern-\displaywidth 
    \rlap{\gather@prefix}% 
    \kern\displaywidth 
  \fi\fi 
  \global\@gather@prefixfalse 
} 
\preto\place@tag{% 
  \if@gather@prefix\iftagsleft@ 
    \kern-\gdisplaywidth@ 
    \rlap{\gather@prefix}% 
    \kern\displaywidth@ 
  \fi\fi 
} 
\appto\place@tag{% 
  \if@gather@prefix\iftagsleft@\else 
    \kern-\displaywidth 
    \rlap{\gather@prefix}% 
    \kern\displaywidth 
  \fi\fi 
  \global\@gather@prefixfalse 
} 
\newcommand*{\beforetext}[1]{% 
  \ifmeasuring@\else
  \gdef\gather@prefix{#1}% 
  \global\@gather@prefixtrue 
  \fi
} 
\makeatother
% \definecolor{black}{rgb}{0,0,0}
% \usepackage[colorlinks, unicode, pagecolor=black]{hyperref}
% \usepackage[unicode]{hyperref} %ссылки
\usepackage{fancyhdr} %загрузим пакет
\pagestyle{fancy} %применим колонтитул
\fancyhead{} %очистим хидер на всякий случай
\fancyhead[R]{Сарафанов Ф.Г.} %номер страницы слева сверху на четных и справа на нечетных
\fancyhead[C]{Механика}
\fancyhead[L]{Иродов -- №1.85} 
\fancyfoot{} %футер будет пустой
% \fancyfoot[CO,CE]{\thepage}
\renewcommand{\labelenumii}{\theenumii)}


\usepackage{tikz}
\usetikzlibrary{scopes}
\usetikzlibrary{%
     decorations.pathreplacing,%
     decorations.pathmorphing,%
    patterns,%
    calc,%
    scopes,%
    positioning,%
    shapes,%
    arrows,%
    % arrows.spaced,%
}

\tikzset{%
  raindrop/.pic={
    code={\tikzset{scale=1/10}
 \clip [preaction={top color=blue!50!cyan!50, bottom color=blue!50!cyan}]
 (0,0)  .. controls ++(0,-1) and ++(0,1) .. (1,-2)
 arc (360:180:1)
 .. controls ++(0,1) and ++(0,-1) .. (0,0) -- cycle;
% %
 \foreach \j in {1,3,...,20}
 \shade [top color=blue!50!cyan!50, shift=(270:0), xscale=1-\j/40,yscale=1-\j/80, white, opacity=1/15]
 [rotate=-\j] (0,0)  .. controls ++(0,-1) and ++(0,1) .. (1,-2)
 arc (360:180:1)
 .. controls ++(0,1) and ++(0,-1) .. (0,0) -- cycle;
  }}}

\begin{document}

\begin{figure}[H]
    \centering
\begin{tikzpicture}[
    force/.style={>=latex,draw=blue,fill=blue},
    % axis/.style={densely dashed,gray,font=\small},
    axis/.style={densely dashed,black!60,font=\small},
    M/.style={rectangle,draw,fill=lightgray,minimum size=0.5cm,thin},
    m2/.style={draw=black!30, rectangle,draw,thin, fill=blue!2, minimum width=0.7cm,minimum height=0.7cm},
    m1/.style={draw=black!30, rectangle,draw,thin, fill=blue!2, minimum width=0.7cm,minimum height=0.7cm},
    plane/.style={draw=black!30, very thick, fill=blue!5, line width=1pt},
    % base/.style={draw=black!70, very thick, fill=blue!4, line width=2pt},
    string/.style={draw=black, thick},
    pulley/.style={thick},
    % interface/.style={draw=gray!60,
    %     % The border decoration is a path replacing decorator. 
    %     % For the interface style we want to draw the original path.
    %     % The postaction option is therefore used to ensure that the
    %     % border decoration is drawn *after* the original path.
    %     postaction={draw=gray!60,decorate,decoration={border,angle=-135,
    %                 amplitude=0.3cm,segment length=2mm}}},
    interface/.style={
        pattern = north east lines,
        draw    = none,
        pattern color=gray!60,          
    },
    plank/.style={
        fill=black!60, 
        draw=black,
        minimum width=3cm,
        inner ysep=0.1cm,
        outer sep=0pt,
        yshift=0.75cm,
        pattern = north east lines,
        pattern color=gray!60, 
    },
    cargo/.style={
        rectangle,
        fill=black!70,              
        inner sep=2.5mm,
    }
]
\draw[draw=none, fill=black!10] (0,-2) rectangle (3,2);
    \draw[thin] (0,-2) -- (0,2);
    \draw[thin] (3,-2) -- (3,2);
    \draw[axis,<->] (0,2.1) -- node[above, black] {$h$} (3,2.1);
    \draw[axis,->] (0,0) -- ++(5,0) node [right] {$x$};
    \draw[force,->] (0,0) -- ++(1,0) node[above] {$\vec{v}_0$};
    \draw[force,->] (0,0) -- ++(-1,0) node[above] {$\vec{f}_R$};
    % \draw[fill=white] (0,0) ++(-0.2,0.2) rectangle ++(0.4,-0.4);
    \draw[fill=white] (0,0) circle (3pt);
    \draw[force,->] (3,0) -- ++(1,0) node[above] {$\vec{v}$};
    \draw[fill=white] (3,0) circle (3pt);


\end{tikzpicture}
% \vspace{-2em}
\end{figure}

Введем ось $x$, сонаправленную с вектором начальной скорости.
Запишем второй закон Ньютона:
\begin{equation*}
    \xdef\km{\frac{k}{m}}\xdef\mk{\frac{m}{k}}\xdef\mgk{\frac{mg}{k}}%
    \begin{aligned}[c]
        m\vec{a}&=-kv^2\frac{\vec{v}}{v}\\
        \text{x: } ma_x&=-kv^2_x\\
    \end{aligned}
        \qquad\qquad
    \begin{aligned}[c]
        \frac{dv_x}{dt}=-\km{}v^2_x\\
        -\mk\frac{dv_x}{v^2_x}=dt\\
    \end{aligned}
\end{equation*}
Можем найти $v_x(t)$ и $x(t)$:
\begin{equation*}
    \begin{aligned}[c]
        \int_0^{t}dt=-\mk\int_{v_0}^{v_x}\frac{dv_x}{v^2_x}\\
        t=\mk\frac{1}{v_x}\bigg|_{v_0}^{v_x}=\mk\left(\frac{1}{v_x}- \frac{1}{v_0}\right)\\
        \frac{m}{kv_x}=t+\frac{m}{kv_0}\\
        \frac{1}{v_x}=\km{t}+\frac{1}{v_0}\\
        v_x=\frac{v_0}{\frac{kv_0}{m}t+1}
    \end{aligned}
        \qquad\qquad
    \begin{aligned}[c]
        \int_0^{h}dx=\int_{0}^{t}\left[\frac{v_0}{\frac{kv_0}{m}t+1}\right]dt\\
        \int_0^{h}dx=\frac{m}{k}\int_{0}^{t}\left[\frac{d[\frac{kv_0}{m}t+1]}{\frac{kv_0}{m}t+1}\right]\\
        h=\mk\ln(\frac{kv_0}{m}t+1)=\mk\ln\frac{v_0}{v}\\
        \mk=h/\ln\frac{v_0}{v}
    \end{aligned}
\end{equation*}
Заметим, что время можно выразить через найденные $\mk$ и $v_x$, $v_0$:
\begin{gather*}
        t=\mk\left(\frac{1}{v_x}- \frac{1}{v_0}\right)=\mk\frac{v_0-v_x}{v_0v_x}\\
        t=\frac{h\cdot(v_0-v_x)}{\ln\frac{v_0}{v_x}\cdot{v_0}\cdot{v_x}}
\end{gather*}


\end{document}