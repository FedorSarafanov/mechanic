\documentclass[a5paper,10pt]{article}\usepackage[usenames,dvipsnames]{color}\usepackage{extsizes,cmap,graphicx,misccorr,indentfirst,makecell,multirow,ulem,geometry,amssymb,amsfonts,amsmath,amsthm,titlesec,float,fancyhdr,wrapfig,tikz}\usepackage[T2A]{fontenc}\usepackage[utf8x]{inputenc}\usepackage[english, russian]{babel}\usetikzlibrary{decorations.pathreplacing,decorations.pathmorphing,patterns,calc,scopes,arrows,through,positioning,shapes.misc}\graphicspath{{img/}}\linespread{1.3}\frenchspacing\geometry{left=1cm, right=1cm, top=2cm, bottom=1cm, bindingoffset=0cm}\pagestyle{fancy}\fancyhead{}\fancyhead[R]{Сарафанов Ф.Г.} 
\fancyhead[C]{Механика}
\fancyhead[L]{№109 -- Яковлев И.А.} 
\fancyfoot{}
\renewcommand{\labelenumii}{\theenumii)}
\tikzset{
	force/.style={>=latex,draw=blue,fill=blue,>=triangle 45},
    axis/.style={densely dashed,black!60,font=\small},
    interface1/.style={draw=gray!60,.
        postaction={draw=gray!60,decorate,decoration={border,angle=-135,
        amplitude=0.3cm,segment length=2mm}}},
    interface/.style={
        pattern = north east lines,
        draw    = none,
        pattern color=gray!60,          
    },
    plank/.style={
        fill=black!60, 
        draw=black,
        minimum width=3cm,
        inner ysep=0.1cm,
        outer sep=0pt,
        yshift=0.75cm,
        pattern = north east lines,
        pattern color=gray!60, 
    },
    cargo/.style={
        rectangle,
        fill=black!70,              
        inner sep=2.5mm,
    }	
}
\begin{document}

\begin{wrapfigure}[13]{l}{0.5\textwidth}
    % \centering
\begin{tikzpicture}
	\def\angle{50}
	% \draw (0,2) coordinate (o) circle (2); 
	% \draw (o) circle (0.5); 
	% \draw (0,0) -- (5,6);
	\draw[interface] (0,6.25) rectangle (6,6);
	\draw[thick] (0,6) --(2.85,6) (3.15,6) --(6,6);
    \draw[interface] (0,6.25) rectangle (6,6);
    \draw [interface, draw=black]
        (2.85,6) rectangle (3.15,4);

    \def\rr{0.8}
    % \draw[axis] (1.2,-3.15) --(1.20,4);
    \draw[axis, ->] (0,4)  -- ++(0,-4) node[below] {$x$};

    % \draw[axis] (3,4) ++ (1.8,0) arc (0:180:1.8); 
    
    \draw[fill=white] (3,4) circle (\rr);
    \path (3,4) -- +(-\rr,0) coordinate (Cl) + (\rr,0) coordinate (Cr);
    \draw[fill=black] (3,4) circle (3pt);
    \draw[black!70] (Cr) -- ++ (0,-3) coordinate (B);
    \draw[black!70] (Cl) -- ++ (0,-3) coordinate (2m);

    \draw[force, ->] (2m) -- ++(0,1) node[left] {$\vec{T}_1$};
    \draw[force, ->] (Cl) -- ++(0,-1) node[left] {$\vec{T}'_1$};
    \draw[force, ->] (Cr) -- ++(0,-1) node[right] {$\vec{T}'_B$};
    \draw[force, ->] (B) -- ++(0,1) node[right] {$\vec{T}_B$};



    \draw[force,->] (2m)-- node[left] {$\vec{u}$} ++(0,0.5);
    \draw[] (B) circle (\rr) node[right] {$B$};
    \path (B) -- +(-\rr,0) coordinate (Bl) + (\rr,0) coordinate (Br);
    \draw[fill=black] (B) circle (3pt);

    \draw[fill=black] (2m) rectangle ++(0.2,-0.6) rectangle ++(-0.4,0.6) node[left] {$2m$};

    \draw[force, ->] (Br) -- ++(0,-0.7) node[right] {$\vec{T}'_3$};
    \draw[force, ->] (Bl) -- ++(0,-0.7) node[left] {$\vec{T}'_2$};




    \draw[black!70] (Br) -- ++ (0,-3) coordinate (o);
    \draw[black!70] (Bl) -- ++ (0,-2) coordinate (m);
    \draw[force,->] (m)-- node[anchor=north east] {$\vec{u}$} ++(0,-1.5);


    \draw[force, ->] (m) -- ++(0,0.5) node[left] {$\vec{T}_2$};
    \draw[force, ->] (o)++(0,0.5) -- ++(0,0.5) node[right] {$\vec{T}_3$};

    \draw[fill=black] (m) rectangle ++(0.1,-0.6) rectangle ++(-0.2,0.6) node[left] {$m$};

    \draw[force,->] ($(o)-(0.2,0)$)--node[left] {$\vec{v}$} ($(o)-(0.2,-1.2)$);
    \node[inner sep=0pt, scale=0.2, xshift=-2em] (russell) at (o)
    {\includegraphics[width=.25\textwidth]{img/monkey.png}}; 
    \node[right, xshift=1em] at (o) {$m$};

    % \draw[force,->] ($(o)-(0.7,1.2)$)--node[left] {$\vec{v}$} ($(o)-(0.7,-1.2)$);


\end{tikzpicture}
\end{wrapfigure}

Из невесомости блоков, III з.Н. и проч.:
\begin{align*}
    T_2=T'_2=T'_3=T_3\equiv{T}\\
    T_1=T'_1=T'_B=T_B\equiv{2T}
\end{align*}
Запишем II з.Н. в проеции на ось $x$:
\begin{align*}
    2m\cdot{a_{1_X}}=2m\cdot{g}-2T\\
    m\cdot{a_{2_X}}=m\cdot{g}-T
\end{align*}
Отсюда $a_{1_X}=a_{2_X}$.
Т.к. нить нерастяжимая, $a_{B_X}=-a_{1_X}$, 
\begin{align*}
    {a_{3_X}}={a'_{3_X}}+a_{B_X}\\
    {a_{2_X}}={a'_{_X}}+a_{B_X}\\
    -a_{B_X}=a_{2_X}\\
    2a_{2_X}=a'_{2_X}\\
\end{align*}
\begin{align*}
    L_2=x'_2+x'_3\\
    L_2/dt=v'_{2_X}+v'_{3_X}\\
    -v=v'_{2_X}+v'_{3_X}\\
\end{align*}
Из нерастяжимости нити $a_{3_X}=a_{2_X}$, тогда
\begin{equation*}
    \begin{aligned}[c]
    a_{B_X}+a'_{3_X}=a'_{2_X}+a_{B_X}\\
    a'_{3_X}=a'_{2_X}        
    \end{aligned}
    \qquad\Rightarrow\qquad
    \begin{aligned}[c]
    v'_{3_X}=v'_{2_X}\\
    -v=-2v'_{2_X}
    \end{aligned}
\end{equation*}
\begin{align*}
    v_{2}=\frac{v}{4}\\
    a_{2_X}=a_{1_X}\rightarrow{}v_{2_X}=v_{1_X}\\
    v_{1}\equiv{}v_{2}=\frac{v}{4}\\
\end{align*}
% \textbf{Случай $I$.} Рассмотрим движение с торможением без поворота:
% \begin{equation*}
%     \begin{aligned}[c]
% 		m\vec{a}=\vec{f}_R\\
% 		\text{x: }ma=-mg\mu\\
% 		\int_{v_0}^{v(t)}dv=\int_0^t -g\mu dt\\
% 		v(t)=v_0-\mu{gt}\\
% 		\int_{0}^{x}dx=\int_0^t [v_0-\mu{gt}] dt\\
% 		x(t)=v_0{t}-\mu{g}\frac{t^2}{2}
%     \end{aligned}
%         \qquad\qquad
%     \begin{aligned}[c]
%     \text{Условие остановки $v=0$ при $t=t^*$:}\\
%     v_\text{ост}=0=v_0-g\mu{}t^*\\
%     t^*=\frac{v_0}{g\mu}\\
%     \text{Тогда пройденное до остановки $R$:}\\
%     R=v_0\cdot{t^*}-\mu{g}\frac{{t^*}^2}{2}\\
%     R=\frac{v_0^2}{2g\mu}
%     \end{aligned}
% \end{equation*}


\end{document}

