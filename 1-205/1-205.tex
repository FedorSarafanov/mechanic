\documentclass[a5paper,10pt]{article}
 
% \usepackage{extsizes}
\usepackage{cmap}
\usepackage[T2A]{fontenc}
\usepackage[utf8x]{inputenc}
\usepackage[english, russian]{babel}

\usepackage{misccorr}

%%%%%%%%%%%%%%%%%%%%%%%%%%%%%%%%%%%%%%%%%%%%%%%%%%%%%%%%%%%%%%%%%%%%%%%%%%%%%%%%%%  
\usepackage{graphicx} % для вставки картинок
\graphicspath{{img/}}
\usepackage{amssymb,amsfonts,amsmath,amsthm} % математические дополнения от АМС

% \usepackage{fontspec}
% \usepackage{unicode-math}

\usepackage{indentfirst} % отделять первую строку раздела абзацным отступом тоже
\usepackage[usenames,dvipsnames]{color} % названия цветов
\usepackage{makecell}
\usepackage{multirow} % улучшенное форматирование таблиц
\usepackage{ulem} % подчеркивания
\linespread{1.3} % полуторный интервал
% \renewcommand{\rmdefault}{ftm} % Times New Roman (не работает)
\frenchspacing
\usepackage{geometry}
\geometry{left=1cm,right=1cm,top=2cm,bottom=1cm,bindingoffset=0cm}
\usepackage{titlesec}
\usepackage{float}
% \definecolor{black}{rgb}{0,0,0}
% \usepackage[colorlinks, unicode, pagecolor=black]{hyperref}
% \usepackage[unicode]{hyperref} %ссылки
\usepackage{fancyhdr} %загрузим пакет
\pagestyle{fancy} %применим колонтитул
\fancyhead{} %очистим хидер на всякий случай
\fancyhead[R]{Сарафанов Ф.Г.} %номер страницы слева сверху на четных и справа на нечетных
\fancyhead[C]{Механика}
\fancyhead[L]{Иродов -- №1.205} 
\fancyfoot{} %футер будет пустой
% \fancyfoot[CO,CE]{\thepage}
\renewcommand{\labelenumii}{\theenumii)}


\usepackage{tikz}
\usepackage{tikz-3dplot}
\usetikzlibrary{scopes}
\usetikzlibrary{%
     decorations.pathreplacing,%
     decorations.pathmorphing,%
    patterns,%
    calc,%
    scopes,%
    arrows,%
    arrows.meta,%
    % arrows.spaced,%
}

\tikzset{
    MyPersp/.style={scale=1.8,x={(1.1cm,-0cm)},y={(0.5cm,1cm)}, z={(0cm,0.8cm)}},
 % MyPersp/.style={scale=1.5,x={(0cm,0cm)},y={(1cm,0cm)}, z={(0cm,1cm)}}, 
 % MyPersp/.style={scale=1.5,x={(1cm,0cm)},y={(0cm,1cm)}, z={(0cm,0cm)}}, 
    MyPoints/.style={fill=black,draw=black},
    force/.style={>=latex,draw=blue,fill=blue},
    % angular/.style={-{Stealth[open, angle=30:5pt,line width=1pt]}, draw=magenta},
    angular/.style={-{Latex[length=3mm, line width=0.4pt,open,magenta, fill=white]}, draw=magenta},
    % axis/.style={densely dashed,gray,font=\small},
    axis/.style={densely dashed,black!60,font=\small},
    interface/.style={
        pattern = north east lines,
        draw    = none,
        pattern color=gray!60,          
    },
    plank/.style={
        fill=black!60, 
        draw=black,
        minimum width=3cm,
        inner ysep=0.1cm,
        outer sep=0pt,
        yshift=0.75cm,
        pattern = north east lines,
        pattern color=gray!60, 
    },
    cargo/.style={
        rectangle,
        fill=black!70,              
        inner sep=2.5mm,
    }
}
\begin{document}

\begin{figure}[H]
    \centering
\begin{tikzpicture}[MyPersp, scale=1.5, font=\Large]

    \def\aphi{15}
    \def\R{2}

    \pgfmathparse{\R*cos(\aphi)}\let\az\pgfmathresult
    \pgfmathparse{\R*sin(\aphi)}\let\d\pgfmathresult
    \pgfmathparse{\R*sqrt(cos(\aphi))/sqrt(2)}\let\r\pgfmathresult
    \pgfmathparse{45-\aphi/2}\let\abeta\pgfmathresult

    \pgfmathparse{\az/(2*\r}\let\sinbeta\pgfmathresult
    \pgfmathparse{sqrt(1-\sinbeta^2))}\let\cosbeta\pgfmathresult

    \pgfmathparse{\r*\cosbeta-\d}\let\bx\pgfmathresult
    \pgfmathparse{\r*\sinbeta}\let\bz\pgfmathresult

    \pgfmathparse{2*\r*\cosbeta-\d}\let\dx\pgfmathresult
    \pgfmathparse{\bz/(\bx+\d)}\let\tgbeta\pgfmathresult

    \coordinate (A) at (0,0,\az);
    \coordinate (B) at (\bx,0,\bz);
    \coordinate (C) at (-\d,0,0);
    \coordinate (D) at (\dx,0,\az); 
    

    \fill[magenta!5,draw=black!90,dashed,opacity=1]
     (C) % elliptical section
        \foreach \t in {1,2,...,360}
            {--({-cos(\t)*(\bx+\d)+\bx},{sin(\t)*(\bx+\d)},{-\tgbeta*cos(\t)+\bz})}--cycle;
    
    \draw[->, >=latex] (A) -- node [black,pos=0.5,left] {$\vec{r}_1$} (C);
    \draw[->, >=latex] (A) -- node [black,pos=0.5,above] {$\vec{r}_2$} (D);

    \draw[MyPoints]  (A) node [above] {$O$} circle (0.5pt);

    \xdef\hh{-0.2}
    % \draw[axis,->] (0,0,\hh) -- ++(0,0,2*\r) node [black,above] {$z$};
    % \draw[axis,->] (-2,0,\hh) -- ++(4,0,0) node [black,right] {$x$};
    % \draw[axis,->] (0,-1,\hh) -- ++(0,1.5,0) node [black,above right] {$y$};

    \draw[axis,<-] (0,0,-1) node [black,below] {$x$} -- ++(0,0,2*\r);
    \draw[axis,->] (-2,0,\hh) -- ++(4,0,0) node [black,right] {$z$};
    \draw[axis,->] (0,-1,\hh) -- ++(0,1.5,0) node [black,above right] {$y$};

    \draw[force,->] (C) -- ++(0,0.5,0) node [black,pos=1,right] {$\vec{v}_1$};
    \draw[force,->] (D) -- ++(0,-0.5,0) node [black,pos=1,below] {$\vec{v}_2$};

    \draw[force,->] (C) -- ++(0,0,-1) node [black,pos=1,below] {$m\vec{g}$};
    \draw[force,->] (D) -- ++(0,0,-1) node [black,pos=1,below] {$m\vec{g}$};

    \draw (C) node[left] {$I$};
    \draw (D) node[right] {$II$};


    \draw[axis] (A) -- ++(-1,0) node[left] {$W_p=0$};

    % \draw[axis,force,->] (B) -- node[above, pos=0.4] {$\vec{r}_1$} (C);
    % \draw[axis,force,->] (B) -- node[above, pos=0.4] {$\vec{r}_2$} (D);

    % \fill[MyPoints] (B) circle (0.5pt) node[below] {$O$};

    \fill[MyPoints] (C) circle (0.5pt);
    \fill[MyPoints] (D) circle (0.5pt);

    \begin{scope}[canvas is xz plane at y=0]
        \draw[->] (0,\az)++(0,-1) node[above, xshift=-0.56em] {$\phi$} arc (270:255:1cm);

    % \draw[angular,->, >=open triangle 60] (C) -- ++(115:0.8) node [black,above] {$\vec{N}_{1_O}$};

    % \draw[angular,->, >=open triangle 60] (D) -- ++(-65:0.8) node [black,below] {$\vec{N}_{2_O}$};

    \draw[angular] (C) -- ++(-25:0.3) node [black,below] {$\vec{N}_{1}$};

    \draw[angular] (D) -- ++(-90:0.3) node [black,below, right] {$\vec{N}_{2}$};


    \end{scope}

\end{tikzpicture}
\end{figure}
Запишем моменты сил, действующих на шарик, в проеции на $x$.
\begin{equation*}
    \begin{aligned}[c]
        [\vec{M}]_x=[\vec{r}_1]_x\times [m\vec{g}]_x\\
        \frac{dN_x}{dt}=0
    \end{aligned}
        \qquad\Longrightarrow\qquad
    \begin{aligned}[c]
        N_x=const
    \end{aligned}
\end{equation*}
Тогда
\begin{equation*}
\left\{
    \begin{aligned}[c]
        N_x={mv_1}l\sin\phi\\
        N_x={mv_2}l
    \end{aligned}\right.
        \qquad\Longrightarrow\qquad
    \begin{aligned}[c]
        v_2=v_1\sin\phi
    \end{aligned}
\end{equation*}
Можно записать З.С.М.Э., выбрав ноль потенциальной энергии
\begin{equation*}
\xdef\mv{\frac{mv_1^2}{2}}%
\xdef\mvv{\frac{mv_2^2}{2}}%
	\begin{aligned}[c]
        \mv-mgl\cos{\phi}=\mvv\\
        \mv-\mvv=mgl\cos{\phi}\\
	\end{aligned}
		\qquad\Longrightarrow\qquad
	\begin{aligned}[c]
        \frac{v_1^2}{2}(1-\sin^2\phi)=gl\cos\phi\\
        v_1^2=\frac{2gl}{\cos\phi}
	\end{aligned}
\end{equation*}
\begin{equation*}
    v_1=\sqrt\frac{2gl}{\cos\phi}
\end{equation*}
% Но движение этой же системы до отрыва можно описать вторым законом Нюьтона:
% \begin{equation*}
% 	\begin{aligned}[c]
% 		&m\vec{a_0}=\vec{f_e}+m\vec{g}\\
% 		\text{x: }&ma_0=mg-k\Delta{}\\
% 	\end{aligned}
% 		\qquad\Longrightarrow\qquad
% 	\begin{aligned}[c]
% 		\Delta{}=\frac{m}{k}(g-a_0)\\
% 		% y&=v/w\\
% 	\end{aligned}
% \end{equation*}
% Тогда по теореме о изменении кинетической энергии можем записать работу сил упругости:
% \begin{equation*}
%     \mvn-\mv=\int_{\Delta{}}^{\delta{}} [mg-kx]\cdot{}dx
% \end{equation*}
% \begin{equation*}
%     -\mv=mg(\delta{}-\Delta{})+\frac{k\Delta{}^2}{2}-\frac{k\delta{}^2}{2}\\
% \end{equation*}
% \begin{equation*}
%     -\mv=-ma_0\Delta{}=mg(\delta{}-\Delta{})+\frac{k\Delta{}^2}{2}-\frac{k\delta{}^2}{2}
% \end{equation*}
% \begin{equation*}
% 	\frac{k}{2}\delta^2-mg\delta+\Delta(m(g-a_0)-\frac{k}{2}\Delta)=0
% \end{equation*}
% И найти удлинение пружины $\delta$:
% \begin{equation*}
% 	\frac{k}{2}\delta^2-mg\delta+\frac{m^2}{2k}(g-a_0)^2=0
% \end{equation*}
% \begin{equation*}
% 	\begin{aligned}[c]
% 		&a=\frac{k}{2}\\
% 		&b=-mg\\
% 		&c=\frac{m^2}{2k}(g-a_0)^2
% 	\end{aligned}
% 		\qquad\Longrightarrow\qquad
% 	\begin{aligned}[c]
% 		D&=b^2-4ac\\
% 		\sqrt{D}&=m\sqrt{2ga_0-a_o^2}
% 		% y&=v/w\\
% 	\end{aligned}
% \end{equation*}
% \begin{equation*}
% 	\begin{aligned}[c]
% 		\delta_1=\frac{mg+{}m\sqrt{2ga_0-a_o^2}}{k}
% 	\end{aligned}
% 		\qquad\qquad
% 	\begin{aligned}[c]
% 		\delta_2=\frac{mg-{}m\sqrt{2ga_0-a_o^2}}{k}
% 	\end{aligned}
% \end{equation*}

% Так как требуется найти $\delta_{max}$, подходит корень $\delta_1$, $\delta_1>\delta_2$.

\end{document}