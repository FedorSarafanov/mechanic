\documentclass[a5paper,10pt]{article}
 
% \usepackage{extsizes}
\usepackage{cmap}
\usepackage[T2A]{fontenc}
\usepackage[utf8x]{inputenc}
\usepackage[english, russian]{babel}

\usepackage{misccorr}

%%%%%%%%%%%%%%%%%%%%%%%%%%%%%%%%%%%%%%%%%%%%%%%%%%%%%%%%%%%%%%%%%%%%%%%%%%%%%%%%%%  
\usepackage{graphicx} % для вставки картинок
\graphicspath{{img/}}
\usepackage{amssymb,amsfonts,amsmath,amsthm} % математические дополнения от АМС

% \usepackage{fontspec}
% \usepackage{unicode-math}

\usepackage{indentfirst} % отделять первую строку раздела абзацным отступом тоже
\usepackage[usenames,dvipsnames]{color} % названия цветов
\usepackage{makecell}
\usepackage{multirow} % улучшенное форматирование таблиц
\usepackage{ulem} % подчеркивания
\linespread{1.3} % полуторный интервал
% \renewcommand{\rmdefault}{ftm} % Times New Roman (не работает)
\frenchspacing
\usepackage[portrait]{geometry}
\geometry{left=1cm,right=1cm,top=2cm,bottom=1cm,bindingoffset=0cm}
\usepackage{titlesec}
\usepackage{float}
% \definecolor{black}{rgb}{0,0,0}
% \usepackage[colorlinks, unicode, pagecolor=black]{hyperref}
% \usepackage[unicode]{hyperref} %ссылки
\usepackage{fancyhdr} %загрузим пакет
\pagestyle{fancy} %применим колонтитул
\fancyhead{} %очистим хидер на всякий случай
\fancyhead[LE,RO]{Сарафанов Ф.Г.} %номер страницы слева сверху на четных и справа на нечетных
\fancyhead[CO, CE]{Механика}
\fancyhead[LO,RE]{№304 -- Яковлев} 
\fancyfoot{} %футер будет пустой
% \fancyfoot[CO,CE]{\thepage}
\renewcommand{\labelenumii}{\theenumii)}


\usepackage{tikz}
\usepackage{etoolbox}
\usetikzlibrary{scopes}
\usetikzlibrary{%
     decorations.pathreplacing,%
     decorations.pathmorphing,%
    patterns,%
    calc,%
    scopes,%
    arrows,%
    % arrows.spaced,%
}
\makeatletter
\newif\if@gather@prefix 
\preto\place@tag@gather{% 
  \if@gather@prefix\iftagsleft@ 
    \kern-\gdisplaywidth@ 
    \rlap{\gather@prefix}% 
    \kern\gdisplaywidth@ 
  \fi\fi 
} 
\appto\place@tag@gather{% 
  \if@gather@prefix\iftagsleft@\else 
    \kern-\displaywidth 
    \rlap{\gather@prefix}% 
    \kern\displaywidth 
  \fi\fi 
  \global\@gather@prefixfalse 
} 
\preto\place@tag{% 
  \if@gather@prefix\iftagsleft@ 
    \kern-\gdisplaywidth@ 
    \rlap{\gather@prefix}% 
    \kern\displaywidth@ 
  \fi\fi 
} 
\appto\place@tag{% 
  \if@gather@prefix\iftagsleft@\else 
    \kern-\displaywidth 
    \rlap{\gather@prefix}% 
    \kern\displaywidth 
  \fi\fi 
  \global\@gather@prefixfalse 
} 
\newcommand*{\beforetext}[1]{% 
  \ifmeasuring@\else
  \gdef\gather@prefix{#1}% 
  \global\@gather@prefixtrue 
  \fi
} 
\makeatother
\begin{document}
\def\iangle{60} % Angle of the inclined plane

\def\down{-90}
\def\arcr{0.5cm} % Radius of the arc used to indicate angles
\def\FIN{\vec{F}_{in}^\text{пост}}
\hspace{-2em}
\begin{tikzpicture}[
    axis/.style={densely dashed,black!60,font=\small},
    force/.style={>=latex,draw=blue,fill=blue},
    % m/.style={rectangle,draw,fill=lightgray,minimum size=0.5cm,thin},
    m/.style={draw=black!30, rectangle,draw,thin, fill=blue!2, minimum size=0.5cm},
    m/.style={draw=black!30, rectangle,draw,thin, fill=blue!2, minimum size=0.5cm},
    interface/.style={draw=gray!60,
        postaction={draw=gray!60,decorate,decoration={border,angle=-135,
                    amplitude=0.3cm,segment length=2mm}}},
    plane/.style={draw=black!30, very thick, fill=blue!5},
    string/.style={draw=black, thick},
    pulley/.style={thick},
]

\matrix[column sep=0.5cm] {
    \begin{scope}[]
        \xdef\r{1.5cm}
        \xdef\rp{3pt}
        \node [shape=circle,draw] (c) at (0,0) [minimum size=\r*2] {};
        \draw [axis, ->] (0,0) -- (60:\r+2cm) node[anchor=south] {$+y$}; 
        \draw (0,0) -- (60:\r) circle (\rp) node[below, pos=1.3] {$z$};
        \draw [thick] (60:\r) +(-135:\rp)--+(45:\rp);
        \draw [thick] (60:\r) +(135:\rp)--+(-45:\rp);

        \draw [axis, ->]  ($(60:\r)!2cm!90:(0,0)$) --($(60:\r)!2cm!270:(0,0)$)  node[anchor=south] {$+x$};

        \draw [axis, ->]  (0,-\r) -- (0,2*\r)  node[anchor=south] {ось};
        \draw [axis, ->]  (-\r,0) -- (1.5*\r,0)  node[below] {экватор};
        \draw [force, ->] (0,\r) -- (0,\r+2em) node[left] {$\vec{\omega}$};
        {[axis,<-] 
            \draw[solid,shorten >=0.5pt] (\iangle:\arcr) arc(\iangle:0:\arcr);
            \node at (0.5*\iangle:1.4*\arcr) {$\phi$};
        }
    \end{scope}
&
    \begin{scope}[rotate=\iangle]

        {[axis,->]
            \draw (0,0) -- (0,2) node[anchor=south] {$+x$};
            \draw (0,0) -- ++(2.2,0) node[above] {$+y$};
            % Indicate angle. The code is a bit awkward.

            \draw[solid,shorten >=0.5pt, ] (90:\arcr)
                arc(90:90-\iangle:\arcr);
            \node at (120-\iangle:0.7cm) {$\phi$};

        }



        % Forces
        {[force,->]
            % Assuming that Mg = 1. The normal force will therefore be cos(alpha)

            \draw [opacity=1] (0,0) -- (0,2em) node[anchor=north east] {$\vec{\omega_\perp}$};

            % \draw (m.center) -- ++(0,{cos(\iangle)}) node[above right] {$\vec{N}$};
            % \draw (m.south) -- ++(-1,0) node[below, pos=1] {$\vec{f}_R$};
            % \draw (m.east) -- ++(1,0) node[above] {$T$};
        }

        {[force]
            \draw (5em,0) circle (\rp);
            \draw [thick] (5em,0)  +(-135:\rp)--+(45:\rp);
            \draw [thick] (5em,0)  +(135:\rp)--+(-45:\rp);
            \node [right] at (5em,0) {$\vec{F}^\text{кор}_\text{in}$};
            \draw [->] (0,0) -- (5em,0) node [below, pos=0.89, xshift=0.5em] {$\vec{v_0}$}; 
            \draw [->] (0,0) -- (3.3em,0) node [below, pos=0.7, xshift=0.5em] {$\vec{\omega_\parallel}$}; 
        }
    \end{scope}

        \draw [force,->, opacity=0.4](0,0) -- (0, 3.6em) node[above] {$\vec{\omega}$};


        {[black!60]
            \draw (0,0) circle (\rp) node[below, pos=1.3] {$+z$};
            \draw [thick] (0,0) +(-135:\rp)--+(45:\rp);
            \draw [thick] (0,0) +(135:\rp)--+(-45:\rp);s
        }
    % \draw[force,double equal sign distance=2pt,->] (m.center) -- ++(1,0) node[below, pos=1.5] {$\FIN$};
    % Draw gravity force. The code is put outside the rotated
    % scope for simplicity. No need to do any angle calculations. 
    % \draw[force,->] (m.center) -- ++(0,-1) node[below] {$m\vec{g}$};
&\begin{scope}[]   
    {[axis,->]
        \draw (0,0) -- (0,2) node[anchor=south] {$+y$};
        \draw (-1,0) -- ++(2.2,0) node[above] {$+z$};
                % Indicate angle. The code is a bit awkward.

                \draw[solid,shorten >=0.5pt, ] (45:\arcr)
                    arc(45:90:\arcr);
                \node at (67:0.7cm) {$\alpha$};
        \draw [-] (0,0) --(45:2);

    }
    {[force,->]
        \draw  (0,0) --(45:4em) node [right] {$\vec{v_0}$};
    }
 \draw (0,0) .. controls (2,2) and (0,3) .. (0,0);  
  \end{scope}

% \node[draw=none,text width=3cm, line width=0mm] at (0,0.5) {
% Возьмем НИСО, связанную с бруском. 
% \begin{gather}
%     \nonumber \FIN=-m\vec{a}_0\\
%     \nonumber \vec{N}=-\vec{P}\\
%     \nonumber N_y={}mg\cos\alpha-ma_{0x}\sin\alpha\\
%     \nonumber f_{Rx}=\\
%     \nonumber = -\mu{}m(g\cos\alpha-a_{0x}\sin\alpha)
% \end{gather}
% };
\\
};
\end{tikzpicture}

Рассмотрим задачу приближенно, исходя из основного движения по оси $y$.
\begin{gather}
    \beforetext{Запишем аналог II з. Н.:}%
    m\vec{a}=m\vec{g}+\vec{F}^\text{кор}_\text{in}+\vec{F}^\text{цб}_\text{in}\\
    \beforetext{Т.к. $\vec{F}^\text{цб}_\text{in}$ мало:}%
    m\vec{a}=m\vec{g}+\vec{F}^\text{кор}_\text{in}\\
    \beforetext{Проекция на y:} ma_y=-mg\\
    \beforetext{Опуская интегрирование:} v'_y(t)=v'_{0y}-gt\\
    \beforetext{Тогда условие остановки:} t_{stop}=\frac{v'_{0y}}{g}\Longrightarrow t_\text{движ}=\frac{2v'_{0y}}{g}\\
    {\omega_\perp}={\omega}\cdot\cos{60}\\
    \beforetext{Проекция на z:} ma^\text{кор}_z=-2m|[\vec{\omega_\perp}\times\vec{v}']|\\
    \beforetext{Тогда} a^\text{кор}_z\approx-2\omega_\perp(v'_{0y}-gt)]\\
%
\end{gather}
Ясно, что должна быть начальная скорость по оси z, чтобы скомпенсировать смещение, обеспеченное силой Кореолиса (из векторного произведения видно, что она направлена против оси - на запад). (правый рисунок - эскиз такого движения)
\begin{gather}
    v'_z-v'_{0z}=-\int_0^t [2\omega_\perp(v'_{0y}-gt)] dt=%
    -2\omega_\perp(v'_{0y}t-\frac{gt^2}{2})\\
    z(t)= \int_0^t[v'_{0z}-2\omega_\perp(v'_{0y}t-\frac{gt^2}{2})] dt=%
    v'_{0z}t-2\omega_\perp(\frac{v'_{0y}t^2}{2}-\frac{gt^3}{6})\\
    z(t=t_\text{движ})=0=-2\omega_\perp(\frac{v'_{0y}t_\text{движ}^2}{2}-\frac{gt_\text{движ}^3}{6})+v'_{0z}t_\text{движ}
%
\end{gather}
\begin{gather}
    v'_{0z}=2\omega_\perp(\frac{v'_{0y}t_\text{движ}}{2}-\frac{gt_\text{движ}^3}{6})=%
    \frac{2\omega_\perp v'^2_{0y}}{3g}\\
    \frac{2\omega_\perp v'_{0y}}{3g}=\frac{v'_{0z}}{v'_{0y}}=\tg{\alpha}\approx\alpha\\
    \cos\alpha=\frac{v'_0}{v'_{0y}}\approx1\Longrightarrow v'_{0y}\approx v'_0\\
    \beforetext{Тогда}
    \alpha\approx \frac{2\cdot{}v'_0\cdot{}\omega\cdot{}\cos\phi }{3g}\approx51'',
%
\end{gather}
а ружье необходимо отклонить на этот угол на восток.

\end{document}