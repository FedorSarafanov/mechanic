\documentclass[a5paper,10pt]{article}\usepackage[usenames,dvipsnames]{color}\usepackage{cmap,graphicx,misccorr,indentfirst,makecell,multirow,ulem,geometry,amssymb,amsfonts,amsmath,amsthm,titlesec,float,fancyhdr,wrapfig,tikz}\usepackage[T2A]{fontenc}\usepackage[utf8x]{inputenc}\usepackage[english, russian]{babel}\usetikzlibrary{decorations.pathreplacing,decorations.pathmorphing,patterns,calc,scopes,arrows,through}\graphicspath{{img/}}\linespread{1.3}\frenchspacing\geometry{left=1cm, right=1cm, top=2cm, bottom=1cm, bindingoffset=0cm}\pagestyle{fancy}\fancyhead{}\fancyhead[R]{Сарафанов Ф.Г.} 
\fancyhead[C]{Механика}
\fancyhead[L]{Задача под запись -- <<клин>>} 
\fancyfoot{}
\begin{document}

\def\iangle{35} % Angle of the inclined plane

\def\down{-90}
\def\arcr{0.5cm} % Radius of the arc used to indicate angles

\begin{figure}[tb]
	\centering
\begin{tikzpicture}[
    force/.style={>=latex,draw=blue,fill=blue},
    axis/.style={densely dashed,gray,font=\small},
    M/.style={rectangle,draw,fill=lightgray,minimum size=0.5cm,thin},
    m/.style={rectangle,draw=black,fill=lightgray,minimum size=0.3cm,thin},
    plane/.style={draw=black,fill=blue!10},
    string/.style={draw=black, thick},
    pulley/.style={thick},
    acceleration/.style={>=open triangle 60,draw=blue,fill=blue},
    inforce/.style={force,double equal sign distance=2pt},
    interface/.style={
        pattern = north east lines,
        draw    = none,
        pattern color=gray!60,          
    },
]

\matrix[column sep=1cm] {

    %% Sketch
    \draw[plane] (0,-1) coordinate (base)
                     -- coordinate[pos=0.5] (mid) ++(\iangle:3) coordinate (top)
                     |- (base) -- cycle;
    \path (mid) node[M,rotate=\iangle,yshift=0.25cm] (M) {};
    % \draw[pulley] (top) -- ++(\iangle:0.25) circle (0.25cm)
                   % ++ (90-\iangle:0.5) coordinate (pulley);
    \draw[string] (M.east) -- ++(\iangle:1.25cm);% arc (90+\iangle:0:0.25);
                  % -- ++(0,-1) node[m]  {};

    \draw[->] (base)++(\arcr,0) arc (0:\iangle:\arcr);
    \path (base)++(\iangle*0.5:\arcr+5pt) node {$\alpha$};
    \draw (top) -- ++({\iangle+90}:0.5);
    %%

    \draw[acceleration, ->] (0,0) -- node[below] {$\vec{a}$} ++(-1,0);

&
    %% Free body diagram of M
    \begin{scope}[rotate=\iangle]
        \node[M,transform shape] (M) {};
        % Draw axes and help lines

        {[axis,->]
            \draw (0,-1) -- (0,2) node[right] {$+y$};
            \draw (M) -- ++(2,0) node[right] {$+x$};
            % Indicate angle. The code is a bit awkward.

            \draw[solid,shorten >=0.5pt] (\down-\iangle:\arcr)
                arc(\down-\iangle:\down:\arcr);
            \node at (\down-0.5*\iangle:1.3*\arcr) {$\alpha$};
        }

        % Forces
        {[force,->]
            % Assuming that Mg = 1. The normal force will therefore be cos(alpha)
            \draw (M.center) -- ++(0,{cos(\iangle)}) node[above right] {$\vec{N}$};
            % \draw (M.west) -- ++(-1,0) node[left] {$\vec{f_R}$};
            \draw (M.east) -- ++(1,0) node[above] {$\vec{T}$};
        }

    \end{scope}
    % Draw gravity force. The code is put outside the rotated
    % scope for simplicity. No need to do any angle calculations. 
    \draw[force,->] (M.center) -- ++(0,-1) node[below] {$M\vec{g}$};
    %%

% &
    %%%
    % Free body diagram of m
    % \node[m] (m) {};
    % \draw[axis,->] (m) -- ++(0,-2) node[left] {$+x$};
    % {[force,->]
    %     \draw (m.north) -- ++(0,1) node[above] {$\vec{T'}$};
    %     \draw (m.south) -- ++(0,-1) node[right] {$m\vec{g}$};
    % }

\\
};
\end{tikzpicture}	
% \vspace{-3em}  
\end{figure}

\def\N{\vec{N}}
\def\T{\vec{T}}
\def\t{\vec{T'}}
\def\F{\vec{f_R}}
\def\P{M\vec{g}}
\def\p{m\vec{g}}

Запишем второй закон Ньютона для груза $M$: 
\begin{gather}
	M\vec{a}=\N+\T+\P
\end{gather}
Запишем проекции на $x$: 
\begin{gather}
	\label{eq1}-Ma\cos\alpha=T-Mg\sin\alpha\\
	\label{eq2}T=M(g\sin\alpha-a\cos\alpha)
\end{gather}
И проекция на $y$:
\begin{gather}
	\label{eq3}Ma\sin\alpha=N-Mg\cos\alpha\\
    P=N=M(g\cos\alpha+a\sin\alpha)
\end{gather}
Рассмотрим полученные значения, имеющие физический смысл:  
\begin{equation}
\left\{
    \begin{aligned}[c]
       T&=M(g\sin\alpha-a\cos\alpha), &\text{ если } \frac{a}{g}<\tg\alpha\\
       T&=0, &\text{ если } \frac{a}{g}\geq\tg\alpha\\
    \end{aligned}
\right.
\end{equation}
\end{document}