\documentclass[a5paper,10pt]{article}
\def\source{/home/lab/tex/templates}

\usepackage{cmap}
\usepackage[T2A]{fontenc}
\usepackage[utf8x]{inputenc}
\usepackage[english, russian]{babel}

\usepackage
	{
		amssymb,
		% misccorr,
		amsfonts,
		amsmath,
		amsthm,
		wrapfig,
		makecell,
		multirow,
		indentfirst,
		ulem,
		graphicx,
		geometry,
		fancyhdr,
		subcaption,
		float,
		tikz,
		csvsimple,
		color,
	}  

\usepackage[outline]{contour}
\usepackage[mode=buildnew]{standalone}


\geometry
	{
		left=1cm,
		right=1cm,
		top=2cm,
		bottom=1cm,
		bindingoffset=0cm,
	}

\linespread{1.3} 
\frenchspacing 


\usetikzlibrary{scopes}
\usetikzlibrary
	{
		decorations.pathreplacing,
		decorations.pathmorphing,
		patterns,
		calc,
		scopes,
		arrows,
		through,
		shapes.misc,
		arrows.meta,
	}


\tikzset{
	force/.style=	{
		>=latex,
		draw=blue,
		fill=blue,
				 	}, 
	%				 	
	axis/.style=	{
		densely dashed,
		gray,
		font=\small,
					},
	%
	acceleration/.style={
		>=open triangle 60,
		draw=blue,
		fill=blue,
					},
	%
	inforce/.style=	{
		force,
		double equal sign distance=2pt,
					},
	%
	interface/.style={
		pattern = north east lines, 
		draw    = none, 
		pattern color=gray!60,
					},
	cross/.style=	{
		cross out, 
		draw=black, 
		minimum size=2*(#1-\pgflinewidth), 
		inner sep=0pt, outer sep=0pt,
					},
	%
	cargo/.style=	{
		rectangle, 
		fill=black!70, 
		inner sep=2.5mm,
					},
	%
	}

\pagestyle{fancy} %применим колонтитул
\fancyhead{} %очистим хидер на всякий случай
\fancyhead[R]{Сарафанов Ф.Г.} %номер страницы слева сверху на четных и справа на нечетных
\fancyhead[C]{Механика}
% \fancyhead[L]{Задача под запись - <<АУУ-2>>} 
\fancyfoot{} %футер будет пустой

\newcommand{\irodov}[1]{\fancyhead[L]{Иродов -- №#1}}
\newcommand{\yakovlev}[1]{\fancyhead[L]{Яковлев -- №#1}}
\newcommand{\wrote}[1]{\fancyhead[L]{Под запись -- <<#1>>}}

\newenvironment{tikzpict}
    {
	    \begin{figure}[htbp]
		\centering
		\begin{tikzpicture}
    }
    { 
		\end{tikzpicture}
		% \caption{caption}
		% \label{fig:label}
		\end{figure}
    }

\newcommand{\vbLabel}[3]{\draw ($(#1,#2)+(0,5pt)$) -- ($(#1,#2)-(0,5pt)$) node[below]{#3}}
\newcommand{\vaLabel}[3]{\draw ($(#1,#2)+(0,5pt)$) node[above]{#3} -- ($(#1,#2)-(0,5pt)$) }

\newcommand{\hrLabel}[3]{\draw ($(#1,#2)+(5pt,0)$) -- ($(#1,#2)-(5pt,0)$) node[right, xshift=1em]{#3}}
\newcommand{\hlLabel}[3]{\draw ($(#1,#2)+(5pt,0)$) node[left, xshift=-1em]{#3} -- ($(#1,#2)-(5pt,0)$) }

% Draw line annotation
% Input:
%   #1 Line offset (optional)
%   #2 Line angle
%   #3 Line length
%   #5 Line label
% Example:
%   \lineann[1]{30}{2}{$L_1$}
\newcommand{\lineann}[4][0.5]{%
    \begin{scope}[rotate=#2, blue,inner sep=2pt, ]
        \draw[dashed, blue!40] (0,0) -- +(0,#1)
            node [coordinate, near end] (a) {};
        \draw[dashed, blue!40] (#3,0) -- +(0,#1)
            node [coordinate, near end] (b) {};
        \draw[|<->|] (a) -- node[fill=white, scale=0.8] {#4} (b);
    \end{scope}
}


\yakovlev{360}

\begin{document}

\begin{tikzpict}

	\draw[line width=2pt] (-0.1,-1,0.2) -- ++(0,1,0);

	\begin{scope}[canvas is yz plane at x=-0.2]
		\draw (0,0) rectangle (3,3);
	\end{scope}	
	\begin{scope}[canvas is yz plane at x=0]
		\draw[fill=white] (0,0) rectangle (3,3);
		\draw[transform shape, fill=magenta] (1.5,2) circle (2pt);
	\end{scope}

	\begin{scope}[canvas is xz plane at y=3]
		\draw[fill=white] (0,0) rectangle (-.2,3);
		\lineann[2]{90}{3}{$a$}
	\end{scope}	

	\begin{scope}[canvas is xz plane at y=1.5]
		\lineann[-1.2]{90}{2}{$\frac{2}{3}a$}
		\draw[force,->] (3,2) -- ++ (-1,0) node [left] {$\vec{v}$};
	\end{scope}	

	\begin{scope}[canvas is xy plane at z=3]
		\draw[fill=white] (0,0) rectangle (-.2,3);
		\lineann[2]{90}{3}{$a$}
	\end{scope}		
	\draw[line width=2pt] (-0.1,3,0.2) -- ++(0,1,0);
	\draw (0,1.5,2) node[above] {$A$};
	\draw[fill=magenta] (3,1.5,2) circle (4pt) node[above, yshift=0.5em] {$m$};
	% \fill[magenta] (0,0) circle (2pt) node[above, yshift=0.5em] {$O$};

	% \draw (0,0)++(-0.2,0.2) rectangle ++ (0.4,-3);
	% \lineann[2]{90}{-2.8}{$L, m$}

	% \draw (0,0) -- ++(-40:2) coordinate (g);

	% \fill[magenta] (g) circle (5pt) node [right, xshift=0.5em] {$m$};
	% \lineann[2]{-40}{2}{$l$}

	% \draw (2,-2) node[blue] {$\bigotimes$} node[right, xshift=.5em] {$\vec{N}$};
\end{tikzpict}
Положим скорость шарика перед ударом $v$, после удара $u$, угловую скорость пластинки после удара $\omega$.

Момент инерции пластинки относительно оси $y$
\begin{equation}
	I=\int\limits_{(m)}r^2\,dm=
	\iint\limits_S x^2 \rho_s dx dy=
	\rho_s\int\limits_0^a x^2dx\int\limits_0^a dy=\frac{\rho_sa^4}{3}=\frac{m_0a^2}{3}
\end{equation}
Запишем ЗСЭ:
\begin{equation}
	\label{eq1}
	\frac{mv^2}{2}=\frac{I\omega^2}{2}+\frac{mu^2}{2}
\end{equation}
Запишем ЗСМИ в проекции на ось $x$:
\begin{equation}
	mv\cdot\frac23a=I\omega_x+mu_x\cdot\frac23a
\end{equation}
Перепишем иначе
\begin{equation}
	\left\{\begin{aligned}
		v^2-u_x^2=&\frac{I\omega^2}{m}\\
		v-u_x=&\frac{3I\omega_x}{2am}
	\end{aligned}\right.
\end{equation}
Откуда
\begin{equation}
	v_x+u=\frac{I\omega^2}{m}\cdot\frac{2am}{3I\omega_x}=\frac{2\omega_x a}{2} \Rightarrow u=\frac{2\omega_x a}{3}-v
\end{equation}
Тогда
\begin{equation}
	\frac43mva=I\omega_x+\frac43am\frac{\omega_x a}{3}
\end{equation}
И
\begin{equation}
	\omega_x\left[I+\frac49ma^2\right]=\frac43mva
\end{equation}
\begin{equation}
	\omega_x=
		\frac43\frac{mva}{I+\frac49ma^2}=
		\frac43\frac{mva}{\frac{m_0a^2}{3}+\frac49ma^2}=\frac{v}{a}\cdot\frac{12m}{3m_0+4m}
\end{equation}
\begin{equation}
	u_x=\frac{2\omega_x a}{3}-v={v}\cdot\frac{8m}{3m_0+4m}-{v}\cdot\frac{3m_0+4m}{3m_0+4m}=v\cdot\frac{8m-3m_0-4m}{3m_0+4m}
\end{equation}
\begin{equation}
	u_x=v\cdot\frac{4m-3m_0}{4m+3m_0}
\end{equation}
\end{document}