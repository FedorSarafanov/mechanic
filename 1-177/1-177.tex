\documentclass[a5paper,10pt]{article}
 
\usepackage{extsizes}
\usepackage{cmap}
\usepackage[T2A]{fontenc}
\usepackage[utf8x]{inputenc}
\usepackage[english, russian]{babel}

\usepackage{misccorr}

%%%%%%%%%%%%%%%%%%%%%%%%%%%%%%%%%%%%%%%%%%%%%%%%%%%%%%%%%%%%%%%%%%%%%%%%%%%%%%%%%%  
\usepackage{graphicx} % для вставки картинок
\graphicspath{{img/}}
\usepackage{amssymb,amsfonts,amsmath,amsthm} % математические дополнения от АМС

% \usepackage{fontspec}
% \usepackage{unicode-math}

\usepackage{indentfirst} % отделять первую строку раздела абзацным отступом тоже
\usepackage[usenames,dvipsnames]{color} % названия цветов
\usepackage{makecell}
\usepackage{multirow} % улучшенное форматирование таблиц
\usepackage{ulem} % подчеркивания
\linespread{1.3} % полуторный интервал
% \renewcommand{\rmdefault}{ftm} % Times New Roman (не работает)
\frenchspacing
\usepackage{geometry}
\geometry{left=1cm,right=1cm,top=2cm,bottom=1cm,bindingoffset=0cm}
\usepackage{titlesec}
\usepackage{float}
% \definecolor{black}{rgb}{0,0,0}
% \usepackage[colorlinks, unicode, pagecolor=black]{hyperref}
% \usepackage[unicode]{hyperref} %ссылки
\usepackage{fancyhdr} %загрузим пакет
\pagestyle{fancy} %применим колонтитул
\fancyhead{} %очистим хидер на всякий случай
\fancyhead[R]{Сарафанов Ф.Г.} %номер страницы слева сверху на четных и справа на нечетных
\fancyhead[C]{Механика}
\fancyhead[L]{Иродов -- №1.177} 
\fancyfoot{} %футер будет пустой
% \fancyfoot[CO,CE]{\thepage}
\renewcommand{\labelenumii}{\theenumii)}


\usepackage{tikz}
\usetikzlibrary{scopes}
\usetikzlibrary{%
     decorations.pathreplacing,%
     decorations.pathmorphing,%
    patterns,%
    calc,%
    scopes,%
    arrows,%
    % arrows.spaced,%
}

\begin{document}

\begin{figure}[H]
    \centering
\begin{tikzpicture}[
    force/.style={>=latex,draw=blue,fill=blue},
    % axis/.style={densely dashed,gray,font=\small},
    axis/.style={densely dashed,black!60,font=\small},
    interface/.style={
        pattern = north east lines,
        draw    = none,
        pattern color=gray!60,          
    },
    cargo/.style={
        rectangle,
        fill=magenta!40,
        draw=black!50,
        inner sep=2.5mm,
    },
    spring/.style={
        decoration={
            aspect=0.3, 
            segment length=.8mm, 
            amplitude=2mm,
            coil},
        decorate,
        draw=magenta!25
    }
]
        \draw[axis,->] (-1,0.5) node[left]{$W_\text{п}=0$} -- ++ (7,0) node[right]{$+x$};
        % \draw[interface] (-0.25,-0.25) rectangle ++(-0.25,0.7);
        \draw[interface] (-0.5,-0) rectangle ++(6,-0.25);
        \draw[thick] (-0.5, -0) -- ++(6,0);

        \draw[fill=magenta!10] (0,0) -- (0,0.5) -- (2,0.5) arc (-90:0:1cm) -- (3.5,1.5) -- (4.5,0.4) -- (4.5,0) -- cycle;

        \draw[fill=magenta!30] (0.25,0.5) rectangle ++(0.5,0.5) node[above, xshift=-0.6em] {$m$};

        \draw[force,->] (0.5,0.72) --++(1,0) node[right] {$\vec{v}_0$};

        \draw (3.5,0.5) node {$M$};


\end{tikzpicture}
% \vspace{-2em}
\end{figure}

Все силы консервативные, трения нет. Пусть скорость платформы в момент отрыва груза - $u$. 

Запишем ЗСИ для начального импульса и импульса непосредственно перед отрывом в проекции на $x$:
\begin{equation}
    m{v_0}=(m+M)u
    \quad\Rightarrow\quad
    u=v_0\frac{m}{m+M}
\end{equation}
Далее запишем ЗСМЭ в начале и в момент отрыва:
\begin{equation}
    \frac{mv_0^2}{2}=\frac{Mu^2}{2}+\frac{m(u^2+v_\perp^2)}{2}
\end{equation}
где $v_\perp$ -- вертикальная составляющая скорости груза, $u$ -- горизонтальная.
Отсюда
\begin{equation}
    \frac{mv_\perp^2}{2}=\frac{mv_0^2}{2}-\frac{(M+m)u^2}{2}=
    \frac{mv_0^2}{2}\left[1-(m+M)\frac{m}{(m+M)^2}\right]=
\end{equation}
\begin{equation}
    = \frac{mv_0^2}{2}\cdot\frac{M}{m+M}
\end{equation}
Но с другой стороны, можем записать ЗСМЭ для груза в момент вылета и в момент наивысшего подъема:
\begin{equation}
    \frac{mv_0^2}{2}\cdot\frac{M}{m+M}=mgh
\end{equation}
Откуда можем записать окончательный ответ
\begin{equation}
    h=\frac{v_0^2}{2g}\cdot\frac{M}{m+M}
\end{equation}
% \begin{equation}
% \label{v20}
% \frac{m_2v_{20}^2}{2}=\frac{\varkappa x^2}{2}
% \end{equation}

% Когда пружина распрямится, скорость $m_2$ будет $v_{20}$, а $m_1$ -- равна нулю.

% Тогда запишем:
% \begin{equation}
%     \label{vc}
%     \vec{v}_c=\frac{m_1\cdot0+m_2\vec{v}_{20}}{m_1+m_2}
% \end{equation}
% Выразим из (\ref{v20}) $v_{20}$ и подставим в (\ref{vc}), перейдя к модулям векторов (можно обойтись без проекций, так как в уравнении (\ref{vc}) связаны два вектора сугубо положительным множителем):
% \begin{equation}
%     v_c=\sqrt\frac{\varkappa x^2}{m_2}\cdot \frac{m_2}{m_1+m_2}
% \end{equation}
% Тогда можем окончательно записать ответ:
% \begin{equation}
%     v_c=\frac{x\sqrt{\varkappa \cdot m_2}}{m_1+m_2}
% \end{equation}

\end{document}