\documentclass[a5paper,10pt]{article}
\def\source{/home/lab/tex/templates}

\usepackage{cmap}
\usepackage[T2A]{fontenc}
\usepackage[utf8x]{inputenc}
\usepackage[english, russian]{babel}

\usepackage
	{
		amssymb,
		% misccorr,
		amsfonts,
		amsmath,
		amsthm,
		wrapfig,
		makecell,
		multirow,
		indentfirst,
		ulem,
		graphicx,
		geometry,
		fancyhdr,
		subcaption,
		float,
		tikz,
		csvsimple,
		color,
	}  

\usepackage[outline]{contour}
\usepackage[mode=buildnew]{standalone}


\geometry
	{
		left=1cm,
		right=1cm,
		top=2cm,
		bottom=1cm,
		bindingoffset=0cm,
	}

\linespread{1.3} 
\frenchspacing 


\usetikzlibrary{scopes}
\usetikzlibrary
	{
		decorations.pathreplacing,
		decorations.pathmorphing,
		patterns,
		calc,
		scopes,
		arrows,
		through,
		shapes.misc,
		arrows.meta,
	}


\tikzset{
	force/.style=	{
		>=latex,
		draw=blue,
		fill=blue,
				 	}, 
	%				 	
	axis/.style=	{
		densely dashed,
		gray,
		font=\small,
					},
	%
	acceleration/.style={
		>=open triangle 60,
		draw=blue,
		fill=blue,
					},
	%
	inforce/.style=	{
		force,
		double equal sign distance=2pt,
					},
	%
	interface/.style={
		pattern = north east lines, 
		draw    = none, 
		pattern color=gray!60,
					},
	cross/.style=	{
		cross out, 
		draw=black, 
		minimum size=2*(#1-\pgflinewidth), 
		inner sep=0pt, outer sep=0pt,
					},
	%
	cargo/.style=	{
		rectangle, 
		fill=black!70, 
		inner sep=2.5mm,
					},
	%
	}

\pagestyle{fancy} %применим колонтитул
\fancyhead{} %очистим хидер на всякий случай
\fancyhead[R]{Сарафанов Ф.Г.} %номер страницы слева сверху на четных и справа на нечетных
\fancyhead[C]{Механика}
% \fancyhead[L]{Задача под запись - <<АУУ-2>>} 
\fancyfoot{} %футер будет пустой

\newcommand{\irodov}[1]{\fancyhead[L]{Иродов -- №#1}}
\newcommand{\yakovlev}[1]{\fancyhead[L]{Яковлев -- №#1}}
\newcommand{\wrote}[1]{\fancyhead[L]{Под запись -- <<#1>>}}

\newenvironment{tikzpict}
    {
	    \begin{figure}[htbp]
		\centering
		\begin{tikzpicture}
    }
    { 
		\end{tikzpicture}
		% \caption{caption}
		% \label{fig:label}
		\end{figure}
    }

\newcommand{\vbLabel}[3]{\draw ($(#1,#2)+(0,5pt)$) -- ($(#1,#2)-(0,5pt)$) node[below]{#3}}
\newcommand{\vaLabel}[3]{\draw ($(#1,#2)+(0,5pt)$) node[above]{#3} -- ($(#1,#2)-(0,5pt)$) }

\newcommand{\hrLabel}[3]{\draw ($(#1,#2)+(5pt,0)$) -- ($(#1,#2)-(5pt,0)$) node[right, xshift=1em]{#3}}
\newcommand{\hlLabel}[3]{\draw ($(#1,#2)+(5pt,0)$) node[left, xshift=-1em]{#3} -- ($(#1,#2)-(5pt,0)$) }

% Draw line annotation
% Input:
%   #1 Line offset (optional)
%   #2 Line angle
%   #3 Line length
%   #5 Line label
% Example:
%   \lineann[1]{30}{2}{$L_1$}
\newcommand{\lineann}[4][0.5]{%
    \begin{scope}[rotate=#2, blue,inner sep=2pt, ]
        \draw[dashed, blue!40] (0,0) -- +(0,#1)
            node [coordinate, near end] (a) {};
        \draw[dashed, blue!40] (#3,0) -- +(0,#1)
            node [coordinate, near end] (b) {};
        \draw[|<->|] (a) -- node[fill=white, scale=0.8] {#4} (b);
    \end{scope}
}

% \irodov{1.273}
\fancyhead{}
\fancyhead[R]{Механика}  
\fancyhead[L]{Иродов -- №1.273}
\fancyfoot{} 
\fancyfoot[C]{\thepage}
\begin{document}

\begin{tikzpict}
	\draw[interface] (-2,0) rectangle ++ (4,0.5);
	\draw[] (-2,0) -- ++(4,0);

	\draw[line width=5pt, black!30] (0,0) -- (0,-1);

	\draw[thick] (0,-1) circle (2pt) node[left, xshift=-0.5em] {$O$} -- ++ (-60:4);

	\draw[axis,->] (0,-1) -- ++ (-60:5) node[below] {$+x$};	

	\draw[line width=5pt] (0,-1) ++ (-60:1.9) coordinate (1) ++ (-60:-0.1) -- ++ (-60:0.2);	
	\draw[axis] (0,-1) -- ++ (-90:4) node[below] {$$};

	\draw[axis,->] (0,-1)++(-60:4) -- ++ (30:2) node[right] {$+\tau$};

	\draw[force,->] (1) -- ++(120:1) node [right, xshift=0.25em] {$d\vec{T}$};
	\draw[inforce,->] (1) -- ++(1,0) node [right] {$d\vec{F}_{in}$};
	\draw[force,->] (1) -- ++(0,-1) node [below] {$dm\vec{g}$};
\contourlength{1mm};
 	\draw (2,-1) node[magenta] {$\bigotimes$} node[right, xshift=.5em, magenta] {$\vec{M}_{g}$};
 	\draw (3.5,-1) node[magenta] {$\bigodot$} node[right, xshift=.5em, magenta] {$\vec{M}_{in}$};

 	\draw[force,->,magenta] (0,-5) --++(0,1) node[above] {\contour{white}{$\vec{\omega}$}};
 	\draw[force,->,magenta] (-0.5,-5) --++(0,1) node[above] {\contour{white}{$\vec{N}$}};

    \coordinate (ooo) at (0,-2);

\draw[line width=5pt] (0,-1) coordinate (o) ++ (-60:1.9) coordinate (oo) ++ (-60:-0.1) -- ++ (-60:0.2);		
    \draw pic["$\alpha$",draw=magenta,<-,angle eccentricity=1.2,angle radius=1.5cm] {angle=ooo--o--oo}; 
\end{tikzpict}
\begin{equation}
	I\vec\gamma=\vec{M}
\end{equation}
Т.к. начальной угловой скорости (поворота стержня) нет, то при $\gamma=0$ стержень будет покоится относительно вращающегося подвеса.

Запишем суммарный момент для участка стержня длиной $dx$, массой $dm=\rho dx$:
\begin{equation}
	\vec{M}=\int\limits_{(m)}\left[[\vec{r},dm\vec{g}]+[\vec{r},dm\omega^2\vec{r}_\perp]\right]=0
\end{equation}
В проекции на ось $z$ (от нас через $O$):
\begin{equation}
	dm\cdot g\cdot\sin\alpha\cdot x=dm\cdot\omega^2\cdot x^2\cdot\sin\alpha\cos\alpha
\end{equation}
Проинтегрируем с учетом $dm=dx\cdot \rho$:
\begin{equation}
	g\sin\alpha \int\limits_0^L x dx = 
		\omega^2\sin\alpha\cos\alpha \int\limits_0^L x^2 dx
\end{equation}
Отсюда
\begin{equation}
	g\frac{L^2}{2}=\omega^2\cos\alpha\frac{L^3}{3}
    \quad\Rightarrow\quad
	\cos\alpha=\frac{3g}{2\omega^2L}
\end{equation}
\end{document}