\documentclass[a5paper,12pt]{article}
 
\usepackage{extsizes}
\usepackage{cmap}
\usepackage[T2A]{fontenc}
\usepackage[utf8x]{inputenc}
\usepackage[english, russian]{babel}

\usepackage{misccorr}

%%%%%%%%%%%%%%%%%%%%%%%%%%%%%%%%%%%%%%%%%%%%%%%%%%%%%%%%%%%%%%%%%%%%%%%%%%%%%%%%%%  
\usepackage{graphicx} % для вставки картинок
\graphicspath{{img/}}
\usepackage{amssymb,amsfonts,amsmath,amsthm} % математические дополнения от АМС

% \usepackage{fontspec}
% \usepackage{unicode-math}

\usepackage{indentfirst} % отделять первую строку раздела абзацным отступом тоже
\usepackage[usenames,dvipsnames]{color} % названия цветов
\usepackage{makecell}
\usepackage{multirow} % улучшенное форматирование таблиц
\usepackage{ulem} % подчеркивания
\linespread{1.3} % полуторный интервал
% \renewcommand{\rmdefault}{ftm} % Times New Roman (не работает)
\frenchspacing
\usepackage{geometry}
\geometry{left=1cm,right=1cm,top=2cm,bottom=1cm,bindingoffset=0cm}
\usepackage{titlesec}
\usepackage{float}
% \definecolor{black}{rgb}{0,0,0}
% \usepackage[colorlinks, unicode, pagecolor=black]{hyperref}
% \usepackage[unicode]{hyperref} %ссылки
\usepackage{fancyhdr} %загрузим пакет
\pagestyle{fancy} %применим колонтитул
\fancyhead{} %очистим хидер на всякий случай
\fancyhead[LE,RO]{Сарафанов Ф.Г.} %номер страницы слева сверху на четных и справа на нечетных
\fancyhead[CO, CE]{Механика}
\fancyhead[LO,RE]{Иродов 1.47 } 
\fancyfoot{} %футер будет пустой
% \fancyfoot[CO,CE]{\thepage}
\renewcommand{\labelenumii}{\theenumii)}


\usepackage{tikz}
\usetikzlibrary{scopes}
\usetikzlibrary{%
     decorations.pathreplacing,%
     decorations.pathmorphing,%
    patterns,%
    calc,%
    scopes,%
    arrows,%
    through,%
    % arrows.spaced,%
}

\begin{document}

\begin{figure}[H]
    \centering
\begin{tikzpicture}[
    force/.style={>=latex,draw=blue,fill=blue},
    % axis/.style={densely dashed,gray,font=\small},
    axis/.style={densely dashed,black!60,font=\small},
    M/.style={rectangle,draw,fill=lightgray,minimum size=0.5cm,thin},
    m2/.style={draw=black!30, rectangle,draw,thin, fill=blue!2, minimum width=0.7cm,minimum height=0.7cm},
    m1/.style={draw=black!30, rectangle,draw,thin, fill=blue!2, minimum width=0.7cm,minimum height=0.7cm},
    plane/.style={draw=black!30, very thick, fill=blue!5, line width=1pt},
    % base/.style={draw=black!70, very thick, fill=blue!4, line width=2pt},
    string/.style={draw=black, thick},
    pulley/.style={thick},
    % interface/.style={draw=gray!60,
    %     % The border decoration is a path replacing decorator. 
    %     % For the interface style we want to draw the original path.
    %     % The postaction option is therefore used to ensure that the
    %     % border decoration is drawn *after* the original path.
    %     postaction={draw=gray!60,decorate,decoration={border,angle=-135,
    %                 amplitude=0.3cm,segment length=2mm}}},
    interface/.style={
        pattern = north east lines,
        draw    = none,
        pattern color=gray!60,          
    },
    plank/.style={
        fill=black!60, 
        draw=black,
        minimum width=3cm,
        inner ysep=0.1cm,
        outer sep=0pt,
        yshift=0.75cm,
        pattern = north east lines,
        pattern color=gray!60, 
    },
    cargo/.style={
        rectangle,
        fill=black!70,              
        inner sep=2.5mm,
    }
]
    % \draw[force,double equal sign distance=2pt,->] (c) -- ++(0,-2) node[below] {$\vec{a}_0$};
\matrix[column sep=0cm, row sep=0cm] {
%%%%%%%%%%%%%%%%%%%%%%%%%%%%%%%%%%%%%%
	\draw[->, line width=0.5mm] (0,-3) arc (-90:0:3cm);
	\draw (0,0) circle (3cm);
	\coordinate (B) at (3,0);

	\draw[axis,->, thick] (3,-3) --(3,3) node[right, black, scale=1.5] {$\tau$};
	\draw[axis,->, thick] (0,0) --(5,0) node[right, black, scale=1.5] {$n$};	
	\def\t{2}
	\def\n{-2}
	\draw[force,->,>=open triangle 60] (B) -- ++(0,\t) node[right, scale=1.5] {$\vec{a}_\tau$};
	\draw[force,->,>=open triangle 60] (B) -- ++(\n,0) node[below, scale=1.5] {$\vec{a}_n$};

	\draw[force,->,>=open triangle 60] (B) -- ++(\n,\t) node[above, scale=1.5] {$\vec{a}$};

    \draw[solid,shorten >=0.5pt, ->] (3,1)
                arc(90:135:1);	

    \path[] (3,0) -- ++ (120:1) node[above, scale=1] {$\phi$};

\\
};
\end{tikzpicture}
\vspace{-2em}
\end{figure}

\begin{equation*}
	\begin{aligned}[c]
		\vec{a}=\vec{a}_n+\vec{a}_\tau\\
		a_\tau=\beta{R}\\
		a_n=\omega^2{R}
	\end{aligned}
		% \qquad\Longrightarrow\qquad
		\qquad\qquad
	\begin{aligned}[c]
		\vec{v} \parallel \vec{a}_\tau
	\end{aligned}
		\qquad\qquad
	\begin{aligned}[c]
		\vec{v}=\vec{\omega}\times\vec{R}\\
		v=v_\tau=\omega{R}
	\end{aligned}	
\end{equation*}
$\beta$ дано и $\beta=\alpha{t}$. Не трудно найти $\omega{t}$:
\begin{gather*}
	\xdef\d{\mathrm{d}}
	\beta=\frac{\d\omega}{\d{t}}\\
	\int\d\omega=\int\alpha{t}\d{t}\\
	\omega=\frac{\alpha{t}^2}{2}
\end{gather*}
\begin{equation*}
	\phi=\vec{v}\ \widehat{\ }\ \vec{a} = \vec{a}_\tau\ \widehat{\ }\ \vec{a}
\end{equation*}
\begin{gather*}
	\tg{\phi}=\frac{a_n}{a_\tau}=\frac{\omega^2{R}}{\beta{R}}=\frac{\omega^2}{\beta}=%
			 =\frac{\alpha^2t^4}{4\alpha{t}}=\frac{\alpha{t^3}}{4}\\
	t=\sqrt[3]{\frac{4\tg{\phi}}{\alpha}}=7\text{ с}\\
\end{gather*}


\end{document}