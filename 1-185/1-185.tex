\documentclass[a5paper,10pt]{article}
 
% \usepackage{extsizes}
\usepackage{cmap}
\usepackage[T2A]{fontenc}
\usepackage[utf8x]{inputenc}
\usepackage[english, russian]{babel}

\usepackage{misccorr}

%%%%%%%%%%%%%%%%%%%%%%%%%%%%%%%%%%%%%%%%%%%%%%%%%%%%%%%%%%%%%%%%%%%%%%%%%%%%%%%%%%  
\usepackage{graphicx} % для вставки картинок
\graphicspath{{img/}}
\usepackage{amssymb,amsfonts,amsmath,amsthm} % математические дополнения от АМС

% \usepackage{fontspec}
% \usepackage{unicode-math}
% \usepackage[outline]{contour}
\usepackage{indentfirst} % отделять первую строку раздела абзацным отступом тоже
\usepackage[usenames,dvipsnames]{color} % названия цветов
\usepackage{makecell}
\usepackage{multirow} % улучшенное форматирование таблиц
\usepackage{ulem} % подчеркивания
\linespread{1.3} % полуторный интервал
% \renewcommand{\rmdefault}{ftm} % Times New Roman (не работает)
\frenchspacing
\usepackage{geometry}
\geometry{left=1cm,right=1cm,top=2cm,bottom=1cm,bindingoffset=0cm}
\usepackage{titlesec}
\usepackage{float}
% \definecolor{black}{rgb}{0,0,0}
% \usepackage[colorlinks, unicode, pagecolor=black]{hyperref}
% \usepackage[unicode]{hyperref} %ссылки
\usepackage{fancyhdr} %загрузим пакет
\pagestyle{fancy} %применим колонтитул
\fancyhead{} %очистим хидер на всякий случай
\fancyhead[R]{Сарафанов Ф.Г.} %номер страницы слева сверху на четных и справа на нечетных
\fancyhead[C]{Механика}
\fancyhead[L]{Иродов -- №1.185} 
\fancyfoot{} %футер будет пустой
% \fancyfoot[CO,CE]{\thepage}
\renewcommand{\labelenumii}{\theenumii)}


\usepackage{tikz}
\usetikzlibrary{scopes}
\usetikzlibrary{%
    decorations.pathreplacing,%
    decorations.pathmorphing,%
    patterns,%
    calc,%
    scopes,%
    arrows,%
    quotes,%
    angles,%
    babel
    % arrows.spaced,%
}

\begin{document}
\def\pIn{\vec{p}_{1\text{н}}}
\def\pIIn{\vec{p}_{2\text{н}}}
\def\mpIn{{p}_{1\text{н}}}
\def\mpIIn{{p}_{2\text{н}}}
\begin{figure}[H]
    \centering
\begin{tikzpicture}[
    force/.style={>=latex,draw=blue,fill=blue},
    % axis/.style={densely dashed,gray,font=\small},
    axis/.style={densely dashed,black!60,font=\small},
    interface/.style={
        pattern = north east lines,
        draw    = none,
        pattern color=gray!60,          
    },
    cargo/.style={
        rectangle,
        fill=magenta!40,
        draw=black!50,
        inner sep=2.5mm,
    },
    spring/.style={
        decoration={
            aspect=0.3, 
            segment length=.8mm, 
            amplitude=2mm,
            coil},
        decorate,
        draw=magenta!70
    }
]

        \draw[force,->] (0,0) coordinate (a) -- 
            node[midway,fill=white!20, opacity=0.9]  
                {$(m_1+m_2)\vec{v}_c$} 
            (4,0);

        \draw[force,->] (0,0)  -- 
            node[midway,fill=white!20, opacity=0.9]  
                {$\pIn$} 
            (2,2) coordinate (b);

        \draw[force,->] (b)  -- 
            node[midway,fill=white!20, opacity=0.9]  
                {$\pIIn$} 
            (4,0) coordinate (c);       

        \draw pic["$\Phi$",draw=magenta,->,angle eccentricity=1.5,angle radius=0.5cm] {angle=a--b--c};                 

\end{tikzpicture}
\end{figure}



По определению,
\begin{equation}
    \vec{v}_c=\vec{v}_1\frac{m_1}{m_1+m_2} 
    \Longrightarrow
    \vec{v}_c\parallel\vec{v}_1
\end{equation}
По условию
\begin{equation}
    \angle(\vec{u}_1,\vec{v}_1)=
    \angle(\vec{u}_2,\vec{v}_1)=
    \frac{1}{2}\Theta
\end{equation}
Откуда следует
\begin{equation}
    \angle(\pIn,\vec{v}_c)=
    \angle(\pIIn,\vec{v}_c)=
    \frac{1}{2}\Theta 
\end{equation}
Из равенства углов следует, что $\mpIn=\mpIIn$. Обозначим $\Phi=\pi-\Theta$.

При ударе выполняется ЗСЭ, откуда
\begin{equation}
    m_1v_1^2=
    m_1u_1^2+
    m_2u_2^2
\end{equation}
Запишем в другом виде:
\begin{equation}
    \frac{m_2}{m_1}u_2^2=v_1^2-u_1^2
\end{equation}
\begin{equation}
    \mpIn=\mpIIn
        \Longrightarrow
    u_2=\frac{m_1}{m_2}u_1
\end{equation}
Запишем теорему косинусов:
\begin{equation}
    (m_1+m_2)^2v_c^2=\mpIn^2+\mpIIn^2-2\mpIn\mpIIn\cos\Phi
\end{equation}
\begin{equation}
    m_1^2v_1^2=m_1^2u_1^2+m_2^2u_2^2-2m_1u_1 \cdot m_2u_2 \cdot \cos\Phi\bigg|\times\frac{1}{m_1^2}
\end{equation}
\begin{equation}
    v_1^2-u_1^2=\frac{m^2_2}{m^2_1}u_2^2-2u_1u_2\frac{m_2}{m_1}\cos\Phi
\end{equation}
\begin{equation}
    \frac{m_2}{m_1}u_2^2=\frac{m^2_2}{m^2_1}u_2^2-2\frac{m^2_2}{m^2_1}u_2^2\cos\Phi
\end{equation}
\begin{equation}
    \frac{m_1}{m_2}=1-2\cos\Phi=1-2\cos(\pi-\Theta)=1+2\cos\Theta
\end{equation}
\begin{equation}
    \frac{m_1}{m_2}=1+2\cos60^\circ=1+2\cdot\frac{1}{2}=2
\end{equation}
\end{document}