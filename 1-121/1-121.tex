\documentclass[a5paper,10pt]{article}\usepackage[usenames,dvipsnames]{color}

\usepackage{cmap,graphicx,etoolbox,misccorr,indentfirst,makecell,multirow,ulem,geometry,amssymb,amsfonts,amsmath,amsthm,titlesec,float,fancyhdr,wrapfig,tikz}

\usepackage[T2A]{fontenc}\usepackage[utf8x]{inputenc}\usepackage[english, russian]{babel}\usetikzlibrary{decorations.pathreplacing,decorations.pathmorphing,patterns,calc,scopes,arrows,through, shapes.misc}\graphicspath{{img/}}\linespread{1.3}\frenchspacing\geometry{left=1cm, right=1cm, top=2cm, bottom=1cm, bindingoffset=0cm}\pagestyle{fancy}\fancyhead{}\fancyhead[R]{Сарафанов Ф.Г.}\fancyhead[C]{Механика}
\fancyhead[L]{Иродов -- №1.121}
\fancyfoot{}

%Команда \beforetext для текста слева от формулы
\makeatletter \newif\if@gather@prefix \preto\place@tag@gather{\if@gather@prefix\iftagsleft@ \kern-\gdisplaywidth@ \rlap{\gather@prefix} \kern\gdisplaywidth@ \fi\fi } \appto\place@tag@gather{\if@gather@prefix\iftagsleft@\else \kern-\displaywidth \rlap{\gather@prefix} \kern\displaywidth \fi\fi \global\@gather@prefixfalse } \preto\place@tag{\if@gather@prefix\iftagsleft@ \kern-\gdisplaywidth@ \rlap{\gather@prefix} \kern\displaywidth@ \fi\fi } \appto\place@tag{\if@gather@prefix\iftagsleft@\else \kern-\displaywidth \rlap{\gather@prefix} \kern\displaywidth \fi\fi \global\@gather@prefixfalse } \newcommand*{\beforetext}[1]{\ifmeasuring@\else \gdef\gather@prefix{#1} \global\@gather@prefixtrue \fi } \makeatother 
\tikzset{force/.style={>=latex,draw=blue,fill=blue}, axis/.style={densely dashed,gray,font=\small}, acceleration/.style={>=open triangle 60,draw=blue,fill=blue}, inforce/.style={force,double equal sign distance=2pt}, interface/.style={pattern = north east lines, draw    = none, pattern color=gray!60, }, cross/.style={cross out, draw=black, minimum size=2*(#1-\pgflinewidth), inner sep=0pt, outer sep=0pt},    cargo/.style={rectangle, fill=black!70, inner sep=2.5mm, }}
\usepackage[outline]{contour}
\begin{document}
\begin{figure}[H]
    \centering
\begin{tikzpicture}

    % \draw[fill=black] (-4,-2) coordinate (I) circle (1pt);
    % \draw[fill=black] (0,0) coordinate   (II) circle (1pt);
    % \draw[fill=black] (4,-2) coordinate  (III) circle (0pt);
    \draw[interface] (0,0) rectangle (12,-0.5);
    \draw[thick] (0,0) -- (12,-0);

    \draw[axis,->] (2.05,0.25) -- ++(0,3) node[above] {$+y$};
    \draw[axis,->] (2.05,0.25) -- ++(3,0) node[right] {$+x$};
    \draw[force,->] (2.05,0.5) -- ++(-1,0) node[left] {$\vec{v}^\text{ п}$};
    \contourlength{1mm};
    \begin{scope}[rotate=-45]        
        \draw[] (1,1.5) rectangle ++(0.3,3);
        \draw[fill=magenta] (1,3) rectangle ++ (0.3,0.4);

        \draw[force,->] (1.15,3.4) -- ++(0,2) node[right] {$\vec{v'}_0$};

        \node[below]  at (1.45,3.4) {{$m$}};

        \draw[fill=magenta] (1.1,1.8) circle (0.5cm);
    \end{scope}

        \node  at (2.05,0.5) {\contour{white}{$M$}};
\end{tikzpicture}
% \vspace{-1em}
\end{figure}

Первым будем считать положение до выстрела, вторым -- после. Все будем рассматривать в проекции на $x$ (Чтобы не изменялся импульс системы. Проекция внешних сил на эту ось равна нулю).
\begin{gather*}
   0 = m\cdot v^\text{с}_x +  M\cdot v^\text{п}_x
\end{gather*}
Причем
\begin{gather*}
    v^\text{с}_x= v^\text{п}_x+v'_{0x}
\end{gather*}
Тогда   
\begin{gather*}
    -M\cdot v^\text{п}_x =  m\cdot (v^\text{п}_x+v'_{0x})
\end{gather*}
Далее
\begin{gather*}
    m\cdot v^\text{п}_x+M\cdot v^\text{п}_x = -m\cdot v'_{0x}
\end{gather*}
И ответ:
\begin{gather*}
    v^\text{п}_x = -\frac{m\cdot v'_{0x}}{m+M}
                 = -\frac{v_0\cdot\cos\alpha}{1+M/m}=
                 = -\frac{v_0\cdot\cos\alpha}{1+\mu}\\
\end{gather*}
\begin{equation*}
    v^\text{п}   = \frac{v_0\cdot\cos\alpha}{1+\mu}\\
\end{equation*}

\end{document}