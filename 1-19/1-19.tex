\documentclass[a5paper,10pt]{article}
 
\usepackage{extsizes}
\usepackage{cmap}
\usepackage[T2A]{fontenc}
\usepackage[utf8x]{inputenc}
\usepackage[english, russian]{babel}

\usepackage{misccorr}

%%%%%%%%%%%%%%%%%%%%%%%%%%%%%%%%%%%%%%%%%%%%%%%%%%%%%%%%%%%%%%%%%%%%%%%%%%%%%%%%%%  
\usepackage{graphicx} % для вставки картинок
\graphicspath{{img/}}
\usepackage{amssymb,amsfonts,amsmath,amsthm} % математические дополнения от АМС

% \usepackage{fontspec}
% \usepackage{unicode-math}

\usepackage{indentfirst} % отделять первую строку раздела абзацным отступом тоже
\usepackage[usenames,dvipsnames]{color} % названия цветов
\usepackage{makecell}
\usepackage{multirow} % улучшенное форматирование таблиц
\usepackage{ulem} % подчеркивания
\linespread{1.3} % полуторный интервал
% \renewcommand{\rmdefault}{ftm} % Times New Roman (не работает)
\frenchspacing
\usepackage{geometry}
\geometry{left=1cm,right=1cm,top=2cm,bottom=1cm,bindingoffset=0cm}
\usepackage{titlesec}
\usepackage{float}
% \definecolor{black}{rgb}{0,0,0}
% \usepackage[colorlinks, unicode, pagecolor=black]{hyperref}
% \usepackage[unicode]{hyperref} %ссылки
\usepackage{fancyhdr} %загрузим пакет
\pagestyle{fancy} %применим колонтитул
\fancyhead{} %очистим хидер на всякий случай
\fancyhead[LE,RO]{Сарафанов Ф.Г.} %номер страницы слева сверху на четных и справа на нечетных
\fancyhead[CO, CE]{Механика}
\fancyhead[LO,RE]{Иродов 1.19} 
\fancyfoot{} %футер будет пустой
% \fancyfoot[CO,CE]{\thepage}
\renewcommand{\labelenumii}{\theenumii)}


\usepackage{tikz}
\usetikzlibrary{scopes}
\usetikzlibrary{%
     decorations.pathreplacing,%
     decorations.pathmorphing,%
    patterns,%
    calc,%
    scopes,%
    arrows,%
    through,%
    % arrows.spaced,%
}
\newcommand{\vangle}{\mathop{\mathstrut^\wedge}\nolimits}
\newcommand{\average}[1]{\langle{#1}\rangle}
\usepackage{wrapfig}
\newcommand{\RN}[1]{%
  \textup{\tiny\uppercase\expandafter{\romannumeral#1}}%
}
\begin{document}

\begin{figure}[H]
    \centering
\begin{tikzpicture}[
    force/.style={>=latex,draw=blue,fill=blue},
    % axis/.style={densely dashed,gray,font=\small},
    axis/.style={densely dashed,black!60,font=\small},
    M/.style={rectangle,draw,fill=lightgray,minimum size=0.5cm,thin},
    m2/.style={draw=black!30, rectangle,draw,thin, fill=blue!2, minimum width=0.7cm,minimum height=0.7cm},
    m1/.style={draw=black!30, rectangle,draw,thin, fill=blue!2, minimum width=0.7cm,minimum height=0.7cm},
    plane/.style={draw=black!30, very thick, fill=blue!5, line width=1pt},
    % base/.style={draw=black!70, very thick, fill=blue!4, line width=2pt},
    string/.style={draw=black, thick},
    pulley/.style={thick},
    interface1/.style={draw=gray!60,
        % The border decoration is a path replacing decorator. 
        % For the interface style we want to draw the original path.
        % The postaction option is therefore used to ensure that the
        % border decoration is drawn *after* the original path.
        postaction={draw=gray!60,decorate,decoration={border,angle=-135,
                    amplitude=0.3cm,segment length=2mm}}},
    interface/.style={
        pattern = north east lines,
        draw    = none,
        pattern color=gray!60,          
    },
    plank/.style={
        fill=black!60, 
        draw=black,
        minimum width=3cm,
        inner ysep=0.1cm,
        outer sep=0pt,
        yshift=0.75cm,
        pattern = north east lines,
        pattern color=gray!60, 
    },
    cargo/.style={
        rectangle,
        fill=black!70,              
        inner sep=2.5mm,
    }
]
	\draw (0,0) ++ (2,0) arc(0:180:2cm);
	\draw[axis] (-180:2) coordinate (II) -- (0:2) coordinate (I);
	\draw (0,0) circle (1pt) -- node[above] {$R$} (I) circle (1pt);
	\draw[force,->,>=open triangle 60] (I) -- ++ (0,1) node[above] {$\vec{a_\tau}$};

	\draw[force,->,>=open triangle 60] (II) -- ++ (0,-1.5) node[below] {$\vec{a_\tau}$};

	\draw[force,->] (II) -- ++ (0,-0.8) node[left] {$\vec{v_\RN{2}}$};


\end{tikzpicture}
\end{figure}

\begin{gather*}
	\average{v}=\frac{{S}}{{t}}=\frac{\pi{R}}{\tau}=
	\frac{3.14\cdot{1.6}}{10}=0.5\text{ m/s}\\
	|\average{\vec{v}}|=|\frac{\Delta{\vec{r}}}{\Delta{t}}|=\frac{2R}{\tau}=
	\frac{2\cdot{1.6}}{10}=0.32\text{ m/s}\\
	|\average{\vec{a}}|=|\frac{\Delta{\vec{v}}}{\Delta{t}}|=\frac{v_{\RN{2}_\tau}}{\tau}
\end{gather*}

\begin{gather*}
	v_\tau=a_\tau\cdot{t}\\
	S=\pi{R}=\int_0^\tau a_\tau{t}\,dt=\frac{a\tau^2}{2}\\
	a_\tau=\frac{2\pi{R}}{\tau^2}\\
	v_{\RN{2}_\tau}=a_\tau\cdot{\tau}=\frac{2\pi{R}}{\tau}\\
	|\average{\vec{a}}|=\frac{v_{\RN{2}_\tau}}{\tau}=\frac{2\pi{R}}{\tau^2}=
	\frac{2\cdot{}3.14\cdot{1.6}}{100}=0.1\text{ m/s$^2$}\\
\end{gather*}

\end{document}