\documentclass[a5paper,10pt]{article}\usepackage[usenames,dvipsnames]{color}\usepackage{extsizes,cmap,graphicx,misccorr,indentfirst,makecell,multirow,ulem,geometry,amssymb,amsfonts,amsmath,amsthm,titlesec,float,fancyhdr,wrapfig,tikz}\usepackage[T2A]{fontenc}\usepackage[utf8x]{inputenc}\usepackage[english, russian]{babel}\usetikzlibrary{decorations.pathreplacing,decorations.pathmorphing,patterns,calc,scopes,arrows,through,positioning,shapes.misc}\graphicspath{{img/}}\linespread{1.3}\frenchspacing\geometry{left=1cm, right=1cm, top=2cm, bottom=1cm, bindingoffset=0cm}\pagestyle{fancy}\fancyhead{}\fancyhead[R]{Сарафанов Ф.Г.} 
\fancyhead[C]{Механика}
\fancyhead[L]{Иродов - 1.370} 
\fancyfoot{}
\renewcommand{\labelenumii}{\theenumii)}
\tikzset{
	force/.style={>=latex,draw=blue,fill=blue,>=triangle 45},
    axis/.style={densely dashed,black!60,font=\small},
    interface1/.style={draw=gray!60,.
        postaction={draw=gray!60,decorate,decoration={border,angle=-135,
        amplitude=0.3cm,segment length=2mm}}},
    interface/.style={
        pattern = north east lines,
        draw    = none,
        pattern color=gray!60,          
    },
    plank/.style={
        fill=black!60, 
        draw=black,
        minimum width=3cm,
        inner ysep=0.1cm,
        outer sep=0pt,
        yshift=0.75cm,
        pattern = north east lines,
        pattern color=gray!60, 
    },
    cargo/.style={
        rectangle,
        fill=black!70,              
        inner sep=2.5mm,
    }	
}

\definecolor{clock0}{cmyk}{1,0,0,0} % cyan
\definecolor{clock1}{cmyk}{0.75,0.25,0,0}
\definecolor{clock2}{cmyk}{0.5,0.5,0,0}
\definecolor{clock3}{cmyk}{0.25,0.75,0,0}
\definecolor{clock4}{cmyk}{0,1,0,0} % magenta
\definecolor{clock5}{cmyk}{0,0.75,0.25,0}
\definecolor{clock6}{cmyk}{0,0,5,0.5,0}
\definecolor{clock7}{cmyk}{0,0.25,0.75,0}
\definecolor{clock8}{cmyk}{0,0,1,0} % yellow
\definecolor{clock9}{cmyk}{0.25,0,0.75,0}
\definecolor{clock10}{cmyk}{0.5,0,0.5,0}
\definecolor{clock11}{cmyk}{0.75,0,0.25,0}

\begin{document}
\newcommand{\clock}[3]{%
  \begin{scope}[xshift=#1cm,yshift=#2cm, scale=0.3]
  \shadedraw [inner color=white, outer color=white, 
    line width=1.1pt] (0,0) circle (1cm); 
  \foreach \angle in {0, 30, ..., 330} 
    \draw[line width=0.5pt] (\angle:0.62cm) -- (\angle:1cm);
  \foreach \angle in {0,90,180,270}
    \draw[line width=1.3pt] (\angle:0.75cm) -- (\angle:1cm);
  \draw[line width=1.6pt] (0,0) -- (90-30*#3:0.6cm); 
  \end{scope}
}

\begin{figure}[H]
    % \centering
\begin{tikzpicture}
% \matrix[column sep=1cm, row sep=0cm] {
	\draw[->, >=triangle 60] (0,0) coordinate (K) -- node [left, xshift=-1em, draw=black,circle] {\,$K$\,} ++ (0,3); 
    \draw[->, >=triangle 60] (K) -- ++ (10,0); 

    \draw[->, >=triangle 60] (1,2) coordinate (K') -- node [left, xshift=-1em, draw=black,circle] {$K'$} ++ (0,3); 
    \draw[force,->, >=latex] ($(K')+(0,2)$) -- ++ (1,0) node [right] {$\vec{u}$};
    \draw[->, >=triangle 60] (K') - ++ (1,0); 

    \draw[->, >=triangle 60] (6,2) coordinate (K') -- node [left, xshift=-1em, draw=black,circle] {$K'$} ++ (0,3); 
    \draw[force,->, >=latex] ($(K')+(0,2)$) -- ++ (1,0) node [right] {$\vec{u}$};
    \draw[->, >=triangle 60] (K') -- ++ (1,0); 

    \clock{1}{0}{0};
    \clock{1}{2}{0};
    \clock{6}{2}{3};
    \clock{6}{0}{4};

% &
% \\
% };
\end{tikzpicture}
\end{figure}
Синхронизируем часы на мюоне и на часах в $K$ -- системе: выставим их на 0 в момент пролета мюона над первыми часами $K$ -- системы.

Время пролета в $K$ $$t=\frac{l}{u}$$

Преобразование Лоренца для времени штрихованной системы:
\begin{gather*}
    t'=\frac{t-ux/c^2}{\sqrt{1-u^2/c^2}}
\end{gather*}
Тогда собственное время жизни $t'$
\begin{gather*}
    t'=\frac{l/u-ul/c^2}{\sqrt{1-u^2/c^2}}=\frac{l}{u}\sqrt{1-u^2/c^2}=1.4249\cdot{10}^{-6}\text{ c}
\end{gather*}
И собственное расстояние пролета $l'$
\begin{gather*}
    l'=u\cdot{t'}=423.2\text{ m}
\end{gather*}
% \begin{equation*}
%     \begin{aligned}[c]
% 		m\vec{a}=\vec{f}_R\\
% 		\text{x: }ma=-mg\mu\\
% 		\int_{v_0}^{v(t)}dv=\int_0^t -g\mu dt\\
% 		v(t)=v_0-\mu{gt}\\
% 		\int_{0}^{x}dx=\int_0^t [v_0-\mu{gt}] dt\\
% 		x(t)=v_0{t}-\mu{g}\frac{t^2}{2}
%     \end{aligned}
%         \qquad\qquad
%     \begin{aligned}[c]
%     \text{Условие остановки $v=0$ при $t=t^*$:}\\
%     v_\text{ост}=0=v_0-g\mu{}t^*\\
%     t^*=\frac{v_0}{g\mu}\\
%     \text{Тогда пройденное до остановки $R$:}\\
%     R=v_0\cdot{t^*}-\mu{g}\frac{{t^*}^2}{2}\\
%     R=\frac{v_0^2}{2g\mu}
%     \end{aligned}
% \end{equation*}


\end{document}

