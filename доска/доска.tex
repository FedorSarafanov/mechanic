\documentclass[a5paper,10pt]{article}
 
\usepackage{extsizes}
\usepackage{cmap}
\usepackage[T2A]{fontenc}
\usepackage[utf8x]{inputenc}
\usepackage[english, russian]{babel}

\usepackage{misccorr}

%%%%%%%%%%%%%%%%%%%%%%%%%%%%%%%%%%%%%%%%%%%%%%%%%%%%%%%%%%%%%%%%%%%%%%%%%%%%%%%%%%  
\usepackage{graphicx} % для вставки картинок
\graphicspath{{img/}}
\usepackage{amssymb,amsfonts,amsmath,amsthm} % математические дополнения от АМС

% \usepackage{fontspec}
% \usepackage{unicode-math}

\usepackage{indentfirst} % отделять первую строку раздела абзацным отступом тоже
\usepackage[usenames,dvipsnames]{color} % названия цветов
\usepackage{makecell}
\usepackage{multirow} % улучшенное форматирование таблиц
\usepackage{ulem} % подчеркивания
\linespread{1.3} % полуторный интервал
% \renewcommand{\rmdefault}{ftm} % Times New Roman (не работает)
\frenchspacing
\usepackage{geometry}
\geometry{left=1cm,right=1cm,top=2cm,bottom=1cm,bindingoffset=0cm}
\usepackage{titlesec}
\usepackage{float}
% \definecolor{black}{rgb}{0,0,0}
% \usepackage[colorlinks, unicode, pagecolor=black]{hyperref}
% \usepackage[unicode]{hyperref} %ссылки
\usepackage{fancyhdr} %загрузим пакет
\pagestyle{fancy} %применим колонтитул
\fancyhead{} %очистим хидер на всякий случай
\fancyhead[LE,RO]{Сарафанов Ф.Г.} %номер страницы слева сверху на четных и справа на нечетных
\fancyhead[CO, CE]{Механика}
\fancyhead[LO,RE]{Задача под запись -- <<доска>>} 
\fancyfoot{} %футер будет пустой
% \fancyfoot[CO,CE]{\thepage}
\renewcommand{\labelenumii}{\theenumii)}


\usepackage{tikz}
\usetikzlibrary{scopes}
\usetikzlibrary{%
     decorations.pathreplacing,%
     decorations.pathmorphing,%
    patterns,%
    calc,%
    scopes,%
    arrows,%
    % arrows.spaced,%
}

\begin{document}

\begin{figure}[H]
    \centering
\begin{tikzpicture}[
    force/.style={>=latex,draw=blue,fill=blue},
    % axis/.style={densely dashed,gray,font=\small},
    axis/.style={densely dashed,black!60,font=\small},
    M/.style={rectangle,draw,fill=lightgray,minimum size=0.5cm,thin},
    m2/.style={draw=black!30, rectangle,draw,thin, fill=blue!2, minimum width=0.7cm,minimum height=0.7cm},
    m1/.style={draw=black!30, rectangle,draw,thin, fill=blue!2, minimum width=0.7cm,minimum height=0.7cm},
    plane/.style={draw=black!30, very thick, fill=blue!5, line width=1pt},
    % base/.style={draw=black!70, very thick, fill=blue!4, line width=2pt},
    string/.style={draw=black, thick},
    pulley/.style={thick},
    % interface/.style={draw=gray!60,
    %     % The border decoration is a path replacing decorator. 
    %     % For the interface style we want to draw the original path.
    %     % The postaction option is therefore used to ensure that the
    %     % border decoration is drawn *after* the original path.
    %     postaction={draw=gray!60,decorate,decoration={border,angle=-135,
    %                 amplitude=0.3cm,segment length=2mm}}},
    interface/.style={
        pattern = north east lines,
        draw    = none,
        pattern color=gray!60,          
    },
    plank/.style={
        fill=black!60, 
        draw=black,
        minimum width=3cm,
        inner ysep=0.1cm,
        outer sep=0pt,
        yshift=0.75cm,
        pattern = north east lines,
        pattern color=gray!60, 
    },
    cargo/.style={
        rectangle,
        fill=black!70,              
        inner sep=2.5mm,
    }
]
    % \draw[force,double equal sign distance=2pt,->] (c) -- ++(0,-2) node[below] {$\vec{a}_0$};
\matrix[column sep=0cm, row sep=0cm] {
%%%%%%%%%%%%%%%%%%%%%%%%%%%%%%%%%%%%%%
    \node[cargo] (b) at (1,2) {} node[right of=b, xshift=-0.5cm] {$m$};
    \node[plank, below of=b, anchor=north] (plank) {};
    % \node[right] at (plank.east) {доска};

    \draw[axis,<->] (plank.north west) -- node[left, black] {$l_0$} ++(0,3.25);

    \node[below] (c) at (plank.south) {};
    % \draw[force,double equal sign distance=2pt,->] (c) -- ++(0,-2) node[below] {$\vec{a}_0$};
    \draw[force,->,>=open triangle 60] (c) -- ++(0,-1) node[below] {$\vec{a}_0$};

    \draw[decoration={aspect=0.3, segment length=1.5mm, amplitude=3mm,coil},decorate] (1,5) -- node[left, black, xshift=-10pt] {$k$} (b); 
    \draw[interface] (-1,5) rectangle (3,5.2);
    \draw[thick] (-1,5) -- (3,5);   
&%%%%%%%%%%%%%%%%%%%%%%%%%%%%%%%%%%%%%%
    %груз и доска
    \node[cargo] (a) at (1,1) {} node[right of=a, xshift=-0.5cm] {$m$};
    \node[plank, below of=a, anchor=north] (plank) {};
    %верх
    \draw[interface] (-1,5) rectangle (3,5.2);
    \draw[thick] (-1,5) -- (3,5);
    %пружинка
    \draw[decoration={aspect=0.3, segment length=3mm, amplitude=3mm,coil},decorate] (1,5) -- node[left, black, xshift=-10pt] {$k$} (a); 

    \draw[axis,<->] (plank.north west) -- node[left, black] {$\Delta{}$} ++(0,1) coordinate (mid);
    \draw[-] ($(mid)-(0.2,0)$) -- ++ (0.4,0);
    \draw[axis,<->] (mid) -- node[left, black] {$l_0$} ++(0,3.25);
    \draw[-] ($(-0.5,5-3.25-1)-(0.2,0)$) -- ++ (0.4,0);

    \node[below] (c) at (plank.south) {};

    \draw[force,->,>=open triangle 60] (c) -- ++(0,-1) node[below] {$\vec{a}_0$};
&%%%%%%%%%%%%%%%%%%%%%%%%%%%%%%%%%%%%%
%груз и доска
    \node[cargo] (a) at (1,0) {} node[right of=a, xshift=-0.5cm] {$m$};
    \node[plank, below of=a, anchor=north, yshift=-0.3cm] (plank) {};
    %верх
    \draw[interface] (-1,5) rectangle (3,5.2);
    \draw[thick] (-1,5) -- (3,5);
    %пружинка
    \draw[decoration={aspect=0.3, segment length=4.5mm, amplitude=3mm,coil},decorate] (1,5) -- node[left, black, xshift=-10pt] {$k$} (a); 

   
    \draw[-] ($(-0.5,5-3.25)-(0.2,0)$) -- ++ (0.4,0);

    \draw[axis,<->] (-0.5,5) -- node[left, black] {$l_0$} ++(0,-3.25);
    \draw[axis,<->] (-0.5,5-3.25) -- node[left, black] {$\Delta{}$} ++(0,-1) coordinate (temp);
    \draw[axis,<->] (temp) -- node[left, black] {$\delta{}$} ($(a.south)-(1.5,0)$);
    \draw[axis] ($(a.south)-(1.5,0)$) --  ($(a.south)+(1.5,0)$);
    \draw[-] ($(a.south)-(1.7,0)$) -- ++ (0.4,0);

    \draw[-] ($(-0.5,5-3.25-1)-(0.2,0)$) -- ++ (0.4,0);

    \node[below] (c) at (plank.south) {};

    \draw[force,->,>=open triangle 60] (c) -- ++(0,-1) node[below] {$\vec{a}_0$};
\\
};
\end{tikzpicture}
\vspace{-3em}
\end{figure}
1
\begin{gather}
    \xdef\mv{\frac{mv^2}{2}}%
    \xdef\mvn{\frac{0^{2}}{2}}%
    m\vec{a_0}=\vec{f_e}+m\vec{g}\\
    x: ma_0=mg-k\Delta{}\\
    \Delta{}=\frac{m}{k}(g-a_0)\\
    \mv-\mvn=\int_0^{\Delta{}} [mg-k\Delta{}]\cdot{}d[\Delta{}]\\
    \mv=mg\Delta{}-k \frac{\Delta{}^2}{2}
\end{gather}
\begin{gather}
    \mvn-\mv=\int_0^{\delta{}} [mg-k(\Delta{}+\delta{})]\cdot{}d[\delta{}]\\
    -\mv=mg\delta{}-k\Delta{}\delta{}-k \frac{\delta{}^2}{2}\\
    k \frac{\Delta{}^2}{2}-mg\Delta{}=mg\delta{}-k\Delta{}\delta{}-k \frac{\delta{}^2}{2}
\end{gather}
% Учитывая, что в данной задаче $m_1=m_2$, можем записать ответ:
% \begin{gather}
%     a_0=g\frac{1-\mu{}}{1+\mu}
% \end{gather}

% \begin{gather}
%     0=-m_1a_0+T-\mu{}gm_1\\
%     0=-m_2g-T-\mu{}m_2a_0\\
%     0=-a_0(m_1+\mu{}m_2)-g(\mu{}m_1-m_2)\\
%     a_0(m_1+\mu{}m_2)=g(m_2-\mu{}m_1)\\
%     a_0=g\frac{m_2-\mu{}m_1}{m_1+\mu{}m_2}
% \end{gather}

% Учитывая, что $m_1=m_2$:
% \begin{gather}
%     a_0=g\frac{1-\mu{}}{1+\mu}
% \end{gather}
\end{document}