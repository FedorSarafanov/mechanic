\documentclass[a5paper,10pt]{article}
 
% \usepackage{extsizes}
\usepackage{cmap}
\usepackage[T2A]{fontenc}
\usepackage[utf8x]{inputenc}
\usepackage[english, russian]{babel}

\usepackage{misccorr}

%%%%%%%%%%%%%%%%%%%%%%%%%%%%%%%%%%%%%%%%%%%%%%%%%%%%%%%%%%%%%%%%%%%%%%%%%%%%%%%%%%  
\usepackage{graphicx} % для вставки картинок
\graphicspath{{img/}}
\usepackage{amssymb,amsfonts,amsmath,amsthm} % математические дополнения от АМС

% \usepackage{fontspec}
% \usepackage{unicode-math}
\usepackage[printwatermark]{xwatermark}
\usepackage{indentfirst} % отделять первую строку раздела абзацным отступом тоже
\usepackage[usenames,dvipsnames]{color} % названия цветов
% \usepackage{xcolor}
\usepackage{makecell}
\usepackage{multirow} % улучшенное форматирование таблиц
\usepackage{ulem} % подчеркивания
\linespread{1.3} % полуторный интервал
% \renewcommand{\rmdefault}{ftm} % Times New Roman (не работает)
\frenchspacing
\usepackage{geometry}
\geometry{left=1cm,right=1cm,top=2cm,bottom=1cm,bindingoffset=0cm}
\usepackage{titlesec}
\usepackage{float}
% \definecolor{black}{rgb}{0,0,0}
% \usepackage[colorlinks, unicode, pagecolor=black]{hyperref}
% \usepackage[unicode]{hyperref} %ссылки
\usepackage{fancyhdr} %загрузим пакет
\pagestyle{fancy} %применим колонтитул
\fancyhead{} %очистим хидер на всякий случай
\fancyhead[R]{Сарафанов Ф.Г.} %номер страницы слева сверху на четных и справа на нечетных
\fancyhead[C]{Механика}
\fancyhead[L]{Яковлев -- №196} 
\fancyfoot{} %футер будет пустой
% \fancyfoot[CO,CE]{\thepage}
\renewcommand{\labelenumii}{\theenumii)}


\usepackage{tikz}
\usetikzlibrary{scopes}
\usetikzlibrary{%
     decorations.pathreplacing,%
     decorations.pathmorphing,%
    patterns,%
    calc,%
    scopes,%
    arrows,%
    % arrows.spaced,%
}
% \newwatermark*[allpages,color =magenta,angle=45,scale=2,xpos=0,ypos=0]{Проверено Дашей}
\begin{document}

\begin{figure}[H]
    \centering
\begin{tikzpicture}[
    force/.style={>=latex,draw=blue,fill=blue},
    % axis/.style={densely dashed,gray,font=\small},
    axis/.style={densely dashed,black!60,font=\small},
    interface/.style={
        pattern = north east lines,
        draw    = none,
        pattern color=gray!60,          
    },
    cargo/.style={
        rectangle,
        fill=magenta!40,
        draw=black!50,
        inner sep=2.5mm,
    },
    spring/.style={
        decoration={
            aspect=0.3, 
            segment length=.8mm, 
            amplitude=2mm,
            coil},
        decorate,
        draw=magenta!70
    }
]

    \begin{scope}[xscale=-1]
        \node[] (b) at (-0.35,0) {};
        \node[] (d) at (7,0) {};
        % \node[above, yshift=1em] at (b) {$m_1$};

        \node[] (c) at (4,0) {};
        \node[above, yshift=1em] at (c) {$M$};
        \node[above, yshift=1em] at (d) {$m$};

        % \draw[interface] (0,2.5) rectangle ++(-0.25,-5);
        \draw[spring, decoration={segment length=1.7mm}] (b) -- node[above, yshift=1em, xshift=-0.25em, black] {$k$} (c); 

        \draw[interface] (-0.25,-0.25) rectangle ++(-0.25,0.7);
        \draw[interface] (-0.5,-0.25) rectangle ++(8,-0.25);
        \draw[thick] (-0.25, -0.25) coordinate (left) -- ++(0,0.7) (left) -- ++(7.75,0);

        \draw[axis, <-] (-0.25,-0.7) node[right] {$+x$} -- ++(7.5,0) ;

        \draw (2.25,-0.8) -- ++ (0,0.2) node [below, yshift=-0.5em] {$x$};

        \draw (4,-0.8) -- ++ (0,0.2) node [below, yshift=-0.5em] {$0$};       
        \draw[force,->] (d.west) -- ++(-1,0) node[right] {$\vec{v}_{m}$};

        \draw[fill=magenta] (c) circle (0.25);
        \draw[fill=magenta] (d) circle (0.25);
    \end{scope}
\end{tikzpicture}
% \vspace{-2em}
\end{figure}

Обозначим скорости до удара $\vec{v}_m\equiv\vec{v}$ и $\vec{v}_M\equiv0$, сразу после удара - $\vec{u}_m$ и $\vec{u}_M$.
Масса легкого шара -- $m$, тяжелого -- $M$, $m<M$.

Можно упростить решение задачи, воспользовавшись выведенными формулами для упругого удара:
\begin{equation}
    \vec{u}_{m}=\frac{2m\vec{v}_{M}+(m-M)\vec{v}_{m}}{m+M}\equiv
    \vec{v}_m\frac{m-M}{m+M}
\end{equation}
Так как по условию $m<M$, то множитель в уравнении выше сугобо отрицателен, т.е. \textbf{скорость легкого шара после удара направлена влево}.

Будем считать известной формулу коэффициента передачи энергии. Начальная энергия легкого шара $\frac{mv_m^2}{2}$. Тяжелому шару он передаст энергию
\begin{equation}
    \frac{4\varkappa}{(1+\varkappa)^2}\cdot \frac{mv_m^2}{2}, \text{ где } \varkappa=\frac{m}{M}
\end{equation}

Максимальное отклонение тяжелого шара будет тогда, когда вся его кинетическая энергия уйдет на сжатие пружины:
\begin{equation}
    \frac{4\varkappa}{(1+\varkappa)^2}\cdot \frac{mv_m^2}{2}=\frac{kA^2}{2}
\end{equation}
Отсюда
\begin{equation}
    A^2=\frac{4m\cdot mv_m^2}{M(1+m/M)^2\cdot k}
\end{equation}
\begin{equation}
    A=2\sqrt\frac{m^2v_m^2}{kM}\frac{M}{m+M}=2\sqrt\frac{M}{k}\frac{mv_m}{m+M}
\end{equation}
\end{document}