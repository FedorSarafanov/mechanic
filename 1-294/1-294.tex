\documentclass[a5paper,10pt]{article}
\def\source{/home/lab/tex/templates}

\usepackage{cmap}
\usepackage[T2A]{fontenc}
\usepackage[utf8x]{inputenc}
\usepackage[english, russian]{babel}

\usepackage
	{
		amssymb,
		% misccorr,
		amsfonts,
		amsmath,
		amsthm,
		wrapfig,
		makecell,
		multirow,
		indentfirst,
		ulem,
		graphicx,
		geometry,
		fancyhdr,
		subcaption,
		float,
		tikz,
		csvsimple,
		color,
	}  

\usepackage[outline]{contour}
\usepackage[mode=buildnew]{standalone}


\geometry
	{
		left=1cm,
		right=1cm,
		top=2cm,
		bottom=1cm,
		bindingoffset=0cm,
	}

\linespread{1.3} 
\frenchspacing 


\usetikzlibrary{scopes}
\usetikzlibrary
	{
		decorations.pathreplacing,
		decorations.pathmorphing,
		patterns,
		calc,
		scopes,
		arrows,
		through,
		shapes.misc,
		arrows.meta,
	}


\tikzset{
	force/.style=	{
		>=latex,
		draw=blue,
		fill=blue,
				 	}, 
	%				 	
	axis/.style=	{
		densely dashed,
		gray,
		font=\small,
					},
	%
	acceleration/.style={
		>=open triangle 60,
		draw=blue,
		fill=blue,
					},
	%
	inforce/.style=	{
		force,
		double equal sign distance=2pt,
					},
	%
	interface/.style={
		pattern = north east lines, 
		draw    = none, 
		pattern color=gray!60,
					},
	cross/.style=	{
		cross out, 
		draw=black, 
		minimum size=2*(#1-\pgflinewidth), 
		inner sep=0pt, outer sep=0pt,
					},
	%
	cargo/.style=	{
		rectangle, 
		fill=black!70, 
		inner sep=2.5mm,
					},
	%
	}

\pagestyle{fancy} %применим колонтитул
\fancyhead{} %очистим хидер на всякий случай
\fancyhead[R]{Сарафанов Ф.Г.} %номер страницы слева сверху на четных и справа на нечетных
\fancyhead[C]{Механика}
% \fancyhead[L]{Задача под запись - <<АУУ-2>>} 
\fancyfoot{} %футер будет пустой

\newcommand{\irodov}[1]{\fancyhead[L]{Иродов -- №#1}}
\newcommand{\yakovlev}[1]{\fancyhead[L]{Яковлев -- №#1}}
\newcommand{\wrote}[1]{\fancyhead[L]{Под запись -- <<#1>>}}

\newenvironment{tikzpict}
    {
	    \begin{figure}[htbp]
		\centering
		\begin{tikzpicture}
    }
    { 
		\end{tikzpicture}
		% \caption{caption}
		% \label{fig:label}
		\end{figure}
    }

\newcommand{\vbLabel}[3]{\draw ($(#1,#2)+(0,5pt)$) -- ($(#1,#2)-(0,5pt)$) node[below]{#3}}
\newcommand{\vaLabel}[3]{\draw ($(#1,#2)+(0,5pt)$) node[above]{#3} -- ($(#1,#2)-(0,5pt)$) }

\newcommand{\hrLabel}[3]{\draw ($(#1,#2)+(5pt,0)$) -- ($(#1,#2)-(5pt,0)$) node[right, xshift=1em]{#3}}
\newcommand{\hlLabel}[3]{\draw ($(#1,#2)+(5pt,0)$) node[left, xshift=-1em]{#3} -- ($(#1,#2)-(5pt,0)$) }

% Draw line annotation
% Input:
%   #1 Line offset (optional)
%   #2 Line angle
%   #3 Line length
%   #5 Line label
% Example:
%   \lineann[1]{30}{2}{$L_1$}
\newcommand{\lineann}[4][0.5]{%
    \begin{scope}[rotate=#2, blue,inner sep=2pt, ]
        \draw[dashed, blue!40] (0,0) -- +(0,#1)
            node [coordinate, near end] (a) {};
        \draw[dashed, blue!40] (#3,0) -- +(0,#1)
            node [coordinate, near end] (b) {};
        \draw[|<->|] (a) -- node[fill=white, scale=0.8] {#4} (b);
    \end{scope}
}


\irodov{1.294}

\begin{document}

\begin{tikzpict}
	\draw[interface] (-2,3) rectangle (2,3.5);
	\draw[thick] (-2,3) -- ++(4,0);
	% \clip (-6,-1) rectangle (6,4.25);

	\draw (0,0) circle (2cm);
	\draw (0,0) circle (1cm);

	\draw (-1,0) coordinate (1) -- ++(0,3) coordinate (2);
	\draw (-2,0) coordinate (3) -- ++(0,-3) coordinate (4);

	\fill[magenta] (4) rectangle ++(-0.25,-1);
	\fill[magenta] (4) rectangle ++(0.25,-1);

	\draw[force,->] (1) -- ++(0,1) node[left] {$\vec{T}$};
	\draw[force,->] (2) -- ++(0,-1) node[left] {$\vec{T'}$};

	\draw[force,->] (3) -- ++(0,-1) node[left] {$\vec{f}$};
	\draw[force,->] (4) -- ++(0,1) node[left] {$\vec{f'}$};

	\draw[] (0,0) coordinate (o) node[magenta, scale=1.5] {$\bigotimes$} node[above, yshift=0.8em] {$+z$};	

	\draw[] (-3,0) coordinate (o) node[magenta, scale=1.5] {$\bigotimes$} node[above, yshift=0.8em] {$\vec{M}_T$};	
	\draw[] (-4,0) coordinate (o) node[magenta, scale=1.5] {$\bigodot$} node[above, yshift=0.8em] {$\vec{M}_f$};	

	\draw[axis,->,black] (3,3) -- (3,-2) node[below] {$+x$};
	% \draw (0,0) circle (2cm);
	% \fill[magenta] (0,0) circle (2pt) coordinate (c);
	% \fill[magenta] (0,3) circle (2pt) coordinate (o);
	% \draw (0,3) circle (1cm);

	% \draw[axis] (0,0) circle (3cm);
	% \draw[axis] (0,0) -- (0,5);

	% \draw (0,0)  ++(150:3) coordinate (oo) circle(1cm);
	% \draw[axis,-<] (0,0) -- ++(150:5)node[left] {$-n$};
	% \draw[axis] (0,0) -- (0,5);
	% \draw[axis] (oo) -- ++(0,3.5);
	% \draw[axis] (oo) -- ++(0,-3.5) coordinate (phi);
	% \draw[force,->] (0,0) ++(150:2) -- ++ (150:1.5)node[above] {$\vec{N}$};

	% \draw[force,->] (0,0) ++(150:3) -- ++ (240:1.5)node[below] {$\vec{v}$};

	% \draw[force,->] (oo) -- ++ (0,-1.5) node[below] {$m\vec{g}$};
	% \fill[magenta] (oo) circle (2pt);


	% \draw pic["$\phi$",draw=magenta,<-,angle eccentricity=1.5,angle radius=.5cm] {angle=phi--oo--c};   
	
	% \draw pic["$\phi$",draw=magenta,<-,angle 
	% eccentricity=1.5,angle radius=0.5cm] {angle=o--c--oo};   

	% \lineann[6]{90}{1.5}{$h^*=h_0\cdot\cos\phi$}
	% \lineann[-5]{90}{3}{$h_0=r+R$}

	% \contourlength{2mm}
	% \draw (2,0) node[] {\contour{white}{$W_\text{п}=0$}};
\end{tikzpict}

Запишем второй закон Ньютона в проекции на ось $x$ (с учетом $f'=f$):
\begin{equation}
	\left\{\begin{aligned}
		Ma_{2x}=Mg+f-T\\
		Ma_{1x}=mg-f
	\end{aligned}\right.
\end{equation}
Уравнение моментов относительно оси $z$:
\begin{equation}
	I\gamma_z=RT-2Rf
\end{equation}
И уравнения кинематической связи (проскальзывания нет):
\begin{equation}
	R\gamma_z=a_{2x}
\end{equation}
\begin{equation}
	2R\gamma_z=-a'_{1x}
\end{equation}
Где (груз движется относительно блока)
\begin{equation}
	a_{1x}=a_{2x}+a'_{1x}
\end{equation}
Отсюда
\begin{equation}
	a_{1x}=-R\gamma_z
\end{equation}
Решим систему, составленную из предыдущих уравнений, относительно $a_x\equiv a_{2x}$
\begin{equation}
	M\gamma_z R-m\gamma_z R=Mg+mg-T
\end{equation}
\begin{equation}
	\gamma_z(MR-mR)=Mg+mg-T
\end{equation}
\begin{equation}
	f=mg+mR\gamma_z
\end{equation}
\begin{equation}
	T=2f+\frac{I\gamma_z}{R}=2mg+2mR\gamma_z+\frac{I\gamma_z}{R}
\end{equation}
\begin{equation}
	\gamma_z(MR-mR+2mR+\frac{I}{R})=g(M-m)
\end{equation}
\begin{equation}
	a_x=-R\gamma_z=\frac{Rg(m-M)}{MR+mR+\frac{I}{R}}
\end{equation}
И окончательный ответ:
\begin{equation}
	a_x=\frac{g(m-M)}{M+m+\frac{I}{R^2}}
\end{equation}
pp
\end{document}