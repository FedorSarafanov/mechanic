\documentclass[a5paper,10pt]{article}
\def\source{/home/lab/tex/templates}

\usepackage{cmap}
\usepackage[T2A]{fontenc}
\usepackage[utf8x]{inputenc}
\usepackage[english, russian]{babel}

\usepackage
	{
		amssymb,
		% misccorr,
		amsfonts,
		amsmath,
		amsthm,
		wrapfig,
		makecell,
		multirow,
		indentfirst,
		ulem,
		graphicx,
		geometry,
		fancyhdr,
		subcaption,
		float,
		tikz,
		csvsimple,
		color,
	}  

\usepackage[outline]{contour}
\usepackage[mode=buildnew]{standalone}


\geometry
	{
		left=1cm,
		right=1cm,
		top=2cm,
		bottom=1cm,
		bindingoffset=0cm,
	}

\linespread{1.3} 
\frenchspacing 


\usetikzlibrary{scopes}
\usetikzlibrary
	{
		decorations.pathreplacing,
		decorations.pathmorphing,
		patterns,
		calc,
		scopes,
		arrows,
		through,
		shapes.misc,
		arrows.meta,
	}


\tikzset{
	force/.style=	{
		>=latex,
		draw=blue,
		fill=blue,
				 	}, 
	%				 	
	axis/.style=	{
		densely dashed,
		gray,
		font=\small,
					},
	%
	acceleration/.style={
		>=open triangle 60,
		draw=blue,
		fill=blue,
					},
	%
	inforce/.style=	{
		force,
		double equal sign distance=2pt,
					},
	%
	interface/.style={
		pattern = north east lines, 
		draw    = none, 
		pattern color=gray!60,
					},
	cross/.style=	{
		cross out, 
		draw=black, 
		minimum size=2*(#1-\pgflinewidth), 
		inner sep=0pt, outer sep=0pt,
					},
	%
	cargo/.style=	{
		rectangle, 
		fill=black!70, 
		inner sep=2.5mm,
					},
	%
	}

\pagestyle{fancy} %применим колонтитул
\fancyhead{} %очистим хидер на всякий случай
\fancyhead[R]{Сарафанов Ф.Г.} %номер страницы слева сверху на четных и справа на нечетных
\fancyhead[C]{Механика}
% \fancyhead[L]{Задача под запись - <<АУУ-2>>} 
\fancyfoot{} %футер будет пустой

\newcommand{\irodov}[1]{\fancyhead[L]{Иродов -- №#1}}
\newcommand{\yakovlev}[1]{\fancyhead[L]{Яковлев -- №#1}}
\newcommand{\wrote}[1]{\fancyhead[L]{Под запись -- <<#1>>}}

\newenvironment{tikzpict}
    {
	    \begin{figure}[htbp]
		\centering
		\begin{tikzpicture}
    }
    { 
		\end{tikzpicture}
		% \caption{caption}
		% \label{fig:label}
		\end{figure}
    }

\newcommand{\vbLabel}[3]{\draw ($(#1,#2)+(0,5pt)$) -- ($(#1,#2)-(0,5pt)$) node[below]{#3}}
\newcommand{\vaLabel}[3]{\draw ($(#1,#2)+(0,5pt)$) node[above]{#3} -- ($(#1,#2)-(0,5pt)$) }

\newcommand{\hrLabel}[3]{\draw ($(#1,#2)+(5pt,0)$) -- ($(#1,#2)-(5pt,0)$) node[right, xshift=1em]{#3}}
\newcommand{\hlLabel}[3]{\draw ($(#1,#2)+(5pt,0)$) node[left, xshift=-1em]{#3} -- ($(#1,#2)-(5pt,0)$) }

% Draw line annotation
% Input:
%   #1 Line offset (optional)
%   #2 Line angle
%   #3 Line length
%   #5 Line label
% Example:
%   \lineann[1]{30}{2}{$L_1$}
\newcommand{\lineann}[4][0.5]{%
    \begin{scope}[rotate=#2, blue,inner sep=2pt, ]
        \draw[dashed, blue!40] (0,0) -- +(0,#1)
            node [coordinate, near end] (a) {};
        \draw[dashed, blue!40] (#3,0) -- +(0,#1)
            node [coordinate, near end] (b) {};
        \draw[|<->|] (a) -- node[fill=white, scale=0.8] {#4} (b);
    \end{scope}
}


\yakovlev{404}

\begin{document}

\begin{tikzpict}
    \contourlength{0.5mm};

	\draw[very thick] (0,-1.5) -- (0,1.5);
    \draw[magenta] (0,0) node[] {\contour{white}{$\bigodot$}} node[right, xshift=.5em] {$O$};

    \draw[magenta] (0,1) node[] {\contour{white}{$\times$}};

	% \draw[interface] (-2,0) rectangle (2,-0.25);
	% \draw[thick] (-2,0) -- (2,0);

	% \draw (0,1) circle (1);



 %    \draw[fill=magenta] (0,0) circle (2pt) node[above, yshift=3pt] {\contour{white}{$B$}};
    \lineann[1.5]{90}{1}{$x$}
 	\begin{scope}[yshift=-1.5cm] 		
    \lineann[1]{90}{3}{$l$}
 	\end{scope}

    \draw[fill=magenta] (2,1) circle (2pt);
    \draw[->] (2,1) -- ++ (-0.5,0) node[left] {$\vec{v}$};

    \draw (-2.5,0) node[blue] {$\bigotimes$} node[right, xshift=.5em] {$\vec{g}$};
\end{tikzpict}
Обозначим скорость шарика перед ударом $v$ (после удара $0$), угловую скорость стержня после удара $\omega$, скорость центра масс стержня $u$.

Момент стержня относительно $O$ считаем известным:
\begin{equation}
	I_o=\frac{Ml^2}{12}
\end{equation}
Запишем ЗСЭ, а также ЗСИ и ЗСМИ (относительно $O$) в проекции на ось $z$ (через точку $O$ на нас):
\begin{equation}
	\left\{
	\begin{aligned}
		mv^2&=Mu^2+I\omega^2\\
		mv&=Mu_x\\
		mvx&=I\omega_z
	\end{aligned}
	\right.
\end{equation}
Откуда
\begin{equation}
	\omega_z=\frac{mvx}{I}%=mvx\cdot\frac{12}{Ml^2}
\end{equation}
\begin{equation}
	u_x=\frac{mv}{M}
\end{equation}
Подставим в формулу ЗСЭ:
\begin{equation}
	mv^2=M\frac{m^2v^2}{M^2}+I\frac{m^2v^2x^2}{I^2}
	\quad\Rightarrow\quad
	1=\frac{m}{M}+\frac{mx^2}{I}
\end{equation}
Откуда после подстановки момента инерции
\begin{equation}
	x=l\sqrt{\frac{1}{12}\left(\frac{M}{m}-1\right)}	
\end{equation}
Отсюда следует, что $M\geq m$.
Логично, что $x\leq \frac{l}{2}$. Тогда
\begin{equation}
	{\frac{1}{12}\left(\frac{M}{m}-1\right)}\leq\frac14
	\quad\Rightarrow\quad
	M\leq4m
\end{equation}
Т.е.
\begin{equation}
	m\leq M\leq 4m
\end{equation}
\end{document}