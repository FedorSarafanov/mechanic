\documentclass[a5paper,10pt]{article}\usepackage[usenames,dvipsnames]{color}

\usepackage{cmap,graphicx,etoolbox,misccorr,indentfirst,makecell,multirow,ulem,geometry,amssymb,amsfonts,amsmath,amsthm,titlesec,float,fancyhdr,wrapfig,tikz}

\usepackage[T2A]{fontenc}\usepackage[utf8x]{inputenc}\usepackage[english, russian]{babel}\usetikzlibrary{decorations.pathreplacing,decorations.pathmorphing,patterns,calc,scopes,arrows,through, shapes.misc}\graphicspath{{img/}}\linespread{1.3}\frenchspacing\geometry{left=1cm, right=1cm, top=2cm, bottom=1cm, bindingoffset=0cm}\pagestyle{fancy}\fancyhead{}\fancyhead[R]{Сарафанов Ф.Г.}\fancyhead[C]{Механика}
\fancyhead[L]{Иродов -- №1.382}
\fancyfoot{}

%Команда \beforetext для текста слева от формулы
\makeatletter \newif\if@gather@prefix \preto\place@tag@gather{\if@gather@prefix\iftagsleft@ \kern-\gdisplaywidth@ \rlap{\gather@prefix} \kern\gdisplaywidth@ \fi\fi } \appto\place@tag@gather{\if@gather@prefix\iftagsleft@\else \kern-\displaywidth \rlap{\gather@prefix} \kern\displaywidth \fi\fi \global\@gather@prefixfalse } \preto\place@tag{\if@gather@prefix\iftagsleft@ \kern-\gdisplaywidth@ \rlap{\gather@prefix} \kern\displaywidth@ \fi\fi } \appto\place@tag{\if@gather@prefix\iftagsleft@\else \kern-\displaywidth \rlap{\gather@prefix} \kern\displaywidth \fi\fi \global\@gather@prefixfalse } \newcommand*{\beforetext}[1]{\ifmeasuring@\else \gdef\gather@prefix{#1} \global\@gather@prefixtrue \fi } \makeatother 
\tikzset{force/.style={>=latex,draw=blue,fill=blue}, axis/.style={densely dashed,gray,font=\small}, acceleration/.style={>=open triangle 60,draw=blue,fill=blue}, inforce/.style={force,double equal sign distance=2pt}, interface/.style={pattern = north east lines, draw    = none, pattern color=gray!60, }, cross/.style={cross out, draw=black, minimum size=2*(#1-\pgflinewidth), inner sep=0pt, outer sep=0pt},    cargo/.style={rectangle, fill=black!70, inner sep=2.5mm, }}

\begin{document}
\begin{figure}[H]
    \centering
\begin{tikzpicture}
    \draw[axis,->] (0,0) -- (5,0) node[right, black] {$x$};
    \draw[force,->] (0,1) -- node[above, black] {$\vec{v}_1$} ++(2,0);
    \draw[force,->] (5,1) -- node[above, black] {$\vec{v}_2$} ++(-2,0);

\end{tikzpicture}
\vspace{-1em}
\end{figure}
В лабораторной системе отсчета:
\begin{gather*}
    v_\text{f}=v_1+v_2=1.25c
\end{gather*}
\begin{figure}[H]
    \centering
\begin{tikzpicture}
    \draw[axis,->] (-1,0) -- (5,0) node[right, black] {$x$};

    \draw[axis,->] (0,1) -- ++(0,1) node[right, black] {$K'$};С
    \draw[axis,->] (-1,0) -- ++(0,3) node[right, black] {$K$};

    \draw[force,->] (0,1) -- node[above, black] {$\vec{v}_1$} ++(2,0);
    \draw[force,->] (5,1) -- node[above, black] {$\vec{v}_2$} ++(-2,0);

\end{tikzpicture}
\vspace{-1em}
\end{figure}
Введем СО, связанную с первой частицей. Тогда относительно неё скорость второй частицы $-v_2$, и по релятивистскому закону сложения скоростей (в проекции на ось $x$: 
\begin{gather*} 
    v'_{f_\text{отн}}=\frac{-v_2-v_1}{1-v_1v_2/c^2}=-\frac{v_1+v_2}{1-v_1v_2/c^2}=0.91c
\end{gather*}

\end{document}