\documentclass[a5paper,10pt]{article}
 
\usepackage{extsizes}
\usepackage{cmap}
\usepackage[T2A]{fontenc}
\usepackage[utf8x]{inputenc}
\usepackage[english, russian]{babel}

\usepackage{misccorr}

%%%%%%%%%%%%%%%%%%%%%%%%%%%%%%%%%%%%%%%%%%%%%%%%%%%%%%%%%%%%%%%%%%%%%%%%%%%%%%%%%%  
\usepackage{graphicx} % для вставки картинок
\graphicspath{{img/}}
\usepackage{amssymb,amsfonts,amsmath,amsthm} % математические дополнения от АМС

% \usepackage{fontspec}
% \usepackage{unicode-math}

\usepackage{indentfirst} % отделять первую строку раздела абзацным отступом тоже
\usepackage[usenames,dvipsnames]{color} % названия цветов
\usepackage{makecell}
\usepackage{multirow} % улучшенное форматирование таблиц
\usepackage{ulem} % подчеркивания
\linespread{1.3} % полуторный интервал
% \renewcommand{\rmdefault}{ftm} % Times New Roman (не работает)
\frenchspacing
\usepackage{geometry}
\geometry{left=1cm,right=1cm,top=2cm,bottom=1cm,bindingoffset=0cm}
\usepackage{titlesec}
\usepackage{float}
% \makeatletter
\newif\if@gather@prefix 
\preto\place@tag@gather{% 
  \if@gather@prefix\iftagsleft@ 
    \kern-\gdisplaywidth@ 
    \rlap{\gather@prefix}% 
    \kern\gdisplaywidth@ 
  \fi\fi 
} 
\appto\place@tag@gather{% 
  \if@gather@prefix\iftagsleft@\else 
    \kern-\displaywidth 
    \rlap{\gather@prefix}% 
    \kern\displaywidth 
  \fi\fi 
  \global\@gather@prefixfalse 
} 
\preto\place@tag{% 
  \if@gather@prefix\iftagsleft@ 
    \kern-\gdisplaywidth@ 
    \rlap{\gather@prefix}% 
    \kern\displaywidth@ 
  \fi\fi 
} 
\appto\place@tag{% 
  \if@gather@prefix\iftagsleft@\else 
    \kern-\displaywidth 
    \rlap{\gather@prefix}% 
    \kern\displaywidth 
  \fi\fi 
  \global\@gather@prefixfalse 
} 
\newcommand*{\beforetext}[1]{% 
  \ifmeasuring@\else
  \gdef\gather@prefix{#1}% 
  \global\@gather@prefixtrue 
  \fi
} 
\makeatother
% \definecolor{black}{rgb}{0,0,0}
% \usepackage[colorlinks, unicode, pagecolor=black]{hyperref}
% \usepackage[unicode]{hyperref} %ссылки
\usepackage{fancyhdr} %загрузим пакет
\pagestyle{fancy} %применим колонтитул
\fancyhead{} %очистим хидер на всякий случай
\fancyhead[LE,RO]{Сарафанов Ф.Г.} %номер страницы слева сверху на четных и справа на нечетных
\fancyhead[CO, CE]{Механика}
\fancyhead[LO,RE]{Яковлев -- №114} 
\fancyfoot{} %футер будет пустой
% \fancyfoot[CO,CE]{\thepage}
\renewcommand{\labelenumii}{\theenumii)}


\usepackage{tikz}
\usetikzlibrary{scopes}
\usetikzlibrary{%
     decorations.pathreplacing,%
     decorations.pathmorphing,%
    patterns,%
    calc,%
    scopes,%
    positioning,%
    shapes,%
    arrows,%
    % arrows.spaced,%
}

\tikzset{%
  raindrop/.pic={
    code={\tikzset{scale=1/10}
 \clip [preaction={top color=blue!50!cyan!50, bottom color=blue!50!cyan}]
 (0,0)  .. controls ++(0,-1) and ++(0,1) .. (1,-2)
 arc (360:180:1)
 .. controls ++(0,1) and ++(0,-1) .. (0,0) -- cycle;
% %
 \foreach \j in {1,3,...,20}
 \shade [top color=blue!50!cyan!50, shift=(270:0), xscale=1-\j/40,yscale=1-\j/80, white, opacity=1/15]
 [rotate=-\j] (0,0)  .. controls ++(0,-1) and ++(0,1) .. (1,-2)
 arc (360:180:1)
 .. controls ++(0,1) and ++(0,-1) .. (0,0) -- cycle;
  }}}

\begin{document}

\begin{figure}[H]
    \centering
\begin{tikzpicture}[
    force/.style={>=latex,draw=blue,fill=blue},
    % axis/.style={densely dashed,gray,font=\small},
    axis/.style={densely dashed,black!60,font=\small},
    M/.style={rectangle,draw,fill=lightgray,minimum size=0.5cm,thin},
    m2/.style={draw=black!30, rectangle,draw,thin, fill=blue!2, minimum width=0.7cm,minimum height=0.7cm},
    m1/.style={draw=black!30, rectangle,draw,thin, fill=blue!2, minimum width=0.7cm,minimum height=0.7cm},
    plane/.style={draw=black!30, very thick, fill=blue!5, line width=1pt},
    % base/.style={draw=black!70, very thick, fill=blue!4, line width=2pt},
    string/.style={draw=black, thick},
    pulley/.style={thick},
    % interface/.style={draw=gray!60,
    %     % The border decoration is a path replacing decorator. 
    %     % For the interface style we want to draw the original path.
    %     % The postaction option is therefore used to ensure that the
    %     % border decoration is drawn *after* the original path.
    %     postaction={draw=gray!60,decorate,decoration={border,angle=-135,
    %                 amplitude=0.3cm,segment length=2mm}}},
    interface/.style={
        pattern = north east lines,
        draw    = none,
        pattern color=gray!60,          
    },
    plank/.style={
        fill=black!60, 
        draw=black,
        minimum width=3cm,
        inner ysep=0.1cm,
        outer sep=0pt,
        yshift=0.75cm,
        pattern = north east lines,
        pattern color=gray!60, 
    },
    cargo/.style={
        rectangle,
        fill=black!70,              
        inner sep=2.5mm,
    }
]


    % \draw[force,->] (0,0) ++(0.05,0) -- ++(0,1) node[above] {$\vec{N}$};

    % \draw[force,->] (0,0) ++(0.05,0) -- ++(0,-1) node[right] {$\vec{P}$};
    \draw[axis,->] (-3,1) -- ++(0,-5) node[below] {${x}$};


    \draw[axis] (1,1) -- ++(0,-5);
    \draw[axis] (-1,1) -- ++(0,-5);
    \draw [decorate,decoration={brace,amplitude=5pt},xshift=0pt,yshift=0pt] (-3,-2) -- (-3,-1) node [black,midway,xshift=-2cm] {$l(t)=x_2(t)-x_1(t)$};

    \draw[axis] (1,-2) -- ++(-4,0);
    \draw[axis] (-1,-1) -- ++(-2,0);

    \node[cloud, fill=white, cloud puffs=12.7, cloud ignores aspect, minimum width=4cm,   minimum height=2cm, align=center, draw] (cloud) at (0cm, 0cm) {};%{Облако};

    \draw[force,->] (1,-2.35) -- ++(0,-1) node[below] {$m\vec{g}$};
    \draw[force,->] (1,-2.35) -- ++(0,1) node[above] {$\vec{f}_R$};
    \draw (1,-2) pic[scale=2] {raindrop};

    \draw[force,->] (-1,-1.35) -- ++(0,-1) node[below] {$m\vec{g}$};
    \draw[force,->] (-1,-1.35) -- ++(0,1) node[above] {$\vec{f}_R$};
    \draw (-1,-1) pic[scale=2] {raindrop};



\end{tikzpicture}
% \vspace{-2em}
\end{figure}

\textbf{Движение без сопротивления.}
Запишем второй закон Ньютона и последовательным интегрированием найдем зависимость координат шариков от времени, и соответственно, и $l(t)$:
\begin{equation*}
    \begin{aligned}[c]
        ma_1&=mg\\
        m\,dv_1&=mg\,dt\\
        \int_0^{v_1(t)}\,dv_1&=\int_0^{t-\tau} g\,dt\\   
        v_1(t)&=g(t-\tau)\\
        \int_0^{x_1(t)}\,dx_1&=\int_0^{t-\tau} gt\,dt\\ 
        x_1(t)&=g\frac{(t-\tau)^2}{2}
    \end{aligned}
        \qquad\qquad
    \begin{aligned}[c]
        ma_2&=mg\\
        m\,dv_2&=mg\,dt\\
        \int_0^{v_2(t)}\,dv_2&=\int_0^t g\,dt\\   
        v_2(t)&=gt\\
        \int_0^{x_2(t)}\,dx_2&=\int_0^t gt\,dt\\ 
        x_2(t)&=\frac{gt^2}{2}
    \end{aligned}
\end{equation*}

\begin{equation*}
     l(t)=g\frac{(t-\tau)^2}{2}-\frac{gt^2}{2}=\frac{g}{2}(2\tau{t}+t^2)=gt(\tau+ \frac{t}{2} )
\end{equation*}

\textbf{Движение с сопротивлением.}
\begin{equation*}
    \xdef\km{\frac{k}{m}}\xdef\mk{\frac{m}{k}}\xdef\mgk{\frac{mg}{k}}%
    \begin{aligned}[c]
        m\vec{a}&=m\vec{g}-k\vec{v}\\
        \text{x: } ma_x&=mg-kv_x\\
    \end{aligned}
        \qquad\qquad
    \begin{aligned}[c]
        \frac{dv_x}{dt}=g-\km{}v_x\\
        \frac{dv_x}{g-\km{}v_x}=dt\\
    \end{aligned}
\end{equation*}
Сделаем замену переменной $c=g-\km{}v_x$ под интегралом, соотвественно новые пределы будут $c[v_x=0]=g$ и $c[v_x=v_x(t)]=g-\km{}v_x(t)$, а дифференциал $dv_x=-\mk{}dc$. Тогда можем найти $v_x(t)$:
\begin{equation*}
    \begin{aligned}[c]
        \int_0^{t}dt=-\mk\int_g^{g-\km{}v_x(t)}\frac{dc}{c}\\
        t=-\mk{\ln{(1-\frac{kv_x(t)}{mg})}}\\
    \end{aligned}
        \qquad\qquad
    \begin{aligned}[c]
        1-\frac{kv_x(t)}{mg}=e^{-\km{t}}\\
        v_x(t)=\mgk(1-e^{-\km{t}})\\
    \end{aligned}
\end{equation*}
Проинтегрировав еще раз, найдем $x(t)$:
\begin{gather*}
    \int_0^{x(t)} dx = \int_0^t \mgk(1-e^{-\km{t}}) dt\\
    x(t)=\mgk(t+\mk{}[e^{-\km{t}}-1])\\
\end{gather*}
Тогда, зная, что движения шариков подобны и отличаются только временем начала движения, найдем $l(t)$:
\begin{gather*}
    l(t)=x(t+\tau)-x(t)=\mgk(\tau+\mk{}[e^{-\km{t}}-1]-\mk{}[e^{-\km{t}}\cdot{}e^{-\km{\tau}}-1])=\\=\mgk(\tau+\mk{e^{-\km{t}}}-\mk-\mk{e^{-\km{t}}\cdot{}e^{-\km{\tau}}}+\mk)=\\
    =\mgk(\tau+\mk{e^{-\km{t}}}[1-e^{-\km{\tau}}])    
\end{gather*}



\end{document}