\documentclass[a5paper,10pt]{article}
 
\usepackage{extsizes}
\usepackage{cmap}
\usepackage[T2A]{fontenc}
\usepackage[utf8x]{inputenc}
\usepackage[english, russian]{babel}

\usepackage{misccorr}

%%%%%%%%%%%%%%%%%%%%%%%%%%%%%%%%%%%%%%%%%%%%%%%%%%%%%%%%%%%%%%%%%%%%%%%%%%%%%%%%%%  
\usepackage{graphicx} % для вставки картинок
\graphicspath{{img/}}
\usepackage{amssymb,amsfonts,amsmath,amsthm} % математические дополнения от АМС

% \usepackage{fontspec}
% \usepackage{unicode-math}

\usepackage{indentfirst} % отделять первую строку раздела абзацным отступом тоже
\usepackage[usenames,dvipsnames]{color} % названия цветов
\usepackage{makecell}
\usepackage{multirow} % улучшенное форматирование таблиц
\usepackage{ulem} % подчеркивания
\linespread{1.3} % полуторный интервал
% \renewcommand{\rmdefault}{ftm} % Times New Roman (не работает)
\frenchspacing
\usepackage{geometry}
\geometry{left=1cm,right=1cm,top=2cm,bottom=1cm,bindingoffset=0cm}
\usepackage{titlesec}
\usepackage{float}
% \definecolor{black}{rgb}{0,0,0}
% \usepackage[colorlinks, unicode, pagecolor=black]{hyperref}
% \usepackage[unicode]{hyperref} %ссылки
\usepackage{fancyhdr} %загрузим пакет
\pagestyle{fancy} %применим колонтитул
\fancyhead{} %очистим хидер на всякий случай
\fancyhead[R]{Сарафанов Ф.Г.} %номер страницы слева сверху на четных и справа на нечетных
\fancyhead[C]{Механика}
\fancyhead[L]{Задача под запись - <<АУУ-1>>} 
\fancyfoot{} %футер будет пустой
% \fancyfoot[CO,CE]{\thepage}
\renewcommand{\labelenumii}{\theenumii)}
\usepackage[outline]{contour}

\usepackage{tikz}
\usetikzlibrary{scopes}
\usetikzlibrary{%
     decorations.pathreplacing,%
     decorations.pathmorphing,%
    patterns,%
    calc,%
    scopes,%
    arrows,%
    % arrows.spaced,%
}

\begin{document}

\begin{figure}[H]
    \centering
\begin{tikzpicture}[
    force/.style={>=latex,draw=blue,fill=blue},
    % axis/.style={densely dashed,gray,font=\small},
    axis/.style={densely dashed,black!60,font=\small},
    interface/.style={
        pattern = north east lines,
        draw    = none,
        pattern color=gray!60,          
    },
    cargo/.style={
        rectangle,
        fill=magenta!40,
        draw=black!50,
        inner sep=2.5mm,
    },
    spring/.style={
        decoration={
            aspect=0.3, 
            segment length=.8mm, 
            amplitude=2mm,
            coil},
        decorate,
        draw=magenta!25
    },
    interface1/.style={draw=gray!60,
        % The border decoration is a path replacing decorator. 
        % For the interface style we want to draw the original path.
        % The postaction option is therefore used to ensure that the
        % border decoration is drawn *after* the original path.
        postaction={draw=gray!60,decorate,decoration={border,angle=-135,
                    amplitude=0.3cm,segment length=2mm}}},    
    scale=0.75
]
\def\angle{41}
%%%%%%%%%%%%%%%%%%%%%%%%%%%%%%%%%%%%%%
    \draw[force,->] (-3,-3) -- ++(1,0) node[right]{$\vec{v}_0$};
    \contourlength{1mm};

    \draw[thick] (0,0) --  ++(0,-3);
    \draw[thick,->] (0,0) -- node[right] {$\vec{l}$} ++(0,-2.65);

    \draw[thick,->] (0,0) -- node[right, above] {$\vec{r}$} ++(-2.8,-2.69);

    \draw[interface] (-3,0) rectangle ++(6,0.5);
    \draw[line width=0.4mm] (-3,0) -- ++(6,0);
    \draw[fill=magenta] (0,0) circle (3pt) node[above, yshift=3pt] {\contour{white}{$O$}};
    
    \draw[fill=magenta] (0,-3) circle (0.35) node {\contour{white}{$m$}};
    \draw[fill=magenta] (-3,-3) circle (0.35) node {\contour{white}{$m$}};

    \draw[axis,->] (-3,-4) -- ++(6,0) node[right] {$+x$};

    \draw[force,->] (0,0) -- (-1,0) node[left, below]{$\vec{f}$};

\end{tikzpicture}
% \vspace{-2em}
\end{figure}
\begin{equation}
    \vec{M}_0=[\vec{0}\,\times\,\vec{f}]=0 \quad \Longrightarrow \vec{N}=\text{const}
\end{equation}
Тогда можно записать закон сохранения момента импульса до удара и сразу после него:
\begin{equation}
    \label{eq1}
    \vec{N}_\text{нач}=
    \vec{N}_\text{кон}
\end{equation}
\begin{equation}
    \vec{N}_\text{нач}=[\vec{r}\,\times\,{m\vec{v}_0}]
    \equiv [\vec{l}\,\times\,{m\vec{v}_0}]
\end{equation}
\begin{equation}
    \vec{N}_\text{кон}=[\vec{l}\,\times\,{m\vec{u}_1}]+
    [\vec{l}\,\times\,{m\vec{u}_2}]
\end{equation}

В проекции на ось $z$ -- к нам:
\begin{equation}
    l\cdot mv_0 = l\cdot mu_{1x}+l\cdot mu_{2x}
\end{equation}
Равенство будет выполнятся и векторно, т.е. выполняется ЗСИ - легко проверить, приведя общие слагаемые в (\ref{eq1}). Отсюда, в частности, следует $u\equiv|u_x|$.

При АУУ сохраняется $W_k$, отсюда
\begin{equation}
    \label{eq2}
    mv_0^2=mu_1^2+mu_2^2
\end{equation}
Решением системы (\ref{eq1},\ref{eq2}), имеющим физический смысл, будет $\vec{u_2}=\vec{v}_0$, $\vec{u}_1=0$.

\begin{equation}
    \Delta p_x=mu_2-mv_0=0
\end{equation}
По определению, $ F=m\frac{dv}{dt}$ ($\vec{F}dt=d\vec{p}$).

Физически, $\Delta t \to dt$, если $t$ -- мало. Тогда средняя сила при ударе
\begin{equation}
    f_{\text{ср}_x}=\frac{\Delta p_x}{\tau}=0
\end{equation}
(Из жизненного опыта --- при ударе палкой по неподвижному предмету можно ударить так, что удар не почувствуется рукой, а можно наоборот - её <<отбить>>)
\end{document}