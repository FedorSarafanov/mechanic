\documentclass[a5paper,10pt]{article}
 
% \usepackage{extsizes}
\usepackage{cmap}
\usepackage[T2A]{fontenc}
\usepackage[utf8x]{inputenc}
\usepackage[english, russian]{babel}

\usepackage{misccorr}

%%%%%%%%%%%%%%%%%%%%%%%%%%%%%%%%%%%%%%%%%%%%%%%%%%%%%%%%%%%%%%%%%%%%%%%%%%%%%%%%%%  
\usepackage{graphicx} % для вставки картинок
\graphicspath{{img/}}
\usepackage{amssymb,amsfonts,amsmath,amsthm} % математические дополнения от АМС

% \usepackage{fontspec}
% \usepackage{unicode-math}

\usepackage{indentfirst} % отделять первую строку раздела абзацным отступом тоже
\usepackage[usenames,dvipsnames]{color} % названия цветов
\usepackage{makecell}
\usepackage{multirow} % улучшенное форматирование таблиц
\usepackage{ulem} % подчеркивания
\linespread{1.3} % полуторный интервал
% \renewcommand{\rmdefault}{ftm} % Times New Roman (не работает)
\frenchspacing
\usepackage[portrait]{geometry}
\geometry{left=1cm,right=1cm,top=2cm,bottom=1cm,bindingoffset=0cm}
\usepackage{titlesec}
\usepackage{float}
% \definecolor{black}{rgb}{0,0,0}
% \usepackage[colorlinks, unicode, pagecolor=black]{hyperref}
% \usepackage[unicode]{hyperref} %ссылки
\usepackage{fancyhdr} %загрузим пакет
\pagestyle{fancy} %применим колонтитул
\fancyhead{} %очистим хидер на всякий случай
\fancyhead[LE,RO]{Сарафанов Ф.Г.} %номер страницы слева сверху на четных и справа на нечетных
\fancyhead[CO, CE]{Механика}
\fancyhead[LO,RE]{№1.108 -- Иродов} 
\fancyfoot{} %футер будет пустой
% \fancyfoot[CO,CE]{\thepage}
\renewcommand{\labelenumii}{\theenumii)}


\usepackage{tikz}
\usepackage{etoolbox}
\usetikzlibrary{scopes}
\usetikzlibrary{%
     decorations.pathreplacing,%
     decorations.pathmorphing,%
    patterns,%
    calc,%
    scopes,%
    arrows,%
    % arrows.spaced,%
}
\makeatletter
\newif\if@gather@prefix 
\preto\place@tag@gather{% 
  \if@gather@prefix\iftagsleft@ 
    \kern-\gdisplaywidth@ 
    \rlap{\gather@prefix}% 
    \kern\gdisplaywidth@ 
  \fi\fi 
} 
\appto\place@tag@gather{% 
  \if@gather@prefix\iftagsleft@\else 
    \kern-\displaywidth 
    \rlap{\gather@prefix}% 
    \kern\displaywidth 
  \fi\fi 
  \global\@gather@prefixfalse 
} 
\preto\place@tag{% 
  \if@gather@prefix\iftagsleft@ 
    \kern-\gdisplaywidth@ 
    \rlap{\gather@prefix}% 
    \kern\displaywidth@ 
  \fi\fi 
} 
\appto\place@tag{% 
  \if@gather@prefix\iftagsleft@\else 
    \kern-\displaywidth 
    \rlap{\gather@prefix}% 
    \kern\displaywidth 
  \fi\fi 
  \global\@gather@prefixfalse 
} 
\newcommand*{\beforetext}[1]{% 
  \ifmeasuring@\else
  \gdef\gather@prefix{#1}% 
  \global\@gather@prefixtrue 
  \fi
} 
\makeatother
% \begin{document}
\def\iangle{60} % Angle of the inclined plane

\def\down{-90}
\def\arcr{0.5cm} % Radius of the arc used to indicate angles
\def\FIN{\vec{F}_{in}^\text{пост}}
% \hspace{-2em}

\usepackage{tikz-3dplot}
\tikzset{
	MyPersp/.style={scale=1.8,x={(-0.8cm,-0.3cm)},y={(0.8cm,-0.3cm)},
    z={(0cm,1cm)}},
    coord/.style={scale=3,x={(1cm,0cm)}, y={(0cm,1cm)}, z={(-0cm,-0.4cm)}},
 % MyPersp/.style={scale=1.5,x={(0cm,0cm)},y={(1cm,0cm)},
   % z={(0cm,1cm)}}, % uncomment the two lines to get a lateral view
	MyPoints/.style={fill=white,draw=black,thick}
		}
\usetikzlibrary{3d}
 \tikzset{zxplane/.style={canvas is zx plane at y=#1}}
 \tikzset{yxplane/.style={canvas is yx plane at z=#1}}
 \tikzset{zyplane/.style={canvas is zy plane at x=#1}}

\def\centerarc[#1](#2)(#3:#4:#5)% Syntax: [draw options] (center) (initial angle:final angle:radius)
    { \draw[#1] ($(#2)+({#5*cos(#3)},{#5*sin(#3)})$) arc (#3:#4:#5); }
\begin{document}

\begin{tikzpicture}[coord,font=\large,
    axis/.style={densely dashed,black!60,font=\small},
    force/.style={>=latex,draw=blue,fill=blue},
    % m/.style={rectangle,draw,fill=lightgray,minimum size=0.5cm,thin},
    m/.style={draw=black!30, rectangle,draw,thin, fill=blue!2, minimum size=0.5cm},
    m/.style={draw=black!30, rectangle,draw,thin, fill=blue!2, minimum size=0.5cm},
    interface/.style={draw=gray!60,
        postaction={draw=gray!60,decorate,decoration={border,angle=-135,
                    amplitude=0.3cm,segment length=2mm}}},
    plane/.style={draw=black!30, very thick, fill=blue!5},
    string/.style={draw=black, thick},
    pulley/.style={thick},
]

    \begin{scope}[zxplane=0]
    \def\angl{30}
     \draw [fill=black!5] (0,0) circle (1cm);
     \draw[force, ->] (\angl:0.3) -- node[above]{$\vec{v'}$}(\angl:0.9);
     \draw[axis, ->] (0,0) -- (\angl:2) node[left]{$+x$};
	\draw[dashed, thick, fill=black] (0,0) -- node[above]{$\vec{r}$}(\angl:0.3) circle(0.6pt);
	% \draw[force, ->] (\angl:0.3) -- node[anchor=south west, yshift=0.5em]{$\vec{F_{kor}^{in}}$}++(90+\angl:0.9);

     % \draw (-1,0) -- (1,0) (0,-1) -- (0,1);
   \end{scope}
	\draw[force, ->] (0,0,0) -- (0,0,-3) node[right] {$\vec{\omega}$};
	
\end{tikzpicture}
\begin{tikzpicture}[font=\large,
    axis/.style={densely dashed,black!60,font=\small},
    force/.style={>=latex,draw=blue,fill=blue},
    % m/.style={rectangle,draw,fill=lightgray,minimum size=0.5cm,thin},
    m/.style={draw=black!30, rectangle,draw,thin, fill=blue!2, minimum size=0.5cm},
    m/.style={draw=black!30, rectangle,draw,thin, fill=blue!2, minimum size=0.5cm},
    interface/.style={draw=gray!60,
        postaction={draw=gray!60,decorate,decoration={border,angle=-135,
                    amplitude=0.3cm,segment length=2mm}}},
    plane/.style={draw=black!30, very thick, fill=blue!5},
    string/.style={draw=black, thick},
    pulley/.style={thick},
]
     \xdef\rp{3pt}
     {[force]
          \draw (0,0)  circle (\rp) node[anchor=south east, pos=1.3] {$\vec{F_{kor}^{in}}$};
        \draw [thick] (0,0) +(-135:\rp)--+(45:\rp);
        \draw [thick] (0,0) +(135:\rp)--+(-45:\rp);
     }

	\draw[white] (-3,0) --(2,0);
	\draw[white] (0,0) --(0,-2.5);
	\draw[interface] (-1,0) --(1,0);
	\draw[force, ->] (0,0) --(0,1) node[above] {$\vec{N}$};
	\draw[force, ->] (0,0) --(-2,0) node[above] {$\vec{f_R}$};
	% \draw[force, ->] (0,0) --(-2,0) node[above] {$\vec{f_R}$};
	\draw[force, ->] (0,0) --(2,0) node[above] {$\vec{F_{post}^{in}}$};
     \draw[axis, ->] (0,0) -- (3,0) node[above]{$+x$};

	\draw[force, ->] (0,0) --(0,-1) node[below] {$m\vec{g}$};
\end{tikzpicture}

Рассмотрим НИСО диска.
\begin{gather}
    \beforetext{Запишем ан. II з. Н.:} m\vec{a}=\vec{f_R}+\vec{N}+\vec{F_{post}^{in}}+\vec{F_{kor}^{in}}+m\vec{g}\\%
    \beforetext{Пусть} \vec{Q}=\vec{N}+\vec{F_{post}^{in}}+\vec{F_{kor}^{in}}\\
    \beforetext{Тогда} Q=\sqrt{(mg)^2+(m\omega^2{R})^2+(2mv'\omega)^2}=\\
    =m\sqrt{g^2+\omega^4{R}^2+4v'^2\omega^2}
\end{gather}

Но $\vec{Q}$ -- это сумма всех сил, действующих на тело. Тогда на диск со стороны тела действует $\vec{F}=-\vec{Q}$, $F=Q$.

Тогда

\begin{equation}
	F=m\sqrt{g^2+\omega^4{R}^2+4v'^2\omega^2}\approx8\text{Н}
\end{equation}
\end{document}