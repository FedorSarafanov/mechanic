\documentclass[a5paper,10pt]{article}
 
\usepackage{extsizes}
\usepackage{cmap}
\usepackage[T2A]{fontenc}
\usepackage[utf8x]{inputenc}
\usepackage[english, russian]{babel}

\usepackage{misccorr}

%%%%%%%%%%%%%%%%%%%%%%%%%%%%%%%%%%%%%%%%%%%%%%%%%%%%%%%%%%%%%%%%%%%%%%%%%%%%%%%%%%  
\usepackage{graphicx} % для вставки картинок
\graphicspath{{img/}}
\usepackage{amssymb,amsfonts,amsmath,amsthm} % математические дополнения от АМС

% \usepackage{fontspec}
% \usepackage{unicode-math}

\usepackage{indentfirst} % отделять первую строку раздела абзацным отступом тоже
\usepackage[usenames,dvipsnames]{color} % названия цветов
\usepackage{makecell}
\usepackage{multirow} % улучшенное форматирование таблиц
\usepackage{ulem} % подчеркивания
\linespread{1.3} % полуторный интервал
% \renewcommand{\rmdefault}{ftm} % Times New Roman (не работает)
\frenchspacing
\usepackage{geometry}
\geometry{left=1cm,right=1cm,top=2cm,bottom=1cm,bindingoffset=0cm}
\usepackage{titlesec}
\usepackage{float}
% \definecolor{black}{rgb}{0,0,0}
% \usepackage[colorlinks, unicode, pagecolor=black]{hyperref}
% \usepackage[unicode]{hyperref} %ссылки
\usepackage{fancyhdr} %загрузим пакет
\pagestyle{fancy} %применим колонтитул
\fancyhead{} %очистим хидер на всякий случай
\fancyhead[LE,RO]{Сарафанов Ф.Г.} %номер страницы слева сверху на четных и справа на нечетных
\fancyhead[CO, CE]{Механика}
\fancyhead[LO,RE]{Яковлев -- №578} 
\fancyfoot{} %футер будет пустой
% \fancyfoot[CO,CE]{\thepage}
\renewcommand{\labelenumii}{\theenumii)}


\usepackage{tikz}
\usetikzlibrary{scopes}
\usetikzlibrary{%
     decorations.pathreplacing,%
     decorations.pathmorphing,%
    patterns,%
    calc,%
    scopes,%
    arrows,%
    % arrows.spaced,%
}

\begin{document}

\begin{figure}[H]
    \centering
\begin{tikzpicture}[
    force/.style={>=latex,draw=blue,fill=blue},
    % axis/.style={densely dashed,gray,font=\small},
    axis/.style={densely dashed,black!60,font=\small},
    M/.style={rectangle,draw,fill=lightgray,minimum size=0.5cm,thin},
    m2/.style={draw=black!30, rectangle,draw,thin, fill=blue!2, minimum width=0.7cm,minimum height=0.7cm},
    m1/.style={draw=black!30, rectangle,draw,thin, fill=blue!2, minimum width=0.7cm,minimum height=0.7cm},
    plane/.style={draw=black!30, very thick, fill=blue!5, line width=1pt},
    % base/.style={draw=black!70, very thick, fill=blue!4, line width=2pt},
    string/.style={draw=black, thick},
    pulley/.style={thick},
    % interface/.style={draw=gray!60,
    %     % The border decoration is a path replacing decorator. 
    %     % For the interface style we want to draw the original path.
    %     % The postaction option is therefore used to ensure that the
    %     % border decoration is drawn *after* the original path.
    %     postaction={draw=gray!60,decorate,decoration={border,angle=-135,
    %                 amplitude=0.3cm,segment length=2mm}}},
    interface/.style={
        pattern = north east lines,
        draw    = none,
        pattern color=gray!60,          
    },
    plank/.style={
        fill=black!60, 
        draw=black,
        minimum width=3cm,
        inner ysep=0.1cm,
        outer sep=0pt,
        yshift=0.75cm,
        pattern = north east lines,
        pattern color=gray!60, 
    },
    cargo/.style={
        rectangle,
        fill=black!70,              
        inner sep=2.5mm,
    }
]

    \node[cargo] (b) at (0,0) {} node[right of=b, xshift=-0.5cm] {$m$};
    \node[plank, below of=b, anchor=north] (plank) {};

    \draw[axis, ->] (-2,2) -- ++(0,-4) node[below] {$x$};
    \draw[black!60] (-2.2,-0.35) node[left, black] {$0$} -- ++(0.4,0) ;

    \node[below] (c) at (plank.south) {};
    \draw[force,->] (b.center) -- ++(0,-1) node[left] {$m\vec{g}$};
    \draw[force,->] (b.south) ++(0.05,0) -- ++(0,1) node[above] {$\vec{N}$};

    \draw[force,->] (b.south) ++(0.05,0) -- ++(0,-1) node[right] {$\vec{P}$};

\end{tikzpicture}
% \vspace{-2em}
\end{figure}

Груз еще будет покоится на доске, когда его вес станет равен нулю, но начнет подскакивать, когда после этого амплитуда станет больше на сколь угодно малую величину.

Доска совершает колебания по закону
\begin{equation*}
     a=\frac{d^2x}{dt^2}=-\omega^2x=-\frac{4\pi^2}{T^2}x
 \end{equation*} 

Запишем второй закон Ньютона для груза:
\begin{equation*}
	\begin{aligned}[c]
		m\vec{a}&=m\vec{g}+\vec{N}\\
        \text{x: }ma_x&=mg-N=mg-P\\
	\end{aligned}
		\qquad\Longrightarrow\qquad
	\begin{aligned}[c]
		P&=m(g-a_x)\\
        P&=m(g+\frac{4\pi^2}{T^2}x)
	\end{aligned}
\end{equation*}
Тогда $P=0$ тогда, когда 
\begin{gather*}
    {}m(g+\frac{4\pi^2}{T^2}x)\leq{0}\\
    g\leq-{\frac{4\pi^2}{T^2}x}
\end{gather*}
Так как $A$ здесь и есть максимальное $|x|$, то 
\begin{gather*}
    A>{g\frac{T^2}{4\pi^2}}\approx6.2\text{ см}
\end{gather*}
При 
\begin{gather*}
    A={g\frac{T^2}{4\pi^2}}
\end{gather*}
будет предельное значение амплитуды, при увеличении которой начнется подскакивание.
% Но движение этой же системы до отрыва можно описать вторым законом Нюьтона:
% \begin{equation*}
% 	\begin{aligned}[c]
% 		&m\vec{a_0}=\vec{f_e}+m\vec{g}\\
% 		\text{x: }&ma_0=mg-k\Delta{}\\
% 	\end{aligned}
% 		\qquad\Longrightarrow\qquad
% 	\begin{aligned}[c]
% 		\Delta{}=\frac{m}{k}(g-a_0)\\
% 		% y&=v/w\\
% 	\end{aligned}
% \end{equation*}
% Тогда по теореме о изменении кинетической энергии можем записать работу сил упругости:
% \begin{equation*}
%     \mvn-\mv=\int_{\Delta{}}^{\delta{}} [mg-kx]\cdot{}dx
% \end{equation*}
% \begin{equation*}
%     -\mv=mg(\delta{}-\Delta{})+\frac{k\Delta{}^2}{2}-\frac{k\delta{}^2}{2}\\
% \end{equation*}
% \begin{equation*}
%     -\mv=-ma_0\Delta{}=mg(\delta{}-\Delta{})+\frac{k\Delta{}^2}{2}-\frac{k\delta{}^2}{2}
% \end{equation*}
% \begin{equation*}
% 	\frac{k}{2}\delta^2-mg\delta+\Delta(m(g-a_0)-\frac{k}{2}\Delta)=0
% \end{equation*}
% И найти удлинение пружины $\delta$:
% \begin{equation*}
% 	\frac{k}{2}\delta^2-mg\delta+\frac{m^2}{2k}(g-a_0)^2=0
% \end{equation*}
% \begin{equation*}
% 	\begin{aligned}[c]
% 		&a=\frac{k}{2}\\
% 		&b=-mg\\
% 		&c=\frac{m^2}{2k}(g-a_0)^2
% 	\end{aligned}
% 		\qquad\Longrightarrow\qquad
% 	\begin{aligned}[c]
% 		D&=b^2-4ac\\
% 		\sqrt{D}&=m\sqrt{2ga_0-a_o^2}
% 		% y&=v/w\\
% 	\end{aligned}
% \end{equation*}
% \begin{equation*}
% 	\begin{aligned}[c]
% 		\delta_1=\frac{mg+{}m\sqrt{2ga_0-a_o^2}}{k}
% 	\end{aligned}
% 		\qquad\qquad
% 	\begin{aligned}[c]
% 		\delta_2=\frac{mg-{}m\sqrt{2ga_0-a_o^2}}{k}
% 	\end{aligned}
% \end{equation*}

% Так как требуется найти $\delta_{max}$, подходит корень $\delta_1$, $\delta_1>\delta_2$.

\end{document}