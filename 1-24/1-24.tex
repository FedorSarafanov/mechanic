\documentclass[a5paper,10pt]{article}
 
\usepackage{extsizes}
\usepackage{cmap}
\usepackage[T2A]{fontenc}
\usepackage[utf8x]{inputenc}
\usepackage[english, russian]{babel}

\usepackage{misccorr}

%%%%%%%%%%%%%%%%%%%%%%%%%%%%%%%%%%%%%%%%%%%%%%%%%%%%%%%%%%%%%%%%%%%%%%%%%%%%%%%%%%  
\usepackage{graphicx} % для вставки картинок
\graphicspath{{img/}}
\usepackage{amssymb,amsfonts,amsmath,amsthm} % математические дополнения от АМС

% \usepackage{fontspec}
% \usepackage{unicode-math}

\usepackage{indentfirst} % отделять первую строку раздела абзацным отступом тоже
\usepackage[usenames,dvipsnames]{color} % названия цветов
\usepackage{makecell}
\usepackage{multirow} % улучшенное форматирование таблиц
\usepackage{ulem} % подчеркивания
\linespread{1.3} % полуторный интервал
% \renewcommand{\rmdefault}{ftm} % Times New Roman (не работает)
\frenchspacing
\usepackage{geometry}
\geometry{left=1cm,right=1cm,top=2cm,bottom=1cm,bindingoffset=0cm}
\usepackage{titlesec}
\usepackage{float}
% \definecolor{black}{rgb}{0,0,0}
% \usepackage[colorlinks, unicode, pagecolor=black]{hyperref}
% \usepackage[unicode]{hyperref} %ссылки
\usepackage{fancyhdr} %загрузим пакет
\pagestyle{fancy} %применим колонтитул
\fancyhead{} %очистим хидер на всякий случай
\fancyhead[LE,RO]{Сарафанов Ф.Г.} %номер страницы слева сверху на четных и справа на нечетных
\fancyhead[CO, CE]{Механика}
\fancyhead[LO,RE]{Иродов 1.24} 
\fancyfoot{} %футер будет пустой
% \fancyfoot[CO,CE]{\thepage}
\renewcommand{\labelenumii}{\theenumii)}


\usepackage{tikz}
\usetikzlibrary{scopes}
\usetikzlibrary{%
     decorations.pathreplacing,%
     decorations.pathmorphing,%
    patterns,%
    calc,%
    scopes,%
    arrows,%
    through,%
    % arrows.spaced,%
}
\newcommand{\vangle}{\mathop{\mathstrut^\wedge}\nolimits}
\usepackage{wrapfig}
\begin{document}

\begin{wrapfigure}{r}{0.5\textwidth}
    \centering
\begin{tikzpicture}[
    force/.style={>=latex,draw=blue,fill=blue},
    % axis/.style={densely dashed,gray,font=\small},
    axis/.style={densely dashed,black!60,font=\small},
    M/.style={rectangle,draw,fill=lightgray,minimum size=0.5cm,thin},
    m2/.style={draw=black!30, rectangle,draw,thin, fill=blue!2, minimum width=0.7cm,minimum height=0.7cm},
    m1/.style={draw=black!30, rectangle,draw,thin, fill=blue!2, minimum width=0.7cm,minimum height=0.7cm},
    plane/.style={draw=black!30, very thick, fill=blue!5, line width=1pt},
    % base/.style={draw=black!70, very thick, fill=blue!4, line width=2pt},
    string/.style={draw=black, thick},
    pulley/.style={thick},
    interface1/.style={draw=gray!60,
        % The border decoration is a path replacing decorator. 
        % For the interface style we want to draw the original path.
        % The postaction option is therefore used to ensure that the
        % border decoration is drawn *after* the original path.
        postaction={draw=gray!60,decorate,decoration={border,angle=-135,
                    amplitude=0.3cm,segment length=2mm}}},
    interface/.style={
        pattern = north east lines,
        draw    = none,
        pattern color=gray!60,          
    },
    plank/.style={
        fill=black!60, 
        draw=black,
        minimum width=3cm,
        inner ysep=0.1cm,
        outer sep=0pt,
        yshift=0.75cm,
        pattern = north east lines,
        pattern color=gray!60, 
    },
    cargo/.style={
        rectangle,
        fill=black!70,              
        inner sep=2.5mm,
    }
]
	\draw [axis, ->] (0,-3) -- (0,3) node[left] {$y$};

	\draw [axis, ->] (-2,0) -- (2,0) node[anchor=north west] {$x$};

	\draw[domain=-1.4:1.4, samples=200,smooth,variable=\x,black] plot ({\x},{\x*\x}) node[anchor=north west] {$\beta>0$};
	\draw[dashed, thick, domain=-1.4:1.4, samples=200,smooth,variable=\x,black] plot ({\x},{-\x*\x}) node[anchor=north west] {$\beta<0$};

\end{tikzpicture}
\caption{График траектории}
\end{wrapfigure}

\begin{gather*}
    \vec{r}=\alpha{t}\vec{i}+\beta{t^2}\vec{j}\\
    x=\alpha{t}\\
    y=\beta{t^2}\\
    \vec{v}=\frac{d\vec{r}}{dt}=\alpha\vec{i}+2\beta{t}\vec{j}\\
    v=\sqrt{\alpha^2+4\beta^2t^2}\\
    \vec{a}=\frac{d\vec{v}}{dt}=0\vec{i}+2\beta{}\vec{j}=2\beta\vec{j}\\
    a=\sqrt{0^2+4\beta^2}=2\beta\\
    \phi=\vec{v}\vangle\vec{a}\\
    % v(t)=\sqrt{x'^2(t)+y'^2(t)}=\sqrt{\alpha^2+4\alpha^2t^2}\\
\end{gather*}


\begin{wrapfigure}{l}{0.37\textwidth}
    \centering
\begin{tikzpicture}[
    force/.style={>=latex,draw=blue,fill=blue},
    % axis/.style={densely dashed,gray,font=\small},
    axis/.style={densely dashed,black!60,font=\small},
    M/.style={rectangle,draw,fill=lightgray,minimum size=0.5cm,thin},
    m2/.style={draw=black!30, rectangle,draw,thin, fill=blue!2, minimum width=0.7cm,minimum height=0.7cm},
    m1/.style={draw=black!30, rectangle,draw,thin, fill=blue!2, minimum width=0.7cm,minimum height=0.7cm},
    plane/.style={draw=black!30, very thick, fill=blue!5, line width=1pt},
    % base/.style={draw=black!70, very thick, fill=blue!4, line width=2pt},
    string/.style={draw=black, thick},
    pulley/.style={thick},
    interface1/.style={draw=gray!60,
        % The border decoration is a path replacing decorator. 
        % For the interface style we want to draw the original path.
        % The postaction option is therefore used to ensure that the
        % border decoration is drawn *after* the original path.
        postaction={draw=gray!60,decorate,decoration={border,angle=-135,
                    amplitude=0.3cm,segment length=2mm}}},
    interface/.style={
        pattern = north east lines,
        draw    = none,
        pattern color=gray!60,          
    },
    plank/.style={
        fill=black!60, 
        draw=black,
        minimum width=3cm,
        inner ysep=0.1cm,
        outer sep=0pt,
        yshift=0.75cm,
        pattern = north east lines,
        pattern color=gray!60, 
    },
    cargo/.style={
        rectangle,
        fill=black!70,              
        inner sep=2.5mm,
    }
]
	\draw [axis, ->] (0,0) -- (0,3) node[left] {$y$};
	\draw [force, ->] (0,0) -- (0,2) node[left] {$\vec{v}_y=2\beta{t}\vec{j}$};

	\draw [axis, ->] (0,0) -- (3,0) node[above] {$x$};
	\draw [force, ->] (0,0) -- (2,0) node[below] {$\vec{v}_x=\alpha\vec{i}$};

	\draw [force, ->] (0,0) -- (2,2) node[anchor=west] {$\vec{v}$};

	\draw [axis] (2,0) --(2,2);
	\draw [axis] (0,2) --(2,2);

	\draw[->] (0,0) ++(45:1) arc (45:90:1);
	\draw (0,0) ++(75:1.18) node[anchor=west] {$\phi$};

\end{tikzpicture}
\end{wrapfigure}

\begin{gather*}
	t=\frac{x}{\alpha}\\
    y=\beta\cdot{\frac{x^2}{\alpha^2}}=\frac{\beta}{\alpha^2}x^2\text{ -- парабола}\\
\end{gather*}

Так как ускорение направлено по оси $y$ ($\vec{a}=2\beta\vec{j}$), то $\vec{v}\vangle\vec{a}=\vec{v}\vangle\vec{v_y}$.
\begin{gather*}
	\tg\phi=\frac{v_x}{v_y}=\frac{\alpha}{2\beta{t}}
\end{gather*}

% \vspace{-2em}
\end{document}