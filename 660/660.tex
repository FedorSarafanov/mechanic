\documentclass[a5paper,10pt]{article}
\def\source{/home/lab/tex/templates}

\usepackage{cmap}
\usepackage[T2A]{fontenc}
\usepackage[utf8x]{inputenc}
\usepackage[english, russian]{babel}

\usepackage
	{
		amssymb,
		% misccorr,
		amsfonts,
		amsmath,
		amsthm,
		wrapfig,
		makecell,
		multirow,
		indentfirst,
		ulem,
		graphicx,
		geometry,
		fancyhdr,
		subcaption,
		float,
		tikz,
		csvsimple,
		color,
	}  

\usepackage[outline]{contour}
\usepackage[mode=buildnew]{standalone}


\geometry
	{
		left=1cm,
		right=1cm,
		top=2cm,
		bottom=1cm,
		bindingoffset=0cm,
	}

\linespread{1.3} 
\frenchspacing 


\usetikzlibrary{scopes}
\usetikzlibrary
	{
		decorations.pathreplacing,
		decorations.pathmorphing,
		patterns,
		calc,
		scopes,
		arrows,
		through,
		shapes.misc,
		arrows.meta,
	}


\tikzset{
	force/.style=	{
		>=latex,
		draw=blue,
		fill=blue,
				 	}, 
	%				 	
	axis/.style=	{
		densely dashed,
		gray,
		font=\small,
					},
	%
	acceleration/.style={
		>=open triangle 60,
		draw=blue,
		fill=blue,
					},
	%
	inforce/.style=	{
		force,
		double equal sign distance=2pt,
					},
	%
	interface/.style={
		pattern = north east lines, 
		draw    = none, 
		pattern color=gray!60,
					},
	cross/.style=	{
		cross out, 
		draw=black, 
		minimum size=2*(#1-\pgflinewidth), 
		inner sep=0pt, outer sep=0pt,
					},
	%
	cargo/.style=	{
		rectangle, 
		fill=black!70, 
		inner sep=2.5mm,
					},
	%
	}

\pagestyle{fancy} %применим колонтитул
\fancyhead{} %очистим хидер на всякий случай
\fancyhead[R]{Сарафанов Ф.Г.} %номер страницы слева сверху на четных и справа на нечетных
\fancyhead[C]{Механика}
% \fancyhead[L]{Задача под запись - <<АУУ-2>>} 
\fancyfoot{} %футер будет пустой

\newcommand{\irodov}[1]{\fancyhead[L]{Иродов -- №#1}}
\newcommand{\yakovlev}[1]{\fancyhead[L]{Яковлев -- №#1}}
\newcommand{\wrote}[1]{\fancyhead[L]{Под запись -- <<#1>>}}

\newenvironment{tikzpict}
    {
	    \begin{figure}[htbp]
		\centering
		\begin{tikzpicture}
    }
    { 
		\end{tikzpicture}
		% \caption{caption}
		% \label{fig:label}
		\end{figure}
    }

\newcommand{\vbLabel}[3]{\draw ($(#1,#2)+(0,5pt)$) -- ($(#1,#2)-(0,5pt)$) node[below]{#3}}
\newcommand{\vaLabel}[3]{\draw ($(#1,#2)+(0,5pt)$) node[above]{#3} -- ($(#1,#2)-(0,5pt)$) }

\newcommand{\hrLabel}[3]{\draw ($(#1,#2)+(5pt,0)$) -- ($(#1,#2)-(5pt,0)$) node[right, xshift=1em]{#3}}
\newcommand{\hlLabel}[3]{\draw ($(#1,#2)+(5pt,0)$) node[left, xshift=-1em]{#3} -- ($(#1,#2)-(5pt,0)$) }

% Draw line annotation
% Input:
%   #1 Line offset (optional)
%   #2 Line angle
%   #3 Line length
%   #5 Line label
% Example:
%   \lineann[1]{30}{2}{$L_1$}
\newcommand{\lineann}[4][0.5]{%
    \begin{scope}[rotate=#2, blue,inner sep=2pt, ]
        \draw[dashed, blue!40] (0,0) -- +(0,#1)
            node [coordinate, near end] (a) {};
        \draw[dashed, blue!40] (#3,0) -- +(0,#1)
            node [coordinate, near end] (b) {};
        \draw[|<->|] (a) -- node[fill=white, scale=0.8] {#4} (b);
    \end{scope}
}


\yakovlev{660}

\begin{document}

\begin{tikzpict}

	\fill[magenta!10, draw=none] (0,0) rectangle ++(2,3);
	\fill[magenta!10, draw=none] (0.8,0) rectangle ++(0.4,-0.5);
	\draw (0,3.2)--(0,0)--(2,0)--(2,3.2);

	\draw[magenta!10] (0.8,0) -- ++(0.4,0);

	\draw[magenta] (1,0.3) circle (2pt) node[above, yshift=0.5em] {$A$};
	\draw[magenta] (1,-0.3) circle (2pt) node[below, yshift=-0.5em] {$B$};

	\draw[magenta,->] (1.4,0.3) -- ++(0,-0.5) node[below] {$\vec{v}$};
	\draw[magenta,->] (0.6,-0.3) -- ++(0,-1) node[below] {$\vec{v}_2$};
% Draw line annotation
% Input:
%   #1 Line offset (optional)
%   #2 Line angle
%   #3 Line length
%   #5 Line label
% Example:
	\lineann[0.5]{90}{3}{$h(t)$}


	\draw[axis,->] (-1,-0.5) -- ++(0,4) node[above] {$+h$};
\end{tikzpict}
Запишем уравнение Бернулли для для линии тока в точках $A$ (нижняя точка сосуда у отверстия) и $B$ (под отверстием):
\begin{equation}
	\label{eq:brn}
	p_a+\frac{\rho v^2}{2}+\rho gh=p_a+\frac{\rho v_2^2}{2}
\end{equation}
Скорость $v_2$ легко найдется из равенства расходов жидкости в сечении $S$ и в сечении $\sigma$:
\begin{equation}
	Sv=\sigma v_2 
		\quad\Rightarrow\quad
	v_2=v\frac{S}{\sigma}
\end{equation}
Тогда можно формулу (\ref{eq:brn}) переписать в следующем виде:
\begin{equation}
	2gh=v^2\left[\frac{S^2}{\sigma^2}-1\right]
\end{equation}
С другой стороны, $v_h=-\frac{dh}{dt}$. Тогда 
\begin{equation}
	-\frac{dh}{dt}=\sqrt{\frac{2gh}{\frac{S^2}{\sigma^2}-1}}
\end{equation}
Проинтегрируем, расставив пределы:
\begin{equation}
	-\int\limits_H^h(t)\frac{dh}{\sqrt{h}}=\int\limits_0^t\sqrt{\frac{2g}{\frac{S^2}{\sigma^2}-1}}
\end{equation}
\begin{equation}
	2\left(\sqrt{H}-\sqrt{h}\right)=t\sqrt{\frac{2g}{\frac{S^2}{\sigma^2}-1}}
\end{equation}
\begin{equation}
	t=2\left(\sqrt{H}-\sqrt{h}\right)\sqrt{\frac{\frac{S^2}{\sigma^2}-1}{2g}}
\end{equation}
Вынесем $\frac{S^2}{\sigma^2}$ за знак корня:
\begin{equation}
	t=2\left(\sqrt{H}-\sqrt{h}\right)\frac{S}{\sigma}\sqrt{\frac{1-\frac{\sigma^2}{S^2}}{2g}}
\end{equation}
Ввиду малости $\sigma$ второго порядка положим $\frac{\sigma^2}{S^2}\approx0$, тогда
\begin{equation}
	t=\left(\sqrt{H}-\sqrt{h}\right)\frac{S}{\sigma}\sqrt{\frac{2}{g}}
\end{equation}
Полное время вытекания будет при $h=0$:
\begin{equation}
	T=t(h=0)=\frac{S}{\sigma}\sqrt{\frac{2}{g}}
\end{equation}
\end{document}