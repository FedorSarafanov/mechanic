\documentclass[a5paper,10pt]{article}
 
\usepackage{extsizes}
\usepackage{cmap}
\usepackage[T2A]{fontenc}
\usepackage[utf8x]{inputenc}
\usepackage[english, russian]{babel}

\usepackage{misccorr}

%%%%%%%%%%%%%%%%%%%%%%%%%%%%%%%%%%%%%%%%%%%%%%%%%%%%%%%%%%%%%%%%%%%%%%%%%%%%%%%%%%  
\usepackage{graphicx} % для вставки картинок
\graphicspath{{img/}}
\usepackage{amssymb,amsfonts,amsmath,amsthm} % математические дополнения от АМС

% \usepackage{fontspec}
% \usepackage{unicode-math}

\usepackage{indentfirst} % отделять первую строку раздела абзацным отступом тоже
\usepackage[usenames,dvipsnames]{color} % названия цветов
\usepackage{makecell}
\usepackage{multirow} % улучшенное форматирование таблиц
\usepackage{ulem} % подчеркивания
\linespread{1.3} % полуторный интервал
% \renewcommand{\rmdefault}{ftm} % Times New Roman (не работает)
\frenchspacing
\usepackage{geometry}
\geometry{left=1cm,right=1cm,top=2cm,bottom=1cm,bindingoffset=0cm}
\usepackage{titlesec}
\usepackage{float}
% \definecolor{black}{rgb}{0,0,0}
% \usepackage[colorlinks, unicode, pagecolor=black]{hyperref}
% \usepackage[unicode]{hyperref} %ссылки
\usepackage{fancyhdr} %загрузим пакет
\pagestyle{fancy} %применим колонтитул
\fancyhead{} %очистим хидер на всякий случай
\fancyhead[LE,RO]{Сарафанов Ф.Г.} %номер страницы слева сверху на четных и справа на нечетных
\fancyhead[CO, CE]{Механика}
\fancyhead[LO,RE]{Яковлев 41} 
\fancyfoot{} %футер будет пустой
% \fancyfoot[CO,CE]{\thepage}
\renewcommand{\labelenumii}{\theenumii)}


\usepackage{tikz}
\usetikzlibrary{scopes}
\usetikzlibrary{%
     decorations.pathreplacing,%
     decorations.pathmorphing,%
    patterns,%
    calc,%
    scopes,%
    arrows,%
    % arrows.spaced,%
}

\begin{document}

\begin{figure}[H]
    \centering
\begin{tikzpicture}[
    force/.style={>=latex,draw=blue,fill=blue},
    % axis/.style={densely dashed,gray,font=\small},
    axis/.style={densely dashed,black!60,font=\small},
    M/.style={rectangle,draw,fill=lightgray,minimum size=0.5cm,thin},
    m2/.style={draw=black!30, rectangle,draw,thin, fill=blue!2, minimum width=0.7cm,minimum height=0.7cm},
    m1/.style={draw=black!30, rectangle,draw,thin, fill=blue!2, minimum width=0.7cm,minimum height=0.7cm},
    plane/.style={draw=black!30, very thick, fill=blue!5, line width=1pt},
    % base/.style={draw=black!70, very thick, fill=blue!4, line width=2pt},
    string/.style={draw=black, thick},
    pulley/.style={thick},
    % interface/.style={draw=gray!60,
    %     % The border decoration is a path replacing decorator. 
    %     % For the interface style we want to draw the original path.
    %     % The postaction option is therefore used to ensure that the
    %     % border decoration is drawn *after* the original path.
    %     postaction={draw=gray!60,decorate,decoration={border,angle=-135,
    %                 amplitude=0.3cm,segment length=2mm}}},
    interface/.style={
        pattern = north east lines,
        draw    = none,
        pattern color=gray!60,          
    },
    plank/.style={
        fill=black!60, 
        draw=black,
        minimum width=3cm,
        inner ysep=0.1cm,
        outer sep=0pt,
        yshift=0.75cm,
        pattern = north east lines,
        pattern color=gray!60, 
    },
    cargo/.style={
        rectangle,
        fill=black!70,              
        inner sep=2.5mm,
    }
]
    % \draw[force,double equal sign distance=2pt,->] (c) -- ++(0,-2) node[below] {$\vec{a}_0$};
\matrix[column sep=0cm, row sep=0cm] {
%%%%%%%%%%%%%%%%%%%%%%%%%%%%%%%%%%%%%%
	\coordinate (step) at (0,40mm);
	\path (-4,0) coordinate (A) -- ++(step) coordinate (A') -- (4,0) coordinate (B) -- ++(step) coordinate (B');

	\draw[thick] (A) -- (B);
	\draw[thick] (A') -- (B');

	\foreach \i in {2,4,...,20}{
		\pgfmathsetmacro{\I}{42 + \i*\i/30}%
        \draw[force, ->] (40 mm,\i mm) -- (\I mm,\i mm);
    }
	\foreach \i in {2,4,...,20}{
		\pgfmathsetmacro{\I}{42 + \i*\i/30}%
        \draw[force, ->] (40 mm, 40 mm - \i mm) -- (\I mm, 40 mm -\i mm);
    }

	\foreach \i in {2,4,...,40}{
        \draw[axis, blue, opacity=0.2] (-40 mm, \i mm) -- (40 mm, \i mm);
    }

    \node [] at (60mm,20mm) {$\vec{u}$};

   
    \draw [axis, black, <->] (-45mm,0) -- node[left] {$d$} (-45mm, 40mm);

    \draw [axis, black, <->] (0mm,45mm) -- node[above] {$h$} (40mm, 45mm);

    \draw [axis, ->] (0mm,0mm) -- (0mm, 50mm) node[right] {$\perp$};
    \draw [axis, ->] (0,0) -- (55mm, 0mm) node[right] {$+\parallel$};



    \draw[fill] (0,0) circle (2pt) node[below, yshift=-1em] {лодка};
    \draw[force,->] (0,0) -- ++ (0,1) node[above] {$\vec{v}$};

	\draw[ very thick,domain=0:9,samples=200,smooth,variable=\x,red] plot ({\x/4},{\x^(1/3)});  
	\draw[ very thick,rotate=180, samples=200, xshift=-4cm, yshift=-4cm, domain=0:9,smooth,variable=\x,red] plot ({\x/4},{\x^(1/3)});    
\\
};
\end{tikzpicture}
\vspace{-2em}
\end{figure}

Так как траектория симметрична, то чтобы найти весь снос $h$, достаточно найти снос до середины реки $h/2$.
До середины реки:
\begin{equation}
	v_\parallel(y)=u(y)=ky^2
\end{equation}
Но
\begin{equation}
	y=v_\perp{t}\Longrightarrow{}v_\parallel(t)=kv^2_\perp{t^2}
\end{equation}
А время движения до середины реки
\begin{equation}
	t^*=\frac{d}{2v_\perp}
\end{equation}
Тогда
\begin{equation}
	\frac{h}{2}=\int_0^{t^*} v_\parallel(t)dt=kv^2_\perp{\frac{t^3}{3}}\bigg|_0^{t^*}=\frac{kd^3}{24v_\perp}
\end{equation}
И
\begin{equation}
	h=\frac{kd^3}{12v_\perp}
\end{equation}
Траекторию легко найти
\begin{gather}
	x(t)=\int_0^t v_\parallel(t)dt=kv^2_\perp{\frac{t^3}{3}} \Longrightarrow t=\sqrt[3]{\frac{3x}{2kv_\perp^2}}
\end{gather}
Тогда
\begin{equation}
	y(x)=v_\perp\cdot{t}=\sqrt[3]{\frac{3v_\perp{}x}{2k}}
\end{equation}
До середины реки, далее - симметрично.
% До отрыва груза от доски система доска-груз движется так, будто дело происходит в потенциальном поле единственной силы $\vec{F}=m\vec{a_0}$.

% Тогда по теореме о изменении кинетической энергии можем найти скорость в момент отрыва $v$:
% \begin{equation*}
% \xdef\mv{\frac{mv^2}{2}}%
% \xdef\mvn{\frac{0^{2}}{2}}%
% 	\begin{aligned}[c]
% 		\mv-\mvn&=A_\text{мех}\\
% 		A_\text{мех}&=\int_0^\Delta{ma_0}\cdot{dx}\\
% 	\end{aligned}
% 		\qquad\Longrightarrow\qquad
% 	\begin{aligned}[c]
% 		\mv&=ma_0\Delta\\
% 		% y&=v/w\\
% 	\end{aligned}
% \end{equation*}
% Но движение этой же системы до отрыва можно описать вторым законом Нюьтона:
% \begin{equation*}
% 	\begin{aligned}[c]
% 		&m\vec{a_0}=\vec{f_e}+m\vec{g}\\
% 		\text{x: }&ma_0=mg-k\Delta{}\\
% 	\end{aligned}
% 		\qquad\Longrightarrow\qquad
% 	\begin{aligned}[c]
% 		\Delta{}=\frac{m}{k}(g-a_0)\\
% 		% y&=v/w\\
% 	\end{aligned}
% \end{equation*}
% Тогда по теореме о изменении кинетической энергии можем записать работу сил упругости:
% \begin{equation*}
%     \mvn-\mv=\int_{\Delta{}}^{\delta{}} [mg-kx]\cdot{}dx
% \end{equation*}
% \begin{equation*}
%     -\mv=mg(\delta{}-\Delta{})+\frac{k\Delta{}^2}{2}-\frac{k\delta{}^2}{2}\\
% \end{equation*}
% \begin{equation*}
%     -\mv=-ma_0\Delta{}=mg(\delta{}-\Delta{})+\frac{k\Delta{}^2}{2}-\frac{k\delta{}^2}{2}
% \end{equation*}
% \begin{equation*}
% 	\frac{k}{2}\delta^2-mg\delta+\Delta(m(g-a_0)-\frac{k}{2}\Delta)=0
% \end{equation*}
% И найти удлинение пружины $\delta$:
% \begin{equation*}
% 	\frac{k}{2}\delta^2-mg\delta+\frac{m^2}{2k}(g-a_0)^2=0
% \end{equation*}
% \begin{equation*}
% 	\begin{aligned}[c]
% 		&a=\frac{k}{2}\\
% 		&b=-mg\\
% 		&c=\frac{m^2}{2k}(g-a_0)^2
% 	\end{aligned}
% 		\qquad\Longrightarrow\qquad
% 	\begin{aligned}[c]
% 		D&=b^2-4ac\\
% 		\sqrt{D}&=m\sqrt{2ga_0-a_o^2}
% 		% y&=v/w\\
% 	\end{aligned}
% \end{equation*}
% \begin{equation*}
% 	\begin{aligned}[c]
% 		\delta_1=\frac{mg+{}m\sqrt{2ga_0-a_o^2}}{k}
% 	\end{aligned}
% 		\qquad\qquad
% 	\begin{aligned}[c]
% 		\delta_2=\frac{mg-{}m\sqrt{2ga_0-a_o^2}}{k}
% 	\end{aligned}
% \end{equation*}

% Так как требуется найти $\delta_{max}$, подходит корень $\delta_1$, $\delta_1>\delta_2$.

\end{document}