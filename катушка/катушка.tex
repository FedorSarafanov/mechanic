\documentclass[a5paper,10pt]{article}
 
\usepackage{extsizes}
\usepackage{cmap}
\usepackage[T2A]{fontenc}
\usepackage[utf8x]{inputenc}
\usepackage[english, russian]{babel}

\usepackage{misccorr}

%%%%%%%%%%%%%%%%%%%%%%%%%%%%%%%%%%%%%%%%%%%%%%%%%%%%%%%%%%%%%%%%%%%%%%%%%%%%%%%%%%  
\usepackage{graphicx} % для вставки картинок
\graphicspath{{img/}}
\usepackage{amssymb,amsfonts,amsmath,amsthm} % математические дополнения от АМС

% \usepackage{fontspec}
% \usepackage{unicode-math}

\usepackage{indentfirst} % отделять первую строку раздела абзацным отступом тоже
\usepackage[usenames,dvipsnames]{color} % названия цветов
\usepackage{makecell}
\usepackage{multirow} % улучшенное форматирование таблиц
\usepackage{ulem} % подчеркивания
\linespread{1.3} % полуторный интервал
% \renewcommand{\rmdefault}{ftm} % Times New Roman (не работает)
\frenchspacing
\usepackage{geometry}
\geometry{left=1cm,right=1cm,top=2cm,bottom=1cm,bindingoffset=0cm}
\usepackage{titlesec}
\usepackage{float}
% \definecolor{black}{rgb}{0,0,0}
% \usepackage[colorlinks, unicode, pagecolor=black]{hyperref}
% \usepackage[unicode]{hyperref} %ссылки
\usepackage{fancyhdr} %загрузим пакет
\pagestyle{fancy} %применим колонтитул
\fancyhead{} %очистим хидер на всякий случай
\fancyhead[LE,RO]{Сарафанов Ф.Г.} %номер страницы слева сверху на четных и справа на нечетных
\fancyhead[CO, CE]{Механика}
\fancyhead[LO,RE]{Задача под запись -- <<катушка>>} 
\fancyfoot{} %футер будет пустой
% \fancyfoot[CO,CE]{\thepage}
\renewcommand{\labelenumii}{\theenumii)}


\usepackage{tikz}
\usetikzlibrary{scopes}
\usetikzlibrary{%
     decorations.pathreplacing,%
     decorations.pathmorphing,%
    patterns,%
    calc,%
    scopes,%
    arrows,%
    through,%
    % arrows.spaced,%
}
\newcommand{\vangle}{\mathop{\mathstrut^\wedge}\nolimits}
\newcommand{\average}[1]{\langle{#1}\rangle}
\usepackage{wrapfig}
\newcommand{\RN}[1]{%
  \textup{\tiny\uppercase\expandafter{\romannumeral#1}}%
}
\usepackage{framed,color}
	\definecolor{shadecolor}{rgb}{1,1,1}
\begin{document}

\begin{figure}[H]
    \centering
\begin{tikzpicture}[
    force/.style={>=latex,draw=blue,fill=blue},
    % axis/.style={densely dashed,gray,font=\small},
    axis/.style={densely dashed,black!60,font=\small},
    M/.style={rectangle,draw,fill=lightgray,minimum size=0.5cm,thin},
    m2/.style={draw=black!30, rectangle,draw,thin, fill=blue!2, minimum width=0.7cm,minimum height=0.7cm},
    m1/.style={draw=black!30, rectangle,draw,thin, fill=blue!2, minimum width=0.7cm,minimum height=0.7cm},
    plane/.style={draw=black!30, very thick, fill=blue!5, line width=1pt},
    % base/.style={draw=black!70, very thick, fill=blue!4, line width=2pt},
    string/.style={draw=black, thick},
    pulley/.style={thick},
    interface1/.style={draw=gray!60,
        % The border decoration is a path replacing decorator. 
        % For the interface style we want to draw the original path.
        % The postaction option is therefore used to ensure that the
        % border decoration is drawn *after* the original path.
        postaction={draw=gray!60,decorate,decoration={border,angle=-135,
                    amplitude=0.3cm,segment length=2mm}}},
    interface/.style={
        pattern = north east lines,
        draw    = none,
        pattern color=gray!60,          
    },
    plank/.style={
        fill=black!60, 
        draw=black,
        minimum width=3cm,
        inner ysep=0.1cm,
        outer sep=0pt,
        yshift=0.75cm,
        pattern = north east lines,
        pattern color=gray!60, 
    },
    cargo/.style={
        rectangle,
        fill=black!70,              
        inner sep=2.5mm,
    }
]
	\draw (0,2) coordinate (o) circle (2); 
	\draw (o) circle (0.5); 
	\draw[interface] (-4,-0.25) rectangle (4,0);
	\draw[thick] (-4,0) --(4,0);

	\draw (0,1.5) -- ++(3,0);
	\draw[force,->] (0,1.5) ++(3,0) -- ++ (1,0) 
					node[right] {$\vec{u}$};

	\draw (o) -- ++ (45:0.5)
		node[right, yshift=3pt] {$r$};

	\draw (o) -- node[left, xshift=-2pt] {$R$} ++ (135:2);

	\draw[fill] (0,4) coordinate (A) circle (2pt) node[above] {$A$};

	\draw[fill] (0,0) coordinate (0) circle (2pt) node[below, yshift=-6pt] {МЦС};


	\draw[force,->] (o) -- ++(1,0)
					node[right] {$\vec{v}_0$};

	\draw[force,->] (A) -- ++(2,0)
					node[right] {$\vec{v}_A$};

	\draw[axis] (0,0) -- (2,4);

\end{tikzpicture}
\end{figure}

Частота вращения вокруг МЦС --- $\omega$:
\begin{gather*}
	\omega=\frac{u}{R-r}
\end{gather*}
Тогда
\begin{equation*}
	v_0=\omega{R}=u \frac{R}{R-r}, \hspace{5ex} v_A=\omega{2R}=2v_0
\end{equation*}
Так как скорость в точке $A$ постоянна, $a=a_n$:
\begin{gather*}
	a_n=\frac{v^2}{R}
\end{gather*}
Где $v$ -- <<собственная>> скорость точки (скорость берез пересносной скорости) $A$, $v=v_A-v_0$
\begin{gather*}
	v=v_A-v_0=v_0\\
	a_A=\frac{v_0^2}{R}=\frac{u^2R}{(R-r)^2}
\end{gather*}
Кстати, отсюда следует, что радиус кривизны $\rho$ в высшей точке равен $4R$:
\begin{gather*}
	a_A=\frac{v_A^2}{\rho}=
	\frac{4v_0^2}{\rho}=\frac{4u^2R}{\rho(R-r)^2}\\
	\frac{4u^2R}{\rho(R-r)^2}=\frac{u^2R}{(R-r)^2}\\
	\rho=4R
\end{gather*}

\end{document}