\documentclass[a5paper,10pt]{article}\usepackage[usenames,dvipsnames]{color}\usepackage{cmap,graphicx,misccorr,indentfirst,makecell,multirow,ulem,geometry,amssymb,amsfonts,amsmath,amsthm,titlesec,float,fancyhdr,wrapfig,tikz}\usepackage[T2A]{fontenc}\usepackage[utf8x]{inputenc}\usepackage[english, russian]{babel}\usetikzlibrary{decorations.pathreplacing,decorations.pathmorphing,patterns,calc,scopes,arrows,through}\graphicspath{{img/}}\linespread{1.3}\frenchspacing\geometry{left=1cm, right=1cm, top=2cm, bottom=1cm, bindingoffset=0cm}\pagestyle{fancy}\fancyhead{}\fancyhead[R]{Сарафанов Ф.Г.} 
\fancyhead[C]{Механика}
\fancyhead[L]{Задача под запись -- <<поле>>} 
\fancyfoot{}
\begin{document}

\begin{figure}[H]
    \centering
\begin{tikzpicture}[
    force/.style={>=latex,draw=blue,fill=blue},
    axis/.style={densely dashed,gray,font=\small},
    M/.style={rectangle,draw,fill=lightgray,minimum size=0.5cm,thin},
    m/.style={rectangle,draw=black,fill=lightgray,minimum size=0.3cm,thin},
    plane/.style={draw=black,fill=blue!10},
    string/.style={draw=black, thick},
    pulley/.style={thick},
    acceleration/.style={>=open triangle 60,draw=blue,fill=blue},
    inforce/.style={force,double equal sign distance=2pt},
    interface/.style={
        pattern = north east lines,
        draw    = none,
        pattern color=gray!60,          
    },
]

\draw[draw=none, pattern=horizontal lines, pattern color=black!6] (0,-2) rectangle (3,2);
    \draw[thin] (0,-2) -- (0,2);

    \draw[force, <-] (0,1.7) -- node[below] {$\vec{F}$} (3,1.7);

    \draw[thin] (3,-2) -- (3,2);


    \draw[axis,<->] (0,2.1) -- node[above, black] {$l$} (3,2.1);

    \draw (0,1) .. controls (1,0.8) and (2,0.5) .. (3,-1);
    \draw (0,1) coordinate (A) -- ++(170:3);
    \draw (3,-1) coordinate (B) -- ++(-55:1.2);

    \draw[axis,->] (A) ++ (-3,0) -- ++(9,0) node [right] {$+x$};
    \draw[axis,->] (A) -- ++(0,-4) node [below] {$+y$};
    \draw[axis] (B) -- ++ (3,0);% -- ++(-9,0);

    % \draw[axis,->] (0,0) -- ++(5,0) node [right] {$x$};
    \draw[force,->] (A) -- ++(-10:1) node[below] {$\vec{v}_0$};
    % \draw[force,->] (A) -- ++(-1,0) node[above] {$\vec{f}_R$};
    % \draw[fill=white] (0,0) ++(-0.2,0.2) rectangle ++(0.4,-0.4);
    \draw[fill=white] (A) circle (3pt);

    \draw[force,->] (B) -- ++(-55:1) node[left, xshift=-1pt] {$\vec{v}_1$};
    \draw[fill=white] (B) circle (3pt);



    \draw[->] (A)++(170:2) arc (170:180:2);
    \path (A)++(175:{2cm+5pt}) node {$\alpha$};

    \draw[->] (B)++(-55:1) arc (-55:0:1);
    \path (B)++(-30:{1cm+5pt}) node {$\beta$};

\end{tikzpicture}
% \vspace{-2em}
\end{figure}

Запишем теорему о изменении кинетической энергии в проекциях на оси:
\begin{gather*}
    \xdef\mv{\frac{mv_{0_y}^2}{2}}\xdef\mvv{\frac{mv_{1_y}^2}{2}}\xdef\mgk{\frac{mg}{k}}%
    \begin{aligned}[c]
        \text{y: } \mvv-\mv=\int_0^l 0\,dl\\
    \end{aligned}
        \qquad\Rightarrow\qquad
    \begin{aligned}[c]
        v_{0_y}=v_{1_y}
    \end{aligned}
\end{gather*}
\begin{gather*}
    \xdef\mv{\frac{mv_{0_x}^2}{2}}\xdef\mvv{\frac{mv_{1_x}^2}{2}}\xdef\mgk{\frac{mg}{k}}%
    \begin{aligned}[c]
        \text{x: } \mvv-\mv=\int_0^l F_l\,dl=-Fl\\
    \end{aligned}
        \qquad\Rightarrow\qquad
    \begin{aligned}[c]
        v_{1_x}=\sqrt{v_{0_x}^2-\frac{2Fl}{m}}
    \end{aligned}\\
    v_{1_x}=\sqrt{(v_0\cos\alpha)^2-\frac{2Fl}{m}}\\
    v_{1_y}=v_{0_y}=v_0\sin\alpha\\
    \tg\beta=\frac{v_{1_y}}{v_{1_x}}=\frac{v_0\sin\alpha}{\sqrt{(v_0\cos\alpha)^2-\frac{2Fl}{m}}}
\end{gather*}

Граничным условием невылета из поля будет равенство $\beta=90^\circ$, или подкоренное выражение знаменателя $\tg\beta$ не имеет смысла в поле действительных чисел:
\begin{gather*}
    \begin{aligned}[c]
        (v_0\cos\alpha)^2-\frac{2Fl}{m}=0
    \end{aligned}
        \qquad\Rightarrow\qquad
    \begin{aligned}[c]
        Fl\geq\frac{mv_0^2\cos^2\alpha}{2}
    \end{aligned}\\
\end{gather*}
Физически это значит, что в таком случае вся кинетическая энергия тела в проекции на $x$ меньше работы, необходимой для преодоления поля в поперечном направлении.

\end{document}