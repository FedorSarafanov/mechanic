\documentclass[a5paper,10pt]{article}
 
\usepackage{extsizes}
\usepackage{cmap}
\usepackage[T2A]{fontenc}
\usepackage[utf8x]{inputenc}
\usepackage[english, russian]{babel}

\usepackage{misccorr}

%%%%%%%%%%%%%%%%%%%%%%%%%%%%%%%%%%%%%%%%%%%%%%%%%%%%%%%%%%%%%%%%%%%%%%%%%%%%%%%%%%  
\usepackage{graphicx} % для вставки картинок
\graphicspath{{img/}}
\usepackage{amssymb,amsfonts,amsmath,amsthm} % математические дополнения от АМС

% \usepackage{fontspec}
% \usepackage{unicode-math}

\usepackage{indentfirst} % отделять первую строку раздела абзацным отступом тоже
\usepackage[usenames,dvipsnames]{color} % названия цветов
\usepackage{makecell}
\usepackage{multirow} % улучшенное форматирование таблиц
\usepackage{ulem} % подчеркивания
\linespread{1.3} % полуторный интервал
% \renewcommand{\rmdefault}{ftm} % Times New Roman (не работает)
\frenchspacing
\usepackage{geometry}
\geometry{left=1cm,right=1cm,top=2cm,bottom=1cm,bindingoffset=0cm}
\usepackage{titlesec}
\usepackage{float}
% \definecolor{black}{rgb}{0,0,0}
% \usepackage[colorlinks, unicode, pagecolor=black]{hyperref}
% \usepackage[unicode]{hyperref} %ссылки
\usepackage{fancyhdr} %загрузим пакет
\pagestyle{fancy} %применим колонтитул
\fancyhead{} %очистим хидер на всякий случай
\fancyhead[LE,RO]{Сарафанов Ф.Г.} %номер страницы слева сверху на четных и справа на нечетных
\fancyhead[CO, CE]{Механика}
\fancyhead[LO,RE]{Иродов -- №1.147} 
\fancyfoot{} %футер будет пустой
% \fancyfoot[CO,CE]{\thepage}
\renewcommand{\labelenumii}{\theenumii)}


\usepackage{tikz}
\usetikzlibrary{scopes}
\usetikzlibrary{%
     decorations.pathreplacing,%
     decorations.pathmorphing,%
    patterns,%
    calc,%
    scopes,%
    arrows,%
    % arrows.spaced,%
}

\begin{document}

\begin{figure}[H]
    \centering
\begin{tikzpicture}[
    force/.style={>=latex,draw=blue,fill=blue},
    % axis/.style={densely dashed,gray,font=\small},
    axis/.style={densely dashed,black!60,font=\small},
    interface/.style={
        pattern = north east lines,
        draw    = none,
        pattern color=gray!60,          
    },
    cargo/.style={
        rectangle,
        fill=magenta!40,
        draw=black!50,
        inner sep=2.5mm,
    },
    spring/.style={
        decoration={
            aspect=0.3, 
            segment length=.8mm, 
            amplitude=2mm,
            coil},
        decorate,
        draw=magenta!25
    },
    interface1/.style={draw=gray!60,
        % The border decoration is a path replacing decorator. 
        % For the interface style we want to draw the original path.
        % The postaction option is therefore used to ensure that the
        % border decoration is drawn *after* the original path.
        postaction={draw=gray!60,decorate,decoration={border,angle=-135,
                    amplitude=0.3cm,segment length=2mm}}},    
]
\def\angle{41}
%%%%%%%%%%%%%%%%%%%%%%%%%%%%%%%%%%%%%%
    \draw[thick, interface] (0,0) rectangle ++(4,-0.25);
    \draw[thick] (0,0) -- ++(4,-0);

    \draw[magenta!50,line join=round,line width=5pt,dash pattern={on 7pt off 2pt on 7pt off 2pt}] (1,2.85pt) -- ++(3,0) -- ++ (2.5pt,0) -- ++(0,-1.1);
   

  
    % \draw[black!10] (4,0) -- ++(0,1);
    \begin{scope}[yshift=0.5cm]
        \draw[axis,->] (-1,1) -- (5,1) 
                        node [right] {$+x$};   
        \draw[black] (1,0.9) -- ++(0,0.2) node[above] {$0$};
        \draw[black] (4,0.9) -- ++(0,0.2) node[above] {$(1-\eta)\cdot l$};    
    \end{scope}

    \draw[axis,->] (6,1) -- ++(0,-2.2) 
                    node [below] {$+y$};

    \draw[black] (5.9,0) -- ++(0.2,0) node[right] {$0$};
    \draw[black] (5.9,-1) -- ++(0.2,0) node[right] {$\eta l$};

    \draw[force,->] (2.5,0) -- ++(-3,0) node[left] {$\vec{f}_R$};
    \draw[force,->] (2.5,0) -- ++(0,-0.8) node[below] {$(1-\eta)m\vec{g}$};
    \draw[force,->] (2.5,0) -- ++(0,0.8) node[above] {$\vec{N}$};

    \draw[force,->] (4.1,-0.5) -- ++(0,1) node[above] {$\vec{T}'$};
    \draw[force,->] (4.1,-0.5) -- ++(0,-1) node[below] {$\eta m \vec{g}$};
    \draw[force,->] (2.5,0.1) -- (4.6,0.1) node[above] {$\vec{T}$};

    % \draw[axis] (4,0) -- ++(1,0);
    % \draw[axis] (4,-1) -- ++(1,0) ;
    % \draw[axis,<->] (5,0) -- node[right,black] {$\eta l, m_2=\eta m$} (5,-1);



\end{tikzpicture}
% \vspace{-2em}
\end{figure}
В начальный момент цепочка покоится, ускорения частей цепочки $\equiv 0$.

II з.Н. для части цепочки на столе:
\begin{equation}
    (1-\eta)m\vec{a}=(1-\eta)m\vec{g}+\vec{N}+\vec{T}+\vec{f}_R
\end{equation}
В проекции на $x$:
\begin{equation}
    0=T-k(1-\eta)mg
\end{equation}
II з.Н. для части цепочки, свисающей со стола (в проекции на $y$):
\begin{equation}
    0=\eta mg-T'
\end{equation}
С учётом того, что ускорения частей равны, и силы натяжения в точке перегиба также равны, получим условие покоя цепочки (при нулевой начальной скорости):
\begin{equation}
    \eta mg=k(1-\eta)mg
\end{equation}
Значит,
\begin{equation}
    k=\frac{\eta}{1-\eta}
\end{equation}
Найдем работу силы трения по определению:
\begin{equation}
    A_\text{тр}=\int_{0}^{(1-\eta)l}(-m\frac{x}{l}g{k})\cdot{dx}
\end{equation}
\begin{equation}
    A_\text{тр}=-mg{k}\frac{(1-\eta)^2l^2}{2l}=
    -mg\frac{\eta}{1-\eta}\frac{(1-\eta)^2l^2}{2l}
\end{equation}
И окончательно
\begin{equation}
    A_\text{тр}=-mgl\frac{\eta(1-\eta)}{2}
\end{equation}
% Спустившись с горки (силы трения с нею нет), груз будет обладать кинетической энергией, которая найдется из ЗСМЭ:
% \begin{equation}
%     \frac{mv^2}{2}=mgh \quad\Longrightarrow \quad v^2=2gh
% \end{equation}
% Рассмотрим систему <<$m,M$>>. В проекции на ось $x$ внешние силы отсутствуют, сила трения -- внутренняя -- импульса не изменит, запишем ЗСИ в проекции на $x$: 
% \begin{equation}
%     mv=(M+m)u
% \end{equation}
% Откуда
% \begin{equation}
%     u=v\frac{m}{m+M}
% \end{equation}
% Тогда найдем работу силы трения по теореме о изменении кинетической энергии (между точкой конца спуска и произвольной точкой после остановки шайбы):
% \begin{equation}
%     A_\text{тр}=\Delta W_k=\frac{(M+m)u^2}{2}-\frac{mv^2}{2}
% \end{equation}
% Подставим $u$:
% \begin{equation}
%     A_\text{тр}=\frac{mv^2}{2}\left(\frac{m}{m+M}-1\right)
% \end{equation}
% И окончательно, подставляя $v^2$
% \begin{equation}
%     A_\text{тр}=-\frac{mM}{m+m}gh
% \end{equation}


\end{document}