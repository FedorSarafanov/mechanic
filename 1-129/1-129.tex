\documentclass[a5paper,10pt]{article}\usepackage[usenames,dvipsnames]{color}

\usepackage{cmap,graphicx,etoolbox,misccorr,indentfirst,makecell,multirow,ulem,geometry,amssymb,amsfonts,amsmath,amsthm,titlesec,float,fancyhdr,wrapfig,tikz}

\usepackage[T2A]{fontenc}\usepackage[utf8x]{inputenc}\usepackage[english, russian]{babel}\usetikzlibrary{decorations.pathreplacing,decorations.pathmorphing,patterns,calc,scopes,arrows,through, shapes.misc}\graphicspath{{img/}}\linespread{1.3}\frenchspacing\geometry{left=1cm, right=1cm, top=2cm, bottom=1cm, bindingoffset=0cm}\pagestyle{fancy}\fancyhead{}\fancyhead[R]{Сарафанов Ф.Г.}\fancyhead[C]{Механика}
\fancyhead[L]{Иродов -- №1.129}
\fancyfoot{}

%Команда \beforetext для текста слева от формулы
\makeatletter \newif\if@gather@prefix \preto\place@tag@gather{\if@gather@prefix\iftagsleft@ \kern-\gdisplaywidth@ \rlap{\gather@prefix} \kern\gdisplaywidth@ \fi\fi } \appto\place@tag@gather{\if@gather@prefix\iftagsleft@\else \kern-\displaywidth \rlap{\gather@prefix} \kern\displaywidth \fi\fi \global\@gather@prefixfalse } \preto\place@tag{\if@gather@prefix\iftagsleft@ \kern-\gdisplaywidth@ \rlap{\gather@prefix} \kern\displaywidth@ \fi\fi } \appto\place@tag{\if@gather@prefix\iftagsleft@\else \kern-\displaywidth \rlap{\gather@prefix} \kern\displaywidth \fi\fi \global\@gather@prefixfalse } \newcommand*{\beforetext}[1]{\ifmeasuring@\else \gdef\gather@prefix{#1} \global\@gather@prefixtrue \fi } \makeatother 
\tikzset{force/.style={>=latex,draw=blue,fill=blue}, axis/.style={densely dashed,gray,font=\small}, acceleration/.style={>=open triangle 60,draw=blue,fill=blue}, inforce/.style={force,double equal sign distance=2pt}, interface/.style={pattern = north east lines, draw    = none, pattern color=gray!60, }, cross/.style={cross out, draw=black, minimum size=2*(#1-\pgflinewidth), inner sep=0pt, outer sep=0pt},    cargo/.style={rectangle, fill=black!70, inner sep=2.5mm, }}

\begin{document}
\begin{figure}[H]
    \centering
\begin{tikzpicture}

    % \draw[fill=black] (-4,-2) coordinate (I) circle (1pt);
    % \draw[fill=black] (0,0) coordinate   (II) circle (1pt);
    \draw[fill=black] (4,-2) coordinate  (III) circle (0pt);
    \draw[interface] (-6,-2) rectangle ($(III)+(2,-0.5)$);
    \draw[fill=black] (-4.5,-1.8) circle (0.20);
    \draw[fill=black] (-2.5,-1.8) circle (0.20);
    \draw[fill=white] (-5,-1.8) rectangle ++(3,0.3);
    \draw[thick] (-6,-2) --  (6,-2);
    \node at (-4,-1.3) {$M$}; 
    \node at (-2.7,-0.2) {$m_1,m_2$}; 

    \draw[fill=magenta!10] (-3,-1.5) rectangle ++(0.5,1);
    \draw[fill=magenta!10] (-2.9,-1.5) rectangle ++(0.5,1);
  
    % \draw[axis,->] (-4,1) -- ++(0,-2) node[below]{$+y$};
    % \draw[axis,->] (-4,1) -- ++(0,-2) node[below]{$+y$};
    % \draw[axis,->] (-4,1) -- ++(2,0) node[right]{$+x$};
    % \draw[fill=black] (III) circle (1pt);

\end{tikzpicture}
% \vspace{-1em}
\end{figure}
1) Одновременно.

Система до и после спрыгивания обоих людей:
\begin{gather*}
    0=M\vec{v}+2m(\vec{u}+\vec{v})\\
    \vec{v}=-\frac{2m\vec{u}}{M+2m}=-\vec{u}\frac{2m}{M+2m}
\end{gather*}

2) Последовательно.

Система до и после спрыгивания первого человека:
\begin{gather*}
    0=(M+m)\vec{v}+m(\vec{u}+\vec{v})\\
    \vec{v}_{1}=-\vec{u}\frac{m}{M+2m}
\end{gather*}

Система до и после спрыгивания второго человека:
\begin{gather*}
    (M+m)\vec{v}_{1}=M\vec{v}_{2}+m(\vec{u}+\vec{v}_{2})\\
    \vec{v}_2=\frac{(M+m)\vec{v}_{1}-m\vec{u}}{M+m}=-\frac{1}{M+m}\left[\frac{(M+m)m\vec{u}}{M+2m}+m\vec{u}\right]=\\=-\frac{1}{M+m}
    \left[\frac{Mm\vec{u}}{M+2m}+\frac{Mm\vec{u}}{M+2m}+\frac{3m^2\vec{u}}{M+2m}\right]=\\=-\vec{u}\frac{m(2M+3m)}{(M+m)(M+2m)}
\end{gather*}

\begin{gather*}
    \frac{\vec{v}_2}{\vec{v}}=\frac{2M+3m}{2(M+m)}=
    \frac{2M+2m+m}{2M+2m}=1+\frac{m}{2}(M+m)>1
\end{gather*}

Таким образом, тележка разгонится сильнее, если спрыгивать последовательно. Эта задача аналогична задаче о разгоне ракеты: мгновенная выработка топлива менее эффективна для разгона, чем растянутая во времени. 

\end{document}