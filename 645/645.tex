\documentclass[a5paper,10pt]{article}
 
% \usepackage{extsizes}
\usepackage{cmap}
\usepackage[T2A]{fontenc}
\usepackage[utf8x]{inputenc}
\usepackage[english, russian]{babel}

\usepackage{misccorr}

%%%%%%%%%%%%%%%%%%%%%%%%%%%%%%%%%%%%%%%%%%%%%%%%%%%%%%%%%%%%%%%%%%%%%%%%%%%%%%%%%%  
\usepackage{graphicx} % для вставки картинок
\graphicspath{{img/}}
\usepackage{amssymb,amsfonts,amsmath,amsthm} % математические дополнения от АМС

% \usepackage{fontspec}
% \usepackage{unicode-math}

\usepackage{indentfirst} % отделять первую строку раздела абзацным отступом тоже
\usepackage[usenames,dvipsnames]{color} % названия цветов
\usepackage{makecell}
\usepackage{multirow} % улучшенное форматирование таблиц
\usepackage{ulem} % подчеркивания
\linespread{1.3} % полуторный интервал
% \renewcommand{\rmdefault}{ftm} % Times New Roman (не работает)
\frenchspacing
\usepackage{geometry}
\geometry{left=1cm,right=1cm,top=2cm,bottom=1cm,bindingoffset=0cm}
\usepackage{titlesec}
\usepackage{float}
% \definecolor{black}{rgb}{0,0,0}
% \usepackage[colorlinks, unicode, pagecolor=black]{hyperref}
% \usepackage[unicode]{hyperref} %ссылки
\usepackage{fancyhdr} %загрузим пакет
\pagestyle{fancy} %применим колонтитул
\fancyhead{} %очистим хидер на всякий случай
\fancyhead[R]{Сарафанов Ф.Г.} %номер страницы слева сверху на четных и справа на нечетных
\fancyhead[C]{Механика}
\fancyhead[L]{Яковлев -- №645} 
\fancyfoot{} %футер будет пустой
% \fancyfoot[CO,CE]{\thepage}
\renewcommand{\labelenumii}{\theenumii)}


\usepackage{tikz}
\usepackage{tikz-3dplot}
\usetikzlibrary{scopes}
\usetikzlibrary{%
     decorations.pathreplacing,%
     decorations.pathmorphing,%
    patterns,%
    calc,%
    scopes,%
    arrows,%
    arrows.meta,%
    % arrows.spaced,%
}

\tikzset{
    % MyPersp/.style={scale=1.8,x={(1.1cm,-0cm)},y={(0.5cm,1cm)}, z={(0cm,0.8cm)}},
 % MyPersp/.style={scale=1.5,x={(0cm,0cm)},y={(1cm,0cm)}, z={(0cm,1cm)}}, 
 % MyPersp/.style={scale=1.5,x={(1cm,0cm)},y={(0cm,1cm)}, z={(0cm,0cm)}}, 
    % MyPoints/.style={fill=black,draw=black},
    force/.style={>=latex,draw=blue,fill=blue},
    % angular/.style={-{Stealth[open, angle=30:5pt,line width=1pt]}, draw=magenta},
    angular/.style={-{Latex[length=3mm, line width=0.4pt,open,magenta, fill=white]}, draw=magenta},
    % axis/.style={densely dashed,gray,font=\small},
    axis/.style={densely dashed,black!60,font=\small},
    interface/.style={
        pattern = north east lines,
        draw    = none,
        pattern color=gray!60,          
    },
    plank/.style={
        fill=black!60, 
        draw=black,
        minimum width=3cm,
        inner ysep=0.1cm,
        outer sep=0pt,
        yshift=0.75cm,
        pattern = north east lines,
        pattern color=gray!60, 
    },
    cargo/.style={
        rectangle,
        fill=black!70,              
        inner sep=2.5mm,
    },
    acceleration/.style={>=open triangle 60,draw=blue,fill=blue},
    inforce/.style={force,double equal sign distance=2pt},
}


\begin{document}
\begin{figure}[H]
	\centering
	\vspace{-2em}
\begin{tikzpicture}[font=\large]
\matrix[column sep=2cm] {
		\draw[<->, axis] (1.5,2.5) -- node[above, black] { $L$} ++(3.5,0);
		\draw[<->, axis] (0,2.5) --  node[above, black] { $l$} ++(1.5,0);

		\draw[axis] (0,0) -- ++(0,2.5) (1.5,0) -- ++(0,2.5) (5,0) -- ++(0,2.5);

		\draw[axis, fill=magenta!10] (1.5,0) rectangle (5,2);
		\draw[] (0,0) rectangle (5,2);
		\coordinate (C) at (3.25,1); 
		\path (0,0) -- coordinate (mid) (5,0);
		\path (1.5,0) -- coordinate (F) ++ (0,2);
		\draw[acceleration, ->] (mid) ++ (-0.5,-0.5) -- node[below] {$\vec{a}_1$}  ++(1,0); 

		\draw[force, ->] (F) -- node[above] {$\vec{N}_1$} ++(0.7,0);
		\draw[force, ->] (F) -- node[above] {$\vec{F}_{d_1}$} ++(-0.7,0);
		\draw[inforce, ->] (C) -- node[above] {$\vec{F}_{in_1}$} ++(-0.7,0);

		\node[shape=rectangle, minimum width=2mm, minimum height=2mm, anchor=center ,fill=black!80] at (1.5,2) {};
   &
		\draw[<->, axis] (1.5,2.5) -- node[above, black] { $L$} ++(3.5,0);
		\draw[<->, axis] (0,2.5) --  node[above, black] { $l$} ++(1.5,0);

		\draw[axis] (0,0) -- ++(0,2.5) (1.5,0) -- ++(0,2.5) (5,0) -- ++(0,2.5);

		\draw[axis, fill=magenta!10] (0,0) rectangle (1.5,2);
		\draw (0,0) rectangle (5,2);
		\coordinate (C) at (0.75,1); 
		\path (0,0) -- coordinate (mid) (5,0);
		\path (1.5,0) -- coordinate (F) ++ (0,2);
		\draw[acceleration, <-] (mid) ++ (-0.5,-0.5) -- node[below] {$\vec{a}_2$}  ++(1,0); 

		\draw[force, ->] (F) ++ (0,0) -- node[above] {$\vec{F}_{d_2}$} ++(0.7,0);
		\draw[force, ->] (F) ++ (0,0.2) -- node[above] {$\vec{N}_2$} ++(-0.7,0);
		\draw[inforce, ->] (C) -- node[below] {$\vec{F}_{in_2}$} ++(0.7,0);

		\node[shape=rectangle, minimum width=2mm, minimum height=2mm, anchor=center ,fill=black!80] at (1.5,2) {};
\\
};
\end{tikzpicture}
	\vspace{-1.em}
\end{figure}

Запишем аналог II з.Н. для НИСО цилиндра. По III з.Н., сила нормальной реакции слоя воды слева от пробки равна и противоположна силе давления слоя справа (массой $m_1=SL\rho$) на слой слева (массой $m_2=Sl\rho$):
\begin{gather*}
    \begin{aligned}[c]
    	\vec{F}_{d_1}=-\vec{N}_{1}\\
        m\vec{a}_1'=0=\vec{N}_1+\vec{F}_{in_1}\\
        \vec{F}_{d_1}=\vec{F}_{in_1}\\
    \end{aligned}
        \qquad,\qquad
    \begin{aligned}[c]
         \vec{F}_{d_1}=-m_1\vec{a_1}\\
        {F}_{d_1}=m_1a_1       
    \end{aligned}\\
\end{gather*}
По закону Паскаля, давление, производимое на жидкость, передается в любую точку без изменений во всех направлениях. Тогда давление на пробке по определению:
\begin{gather*}
	P=\frac{F_{d_1}}{S}\\
        P=\frac{m_1a_1}{S}=\frac{SL\rho{a_1}}{S}=\rho{La_1}\\
        a_1=\frac{P}{\rho{L}}\approx5\text{ m/s$^2$}\\
        % \text{пробка вылетит при } \forall a\geq{a_1}.
\end{gather*}
Подобные рассуждения дают ответ для ускорения в противоположном направлении: 
\begin{gather*}
	P=\frac{F_{d_2}}{S}\\
        P=\frac{m_2a_2}{S}=\frac{Sl\rho{a_2}}{S}=\rho{la_2}\\
        a_2=\frac{P}{\rho{l}}\approx50\text{ m/s$^2$}\\
        % \text{пробка вылетит при } \forall a\geq{a_2}.
\end{gather*}
\end{document}