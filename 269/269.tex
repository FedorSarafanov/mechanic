\documentclass[a5paper,10pt]{article}\usepackage[usenames,dvipsnames]{color}\usepackage{extsizes,cmap,graphicx,misccorr,indentfirst,makecell,multirow,ulem,geometry,amssymb,amsfonts,amsmath,amsthm,titlesec,float,fancyhdr,wrapfig,tikz}\usepackage[T2A]{fontenc}\usepackage[utf8x]{inputenc}\usepackage[english, russian]{babel}\usetikzlibrary{decorations.pathreplacing,decorations.pathmorphing,patterns,calc,scopes,arrows,through}\graphicspath{{img/}}\linespread{1.3}\frenchspacing\geometry{left=1cm, right=1cm, top=2cm, bottom=1cm, bindingoffset=0cm}\pagestyle{fancy}\fancyhead{}\fancyhead[R]{Сарафанов Ф.Г.} 
\fancyhead[C]{Механика}
\fancyhead[L]{№269 -- Яковлев И.А.} 
\fancyfoot{}
\renewcommand{\labelenumii}{\theenumii)}
\tikzset{
	force/.style={>=latex,draw=blue,fill=blue,>=triangle 45},
    axis/.style={densely dashed,black!60,font=\small},
    interface1/.style={draw=gray!60,.
        postaction={draw=gray!60,decorate,decoration={border,angle=-135,
        amplitude=0.3cm,segment length=2mm}}},
    interface/.style={
        pattern = north east lines,
        draw    = none,
        pattern color=gray!60,          
    },
    plank/.style={
        fill=black!60, 
        draw=black,
        minimum width=3cm,
        inner ysep=0.1cm,
        outer sep=0pt,
        yshift=0.75cm,
        pattern = north east lines,
        pattern color=gray!60, 
    },
    cargo/.style={
        rectangle,
        fill=black!70,              
        inner sep=2.5mm,
    }	
}
\begin{document}

\begin{figure}[H]
    \centering
\begin{tikzpicture}
	\def\angle{50}
	% \draw (0,2) coordinate (o) circle (2); 
	% \draw (o) circle (0.5); 
	% \draw (0,0) -- (5,6);
	\draw[interface] (-6,6.25) rectangle (6,6);
	\draw[thick] (-6,6) --(6,6);

	\coordinate (cc) at (-6,0);
	\draw[fill=white] (cc) circle (2pt);
	\draw[axis, ->] (cc) -- ++ (\angle:7) node[above,black] {$+n$};

	\draw[axis, ->] (0,0) -- + (0,7) node[above,black] {$+x$};
	\draw[black, dotted] (0,0) coordinate (0) arc (0:90:6);

	\draw[fill,black] (0) ++ (0,4) coordinate (m1) circle (3pt);
	\draw[fill,gray] (0) circle (3pt);



	\draw[fill,black] (cc) ++ (\angle:6) coordinate (m2) circle (3pt);

	\draw[force,->] (m1) -- ++ (0,1) node[right] {$\vec{v}$};
	\draw[force,->] (m1) -- ++ (0,-1) node[right] {$\vec{f}_R$};

	\draw[force,->] (m2) -- ++ (90+\angle:1) node[above] {$\vec{v}$};
	\draw[force,->] (m2) -- ++ (180+\angle:1) node[below] {$\vec{f}_R$};

	\draw[axis, ->] (m2) -- ++ (\angle+90:3.5) node[above,black] {$+\tau$};
	\draw (-5,5) node[scale=1.5] {II};
	\draw (5,5) node[scale=1.5] {I};

	\draw[axis, <->] (4,0) -- node[right, black] {$R$} (4,6);
	\draw[axis] (-6,0) --(6,0);


\end{tikzpicture}
\end{figure}
\textbf{Случай $I$.} Рассмотрим движение с торможением без поворота:
\begin{equation*}
    \begin{aligned}[c]
		m\vec{a}=\vec{f}_R\\
		\text{x: }ma=-mg\mu\\
		\int_{v_0}^{v(t)}dv=\int_0^t -g\mu dt\\
		v(t)=v_0-\mu{gt}\\
		\int_{0}^{x}dx=\int_0^t [v_0-\mu{gt}] dt\\
		x(t)=v_0{t}-\mu{g}\frac{t^2}{2}
    \end{aligned}
        \qquad\qquad
    \begin{aligned}[c]
    \text{Условие остановки $v=0$ при $t=t^*$:}\\
    v_\text{ост}=0=v_0-g\mu{}t^*\\
    t^*=\frac{v_0}{g\mu}\\
    \text{Тогда пройденное до остановки $R$:}\\
    R=v_0\cdot{t^*}-\mu{g}\frac{t^2}{2}\\
    R=\frac{v_0^2}{2g\mu}
    \end{aligned}
\end{equation*}

\textbf{Случай $II$.} Поворот без торможения.
\begin{equation*}
    \begin{aligned}[c]
	m\vec{a}=\vec{f}_R\\
	v_\tau=const \Longrightarrow a_\tau=0 \\ a=a_n\\
    \end{aligned}
        \qquad\qquad
    \begin{aligned}[c]
	\text{n: } ma_n=-mg\mu\\
	\frac{v_0^2}{R}=g\mu\\
	R=\frac{v_0^2}{g\mu}
    \end{aligned}
\end{equation*}

\textbf{Вывод.} Путь до остановки с торможением без поворота  вдвое короче, чем при повороте без торможения.

\end{document}

