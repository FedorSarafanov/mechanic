\documentclass[a5paper,10pt]{article}
 
\usepackage{extsizes}
\usepackage{cmap}
\usepackage[T2A]{fontenc}
\usepackage[utf8x]{inputenc}
\usepackage[english, russian]{babel}

\usepackage{misccorr}

%%%%%%%%%%%%%%%%%%%%%%%%%%%%%%%%%%%%%%%%%%%%%%%%%%%%%%%%%%%%%%%%%%%%%%%%%%%%%%%%%%  
\usepackage{graphicx} % для вставки картинок
\graphicspath{{img/}}
\usepackage{amssymb,amsfonts,amsmath,amsthm} % математические дополнения от АМС

% \usepackage{fontspec}
% \usepackage{unicode-math}

\usepackage{indentfirst} % отделять первую строку раздела абзацным отступом тоже
\usepackage[usenames,dvipsnames]{color} % названия цветов
\usepackage{makecell}
\usepackage{multirow} % улучшенное форматирование таблиц
\usepackage{ulem} % подчеркивания
\linespread{1.3} % полуторный интервал
% \renewcommand{\rmdefault}{ftm} % Times New Roman (не работает)
\frenchspacing
\usepackage{geometry}
\geometry{left=1cm,right=1cm,top=2cm,bottom=1cm,bindingoffset=0cm}
\usepackage{titlesec}
\usepackage{float}
% \definecolor{black}{rgb}{0,0,0}
% \usepackage[colorlinks, unicode, pagecolor=black]{hyperref}
% \usepackage[unicode]{hyperref} %ссылки
\usepackage{fancyhdr} %загрузим пакет
\pagestyle{fancy} %применим колонтитул
\fancyhead{} %очистим хидер на всякий случай
\fancyhead[R]{Сарафанов Ф.Г.} %номер страницы слева сверху на четных и справа на нечетных
\fancyhead[C]{Механика}
\fancyhead[L]{Яковлев - №178} 
\fancyfoot{} %футер будет пустой
% \fancyfoot[CO,CE]{\thepage}
\renewcommand{\labelenumii}{\theenumii)}


\usepackage{tikz}
\usetikzlibrary{scopes}
\usetikzlibrary{%
     decorations.pathreplacing,%
     decorations.pathmorphing,%
    patterns,%
    calc,%
    scopes,%
    arrows,%
    % arrows.spaced,%
}

\begin{document}

\begin{figure}[H]
    \centering
\begin{tikzpicture}[
    force/.style={>=latex,draw=blue,fill=blue},
    % axis/.style={densely dashed,gray,font=\small},
    axis/.style={densely dashed,black!60,font=\small},
    M/.style={rectangle,draw,fill=lightgray,minimum size=0.5cm,thin},
    m2/.style={draw=black!30, rectangle,draw,thin, fill=blue!2, minimum width=0.7cm,minimum height=0.7cm},
    m1/.style={draw=black!30, rectangle,draw,thin, fill=blue!2, minimum width=0.7cm,minimum height=0.7cm},
    plane/.style={draw=black!30, very thick, fill=blue!5, line width=1pt},
    % base/.style={draw=black!70, very thick, fill=blue!4, line width=2pt},
    string/.style={draw=black, thick},
    pulley/.style={thick},
    % interface/.style={draw=gray!60,
    %     % The border decoration is a path replacing decorator. 
    %     % For the interface style we want to draw the original path.
    %     % The postaction option is therefore used to ensure that the
    %     % border decoration is drawn *after* the original path.
    %     postaction={draw=gray!60,decorate,decoration={border,angle=-135,
    %                 amplitude=0.3cm,segment length=2mm}}},
    interface/.style={
        pattern = north east lines,
        draw    = none,
        pattern color=gray!60,          
    },
    plank/.style={
        fill=black!60, 
        draw=black,
        minimum width=3cm,
        inner ysep=0.1cm,
        outer sep=0pt,
        yshift=0.75cm,
        pattern = north east lines,
        pattern color=gray!60, 
    },
    cargo/.style={
        rectangle,
        fill=black!10,
        draw=black,              
        inner sep=2.88mm,
    }
]
    % \draw[force,double equal sign distance=2pt,->] (c) -- ++(0,-2) node[below] {$\vec{a}_0$};

%%%%%%%%%%%%%%%%%%%%%%%%%%%%%%%%%%%%%%
   % \draw[interface,fill=white!40, draw=black] (0,0.1) rectangle ++(5,-0.2);
    \draw[draw=black!80,decoration={aspect=0.3, segment length=1mm, amplitude=2mm,coil},decorate] (0,-1) -- node[above, black, pos=0.5, yshift=1em] {} ++(2,0); 
    \draw[fill=black!40] (0,0) circle (1pt) ++(4,0) circle (1pt);
        

    \draw[draw=black!80,decoration={aspect=0.3, segment length=2mm, amplitude=2mm,coil},decorate] (0,0) -- node[above, black, pos=0.5, yshift=1em] {} ++(4,0); 
    \draw[fill=black!40] (0,-1) circle (1pt) ++(2,0) circle (1pt);
    
    \draw[force,->] (2,-1) -- ++ (1,0) node[right] {$\vec{f}_e$};

    % \draw[] (0,-2.1) -- ++(0,0.2) node[below, yshift=-5pt, black] {$0$};
    \draw[] (2,-2.1) -- ++(0,0.2) coordinate (c) node[below, yshift=-5pt, black] {$x$} node[below of=c, black] {$W_{p_2}$};
    \draw[] (4,-2.1) -- ++(0,0.2) coordinate (c) node[below, yshift=-5pt, black] {$0$} node[below of=c, black] {$W_{p_1}=0$};

    % \draw [decorate,decoration={brace,amplitude=5pt},xshift=0pt,yshift=0pt] (2,-2) -- node [black,above, yshift=0.5em] {$\delta$} (4,-2);

    \draw[axis,<-] (0,-2) node[left, black] {$x$} -- ++(5,0) ;
    % \draw[force,->, very thick] (b.center) -- ++(-1,0) node[above, yshift=0.5em, black] {$\vec{f}_e$};

    % \node[left] at (-0.4,-0.4) {$\bigotimes\vec\omega$}; 


\end{tikzpicture}
% \vspace{-2em}
\end{figure}
По определению, 
\begin{equation*}
    A=-\Delta{W_p}
\end{equation*}
Тогда
\begin{gather*}
    \xdef\wpp{W_{p_2}}\xdef\fex{f_{e_x}}%
    A=-(\wpp-W_{p_1})=-\wpp\\
    \wpp=-A=-\int_0^x \fex dx\\
\end{gather*}
И можем найти функцию потенциальной энергии сжатой пружины от величины деформации $x$:
\begin{gather*}
    f_e=\beta{x^3}\\
    \fex=-\beta{x^3}\\
    \wpp=\int_0^x \beta{x^3} dx\\
    \wpp=\frac{\beta{x^4}}{4}
\end{gather*}

\end{document}