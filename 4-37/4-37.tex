\documentclass[a5paper,10pt]{article}
 
\usepackage{extsizes}
\usepackage{cmap}
\usepackage[T2A]{fontenc}
\usepackage[utf8x]{inputenc}
\usepackage[english, russian]{babel}

\usepackage{misccorr}

%%%%%%%%%%%%%%%%%%%%%%%%%%%%%%%%%%%%%%%%%%%%%%%%%%%%%%%%%%%%%%%%%%%%%%%%%%%%%%%%%%  
\usepackage{graphicx} % для вставки картинок
\graphicspath{{img/}}
\usepackage{amssymb,amsfonts,amsmath,amsthm} % математические дополнения от АМС

% \usepackage{fontspec}
% \usepackage{unicode-math}

\usepackage{indentfirst} % отделять первую строку раздела абзацным отступом тоже
\usepackage[usenames,dvipsnames]{color} % названия цветов
\usepackage{makecell}
\usepackage{multirow} % улучшенное форматирование таблиц
\usepackage{ulem} % подчеркивания
\linespread{1.3} % полуторный интервал
% \renewcommand{\rmdefault}{ftm} % Times New Roman (не работает)
\frenchspacing
\usepackage{geometry}
\geometry{left=1cm,right=1cm,top=2cm,bottom=1cm,bindingoffset=0cm}
\usepackage{titlesec}
\usepackage{float}
% \definecolor{black}{rgb}{0,0,0}
% \usepackage[colorlinks, unicode, pagecolor=black]{hyperref}
% \usepackage[unicode]{hyperref} %ссылки
\usepackage{fancyhdr} %загрузим пакет
\pagestyle{fancy} %применим колонтитул
\fancyhead{} %очистим хидер на всякий случай
\fancyhead[R]{Сарафанов Ф.Г.} %номер страницы слева сверху на четных и справа на нечетных
\fancyhead[C]{Механика}
\fancyhead[L]{Иродов - №4.37} 
\fancyfoot{} %футер будет пустой
% \fancyfoot[CO,CE]{\thepage}
\renewcommand{\labelenumii}{\theenumii)}


\usepackage{tikz}
\usetikzlibrary{scopes}
\usetikzlibrary{%
     decorations.pathreplacing,%
     decorations.pathmorphing,%
    patterns,%
    calc,%
    scopes,%
    arrows,%
    % arrows.spaced,%
}

\begin{document}

\begin{figure}[H]
    \centering
\begin{tikzpicture}[
    force/.style={>=latex,draw=blue,fill=blue},
    % axis/.style={densely dashed,gray,font=\small},
    axis/.style={densely dashed,black!60,font=\small},
    M/.style={rectangle,draw,fill=lightgray,minimum size=0.5cm,thin},
    m2/.style={draw=black!30, rectangle,draw,thin, fill=blue!2, minimum width=0.7cm,minimum height=0.7cm},
    m1/.style={draw=black!30, rectangle,draw,thin, fill=blue!2, minimum width=0.7cm,minimum height=0.7cm},
    plane/.style={draw=black!30, very thick, fill=blue!5, line width=1pt},
    % base/.style={draw=black!70, very thick, fill=blue!4, line width=2pt},
    string/.style={draw=black, thick},
    pulley/.style={thick},
    % interface/.style={draw=gray!60,
    %     % The border decoration is a path replacing decorator. 
    %     % For the interface style we want to draw the original path.
    %     % The postaction option is therefore used to ensure that the
    %     % border decoration is drawn *after* the original path.
    %     postaction={draw=gray!60,decorate,decoration={border,angle=-135,
    %                 amplitude=0.3cm,segment length=2mm}}},
    interface/.style={
        pattern = north east lines,
        draw    = none,
        pattern color=gray!60,          
    },
    plank/.style={
        fill=black!60, 
        draw=black,
        minimum width=3cm,
        inner ysep=0.1cm,
        outer sep=0pt,
        yshift=0.75cm,
        pattern = north east lines,
        pattern color=gray!60, 
    },
    cargo/.style={
        rectangle,
        fill=black!10,
        draw=black,              
        inner sep=2.88mm,
    }
]
    % \draw[force,double equal sign distance=2pt,->] (c) -- ++(0,-2) node[below] {$\vec{a}_0$};

%%%%%%%%%%%%%%%%%%%%%%%%%%%%%%%%%%%%%%
    \begin{scope}[opacity=0.4]
    \node[cargo] (II) at (4,0) {} node[yshift=-1.3em] {};
    \draw[draw=black!30,decoration={aspect=0.3, segment length=3.2mm, amplitude=0.91mm,coil},decorate] (0,0) -- node[above, black, pos=0.2, yshift=1em] {} (II); 

    \draw[force,->] (II.center) -- ++(1,0) node[above, black] {$\vec{F}$};
    \draw[force,->] (II.center) -- ++(-1,0) node[above, black] {$\vec{f}_e$};
    \draw[fill=black] (II.center) circle (1pt);
        
    \end{scope}

    \node[cargo] (b) at (2,0) {} node[above of=b, yshift=-1.3em] {$m$};

    \draw[axis,->] (0,-1) -- ++(5,0) node[right, black] {$x$};
    \draw[] (2,-1.1) -- ++(0,0.2) node[below, yshift=-5pt, black] {$0$};
    \draw[] (4,-1.1) -- ++(0,0.2) node[below, yshift=-7pt, black] {$x$};


    \draw[draw=black!30,decoration={aspect=0.3, segment length=1.5mm, amplitude=0.91mm,coil},decorate] (0,0) -- node[above, black, pos=0.2, yshift=1em] {$k$} (b); 


   \draw[interface] (-0.2,-2) rectangle (0,2);
   \draw[interface] (0,-0.3) rectangle (5,-0.5);
        \draw[thick] (0,-2) -- (0,2);   
        \draw[thick] (0,-0.3) -- (5,-0.3);   


\end{tikzpicture}
\vspace{-1em}
\end{figure}
Выберем ось $x$, сонаправленную с силой $\vec{F}$. Запишем в проекции на неё второй закон Ньютона:
\begin{equation*}
	\begin{aligned}[c]
        m\vec{a}&=\vec{F}+\vec{f}_e\\
        \text{x: } ma&=F-kx\\
        \ddot{x}&+\frac{k}{m}x=\frac{F}{m}
	\end{aligned}
		\qquad\Rightarrow\qquad
    \begin{aligned}[c]
        \omega^2&=\frac{k}{m}\\
        x&=A\cdot\cos{(\omega{t}+\phi_0)}+\frac{F}{k}\\
        \dot{x}&=-\omega{A}\sin{(\omega{t}+\phi_0)}
	\end{aligned}
\end{equation*}
Начальные условия $x(t=0)=0$, \ $\dot{x}(t=0)=0$:
\begin{equation*}
\left\{\begin{aligned}
    0 &= A\cdot\cos{\phi_0}+\frac{F}{k}\\
    0 &= -\omega{A}\sin{\phi_0}
    \end{aligned}\right. \qquad\Rightarrow\qquad\left\{
    \begin{aligned}[c]
        \phi_0=\pi\\
        A=\frac{F}{k}
    \end{aligned}\right.
\end{equation*}
Условие остановки $v(t=t_{stop})=0$:
\begin{gather*}
    \dot{x}=0=-\omega{\frac{F}{k}}\sin{(\omega{t}+\phi_0)}=\omega{\frac{F}{k}}\sin\omega{t}\\
    x=\frac{F}{k}\left[1-\cos\omega{t}\right]
\end{gather*}
Тогда решением этих уравнений будет
\begin{gather*}
    t_{stop}=\frac{\pi}{\omega}=\pi\sqrt\frac{m}{k}\\
    S=x(t=t_{stop})=\frac{2F}{k}
\end{gather*}
% Но движение этой же системы до отрыва можно описать вторым законом Нюьтона:
% \begin{equation*}
% 	\begin{aligned}[c]
% 		&m\vec{a_0}=\vec{f_e}+m\vec{g}\\
% 		\text{x: }&ma_0=mg-k\Delta{}\\
% 	\end{aligned}
% 		\qquad\Longrightarrow\qquad
% 	\begin{aligned}[c]
% 		\Delta{}=\frac{m}{k}(g-a_0)\\
% 		% y&=v/w\\
% 	\end{aligned}
% \end{equation*}
% Тогда по теореме о изменении кинетической энергии можем записать работу сил упругости:
% \begin{equation*}
%     \mvn-\mv=\int_{\Delta{}}^{\delta{}} [mg-kx]\cdot{}dx
% \end{equation*}
% \begin{equation*}
%     -\mv=mg(\delta{}-\Delta{})+\frac{k\Delta{}^2}{2}-\frac{k\delta{}^2}{2}\\
% \end{equation*}
% \begin{equation*}
%     -\mv=-ma_0\Delta{}=mg(\delta{}-\Delta{})+\frac{k\Delta{}^2}{2}-\frac{k\delta{}^2}{2}
% \end{equation*}
% \begin{equation*}
% 	\frac{k}{2}\delta^2-mg\delta+\Delta(m(g-a_0)-\frac{k}{2}\Delta)=0
% \end{equation*}
% И найти удлинение пружины $\delta$:
% \begin{equation*}
% 	\frac{k}{2}\delta^2-mg\delta+\frac{m^2}{2k}(g-a_0)^2=0
% \end{equation*}
% \begin{equation*}
% 	\begin{aligned}[c]
% 		&a=\frac{k}{2}\\
% 		&b=-mg\\
% 		&c=\frac{m^2}{2k}(g-a_0)^2
% 	\end{aligned}
% 		\qquad\Longrightarrow\qquad
% 	\begin{aligned}[c]
% 		D&=b^2-4ac\\
% 		\sqrt{D}&=m\sqrt{2ga_0-a_o^2}
% 		% y&=v/w\\
% 	\end{aligned}
% \end{equation*}
% \begin{equation*}
% 	\begin{aligned}[c]
% 		\delta_1=\frac{mg+{}m\sqrt{2ga_0-a_o^2}}{k}
% 	\end{aligned}
% 		\qquad\qquad
% 	\begin{aligned}[c]
% 		\delta_2=\frac{mg-{}m\sqrt{2ga_0-a_o^2}}{k}
% 	\end{aligned}
% \end{equation*}

% Так как требуется найти $\delta_{max}$, подходит корень $\delta_1$, $\delta_1>\delta_2$.

\end{document}