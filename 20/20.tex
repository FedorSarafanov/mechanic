\documentclass[a5paper,10pt]{article}
 
\usepackage{extsizes}
\usepackage{cmap}
\usepackage[T2A]{fontenc}
\usepackage[utf8x]{inputenc}
\usepackage[english, russian]{babel}

\usepackage{misccorr}

%%%%%%%%%%%%%%%%%%%%%%%%%%%%%%%%%%%%%%%%%%%%%%%%%%%%%%%%%%%%%%%%%%%%%%%%%%%%%%%%%%  
\usepackage{graphicx} % для вставки картинок
\graphicspath{{img/}}
\usepackage{amssymb,amsfonts,amsmath,amsthm} % математические дополнения от АМС

% \usepackage{fontspec}
% \usepackage{unicode-math}

\usepackage{indentfirst} % отделять первую строку раздела абзацным отступом тоже
\usepackage[usenames,dvipsnames]{color} % названия цветов
\usepackage{makecell}
\usepackage{multirow} % улучшенное форматирование таблиц
\usepackage{ulem} % подчеркивания
\linespread{1.3} % полуторный интервал
% \renewcommand{\rmdefault}{ftm} % Times New Roman (не работает)
\frenchspacing
\usepackage{geometry}
\geometry{left=1cm,right=1cm,top=2cm,bottom=1cm,bindingoffset=0cm}
\usepackage{titlesec}
\usepackage{float}
% \definecolor{black}{rgb}{0,0,0}
% \usepackage[colorlinks, unicode, pagecolor=black]{hyperref}
% \usepackage[unicode]{hyperref} %ссылки
\usepackage{fancyhdr} %загрузим пакет
\pagestyle{fancy} %применим колонтитул
\fancyhead{} %очистим хидер на всякий случай
\fancyhead[R]{Сарафанов Ф.Г.} %номер страницы слева сверху на четных и справа на нечетных
\fancyhead[C]{Механика}
\fancyhead[L]{Яковлев -- №20} 
\fancyfoot{} %футер будет пустой
% \fancyfoot[CO,CE]{\thepage}
\renewcommand{\labelenumii}{\theenumii)}


\usepackage{tikz}
\usetikzlibrary{scopes}
\usetikzlibrary{%
     decorations.pathreplacing,%
     decorations.pathmorphing,%
    patterns,%
    calc,%
    scopes,%
    arrows,%
    % arrows.spaced,%
}

\begin{document}

\begin{figure}[H]
    \centering
\begin{tikzpicture}[
    force/.style={>=latex,draw=blue,fill=blue},
    % axis/.style={densely dashed,gray,font=\small},
    axis/.style={densely dashed,black!60,font=\small},
    M/.style={rectangle,draw,fill=lightgray,minimum size=0.5cm,thin},
    m2/.style={draw=black!30, rectangle,draw,thin, fill=blue!2, minimum width=0.7cm,minimum height=0.7cm},
    m1/.style={draw=black!30, rectangle,draw,thin, fill=blue!2, minimum width=0.7cm,minimum height=0.7cm},
    plane/.style={draw=black!30, very thick, fill=blue!5, line width=1pt},
    % base/.style={draw=black!70, very thick, fill=blue!4, line width=2pt},
    string/.style={draw=black, thick},
    pulley/.style={thick},
    acceleration/.style={>=open triangle 60,draw=blue,fill=blue},
    inforce/.style={force,double equal sign distance=2pt},
    interface/.style={
        pattern = north east lines,
        draw    = none,
        pattern color=gray!60,          
    },
    plank/.style={
        fill=black!60, 
        draw=black,
        minimum width=3cm,
        inner ysep=0.1cm,
        outer sep=0pt,
        yshift=0.75cm,
        pattern = north east lines,
        pattern color=gray!60, 
    },
    cargo/.style={
        rectangle,
        fill=black!70,              
        inner sep=2.5mm,
    }
]

\draw[scale=0.5,domain=0:2,smooth,variable=\x,blue] plot ({\x},{-\x*\x});
\draw[scale=0.5,domain=-2:2,smooth,variable=\x,blue] plot ({\x+4},{-0.8-0.8*\x*\x});
\draw[scale=0.5,domain=-2:2,smooth,variable=\x,blue] plot ({\x+8},{-1.6-0.6*\x*\x});
\draw[scale=0.5,domain=-2:2,smooth,variable=\x,blue] plot ({\x+12},{-2.4-0.4*\x*\x});
\draw[dashed,scale=0.5,domain=-2:2,smooth,variable=\x,blue] plot ({\x+16},{-3.2-0.2*\x*\x});

\draw[axis, ->] (0,-2) -- ++(10,0) node[right] {$+x$};
\draw[axis, ->] (0,-2) -- ++(0,3) node[right] {$+y$};
\draw[axis] (0,-2) -- ++(-0.5,0);
\draw[axis] (0,0) coordinate (0) -- ++(-0.5,0);

\draw[<->] (-0.5,-2) -- node[left] {$h$} ++ (0,2);
\draw[force,->] (0) -- ++(0.5,0) node[right] {$\vec{v}_0$};

\draw[acceleration, ->] (4,1) -- node[right] {$\vec{g}$} (4,0);
\end{tikzpicture}
% \vspace{-2em}
\end{figure}

\begin{gather*}
    \intertext{Найдем верт. составляющую скорости шарика и его смещение относительно начала координат в момент первого падения:}
    v_y=-gt\\
    y=h-\frac{gt^2}{2}\\
    h=\frac{gt_0^2}{2}\\
    t_0=\sqrt{\frac{2h}{g}}\\
    \Delta{}x_0=vt_0\\
    \intertext{Скорость (по модулю) в момент его падения:}
    v_{y_0}=gt_0\\
    \intertext{Скорость в момент его первого подъема:}
    v_{y_1}=\alpha{}gt_0\\
    \intertext{Так как верт. составляющая скорости шарика в верхней точке нулевая, найдем время подъема (время периода двое больше):}
    v(t)=v_{y_1}-gt_p\\
    t_1=2t_p=2\alpha{t_0}\\
    \intertext{Так как горизонтальная скорость постоянна, найдем смещение шарика за один период:}
    \Delta{}x_1=vt_1=2\alpha{vt_0}\\
    \intertext{Аналогично далее:}
    \Delta{}x_n=2\alpha^nvt_0\\
    \intertext{Так как все смещения положительны, найдем расстояние на оси $x$ от места броска как сумму всех смещений:}
    x=\Delta{x_0}+\Delta{x_1}+\Delta{x_2}+\ldots+\Delta{x_n}\\
    x=vt_0+2\alpha{vt_0}+2\alpha^2{vt_0}+\ldots+2\alpha^n{v_0t}\\
    x=vt_0+2vt_0(\alpha+\alpha^2+\ldots+\alpha^n)\\
    \intertext{Воспользуемся свойством бесконечной геометрической прогрессии, учитывая, что её знаменатель $\alpha<1$:}
    x=vt_0(1+2\frac{1}{1-\alpha})=vt_0(\frac{1-\alpha}{1-\alpha}+\frac{2}{1-\alpha})=vt_0\frac{1+\alpha}{1-\alpha}=v\sqrt{\frac{2h}{g}}\cdot\frac{1+\alpha}{1-\alpha}
\end{gather*}

\end{document}