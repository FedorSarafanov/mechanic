\documentclass[a5paper,10pt]{article}
 
\usepackage{extsizes}
\usepackage{cmap}
\usepackage[T2A]{fontenc}
\usepackage[utf8x]{inputenc}
\usepackage[english, russian]{babel}

\usepackage{misccorr}

%%%%%%%%%%%%%%%%%%%%%%%%%%%%%%%%%%%%%%%%%%%%%%%%%%%%%%%%%%%%%%%%%%%%%%%%%%%%%%%%%%  
\usepackage{graphicx} % для вставки картинок
\graphicspath{{img/}}
\usepackage{amssymb,amsfonts,amsmath,amsthm} % математические дополнения от АМС

% \usepackage{fontspec}
% \usepackage{unicode-math}

\usepackage{indentfirst} % отделять первую строку раздела абзацным отступом тоже
\usepackage[usenames,dvipsnames]{color} % названия цветов
\usepackage{makecell}
\usepackage{multirow} % улучшенное форматирование таблиц
\usepackage{ulem} % подчеркивания
\linespread{1.3} % полуторный интервал
% \renewcommand{\rmdefault}{ftm} % Times New Roman (не работает)
\frenchspacing
\usepackage{geometry}
\geometry{left=1cm,right=1cm,top=2cm,bottom=1cm,bindingoffset=0cm}
\usepackage{titlesec}
\usepackage{float}
% \definecolor{black}{rgb}{0,0,0}
% \usepackage[colorlinks, unicode, pagecolor=black]{hyperref}
% \usepackage[unicode]{hyperref} %ссылки
\usepackage{fancyhdr} %загрузим пакет
\pagestyle{fancy} %применим колонтитул
\fancyhead{} %очистим хидер на всякий случай
\fancyhead[R]{Сарафанов Ф.Г.} %номер страницы слева сверху на четных и справа на нечетных
\fancyhead[C]{Механика}
\fancyhead[L]{Иродов - №1.163} 
\fancyfoot{} %футер будет пустой
% \fancyfoot[CO,CE]{\thepage}
\renewcommand{\labelenumii}{\theenumii)}


\usepackage{tikz}
\usetikzlibrary{scopes}
\usetikzlibrary{%
     decorations.pathreplacing,%
     decorations.pathmorphing,%
    patterns,%
    calc,%
    scopes,%
    arrows,%
    % arrows.spaced,%
}

\begin{document}

\begin{figure}[H]
    \centering
\begin{tikzpicture}[
    force/.style={>=latex,draw=blue,fill=blue},
    % axis/.style={densely dashed,gray,font=\small},
    axis/.style={densely dashed,black!60,font=\small},
    M/.style={rectangle,draw,fill=lightgray,minimum size=0.5cm,thin},
    m2/.style={draw=black!30, rectangle,draw,thin, fill=blue!2, minimum width=0.7cm,minimum height=0.7cm},
    m1/.style={draw=black!30, rectangle,draw,thin, fill=blue!2, minimum width=0.7cm,minimum height=0.7cm},
    plane/.style={draw=black!30, very thick, fill=blue!5, line width=1pt},
    % base/.style={draw=black!70, very thick, fill=blue!4, line width=2pt},
    string/.style={draw=black, thick},
    pulley/.style={thick},
    % interface/.style={draw=gray!60,
    %     % The border decoration is a path replacing decorator. 
    %     % For the interface style we want to draw the original path.
    %     % The postaction option is therefore used to ensure that the
    %     % border decoration is drawn *after* the original path.
    %     postaction={draw=gray!60,decorate,decoration={border,angle=-135,
    %                 amplitude=0.3cm,segment length=2mm}}},
    interface/.style={
        pattern = north east lines,
        draw    = none,
        pattern color=gray!60,          
    },
    plank/.style={
        fill=black!60, 
        draw=black,
        minimum width=3cm,
        inner ysep=0.1cm,
        outer sep=0pt,
        yshift=0.75cm,
        pattern = north east lines,
        pattern color=gray!60, 
    },
    cargo/.style={
        rectangle,
        fill=black!10,
        draw=black,              
        inner sep=2.88mm,
    }
]
    % \draw[force,double equal sign distance=2pt,->] (c) -- ++(0,-2) node[below] {$\vec{a}_0$};

%%%%%%%%%%%%%%%%%%%%%%%%%%%%%%%%%%%%%%
    \begin{scope}[opacity=0.4]
    \node[cargo] (I) at (2,0) {} node[above of=I, yshift=-1.3em] {};
    \draw[draw=black!80,decoration={aspect=0.3, segment length=1mm, amplitude=2mm,coil},decorate] (0,0) -- node[above, black, pos=0.5, yshift=1em] {} (I); 
    \draw[] (4,-1.1) -- ++(0,0.2) node[below, yshift=-5pt, black] {$l$};
        
    \end{scope}
   \draw[interface,fill=white!40, draw=black] (0,0.1) rectangle ++(5,-0.2);

    \node[cargo] (b) at (4,0) {} node[above of=b, yshift=-1.3em] {$m$};
    \draw[draw=black!80,decoration={aspect=0.3, segment length=1.5mm, amplitude=2mm,coil},decorate] (0,0) -- node[above, black, pos=0.5, yshift=1em] {$k$} (b); 
    \draw[] (4,-1.1) -- ++(0,0.2) node[below, yshift=-5pt, black] {$l$};
    \draw[] (2,-1.1) -- ++(0,0.2) node[below, yshift=-5pt, black] {$l_0$};


    \draw[axis,->] (0,-1) -- ++(5,0) node[right, black] {$x$};
    \draw[fill=black] (0,0) circle (0.1);

    \draw[force,->, very thick] (b.center) -- ++(-1,0) node[above, yshift=0.5em, black] {$\vec{f}_e$};

    \node[left] at (-0.4,-0.4) {$\bigotimes\vec\omega$}; 


\end{tikzpicture}
\vspace{-1em}
\end{figure}
Запишем второй закон Ньютона, и найдем расстояние от оси вращения до муфточки $l$: 
\begin{equation*}
	\begin{aligned}[c]
        m\vec{a}&=\vec{f}_e\\
        \text{x: } ma_n&=-k(l-l_0)\\
        m\omega^2l&=k(l-l_0)\\
        l&=\frac{l_0}{1-\frac{m\omega^2}{k}}\\
        % \ddot{x}'&+\frac{k}{m}x'=0
	\end{aligned}
        % \qquad\Rightarrow\qquad
    % \begin{aligned}[c]
    % \end{aligned}
\end{equation*}
Запишем работу через закон сохранения механической энергии:
\begin{gather*}
        A=W_1-W_0\\
        A=\frac{mv^2}{2}+\frac{k(l-l_0)^2}{2}-0=\\
        =\frac{m\omega^2l^2}{2}+\frac{k(l-l_0)^2}{2}
\end{gather*}
Сделаем замену $\phi=\frac{m\omega^2}{k}$.
\begin{gather*}
\frac{k(l-l_0)^2}{2}=\frac{k}{2}\left[
    \frac{l_0}{1-\frac{m\omega^2}{k}}-l_0
\right]^2=\frac{kl_0^2}{2}\left[\frac{1}{1-\phi}-\frac{1-\phi}{1-\phi}\right]=\frac{kl_0^2\phi^2}{2({1-\phi})^2}\\
\frac{m\omega^2l^2}{2}=\frac{k\frac{m\omega^2}{k}l^2}{2}=
\frac{k\phi l_0^2}{2(1-\phi)^2}\\
    A=\frac{k\phi{}l_0^2}{2(1-\phi)^2}+
    \frac{kl_0^2\phi^2}{2(1-\phi)^2}=\\
    =\frac{kl_0^2\phi}{2(1-\phi)^2}\left[1+\phi\right],
\end{gather*}
где $\phi=\frac{m\omega^2}{k}$
% Начальные условия $x(t=0)=0$, \ $v'(t=0)=-v_0$:
% \begin{equation*}
% \left\{\begin{aligned}
%     0 &= A\cdot\cos{\phi_0}\\
%     v_0 &= \omega{A}\sin{\phi_0}
%     \end{aligned}\right. \qquad\Rightarrow\qquad\left\{
%     \begin{aligned}[c]
%         \phi_0=\frac\pi2\\
%         A=\frac{v_0}{\omega}
%     \end{aligned}\right.
% \end{equation*}
% Условие остановки $v=0$.
% Перейдем в лабораторную ИСО. Согласно преобразованием Галлилея, $v=v_0+v'$. Тогда условие остановки будет $v(t_{stop})=0$ и отсюда $v'(t_{stop})=-v_0$, а чтобы брусок после остановки не возобновил движение, нулевым должно быть также и его ускорение:
% \begin{gather*}
%     \left\{\begin{aligned}
%         {v_0}\cos{\omega{t_{stop}}}=-v_0\\
%         \omega{v_0}\sin{\omega{t_{stop}}}=0
%     \end{aligned}\right. \qquad\Rightarrow\qquad\left\{
%     \begin{aligned}[c]
%         \cos\omega{t_{stop}}=1\\
%         \sin\omega{t_{stop}}=0\\
%     \end{aligned}\right.\qquad\Rightarrow\qquad
%     t_{stop}=\frac{2\pi}\omega=2\pi\sqrt\frac{m}{k}
% \end{gather*}
% Тогда решением этих уравнений будет

\end{document}