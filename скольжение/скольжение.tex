\documentclass[a5paper,10pt]{article}
\def\source{/home/lab/tex/templates}

\usepackage{cmap}
\usepackage[T2A]{fontenc}
\usepackage[utf8x]{inputenc}
\usepackage[english, russian]{babel}

\usepackage
	{
		amssymb,
		% misccorr,
		amsfonts,
		amsmath,
		amsthm,
		wrapfig,
		makecell,
		multirow,
		indentfirst,
		ulem,
		graphicx,
		geometry,
		fancyhdr,
		subcaption,
		float,
		tikz,
		csvsimple,
		color,
	}  

\usepackage[outline]{contour}
\usepackage[mode=buildnew]{standalone}


\geometry
	{
		left=1cm,
		right=1cm,
		top=2cm,
		bottom=1cm,
		bindingoffset=0cm,
	}

\linespread{1.3} 
\frenchspacing 


\usetikzlibrary{scopes}
\usetikzlibrary
	{
		decorations.pathreplacing,
		decorations.pathmorphing,
		patterns,
		calc,
		scopes,
		arrows,
		through,
		shapes.misc,
		arrows.meta,
	}


\tikzset{
	force/.style=	{
		>=latex,
		draw=blue,
		fill=blue,
				 	}, 
	%				 	
	axis/.style=	{
		densely dashed,
		gray,
		font=\small,
					},
	%
	acceleration/.style={
		>=open triangle 60,
		draw=blue,
		fill=blue,
					},
	%
	inforce/.style=	{
		force,
		double equal sign distance=2pt,
					},
	%
	interface/.style={
		pattern = north east lines, 
		draw    = none, 
		pattern color=gray!60,
					},
	cross/.style=	{
		cross out, 
		draw=black, 
		minimum size=2*(#1-\pgflinewidth), 
		inner sep=0pt, outer sep=0pt,
					},
	%
	cargo/.style=	{
		rectangle, 
		fill=black!70, 
		inner sep=2.5mm,
					},
	%
	}

\pagestyle{fancy} %применим колонтитул
\fancyhead{} %очистим хидер на всякий случай
\fancyhead[R]{Сарафанов Ф.Г.} %номер страницы слева сверху на четных и справа на нечетных
\fancyhead[C]{Механика}
% \fancyhead[L]{Задача под запись - <<АУУ-2>>} 
\fancyfoot{} %футер будет пустой

\newcommand{\irodov}[1]{\fancyhead[L]{Иродов -- №#1}}
\newcommand{\yakovlev}[1]{\fancyhead[L]{Яковлев -- №#1}}
\newcommand{\wrote}[1]{\fancyhead[L]{Под запись -- <<#1>>}}

\newenvironment{tikzpict}
    {
	    \begin{figure}[htbp]
		\centering
		\begin{tikzpicture}
    }
    { 
		\end{tikzpicture}
		% \caption{caption}
		% \label{fig:label}
		\end{figure}
    }

\newcommand{\vbLabel}[3]{\draw ($(#1,#2)+(0,5pt)$) -- ($(#1,#2)-(0,5pt)$) node[below]{#3}}
\newcommand{\vaLabel}[3]{\draw ($(#1,#2)+(0,5pt)$) node[above]{#3} -- ($(#1,#2)-(0,5pt)$) }

\newcommand{\hrLabel}[3]{\draw ($(#1,#2)+(5pt,0)$) -- ($(#1,#2)-(5pt,0)$) node[right, xshift=1em]{#3}}
\newcommand{\hlLabel}[3]{\draw ($(#1,#2)+(5pt,0)$) node[left, xshift=-1em]{#3} -- ($(#1,#2)-(5pt,0)$) }

% Draw line annotation
% Input:
%   #1 Line offset (optional)
%   #2 Line angle
%   #3 Line length
%   #5 Line label
% Example:
%   \lineann[1]{30}{2}{$L_1$}
\newcommand{\lineann}[4][0.5]{%
    \begin{scope}[rotate=#2, blue,inner sep=2pt, ]
        \draw[dashed, blue!40] (0,0) -- +(0,#1)
            node [coordinate, near end] (a) {};
        \draw[dashed, blue!40] (#3,0) -- +(0,#1)
            node [coordinate, near end] (b) {};
        \draw[|<->|] (a) -- node[fill=white, scale=0.8] {#4} (b);
    \end{scope}
}


\wrote{Скольжение}

\begin{document}

\begin{tikzpict}
	\draw[interface] (-2,0) rectangle ++ (3,-0.5);
	\draw[interface, pattern color=magenta, pattern=north west lines] (1,0) rectangle ++ (6,-0.5);
	\draw[] (-2,0) -- ++(5,0);

	% \draw[interface] (3,-0.5) rectangle ++ (0.5,5);
	% \draw[] (3,0) -- ++(0,4.5);

	\draw[] (1,2) coordinate (o) circle (2) node[magenta, scale=1.5] {$\bigotimes$} node[above, yshift=0.8em] {$+z$};

	% \draw (o) pic[<-, magenta,]{carc=100:180:1cm};

	\coordinate (1) at (3,2);
	\coordinate (2) at (1,0);

	\draw[force, thick, ->] (2) -- ++ (-1.5,0) node[above] {$\vec{f}\,\,\,$};

	\draw[force, thick, ->] (o) -- ++ (0,-0.65) node[left] {$m\vec{g}$};
	\draw[force, thick, ->] (o) -- ++ (3,0) node[right] {$\vec{v}_0$};
	\draw[force, thick, ->] (2) -- ++ (0,0.65) node[left] {$\vec{N}$};


	\draw[axis,->] (2,0) -- ++ (6,0) node[right] {$+x$};
	% \draw[axis,->] (3,0) -- ++ (0,5) node[right] {$+y$};





	\draw[fill=white, draw=none] (1,2) coordinate (o) circle (2.7mm);
	\draw[] (1,2) coordinate (o) circle (2) node[magenta, scale=1.5] {$\bigotimes$};	

	\draw[] (-2,2) node[magenta, scale=1.5] {$\bigotimes$} node[right, xshift=0.5em] {$\vec{M}_f$};	

\end{tikzpict}
При скольжении сила трения скольжения постоянна:
\begin{equation}
	f_x=-kN=-kmg
\end{equation}
Запишем второй закон Ньютона в проекции на ось $x$:
\begin{equation}
	ma_x=-kmg
\end{equation}
\begin{equation}
	\int\limits_{v_0}^{v(t)}dv=-\int\limits_{0}^{t} kg\, dt
\end{equation}
\begin{equation}
	v(t)=v_0-kg\cdot t
\end{equation}
Запишем уравнение моментов в проекции на ось $z$:
\begin{equation}
	I\gamma_z=kmg\cdot R
\end{equation}
\begin{equation}
	\int\limits_{0}^{\omega_z(t)}dv=\int\limits_{0}^{t} \frac{kmgR}{I}\, dt
\end{equation}
\begin{equation}
	\omega_z(t)=\frac{kmgR}{I}\cdot t=\frac{2kg}{R}\cdot t
\end{equation}
Из условия непроскальзывания найдем момент окончания такового:
\begin{equation}

	v(t^*)=\omega(t^*)\cdot R
\end{equation}
\begin{equation}
	v_0-kg\cdot t^*={2kg}\cdot t^*
\end{equation}
\begin{equation}
	t^*=\frac{v_0}{3kg}
\end{equation}
\begin{equation}
	v=v(t^*)=\frac{2}{3}v_0
\end{equation}
Сила трения - диссипативная, и количество теплоты, выделяющееся при её действии, равно по модулю её работе. 

Тогда
\begin{equation}
	Q=|A_f|=|\Delta W_k|=\frac{mv_0^2}{2}-\left(\frac{mv^2}{2}+\frac{I\omega^2}{2}\right)
\end{equation}
\begin{gather}
	Q=\frac{mv_0^2}{2}-\left(\frac{mv^2}{2}+\frac{Iv^2}{2R^2}\right)=\\=
	\frac{mv_0^2}{2}-\frac{mv_0^2}{2}\cdot\frac{4}{9}-\frac{mR^2}{2}\frac{v_0^2}{2R^2}\frac{4}{9}=\frac{mv_0^2}{6}
\end{gather}
А отношение теплоты к начальной энергии
\begin{equation}
	\eta=\frac{Q}{W_0}=\frac{mv_0^2}{6}\cdot\frac{2}{mv_0^2}=\frac13
\end{equation}
\end{document}