\documentclass[a5paper,10pt]{article}\usepackage[usenames,dvipsnames]{color}

\usepackage{cmap,graphicx,etoolbox,misccorr,indentfirst,makecell,multirow,ulem,geometry,amssymb,amsfonts,amsmath,amsthm,titlesec,float,fancyhdr,wrapfig,tikz}

\usepackage[T2A]{fontenc}\usepackage[utf8x]{inputenc}\usepackage[english, russian]{babel}\usetikzlibrary{decorations.pathreplacing,decorations.pathmorphing,patterns,calc,scopes,arrows,through, shapes.misc}\graphicspath{{img/}}\linespread{1.3}\frenchspacing\geometry{left=1cm, right=1cm, top=2cm, bottom=1cm, bindingoffset=0cm}\pagestyle{fancy}\fancyhead{}\fancyhead[R]{Сарафанов Ф.Г.}\fancyhead[C]{Механика}
\fancyhead[L]{Иродов -- №1.380}
\fancyfoot{}

%Команда \beforetext для текста слева от формулы
\makeatletter \newif\if@gather@prefix \preto\place@tag@gather{\if@gather@prefix\iftagsleft@ \kern-\gdisplaywidth@ \rlap{\gather@prefix} \kern\gdisplaywidth@ \fi\fi } \appto\place@tag@gather{\if@gather@prefix\iftagsleft@\else \kern-\displaywidth \rlap{\gather@prefix} \kern\displaywidth \fi\fi \global\@gather@prefixfalse } \preto\place@tag{\if@gather@prefix\iftagsleft@ \kern-\gdisplaywidth@ \rlap{\gather@prefix} \kern\displaywidth@ \fi\fi } \appto\place@tag{\if@gather@prefix\iftagsleft@\else \kern-\displaywidth \rlap{\gather@prefix} \kern\displaywidth \fi\fi \global\@gather@prefixfalse } \newcommand*{\beforetext}[1]{\ifmeasuring@\else \gdef\gather@prefix{#1} \global\@gather@prefixtrue \fi } \makeatother 
\tikzset{force/.style={>=latex,draw=blue,fill=blue}, axis/.style={densely dashed,gray,font=\small}, acceleration/.style={>=open triangle 60,draw=blue,fill=blue}, inforce/.style={force,double equal sign distance=2pt}, interface/.style={pattern = north east lines, draw    = none, pattern color=gray!60, }, cross/.style={cross out, draw=black, minimum size=2*(#1-\pgflinewidth), inner sep=0pt, outer sep=0pt},    cargo/.style={rectangle, fill=black!70, inner sep=2.5mm, }}

\begin{document}
\begin{figure}[H]
    \centering
\begin{tikzpicture}
    \draw[axis,->] (0,0) -- (5,0) node[right, black] {$x$};
    \draw[axis,->] (0,0) -- (0,5) node[above, black] {$ct$};
    \draw[step=.5,axis,xshift=0cm,yshift=0cm] (0,0) grid (4,4);

    \def\step{0.07}
    \draw[white] (-2.4,0);
   \foreach \i in {0,1,...,8} {
        \draw [very thin] (\i/2,\step) -- (\i/2,-\step)  node [below] {$\i$};
    }
    \foreach \i in {0,1,...,8} {
        \draw [very thin] (\step,\i/2) -- (-\step,\i/2)  node [left] {$\i$};
    }   

    \draw[fill=black] (1,0.5) circle (2pt) node[above,xshift=7pt] {$A$};       
    \draw[fill=black] (2.5,3) circle (2pt) node[above,xshift=7pt] {$B$};       
    \draw[fill=black] (3.5,2) circle (2pt) node[above,xshift=7pt] {$C$};       
          % \draw[axis] (0,4) -- ++(4,0) node[right, black] {$\rho\omega^2
        % \frac{l^2}{8} $};

          % \draw[] (0,0.2) -- ++(0,-0.4) node[below, black] {$0$};
          % \draw[] (3,0.2) -- ++(0,-0.4) node[below, black] {$\frac{l}{2}$};
          % \draw[] (-3,0.2) -- ++(0,-0.4) node[below, black] {$\frac{l}{2}$};

          % \draw[scale=1,domain=-2:2,smooth,variable=\x, magenta]  plot ({1.5*\x},{1*(4-\x*\x)});


\end{tikzpicture}
\vspace{-1em}
\end{figure}
Воспользуемся инвариантностью интервала, учитывая, что разность координат в штрихованной системе равна нулю:
\begin{gather*}
    s^2=c^2(t_B-t_A)^2-(x_B-x_A)^2=c^2(t'_B-t'_A)^2\\
    (6-1)^2-(5-2)^2=c^2\Delta{t'}^2\\
    \Delta{t'}=\frac{\sqrt{25-9}}{c}\\
    \Delta{t'}=
    1.33\cdot10^{-8}\approx13\cdot10^{-9}\text{ с} = 13\text{ нс }
\end{gather*}
Аналогично. только теперь в штрихованной системе равна нулю разность времен:
\begin{gather*} 
    s^2=c^2(t_C-t_A)^2-(x_C-x_A)^2=-(x'_C-x'_A)^2\\
    (4-1)^2-(7-2)^2=-\Delta{x'}^2\\
    \Delta{x'}={\sqrt{25-9}}=4\text{ м }\\
\end{gather*}

\end{document}