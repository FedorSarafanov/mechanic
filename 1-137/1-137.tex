\documentclass[a5paper,10pt]{article}\usepackage[usenames,dvipsnames]{color}
% \usepackage{poisson}
\usepackage{cmap,graphicx,etoolbox,misccorr,indentfirst,makecell,multirow,ulem,geometry,amssymb,amsfonts,amsmath,amsthm,titlesec,float,fancyhdr,wrapfig,tikz}

\usepackage[T2A]{fontenc}\usepackage[utf8x]{inputenc}\usepackage[english, russian]{babel}\usetikzlibrary{decorations.pathreplacing,decorations.pathmorphing,patterns,calc,scopes,arrows,through, shapes.misc}\graphicspath{{img/}}\linespread{1.3}\frenchspacing\geometry{left=1cm, right=1cm, top=2cm, bottom=1cm, bindingoffset=0cm}\pagestyle{fancy}\fancyhead{}\fancyhead[R]{Сарафанов Ф.Г.}\fancyhead[C]{Механика}
\fancyhead[L]{Иродов -- №1.137}
\fancyfoot{}

%Команда \beforetext для текста слева от формулы
\makeatletter \newif\if@gather@prefix \preto\place@tag@gather{\if@gather@prefix\iftagsleft@ \kern-\gdisplaywidth@ \rlap{\gather@prefix} \kern\gdisplaywidth@ \fi\fi } \appto\place@tag@gather{\if@gather@prefix\iftagsleft@\else \kern-\displaywidth \rlap{\gather@prefix} \kern\displaywidth \fi\fi \global\@gather@prefixfalse } \preto\place@tag{\if@gather@prefix\iftagsleft@ \kern-\gdisplaywidth@ \rlap{\gather@prefix} \kern\displaywidth@ \fi\fi } \appto\place@tag{\if@gather@prefix\iftagsleft@\else \kern-\displaywidth \rlap{\gather@prefix} \kern\displaywidth \fi\fi \global\@gather@prefixfalse } \newcommand*{\beforetext}[1]{\ifmeasuring@\else \gdef\gather@prefix{#1} \global\@gather@prefixtrue \fi } \makeatother 
\tikzset{force/.style={>=latex,draw=blue,fill=blue}, axis/.style={densely dashed,gray,font=\small}, acceleration/.style={>=open triangle 60,draw=blue,fill=blue}, inforce/.style={force,double equal sign distance=2pt}, interface/.style={pattern = north east lines, draw    = none, pattern color=gray!60, }, cross/.style={cross out, draw=black, minimum size=2*(#1-\pgflinewidth), inner sep=0pt, outer sep=0pt},    cargo/.style={rectangle, fill=black!70, inner sep=2.5mm, },   pics/carc/.style args={#1:#2:#3}{code={\draw[pic actions] (#1:#3) arc(#1:#2:#3); } }} 

\pgfdeclarepatternformonly{my crosshatch dots}{\pgfqpoint{-1pt}{-1pt}}{\pgfqpoint{5pt}{5pt}}{\pgfqpoint{1.5pt}{1.5pt}}%
{
    \pgfpathcircle{\pgfqpoint{0pt}{0pt}}{.2pt}
    \pgfpathcircle{\pgfqpoint{3pt}{3pt}}{.2pt}
    \pgfusepath{fill}
}

\begin{document}
% \edef\mylist{\poissonpointslist{8}{4}{0.1}{20}}
\begin{figure}[H]
    \centering
\begin{tikzpicture}

    \draw[interface] (0,-0.25) rectangle (5,0);
    \draw[thick] (0,0) -- (5,0);

    \draw (1,.15) circle (0.15) circle(1pt) -- ++(.35,0) circle (0.15) circle(1pt);
    \draw (3,.15) circle (0.15) circle(1pt) -- ++(.35,0) circle (0.15) circle(1pt);

    \draw (0.5,.3) rectangle ++(3.35,.30);

    \draw[force,->] (3.85,.45) -- ++(1,0) node[right] {$\vec{F}$};
    
    % \draw[interface, pattern=my crosshatch dots] (3,1.5) rectangle ++(1.5,0.5);
    \draw[thick] (3,1.5) -- (3.70,1.5) (3.80,1.5) -- (4.5,1.5);

    \draw[draw=none, pattern=my crosshatch dots] (4.5,1.5) arc (0:180:0.75) -- cycle;

    \draw[draw=none, pattern=my crosshatch dots] (3.7,1.5) rectangle ++(0.1,-0.9);


    % \draw[fill=white, pattern=dots, draw=none] (3.35,1.5) -- (3.35,2) -- (4.15,2) -- (4.15,1.5) -- ++ (-0.25,-0.25) -- ++(0,-0.65) -- ++(-0.25,0)-- (3.65,1.25) --cycle;% -- ++ (-0,0.25) -- cycle;
    \draw [axis,->] (0,0) -- (6,0) node[right] {$+x$};
    

\end{tikzpicture}
% \vspace{-1em}
\end{figure}

Запишем теорему о изменении импульса СМТ в интегральной форме:
\begin{equation}
    \vec{p}_c(t)-\vec{p}_c(t_0)=\int_{t_0}^t \vec{F}^\text{внеш}_c dt
\end{equation}

В проеции на $x$, с учетом того, что ${p}_{cx}(0)=0$:
\begin{equation}
    {p}_{cx}(t)=\int_{0}^t {F}^\text{внеш}_{cx} dt
\end{equation}

Так как масса меняется по закону 
\begin{equation}
    m=m_0+\mu t,
\end{equation}
а сила постоянна, то
\begin{equation}
    (m_0+\mu t)\cdot v_x(t)=F\cdot t
\end{equation}
Отсюда
\begin{equation}
    v_x(t)=\frac{Ft}{m_0+\mu t} 
\end{equation}
Взяв производную по $t$, найдем ускорение тележки:
\begin{equation}
    v'_x(t)=\frac{Fm_0+F\mu t - \mu t F}{(m_0+\mu t)^2}
\end{equation}
Перепишем окончательный ответ для ускорения:
\begin{equation}
    a_x=\frac{F}{m_0(1+t\frac{\mu}{m_0})^2}
\end{equation}
% Рассмотрим вариант движения по окружности с постоянной скоростью. Запишем уравнение Мещерского:
% \begin{equation}
%     m \frac{d\vec{v}}{dt}=-\frac{dm_1}{dt}{\vec{u}_1}+\frac{dm_2}{dt}{\vec{u}_2}
% \end{equation}
% Учитывая, что $dm_1=-dm$ и к ракете ничего не прилипает, перепишем уравнение Мещерского:
% \begin{equation}
%     m \frac{d\vec{v}}{dt}=\frac{dm}{dt}{\vec{u}_1}
% \end{equation}
% Спроецируем на ось $y$:
% \begin{equation}
%     m \frac{d{v}_y}{dt}=-\frac{dm}{dt}{{u}_1}    
% \end{equation}
% Учитывая, что $dv_y=v_0\cdot d\alpha$, решим дифференциальное уравнение:
% \begin{equation}
%     -\frac{1}{u_1}\cdot{d{v}_y}=\frac{dm}{m}
% \end{equation}
% \begin{equation}
%     % -\frac{m}{u_1}\cdot{d{v}_y}=
%     -\frac{v_0}{u_1}\cdot\int_0^\pi{d\alpha}=\int_{m_0}^{m^*}\frac{dm}{m}
% \end{equation}
% \begin{equation}
%     -\frac{\pi v_0}{u_1}=\ln\frac{m^*}{m_0}
% \end{equation}
% Откуда, потенцируя, найдем массу после разворота:
% \begin{equation}
%     m^*=m_0\cdot e^{-\frac{\pi v_0}{u_1}}
% \end{equation}
% Значит, расход топлива на разворот ракеты был:
% \begin{equation}
%     \label{eq:dm1}
%     \Delta m = m^*-m_0 = m_0(1-e^{-\frac{\pi v_0}{u_1}})
% \end{equation}


\end{document}