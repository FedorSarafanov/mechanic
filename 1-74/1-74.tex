\documentclass[a5paper,10pt]{article}\usepackage[usenames,dvipsnames]{color}\usepackage{extsizes,cmap,graphicx,misccorr,indentfirst,makecell,multirow,ulem,geometry,amssymb,amsfonts,amsmath,amsthm,titlesec,float,fancyhdr,wrapfig,tikz}\usepackage[T2A]{fontenc}\usepackage[utf8x]{inputenc}\usepackage[english, russian]{babel}\usetikzlibrary{decorations.pathreplacing,decorations.pathmorphing,patterns,calc,scopes,arrows,through,positioning,shapes.misc}\graphicspath{{img/}}\linespread{1.3}\frenchspacing\geometry{left=1cm, right=1cm, top=2cm, bottom=1cm, bindingoffset=0cm}\pagestyle{fancy}\fancyhead{}\fancyhead[R]{Сарафанов Ф.Г.} 
\fancyhead[C]{Механика}
\fancyhead[L]{№1.74 -- Иродов} 
\fancyfoot{}
\renewcommand{\labelenumii}{\theenumii)}
\tikzset{
	force/.style={>=latex,draw=blue,fill=blue,>=triangle 45},
    axis/.style={densely dashed,black!60,font=\small},
    interface1/.style={draw=gray!60,.
        postaction={draw=gray!60,decorate,decoration={border,angle=-135,
        amplitude=0.3cm,segment length=m0m}}},
    interface/.style={
        pattern = north east lines,
        draw    = none,
        pattern color=gray!60,          
    },
    plank/.style={
        fill=black!60, 
        draw=black,
        minimum width=3cm,
        inner ysep=0.1cm,
        outer sep=0pt,
        yshift=0.75cm,
        pattern = north east lines,
        pattern color=gray!60, 
    },
    cargo/.style={
        rectangle,
        fill=black!70,              
        inner sep=2.5mm,
    }	
}
\begin{document}

\begin{wrapfigure}[12]{l}{0.4\textwidth}
    % \centering
\begin{tikzpicture}
    \def\rr{0.4}
    \def\D{4.3}

	\draw[interface] (0,0) rectangle (3,0);
    \draw[interface] (0,-0.25) rectangle (3,0);
	\draw[thick] (0,0) -- (3,0);


    \draw[axis] (0,\D*\rr) --(3,\D*\rr);
    \draw[axis] (3,\D*\rr) ++ (\D*\rr,-\D*\rr) arc (0:90:\D*\rr);
    \draw[axis, ->] ($(3,0)+(\D*\rr,0)$) -- ++(0,-7) node[below] {$x$};
    
    %Верхний блок (с)
    \draw[fill=white] (3,0) coordinate (0) circle (\rr);
    \path (0) -- +(0,\rr) coordinate (Cl) + (\rr,0) coordinate (Cr);
    \draw[fill=black] (0) circle (2pt);
    \draw[black!70] (Cr) -- ++ (0,-2) coordinate (B);
    \draw[black!70] (Cl) -- ++ (-2,0) coordinate (m0);

    %Груз m_0
    \draw[fill=black!10] ($(m0)+(0,\rr)$) rectangle node[] {$m_0$} ($(m0)-(1,\rr)$) ;
    

    \draw (B) circle (\rr);
    \path (B) -- +(-\rr,0) coordinate (Bl) + (\rr,0) coordinate (Br);
    \draw[fill=black] (B) circle (2pt);

    \draw[black!70] (Br) -- ++ (0,-3.5) coordinate (m2);
    \draw[black!70] (Bl) -- ++ (0,-2) coordinate (m1);

    \draw[fill=black!10] ($(m1)+(0.7*\rr,0)$) rectangle node[] {$m_1$} ($(m1)-(0.7*\rr,1)$) ;

    \draw[fill=black!10] ($(m2)+(0.7*\rr,0)$) rectangle node[] {$m_2$} ($(m2)-(0.7*\rr,1)$) ;
 
    \coordinate (m1b) at ($(m1)-(0,1)$);
    \coordinate (m2b) at ($(m2)-(0,1)$);

    \draw[force,->] (m0)-- node[above] {$\vec{T}_1$} ++(0.7,0);

    \draw[force,->] (Cl)-- node[above] {$\vec{T}_2$} ++(-0.7,0);
    \draw[force,->] (Cr)-- node[right] {$\vec{T}_3$} ++(0,-0.7);

    \draw[force,->] (B)-- node[right, anchor=south west] {$\vec{T}_4$} ++(0,0.7);

    \draw[force,->] (Bl)-- node[left] {$\vec{T}_5$} ++(0,-0.7);
    \draw[force,->] (Br)-- node[right] {$\vec{T}_6$} ++(0,-0.7);

    \draw[force,->] (m1)-- node[left] {$\vec{T}_7$} ++(0,0.7);
    \draw[force,->] (m2)-- node[right] {$\vec{T}_8$} ++(0,0.7);

    \draw[force,->] (m1b)-- node[left] {$m_1\vec{g}$} ++(0,-0.7);
    \draw[force,->] (m2b)-- node[right] {$m_2\vec{g}$} ++(0,-0.7);

\end{tikzpicture}
\end{wrapfigure}

Следует из невесомости нитей:
\begin{gather*}
    T_1=T_2\\
    T_3=T_4\\
    T_5=T_6=T_7=T_8(=T)
\end{gather*}
II-й закон для невесомых блоков:
\begin{gather*}
    0=\vec{T}_3+\vec{T}_2\\
    0=\vec{T}_4+\vec{T}_5+\vec{T}_6
\end{gather*}
Отсюда 
\begin{gather*}
    T_3=T_2\\
    -\vec{T}_4=\vec{T}_5+\vec{T}_6
\end{gather*}
\\
II-й закон для грузов:
\begin{gather*}
    % m_0\vec{a}_0=\vec{T}_1=-\vec{T}_2=\vec{T}_3=-\vec{T}_4=\vec{T}_5+\vec{T}_6=-\vec{T}_7+(-\vec{T}_8)\\
    % m_1\vec{a_1}=m_1\vec{g}+\vec{T}_7\\
    m_0a_{0x}=2T\\
    m_1a'_{1x}=m_1g-T\\
    % m_2\vec{a_2}=m_2\vec{g}+\vec{T}_8\\
    m_2a'_{2x}=m_2g-T
    % m_0a_0=m_1(g-a'_1)+m_2(g-a'_2)=m_1(g-a'_1)+m_2(g+a'_1)
\end{gather*}
Еще
\begin{gather*}
    a'_{2x}=-a'_{1x}
\end{gather*}
Тогда
\begin{gather*}
    % m_0\vec{a}_0=\vec{T}_1=-\vec{T}_2=\vec{T}_3=-\vec{T}_4=\vec{T}_5+\vec{T}_6=-\vec{T}_7+(-\vec{T}_8)\\
    % m_1\vec{a_1}=m_1\vec{g}+\vec{T}_7\\
    g-T/m_1=T/m_2-g\\
    2g=T/m_2+T/m_1=T\frac{m_1+m_2}{m_1m_2}\\
    T=2g\frac{m_1m_2}{m_1+m_2}\\
    m_0a_{0x}=2T\\
    a_{0x}=4g\frac{m_1m_2}{m_0(m_1+m_2)}\\
    a'_{1x}=g-2g\frac{m_1m_2}{m_1(m_1+m_2)}\\
    % =g\frac{m_1^2-4m_1m_2}{m_1(m_1-m_2)}
    a_{1x}=\frac{2T}{m_0}+g-\frac{T}{m_1}=\frac{2Tm_1+gm_0m_1-Tm_0}{m_0m_1}
    % 4g\frac{m_1m_2}{m_0(m_1-m_2)}+g-2g\frac{m_1m_2}{m_1(m_1-m_2)}
\end{gather*}
% \begin{align*}
% \end{align*}

% \textbf{Случай $I$.} Рассмотрим движение с торможением без поворота:
% \begin{equation*}
%     \begin{aligned}[c]
% 		m\vec{a}=\vec{f}_R\\
% 		\text{x: }ma=-mg\mu\\
% 		\int_{v_0}^{v(t)}dv=\int_0^t -g\mu dt\\
% 		v(t)=v_0-\mu{gt}\\
% 		\int_{0}^{x}dx=\int_0^t [v_0-\mu{gt}] dt\\
% 		x(t)=v_0{t}-\mu{g}\frac{t^2}{2}
%     \end{aligned}
%         \qquad\qquad
%     \begin{aligned}[c]
%     \text{Условие остановки $v=0$ при $t=t^*$:}\\
%     v_\text{ост}=0=v_0-g\mu{}t^*\\
%     t^*=\frac{v_0}{g\mu}\\
%     \text{Тогда пройденное до остановки $R$:}\\
%     R=v_0\cdot{t^*}-\mu{g}\frac{{t^*}^2}{2}\\
%     R=\frac{v_0^2}{2g\mu}
%     \end{aligned}
% \end{equation*}


\end{document}

