\documentclass[a5paper,10pt]{article}
 
\usepackage{extsizes}
\usepackage{cmap}
\usepackage[T2A]{fontenc}
\usepackage[utf8x]{inputenc}
\usepackage[english, russian]{babel}

\usepackage{misccorr}

%%%%%%%%%%%%%%%%%%%%%%%%%%%%%%%%%%%%%%%%%%%%%%%%%%%%%%%%%%%%%%%%%%%%%%%%%%%%%%%%%%  
\usepackage{graphicx} % для вставки картинок
\graphicspath{{img/}}
\usepackage{amssymb,amsfonts,amsmath,amsthm} % математические дополнения от АМС

% \usepackage{fontspec}
% \usepackage{unicode-math}

\usepackage{indentfirst} % отделять первую строку раздела абзацным отступом тоже
\usepackage[usenames,dvipsnames]{color} % названия цветов
\usepackage{makecell}
\usepackage{multirow} % улучшенное форматирование таблиц
\usepackage{ulem} % подчеркивания
\linespread{1.3} % полуторный интервал
% \renewcommand{\rmdefault}{ftm} % Times New Roman (не работает)
\frenchspacing
\usepackage{geometry}
\geometry{left=1cm,right=1cm,top=2cm,bottom=1cm,bindingoffset=0cm}
\usepackage{titlesec}
\usepackage{float}
% \definecolor{black}{rgb}{0,0,0}
% \usepackage[colorlinks, unicode, pagecolor=black]{hyperref}
% \usepackage[unicode]{hyperref} %ссылки
\usepackage{fancyhdr} %загрузим пакет
\pagestyle{fancy} %применим колонтитул
\fancyhead{} %очистим хидер на всякий случай
\fancyhead[LE,RO]{Сарафанов Ф.Г.} %номер страницы слева сверху на четных и справа на нечетных
\fancyhead[CO, CE]{Механика}
\fancyhead[LO,RE]{№1.75 -- Иродов И.Е.} 
\fancyfoot{} %футер будет пустой
% \fancyfoot[CO,CE]{\thepage}
\renewcommand{\labelenumii}{\theenumii)}


\usepackage{tikz}
\usetikzlibrary{scopes}
\usetikzlibrary{%
     decorations.pathreplacing,%
     decorations.pathmorphing,%
    patterns,%
    calc,%
    scopes,%
    arrows,%
    % arrows.spaced,%
}

\begin{document}

\begin{figure}[H]
    \centering
\begin{tikzpicture}[
    force/.style={>=latex,draw=blue,fill=blue},
    % axis/.style={densely dashed,gray,font=\small},
    axis/.style={densely dashed,black!60,font=\small},
    M/.style={rectangle,draw,fill=lightgray,minimum size=0.5cm,thin},
    m2/.style={draw=black!30, rectangle,draw,thin, fill=blue!2, minimum width=0.7cm,minimum height=0.7cm},
    m1/.style={draw=black!30, rectangle,draw,thin, fill=blue!2, minimum width=0.7cm,minimum height=0.7cm},
    plane/.style={draw=black!30, very thick, fill=blue!5, line width=1pt},
    % base/.style={draw=black!70, very thick, fill=blue!4, line width=2pt},
    string/.style={draw=black, thick},
    pulley/.style={thick},
    interface/.style={draw=gray!60,
        % The border decoration is a path replacing decorator. 
        % For the interface style we want to draw the original path.
        % The postaction option is therefore used to ensure that the
        % border decoration is drawn *after* the original path.
        postaction={draw=gray!60,decorate,decoration={border,angle=-135,
                    amplitude=0.3cm,segment length=2mm}}},
]
\def\height{1.5cm}
\def\width{3cm}
\def\Ti{\vec{T}_1}
\def\Tii{\vec{T}_2}
\def\FINii{\vec{F}_{in\,2}^\text{пост}}
\def\FINi{\vec{F}_{in\,1}^\text{пост}}
\def\radius{0.35cm}    
\matrix[column sep=1cm, row sep=0cm] {

% \draw[use as bounding box] (0,0) rectangle (1,1);

\begin{scope}%
            [scale=1.2, transform shape]
    \coordinate (base left) at (-\width/2,-\height/2);
    \coordinate (top left) at (-\width/2,\height/2);
    \coordinate (top right) at (\width/2,\height/2);
    \coordinate (base right) at (\width/2,-\height/2);
    \coordinate (top mid) at ($(top left)!0.5!(top right)$);

    \draw[axis,->] (base left)  -- ($(base right)+(0.3,0)$) node[right] {$+y'$};
    \draw[axis,->] (base left)  -- ($(top left)+(0,0.3)$) node[right] {$+x'$};

    \draw[plane]  (base left) -- (base right) --  (top right)  -- (top left) -- cycle;
    \draw[interface]  ($(base left)-(0.3,0)$) -- ($(base right)+(0.3,0)$);

    \path (top mid) node[m1, yshift=0.35cm] (m1) {$m_1$};

    \path ($(base left)!0.5!(top right)$) node[yshift=0cm] (A) {$A$};

    \draw[->,>=open triangle 60] (0,2) -- node[above,pos=0.5] {$\vec{a_0}$} (1,2);

    \draw[pulley] ($(top right)-(0,\radius)$) arc (-90:180:\radius) coordinate (pulley);
    \filldraw[fill=black] (top right) circle(0.05cm);

    \draw[string] (m1.east) -- ++(1.15,0) arc (90:0:\radius)
                  -- ++(0,-1) node[m2] {$m_2$};       
\end{scope} 
&
\begin{scope}[scale=1.2, transform shape]
    
   \node[m1,transform shape] (m1) {};

    {[axis,->]
        \draw (0,-1) -- (0,2) node[right] {$+y'$};
        \draw (m1.south) -- ++(2,0) node[right] {$+x'$};
    }

    % Forces
    {
        \draw[force,->] (m1.south) -- ++(0,1.5) node[above right] {$\vec{N}_1$};
        \draw[force,->](m1.south) -- ++(-1,0) node[below] {$\vec{f}_{R_{1A}}$};
        \draw[force,->] (m1.east) -- ++(1,0) node[above] {$\vec{T}_1$};
        \draw[force,double equal sign distance=2pt,->] (m1.west) -- ++(-1,0) node[above] {$\FINi$};
    }

    \draw[force,->] (m1.center) -- ++(0,-1) node[below] {$m_1\vec{g}$};
 \end{scope}
\\
\node[draw=none,text width=5cm, line width=0mm] at (0,0.5) {
Возьмем НИСО, связанную с блоком $A$. 
\begin{gather}
    \nonumber \text{Нить идеальная. По III з.Н.:}\\
    \Ti=-\Tii\\
    \nonumber \text{Инерциальные силы:}\\
    \FINi=-m_1\vec{a}_0\\
    \FINii=-m_2\vec{a}_0
\end{gather}

};
&
\begin{scope}[scale=1.2, transform shape]

    \node[m2,transform shape] (m2) {};
    % Draw axes and help lines

    {[axis,->]
        \draw (0,-1) -- (0,2) node[right] {$+y'$};
        \draw (m2.south) -- ++(2,0) node[right] {$+x'$};
    }

    % Forces
    {
        \draw[force,->] (m2.center) -- ++(0,1) node[above right] {$\vec{T}_2$};
        \draw[force,->](m2.west) -- ++(0,1) node[left] {$\vec{f}_{R_{2A}}$};
        \draw[force,->](m2.west) -- ++(1.5,0) node[right] {$\vec{N}_2$};
        \draw[force,double equal sign distance=2pt,->] (m2.west) -- ++(-1,0) node[below] {$\FINii$};
        \draw[force,->] (m2.center) -- ++(0,-1) node[below] {$m_2\vec{g}$};
    }
 \end{scope}
 \\
 %
};
\end{tikzpicture}
\vspace{-3em}
\end{figure}
\begin{flalign}
\label{eq:m1x} \text{Блок $m_1$, проекция на x:} && 0=-m_1a_0+T_1-\mu{}gm_1 &&\\
\label{eq:m1y}    \text{y:} && 0=N_1-m_1g \hspace{4.7em}& \\
\label{eq:m2x} \text{Блок $m_2$, проекция на x:} && 0=-m_2a_0+N_2  \hspace{1.2cm} &&\\
\label{eq:m2y}      \text{y:} && 0=-m_2g+T_2-\mu{}m_2a_0 &%
\vspace{-1em}
\end{flalign}
(Условие неподвижности - $a'_{1x}=a'_{2y}=0$). Можем найти $a_0$:
\begin{flalign}
\label{eq:m} \text{Сложим (\ref{eq:m1x}) и (\ref{eq:m2y}):} && 0=-a_0(m_1+\mu{}m_2)-g(\mu{}m_1-m_2) \hspace{1.1em}
\end{flalign}
\begin{gather}
     a_0(m_1+\mu{}m_2)=g(m_2-\mu{}m_1)\\
    a_0=g\frac{m_2-\mu{}m_1}{m_1+\mu{}m_2}   
\end{gather}
Учитывая, что в данной задаче $m_1=m_2$, можем записать ответ:
\begin{gather}
    a_0=g\frac{1-\mu{}}{1+\mu}
\end{gather}

% \begin{gather}
%     0=-m_1a_0+T-\mu{}gm_1\\
%     0=-m_2g-T-\mu{}m_2a_0\\
%     0=-a_0(m_1+\mu{}m_2)-g(\mu{}m_1-m_2)\\
%     a_0(m_1+\mu{}m_2)=g(m_2-\mu{}m_1)\\
%     a_0=g\frac{m_2-\mu{}m_1}{m_1+\mu{}m_2}
% \end{gather}

% Учитывая, что $m_1=m_2$:
% \begin{gather}
%     a_0=g\frac{1-\mu{}}{1+\mu}
% \end{gather}
\end{document}