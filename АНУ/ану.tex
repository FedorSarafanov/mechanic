\documentclass[a5paper,10pt]{article}
 
\usepackage{extsizes}
\usepackage{cmap}
\usepackage[T2A]{fontenc}
\usepackage[utf8x]{inputenc}
\usepackage[english, russian]{babel}

\usepackage{misccorr}

%%%%%%%%%%%%%%%%%%%%%%%%%%%%%%%%%%%%%%%%%%%%%%%%%%%%%%%%%%%%%%%%%%%%%%%%%%%%%%%%%%  
\usepackage{graphicx} % для вставки картинок
\graphicspath{{img/}}
\usepackage{amssymb,amsfonts,amsmath,amsthm} % математические дополнения от АМС

% \usepackage{fontspec}
% \usepackage{unicode-math}

\usepackage{indentfirst} % отделять первую строку раздела абзацным отступом тоже
\usepackage[usenames,dvipsnames]{color} % названия цветов
\usepackage{makecell}
\usepackage{multirow} % улучшенное форматирование таблиц
\usepackage{ulem} % подчеркивания
\linespread{1.3} % полуторный интервал
% \renewcommand{\rmdefault}{ftm} % Times New Roman (не работает)
\frenchspacing
\usepackage{geometry}
\geometry{left=1cm,right=1cm,top=2cm,bottom=1cm,bindingoffset=0cm}
\usepackage{titlesec}
\usepackage{float}
% \definecolor{black}{rgb}{0,0,0}
% \usepackage[colorlinks, unicode, pagecolor=black]{hyperref}
% \usepackage[unicode]{hyperref} %ссылки
\usepackage{fancyhdr} %загрузим пакет
\pagestyle{fancy} %применим колонтитул
\fancyhead{} %очистим хидер на всякий случай
\fancyhead[R]{Сарафанов Ф.Г.} %номер страницы слева сверху на четных и справа на нечетных
\fancyhead[C]{Механика}
\fancyhead[L]{Задача под запись - <<АНУ>>} 
\fancyfoot{} %футер будет пустой
% \fancyfoot[CO,CE]{\thepage}
\renewcommand{\labelenumii}{\theenumii)}
\usepackage[outline]{contour}

\usepackage{tikz}
\usetikzlibrary{scopes}
\usetikzlibrary{%
     decorations.pathreplacing,%
     decorations.pathmorphing,%
    patterns,%
    calc,%
    scopes,%
    arrows,%
    % arrows.spaced,%
}

\begin{document}

\begin{figure}[H]
    \centering
\begin{tikzpicture}[
    force/.style={>=latex,draw=blue,fill=blue},
    % axis/.style={densely dashed,gray,font=\small},
    axis/.style={densely dashed,black!60,font=\small},
    interface/.style={
        pattern = north east lines,
        draw    = none,
        pattern color=gray!60,          
    },
    cargo/.style={
        rectangle,
        fill=magenta!40,
        draw=black!50,
        inner sep=2.5mm,
    },
    spring/.style={
        decoration={
            aspect=0.3, 
            segment length=.8mm, 
            amplitude=2mm,
            coil},
        decorate,
        draw=magenta!25
    },
    interface1/.style={draw=gray!60,
        % The border decoration is a path replacing decorator. 
        % For the interface style we want to draw the original path.
        % The postaction option is therefore used to ensure that the
        % border decoration is drawn *after* the original path.
        postaction={draw=gray!60,decorate,decoration={border,angle=-135,
                    amplitude=0.3cm,segment length=2mm}}},    
    % scale=0.95
]
\def\angle{41}
%%%%%%%%%%%%%%%%%%%%%%%%%%%%%%%%%%%%%%
    \draw[force,->] (-3,0) -- ++(1,0) node[right]{$\vec{v}_0$};
    \contourlength{1mm};

    % \draw[thick] (0,0) --  ++(0,-3);
    \draw[thick,->] (0,-1.5) -- node[left,xshift=-5pt] {$\vec{l}_1$} ++(0,1.15);
    \draw[thick,->] (0,-1.5) -- node[left,xshift=-5pt] {$\vec{l}_2$} ++(0,-1.15);
    % \draw[thick,->] (0,-1.325) -- node[right, xshift=5pt] {$\vec{l}/2$} ++(0,-1.325);

    % \draw[thick,->] (0,0) -- node[right, below] {$\vec{r}$} ++(-2.8,-2.69);

    % \draw[interface] (-3,0) rectangle ++(6,0.5);
    % \draw[line width=0.4mm] (-3,0) -- ++(6,0);
    % \draw[fill=magenta] (0,0) circle (3pt) node[above, yshift=3pt] {\contour{white}{$O$}};
    \draw[thick] (0,0) -- (0,-3);

    \draw (0,-1.2) node[scale=1.5, magenta] {$\times$} node[right, xshift=3pt, yshift=-1pt] {ц.м.};
    
    \draw[fill=magenta] (-3,0) circle (0.35) node {\contour{white}{$m_1$}};
    \draw[fill=magenta] (0,-3) circle (0.35) node {\contour{white}{$m_2$}};
    \draw[fill=magenta] (0,0) circle (0.35) node {\contour{white}{$m_3$}};


    \draw[axis,->] (-3,-4) -- ++(6,0) node[right] {$+x$};
    \draw[axis,->] (-1.3,-4) -- ++(0,4) node[right] {$+y$};

    \draw (2,0) node[blue] {$\bigotimes$} node[right, xshift=.5em] {$\vec{g}$};
    \draw (2,-2) node[magenta] {$\bigotimes$} node[right, xshift=.5em, magenta] {$\vec{N}_c^*$};
    % \draw[force,->] (0,0) -- (-1,0) node[left, below]{$\vec{f}$};

\end{tikzpicture}
% \vspace{-2em}
\end{figure}
\begin{equation*}
    m_1=m_2=m_3\equiv m
\end{equation*}
% \begin{equation}
%     \vec{M}_0=[\vec{0}\,\times\,\vec{f}]=0 \quad \Longrightarrow \vec{N}=\text{const}
% \end{equation}
ЗСМИ:
\begin{equation}
    \label{eq1}
    [\vec{r}\,\times\,{m\vec{v}_0}]
    \equiv [\vec{l}\,\times\,{m\vec{v}_0}]=
    [\vec{0}\,\times\,{m\vec{u}_3}]+
    [\vec{l}\,\times\,{(m+m)\vec{u}_3}]
    % \frac{1}{2}[\vec{l}\,\times\,{m\vec{u}_3}]
\end{equation}
ЗСМИ в проекции на ось, проходящую через $m_2$ перпендикулярно столу:
\begin{equation}
    \label{eq:zsmi}
    l\cdot mv_0 = 
                l\cdot 2m\cdot u_{3x}
                \Rightarrow
                u_{3x}=\frac{v_0}{2}
\end{equation}
ЗСИ в проекции на $x$:
\begin{equation}
    mv_0=
        2m\cdot u_{3x}+mu_{2x}
        \Rightarrow
        u_{2}=0 
\end{equation}
\begin{equation}
    v_{cx}=
            \frac{2mu_{3x}+
            m\cdot0}{3m}=
            \frac{v_0}{3}
\end{equation}
\begin{equation}
    u_{3x}^*=
        u_{3x}-v_{cx}=
        \frac{v_0}{6}
\end{equation}
\begin{equation}
    u_{2x}^*=
            u_{2x}-v_{cx}=
            -\frac{v_0}{3}
\end{equation}
Отсюда
\begin{gather}
    \vec{N}_c^*=
        \left[{\vec{l}_1} \,\times\, 2m\cdot \vec{u}_{3}^*\right]+
        \left[{\vec{l}_2} \,\times\, m\cdot \vec{u}_{2}^*\right]
        % \equiv   
        % \frac{1}{2}\left[m\cdot \vec{u}_{2}^* \,\times\, \vec{l}\,\right]
\end{gather}
Найдем координаты центра масс относительно $m_2$:
\begin{equation}
    y_c=\frac{m_2\cdot0+2m\cdot l}{3m}=\frac23l
\end{equation}
Отсюда следует, что относительно центра масс
\begin{equation}
    \vec{l}_1=\frac13\vec{l}
\end{equation}
и
\begin{equation}
    \vec{l}_2=-\frac23\vec{l}
\end{equation}
\begin{gather}
    N_{cZ}^*=\frac{l}{3}\cdot2m\cdot\frac16v_0+\left(-\frac{2l}3\right)\cdot m\cdot\left(-\frac13v_0\right)=\\=
    \frac19lmv_0+\frac29mlv_0=\frac13lmv_0\\
    \vec{N}_c^*=\vec{k}N_{cZ}^*=\vec{k}\frac13lmv_0
\end{gather}

\end{document}