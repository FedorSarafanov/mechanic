\documentclass[a5paper,10pt]{article}
 
\usepackage{extsizes}
\usepackage{cmap}
\usepackage[T2A]{fontenc}
\usepackage[utf8x]{inputenc}
\usepackage[english, russian]{babel}

\usepackage{misccorr}

%%%%%%%%%%%%%%%%%%%%%%%%%%%%%%%%%%%%%%%%%%%%%%%%%%%%%%%%%%%%%%%%%%%%%%%%%%%%%%%%%%  
\usepackage{graphicx} % для вставки картинок
\graphicspath{{img/}}
\usepackage{amssymb,amsfonts,amsmath,amsthm} % математические дополнения от АМС

% \usepackage{fontspec}
% \usepackage{unicode-math}

\usepackage{indentfirst} % отделять первую строку раздела абзацным отступом тоже
\usepackage[usenames,dvipsnames]{color} % названия цветов
\usepackage{makecell}
\usepackage{multirow} % улучшенное форматирование таблиц
\usepackage{ulem} % подчеркивания
\linespread{1.3} % полуторный интервал
% \renewcommand{\rmdefault}{ftm} % Times New Roman (не работает)
\frenchspacing
\usepackage{geometry}
\geometry{left=1cm,right=1cm,top=2cm,bottom=1cm,bindingoffset=0cm}
\usepackage{titlesec}
\usepackage{float}
% \definecolor{black}{rgb}{0,0,0}
% \usepackage[colorlinks, unicode, pagecolor=black]{hyperref}
% \usepackage[unicode]{hyperref} %ссылки
\usepackage{fancyhdr} %загрузим пакет
\pagestyle{fancy} %применим колонтитул
\fancyhead{} %очистим хидер на всякий случай
\fancyhead[LE,RO]{Сарафанов Ф.Г.} %номер страницы слева сверху на четных и справа на нечетных
\fancyhead[CO, CE]{Механика}
\fancyhead[LO,RE]{Яковлев 262} 
\fancyfoot{} %футер будет пустой
% \fancyfoot[CO,CE]{\thepage}
\renewcommand{\labelenumii}{\theenumii)}


\usepackage{tikz}
\usetikzlibrary{scopes}
\usetikzlibrary{%
     decorations.pathreplacing,%
     decorations.pathmorphing,%
    patterns,%
    calc,%
    scopes,%
    arrows,%
    through,%
    % arrows.spaced,%
}

\begin{document}

\begin{figure}[H]
    \centering
\begin{tikzpicture}[
    force/.style={>=latex,draw=blue,fill=blue},
    % axis/.style={densely dashed,gray,font=\small},
    axis/.style={densely dashed,black!60,font=\small},
    M/.style={rectangle,draw,fill=lightgray,minimum size=0.5cm,thin},
    m2/.style={draw=black!30, rectangle,draw,thin, fill=blue!2, minimum width=0.7cm,minimum height=0.7cm},
    m1/.style={draw=black!30, rectangle,draw,thin, fill=blue!2, minimum width=0.7cm,minimum height=0.7cm},
    plane/.style={draw=black!30, very thick, fill=blue!5, line width=1pt},
    % base/.style={draw=black!70, very thick, fill=blue!4, line width=2pt},
    string/.style={draw=black, thick},
    pulley/.style={thick},
    interface1/.style={draw=gray!60,
        % The border decoration is a path replacing decorator. 
        % For the interface style we want to draw the original path.
        % The postaction option is therefore used to ensure that the
        % border decoration is drawn *after* the original path.
        postaction={draw=gray!60,decorate,decoration={border,angle=-135,
                    amplitude=0.3cm,segment length=2mm}}},
    interface/.style={
        pattern = north east lines,
        draw    = none,
        pattern color=gray!60,          
    },
    plank/.style={
        fill=black!60, 
        draw=black,
        minimum width=3cm,
        inner ysep=0.1cm,
        outer sep=0pt,
        yshift=0.75cm,
        pattern = north east lines,
        pattern color=gray!60, 
    },
    cargo/.style={
        rectangle,
        fill=black!70,              
        inner sep=2.5mm,
    }
]
    % \draw[force,double equal sign distance=2pt,->] (c) -- ++(0,-2) node[below] {$\vec{a}_0$};
\matrix[column sep=0cm, row sep=0cm] {
\def\angle{41}
%%%%%%%%%%%%%%%%%%%%%%%%%%%%%%%%%%%%%%
	\draw[thick, interface1] (-1,0) -- (8,0);
	\draw[thick,] (5,0) arc (-90:270:1.5cm);
	\draw[thick] (0,5) coordinate (a) .. controls (3,5) and (2.6,0) .. (5,0);
	\draw[axis,<->] (0,0) -- node[left,black] {$h$} (a);
	\draw[axis,<->] (7,0) -- node[right,black] {$R$} ++(0,1.5);

    \draw[axis] (0,1.5) -- (7,1.5);

    \draw[axis,fill] (5,1.5) -- ++(\angle:1.5) coordinate (m) circle(3pt);

    \draw[fill,axis] (0,5)  circle(3pt);

    \draw[fill=white] (5,3)  circle(2pt) node[above] {$b$};


    \draw[axis] (5,1.5) -- ++(\angle:1.5);
    \path[draw,->] (5,1.5) ++ (0.7,0) arc(0:\angle:0.7);

    \path[] (5,1.5) -- ++ (30:0.5) node[right, xshift=5pt, scale=1] {$\phi$};

   \draw[force,->] (m) -- ++(\angle:0.7) node[below] {$\vec{P}$};
    \draw[force,->] (m) -- ++(\angle:-0.7) node[above, yshift=5pt] {$\vec{N}$};
    \draw[force,->] (m) -- ++(90:-1.4) node[below, xshift=-5pt] {$m\vec{g}$};

\\
};
\end{tikzpicture}
% \vspace{-2em}
\end{figure}
\paragraph{1. Удержание на рельсах}
\begin{equation*}
    \begin{aligned}[c]
        m\vec{a}=\vec{N}+m\vec{g}\\
        -ma_n=-N-mg\sin{\phi}\\
        N=m{}\frac{u^2}{R}-mg\sin\phi\\
    \end{aligned}
        \qquad\qquad
    \begin{aligned}[c]
        \text{Условие удержания: $P=N,\ >0$}\\
        m{}\frac{u^2}{R}>mg\sin\phi\\
        \frac{mu^2}{2}>\frac{1}{2}mgR\sin\phi\\
    \end{aligned}
\end{equation*}
где $u$ -- скорость на высоте $x=R+R\sin\phi$. В точке $b$ скорость $u=v_b$, $\sin\phi=1$.

Применим ЗСЭ:
\begin{equation*}
    \begin{aligned}[c]
        mgh=2mgR+\frac{mv_b^2}{2}\\
        \frac{mv_b^2}{2}=mg(h-2R)\\
    \end{aligned}
        \qquad\qquad
    \begin{aligned}[c]
        mg(h-2R)>\frac{1}{2}mgR\\
        h>\frac{5}{2}R
    \end{aligned}
\end{equation*}
\textbf{2. Силы в точке $b$}
\begin{equation*}
    \begin{aligned}[c]
        N=m{}\frac{v_b^2}{R}-mg\\
    \end{aligned}
        \qquad\qquad
    \begin{aligned}[c]
        F_g=mg\\
    \end{aligned}
\end{equation*}
\textbf{3. Траектория при $h\leq\frac{5}{2}R$}

$R\leq{}h\leq\frac{5}{2}R$: Тележка оторвется, не доезжая до точки $b$, и полетит по параболе и столкнется с рельсами.

$h\leq{}R$: Тележка остановится, поедет обратно и может бесконечно перекатываться при отсутствии трения.
\end{document}