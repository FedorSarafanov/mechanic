\documentclass[a5paper,10pt]{article}
\def\source{/home/lab/tex/templates}

\usepackage{cmap}
\usepackage[T2A]{fontenc}
\usepackage[utf8x]{inputenc}
\usepackage[english, russian]{babel}

\usepackage
	{
		amssymb,
		% misccorr,
		amsfonts,
		amsmath,
		amsthm,
		wrapfig,
		makecell,
		multirow,
		indentfirst,
		ulem,
		graphicx,
		geometry,
		fancyhdr,
		subcaption,
		float,
		tikz,
		csvsimple,
		color,
	}  

\usepackage[outline]{contour}
\usepackage[mode=buildnew]{standalone}


\geometry
	{
		left=1cm,
		right=1cm,
		top=2cm,
		bottom=1cm,
		bindingoffset=0cm,
	}

\linespread{1.3} 
\frenchspacing 


\usetikzlibrary{scopes}
\usetikzlibrary
	{
		decorations.pathreplacing,
		decorations.pathmorphing,
		patterns,
		calc,
		scopes,
		arrows,
		through,
		shapes.misc,
		arrows.meta,
	}


\tikzset{
	force/.style=	{
		>=latex,
		draw=blue,
		fill=blue,
				 	}, 
	%				 	
	axis/.style=	{
		densely dashed,
		gray,
		font=\small,
					},
	%
	acceleration/.style={
		>=open triangle 60,
		draw=blue,
		fill=blue,
					},
	%
	inforce/.style=	{
		force,
		double equal sign distance=2pt,
					},
	%
	interface/.style={
		pattern = north east lines, 
		draw    = none, 
		pattern color=gray!60,
					},
	cross/.style=	{
		cross out, 
		draw=black, 
		minimum size=2*(#1-\pgflinewidth), 
		inner sep=0pt, outer sep=0pt,
					},
	%
	cargo/.style=	{
		rectangle, 
		fill=black!70, 
		inner sep=2.5mm,
					},
	%
	}

\pagestyle{fancy} %применим колонтитул
\fancyhead{} %очистим хидер на всякий случай
\fancyhead[R]{Сарафанов Ф.Г.} %номер страницы слева сверху на четных и справа на нечетных
\fancyhead[C]{Механика}
% \fancyhead[L]{Задача под запись - <<АУУ-2>>} 
\fancyfoot{} %футер будет пустой

\newcommand{\irodov}[1]{\fancyhead[L]{Иродов -- №#1}}
\newcommand{\yakovlev}[1]{\fancyhead[L]{Яковлев -- №#1}}
\newcommand{\wrote}[1]{\fancyhead[L]{Под запись -- <<#1>>}}

\newenvironment{tikzpict}
    {
	    \begin{figure}[htbp]
		\centering
		\begin{tikzpicture}
    }
    { 
		\end{tikzpicture}
		% \caption{caption}
		% \label{fig:label}
		\end{figure}
    }

\newcommand{\vbLabel}[3]{\draw ($(#1,#2)+(0,5pt)$) -- ($(#1,#2)-(0,5pt)$) node[below]{#3}}
\newcommand{\vaLabel}[3]{\draw ($(#1,#2)+(0,5pt)$) node[above]{#3} -- ($(#1,#2)-(0,5pt)$) }

\newcommand{\hrLabel}[3]{\draw ($(#1,#2)+(5pt,0)$) -- ($(#1,#2)-(5pt,0)$) node[right, xshift=1em]{#3}}
\newcommand{\hlLabel}[3]{\draw ($(#1,#2)+(5pt,0)$) node[left, xshift=-1em]{#3} -- ($(#1,#2)-(5pt,0)$) }

% Draw line annotation
% Input:
%   #1 Line offset (optional)
%   #2 Line angle
%   #3 Line length
%   #5 Line label
% Example:
%   \lineann[1]{30}{2}{$L_1$}
\newcommand{\lineann}[4][0.5]{%
    \begin{scope}[rotate=#2, blue,inner sep=2pt, ]
        \draw[dashed, blue!40] (0,0) -- +(0,#1)
            node [coordinate, near end] (a) {};
        \draw[dashed, blue!40] (#3,0) -- +(0,#1)
            node [coordinate, near end] (b) {};
        \draw[|<->|] (a) -- node[fill=white, scale=0.8] {#4} (b);
    \end{scope}
}


\irodov{1.298}

\begin{document}

\begin{tikzpict}
	\clip (-6,-1) rectangle (6,4.25);

	\draw (0,0) circle (2cm);
	\fill[magenta] (0,0) circle (2pt) coordinate (c);
	\fill[magenta] (0,3) circle (2pt) coordinate (o);
	\draw (0,3) circle (1cm);

	\draw[axis] (0,0) circle (3cm);
	\draw[axis] (0,0) -- (0,5);

	\draw (0,0)  ++(150:3) coordinate (oo) circle(1cm);
	\draw[axis,-<] (0,0) -- ++(150:5)node[left] {$-n$};
	\draw[axis] (0,0) -- (0,5);
	\draw[axis] (oo) -- ++(0,3.5);
	\draw[axis] (oo) -- ++(0,-3.5) coordinate (phi);
	\draw[force,->] (0,0) ++(150:2) -- ++ (150:1.5)node[above] {$\vec{N}$};

	\draw[force,->] (0,0) ++(150:3) -- ++ (240:1.5)node[below] {$\vec{v}$};

	\draw[force,->] (oo) -- ++ (0,-1.5) node[below] {$m\vec{g}$};
	\fill[magenta] (oo) circle (2pt);


	\draw pic["$\phi$",draw=magenta,<-,angle eccentricity=1.5,angle radius=.5cm] {angle=phi--oo--c};   
	
	\draw pic["$\phi$",draw=magenta,<-,angle 
	eccentricity=1.5,angle radius=0.5cm] {angle=o--c--oo};   

	\lineann[6]{90}{1.5}{$h^*=h_0\cdot\cos\phi$}
	\lineann[-5]{90}{3}{$h_0=r+R$}

	\contourlength{2mm}
	\draw (2,0) node[] {\contour{white}{$W_\text{п}=0$}};
\end{tikzpict}
Начальная энергия
\begin{equation}
	W_0=mgh_0=mg(r+R)
\end{equation}
Условие отрыва -- $N=0$. Запишем второй закон Ньютона в проекции на радиальную ось $n$:
\begin{equation}
	ma_n=\frac{mv^2}{r+R}=mg\cos\phi-N
\end{equation}
Отсюда
\begin{equation}
	\cos\phi=\frac{v^2}{g(r+R)}, 
		\qquad
	h^*=\frac{v^2}{g}
\end{equation}
Полная энергия в момент отрыва
\begin{equation}
	W=mgh^*+\frac{mv^2}{2}+\frac{I\omega^2}{2}
\end{equation}
Так как движение происходит без проскальзывания, то все время до отрыва
\begin{equation}
	v=\omega r
\end{equation}
Тогда
\begin{equation}
	mg(r+R)=mg(r+R)\frac{\omega^2r^2}{g}+\frac{m\omega^2r^2}{2}+\frac{I\omega^2}{2}
\end{equation}
\begin{equation}
	\omega^2=\frac{g(r+R)}{mr^2+\frac{mr^2}{2}+\frac{2}{10}mr^2}
\end{equation}
Откуда, наконец, выражается $\omega$  в момент отрыва:
\begin{equation}
	\omega=\sqrt{\frac{10g(r+R)}{17r^2}}
\end{equation}
% Из условия непроскальзывания найдем момент окончания такового:
% \begin{equation}
% 	v(t^*)=\omega(t^*)\cdot R
% \end{equation}
% \begin{equation}
% 	v_0-kg\cdot t^*={2kg}\cdot t^*
% \end{equation}
% \begin{equation}
% 	t^*=\frac{2}{3}v_0
% \end{equation}
% Сила трения - диссипативная, и количество теплоты, выделяющееся при её действии, равно по модулю её работе. 

% Тогда
% \begin{equation}
% 	Q=|A_f|=|\Delta W_k|=\frac{mv_0^2}{2}-\left(\frac{mv^2}{2}+\frac{I\omega^2}{2}\right)
% \end{equation}
% \begin{gather}
% 	Q=\frac{mv_0^2}{2}-\left(\frac{mv^2}{2}+\frac{Iv^2}{2R^2}\right)=\\=
% 	\frac{mv_0^2}{2}-\frac{mv_0^2}{2}\cdot\frac{4}{9}-\frac{mR^2}{2}\frac{v_0^2}{2R^2}\frac{4}{9}=\frac{mv_0^2}{6}
% \end{gather}
% А отношение теплоты к начальной энергии
% \begin{equation}
% 	\eta=\frac{Q}{W_0}=\frac{mv_0^2}{6}\cdot\frac{2}{mv_0^2}=\frac13
% \end{equation}
\end{document}