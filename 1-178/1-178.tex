\documentclass[a5paper,10pt]{article}
 
\usepackage{extsizes}
\usepackage{cmap}
\usepackage[T2A]{fontenc}
\usepackage[utf8x]{inputenc}
\usepackage[english, russian]{babel}

\usepackage{misccorr}

%%%%%%%%%%%%%%%%%%%%%%%%%%%%%%%%%%%%%%%%%%%%%%%%%%%%%%%%%%%%%%%%%%%%%%%%%%%%%%%%%%  
\usepackage{graphicx} % для вставки картинок
\graphicspath{{img/}}
\usepackage{amssymb,amsfonts,amsmath,amsthm} % математические дополнения от АМС

% \usepackage{fontspec}
% \usepackage{unicode-math}

\usepackage{indentfirst} % отделять первую строку раздела абзацным отступом тоже
\usepackage[usenames,dvipsnames]{color} % названия цветов
\usepackage{makecell}
\usepackage{multirow} % улучшенное форматирование таблиц
\usepackage{ulem} % подчеркивания
\linespread{1.3} % полуторный интервал
% \renewcommand{\rmdefault}{ftm} % Times New Roman (не работает)
\frenchspacing
\usepackage{geometry}
\geometry{left=1cm,right=1cm,top=2cm,bottom=1cm,bindingoffset=0cm}
\usepackage{titlesec}
\usepackage{float}
% \definecolor{black}{rgb}{0,0,0}
% \usepackage[colorlinks, unicode, pagecolor=black]{hyperref}
% \usepackage[unicode]{hyperref} %ссылки
\usepackage{fancyhdr} %загрузим пакет
\pagestyle{fancy} %применим колонтитул
\fancyhead{} %очистим хидер на всякий случай
\fancyhead[LE,RO]{Сарафанов Ф.Г.} %номер страницы слева сверху на четных и справа на нечетных
\fancyhead[CO, CE]{Механика}
\fancyhead[LO,RE]{Иродов -- №1.178} 
\fancyfoot{} %футер будет пустой
% \fancyfoot[CO,CE]{\thepage}
\renewcommand{\labelenumii}{\theenumii)}


\usepackage{tikz}
\usetikzlibrary{scopes}
\usetikzlibrary{%
     decorations.pathreplacing,%
     decorations.pathmorphing,%
    patterns,%
    calc,%
    scopes,%
    arrows,%
    % arrows.spaced,%
}

\begin{document}

\begin{figure}[H]
    \centering
\begin{tikzpicture}[
    force/.style={>=latex,draw=blue,fill=blue},
    % axis/.style={densely dashed,gray,font=\small},
    axis/.style={densely dashed,black!60,font=\small},
    interface/.style={
        pattern = north east lines,
        draw    = none,
        pattern color=gray!60,          
    },
    cargo/.style={
        rectangle,
        fill=magenta!40,
        draw=black!50,
        inner sep=2.5mm,
    },
    spring/.style={
        decoration={
            aspect=0.3, 
            segment length=.8mm, 
            amplitude=2mm,
            coil},
        decorate,
        draw=magenta!25
    },
    interface1/.style={draw=gray!60,
        % The border decoration is a path replacing decorator. 
        % For the interface style we want to draw the original path.
        % The postaction option is therefore used to ensure that the
        % border decoration is drawn *after* the original path.
        postaction={draw=gray!60,decorate,decoration={border,angle=-135,
                    amplitude=0.3cm,segment length=2mm}}},    
]
\def\angle{41}
%%%%%%%%%%%%%%%%%%%%%%%%%%%%%%%%%%%%%%
    \draw[thick, interface1] (5,-0.25) -- ++(4,0);
    % \draw[thick,] (5,0) arc (-90:270:1.5cm);
    \draw[thick, interface1] (0,3) coordinate (a) .. controls (3,3) and (2.6,0) .. (5,0);
    \draw[axis,<->] (0,0) -- node[left,black] {$h$} (a);
    % \draw[axis,<->] (7,0) -- ++(0,1.5);
    \draw[axis,->] (0,0) --  ++(9,0) node[right] {$+x$};
    \draw[draw=black!30, fill=magenta!20] (5,0) rectangle node[above, yshift=0.5em] {$M$} ++(3,-0.25);
    

    \draw[fill=magenta] (0,3) circle (2pt) node [above]{$m$};


\end{tikzpicture}
% \vspace{-2em}
\end{figure}
Спустившись с горки (силы трения с нею нет), груз будет обладать кинетической энергией, которая найдется из ЗСМЭ:
\begin{equation}
    \frac{mv^2}{2}=mgh \quad\Longrightarrow \quad v^2=2gh
\end{equation}
Рассмотрим систему <<$m,M$>>. В проекции на ось $x$ внешние силы отсутствуют, сила трения -- внутренняя -- импульса не изменит, запишем ЗСИ в проекции на $x$: 
\begin{equation}
    mv=(M+m)u
\end{equation}
Откуда
\begin{equation}
    u=v\frac{m}{m+M}
\end{equation}
Тогда найдем работу силы трения по теореме о изменении кинетической энергии (между точкой конца спуска и произвольной точкой после остановки шайбы):
\begin{equation}
    A_\text{тр}=\Delta W_k=\frac{(M+m)u^2}{2}-\frac{mv^2}{2}
\end{equation}
Подставим $u$:
\begin{equation}
    A_\text{тр}=\frac{mv^2}{2}\left(\frac{m}{m+M}-1\right)
\end{equation}
И окончательно, подставляя $v^2$
\begin{equation}
    A_\text{тр}=-\frac{mM}{m+m}gh
\end{equation}


\end{document}