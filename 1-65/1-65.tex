\documentclass[a5paper,10pt]{article}\usepackage[usenames,dvipsnames]{color}\usepackage{extsizes,cmap,graphicx,misccorr,indentfirst,makecell,multirow,ulem,geometry,amssymb,amsfonts,amsmath,amsthm,titlesec,float,fancyhdr,wrapfig,tikz}\usepackage[T2A]{fontenc}\usepackage[utf8x]{inputenc}\usepackage[english, russian]{babel}\usetikzlibrary{decorations.pathreplacing,decorations.pathmorphing,patterns,calc,scopes,arrows,through}\graphicspath{{img/}}\linespread{1.3}\frenchspacing\geometry{left=1cm, right=1cm, top=2cm, bottom=1cm, bindingoffset=0cm}\pagestyle{fancy}\fancyhead{}\fancyhead[R]{Сарафанов Ф.Г.} 
\fancyhead[C]{Механика}
\fancyhead[L]{1.65 -- Иродов} 
\fancyfoot{}
\begin{document}

\def\iangle{35} % Angle of the inclined plane

\def\down{-90}
\def\arcr{0.5cm} % Radius of the arc used to indicate angles

\begin{figure}[tb]
	\centering
\begin{tikzpicture}[
    force/.style={>=latex,draw=blue,fill=blue},
    axis/.style={densely dashed,gray,font=\small},
    M/.style={rectangle,draw,fill=lightgray,minimum size=0.5cm,thin},
    m/.style={rectangle,draw=black,fill=lightgray,minimum size=0.3cm,thin},
    plane/.style={draw=black,fill=blue!10},
    string/.style={draw=black, thick},
    pulley/.style={thick},
]

\matrix[column sep=1cm] {
    %% Sketch
    \draw[plane] (0,-1) coordinate (base)
                     -- coordinate[pos=0.5] (mid) ++(\iangle:3) coordinate (top)
                     |- (base) -- cycle;
    \path (mid) node[M,rotate=\iangle,yshift=0.25cm] (M) {};
    \draw[pulley] (top) -- ++(\iangle:0.25) circle (0.25cm)
                   ++ (90-\iangle:0.5) coordinate (pulley);
    \draw[string] (M.east) -- ++(\iangle:1.5cm) arc (90+\iangle:0:0.25)
                  -- ++(0,-1) node[m]  {};

    \draw[->] (base)++(\arcr,0) arc (0:\iangle:\arcr);
    \path (base)++(\iangle*0.5:\arcr+5pt) node {$\alpha$};
    %%

&
    %% Free body diagram of M
    \begin{scope}[rotate=\iangle]
        \node[M,transform shape] (M) {};
        % Draw axes and help lines

        {[axis,->]
            \draw (0,-1) -- (0,2) node[right] {$+y$};
            \draw (M) -- ++(2,0) node[right] {$+x$};
            % Indicate angle. The code is a bit awkward.

            \draw[solid,shorten >=0.5pt] (\down-\iangle:\arcr)
                arc(\down-\iangle:\down:\arcr);
            \node at (\down-0.5*\iangle:1.3*\arcr) {$\alpha$};
        }

        % Forces
        {[force,->]
            % Assuming that Mg = 1. The normal force will therefore be cos(alpha)
            \draw (M.center) -- ++(0,{cos(\iangle)}) node[above right] {$\vec{N}$};
            \draw (M.west) -- ++(-1,0) node[left] {$\vec{f_R}$};
            \draw (M.east) -- ++(1,0) node[above] {$\vec{T}$};
        }

    \end{scope}
    % Draw gravity force. The code is put outside the rotated
    % scope for simplicity. No need to do any angle calculations. 
    \draw[force,->] (M.center) -- ++(0,-1) node[below] {$M\vec{g}$};
    %%

&
    %%%
    % Free body diagram of m
    \node[m] (m) {};
    \draw[axis,->] (m) -- ++(0,-2) node[left] {$+x$};
    {[force,->]
        \draw (m.north) -- ++(0,1) node[above] {$\vec{T'}$};
        \draw (m.south) -- ++(0,-1) node[right] {$m\vec{g}$};
    }

\\
};
\end{tikzpicture}	
\vspace{-3em}  
\end{figure}

\def\N{\vec{N}}
\def\T{\vec{T}}
\def\t{\vec{T'}}
\def\F{\vec{f_R}}
\def\P{M\vec{g}}
\def\p{m\vec{g}}

Запишем второй закон Ньютона для грузов $M$ и $m$: 
\begin{gather}
	M\vec{a}=\N+\T+\F+\P\\
	m\vec{a}=\t+\p
\end{gather}
Учитывая, что нить и блок идеальная  (отсюда $|\T|=|\t|$, $a_m=a_M$), запишем проекции на $x$: 
\begin{gather}
	\label{eq1}Ma=T+{f_R}_x-Mg\sin(\alpha)\\
	\label{eq2}ma=mg-T
\end{gather}
И проекция на $y$:
\begin{gather}
	\label{eq3}N=Mg\cos(\alpha)
\end{gather}
Решив систему (\ref{eq1}), (\ref{eq2}), (\ref{eq3}) относительно $a$, получаем
\begin{gather}
	\label{eq4}a=g\frac{(1-(\sin{\alpha}-\frac{{f_R}_x}{Mg})\frac{M}{m})}{\frac{M}{m}+1}
\end{gather}
Если грузы катятся вправо, то $a>0$ и ${f_R}_x=Mg\mu\cos\alpha$. Тогда
\begin{gather}
	a>0, \hspace{1em}\text{если}\hspace{1em} \frac{m}{M}>\mu\cos{\alpha}+\sin{\alpha}
\end{gather}
Если влево, то $a<0$ и ${f_R}_x=-Mg\mu\cos\alpha$
\begin{equation}
        a<0, \hspace{1em}\text{если}\hspace{1em} \frac{m}{M}<-\mu\cos{\alpha}+\sin{\alpha}
\end{equation}
\end{document}