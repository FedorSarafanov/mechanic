\documentclass[a5paper,10pt]{article}\usepackage[usenames,dvipsnames]{color}

\usepackage{cmap,graphicx,etoolbox,misccorr,indentfirst,makecell,multirow,ulem,geometry,amssymb,amsfonts,amsmath,amsthm,titlesec,float,fancyhdr,wrapfig,tikz}

\usepackage[T2A]{fontenc}\usepackage[utf8x]{inputenc}\usepackage[english, russian]{babel}\usetikzlibrary{decorations.pathreplacing,decorations.pathmorphing,patterns,calc,scopes,arrows,through, shapes.misc}\graphicspath{{img/}}\linespread{1.3}\frenchspacing\geometry{left=1cm, right=1cm, top=2cm, bottom=1cm, bindingoffset=0cm}\pagestyle{fancy}\fancyhead{}\fancyhead[R]{Сарафанов Ф.Г.}\fancyhead[C]{Механика}
\fancyhead[L]{Иродов -- №1.399}
\fancyfoot{}

%Команда \beforetext для текста слева от формулы
\makeatletter \newif\if@gather@prefix \preto\place@tag@gather{\if@gather@prefix\iftagsleft@ \kern-\gdisplaywidth@ \rlap{\gather@prefix} \kern\gdisplaywidth@ \fi\fi } \appto\place@tag@gather{\if@gather@prefix\iftagsleft@\else \kern-\displaywidth \rlap{\gather@prefix} \kern\displaywidth \fi\fi \global\@gather@prefixfalse } \preto\place@tag{\if@gather@prefix\iftagsleft@ \kern-\gdisplaywidth@ \rlap{\gather@prefix} \kern\displaywidth@ \fi\fi } \appto\place@tag{\if@gather@prefix\iftagsleft@\else \kern-\displaywidth \rlap{\gather@prefix} \kern\displaywidth \fi\fi \global\@gather@prefixfalse } \newcommand*{\beforetext}[1]{\ifmeasuring@\else \gdef\gather@prefix{#1} \global\@gather@prefixtrue \fi } \makeatother 
\tikzset{force/.style={>=latex,draw=blue,fill=blue}, axis/.style={densely dashed,gray,font=\small}, acceleration/.style={>=open triangle 60,draw=blue,fill=blue}, inforce/.style={force,double equal sign distance=2pt}, interface/.style={pattern = north east lines, draw    = none, pattern color=gray!60, }, cross/.style={cross out, draw=black, minimum size=2*(#1-\pgflinewidth), inner sep=0pt, outer sep=0pt},    cargo/.style={rectangle, fill=black!70, inner sep=2.5mm, }}

\begin{document}

Между релятивистскими энергией и импульсом существует следующая связь:
\begin{gather*}
    E^2-{p}^2c^2=m_0^2c^4\\
    \intertext{где можно расписать энергию через импульс как}
    E=mc^2\equiv\frac{pc^2}{u}\\
    % p=mu\equiv\frac{E}{c^2}u\\
    \intertext{Тогда можно выразить собственную массу через известные величины:}
    T=mc^2-m_0c^2=\frac{pc^2}{u}-m_0c^2\\
    m_0c^2=\frac{pc^2}{u}-T\\
    \intertext{Далее подстановка в вышеприведенную формулу дает ответ:}
    \frac{p^2c^4}{u^2}-p^2c^2=(\frac{pc^2}{u}-T)^2\\
    -p^2c^2=-\frac{2pTc^2}{u}+T^2\\
    \frac{2pTc^2}{u}=p^2c^2+T^2\\
    u=\frac{2pTc^2}{p^2c^2+T^2}=\frac{2pT}{p^2+T^2/c^2}=0.87c
    % E^2-E^2\frac{u^2}{c^2}=(E-T)^2\\
    % E^2-E^2\frac{u^2}{c^2}=E^2-2ET+T^2\\
\end{gather*}
% \begin{gather*} 
%     v'=\frac{-v-v}{1+v^2/c^2}=-\frac{2v}{1+v^2/c^2}
% \end{gather*}
% Но тогда можно записать релятивистское сокращение длины движущегося стержня:
% \begin{gather*} 
%     l=l_0\sqrt{1-v'^2/c^2}\\
%     l=l_0\sqrt{1-\frac{4v^2}{c^2(1+v^2/c^2)^2}}=
%     l_0\sqrt{1-\frac{4v^2c^2}{(c^2+v^2)^2}}=\\
%     =l_0\sqrt{\frac{c^4+2c^2v^2+v^2}{(c^2+v^2)^2}-\frac{4v^2c^2}{(c^2+v^2)^2}}=\\=
%     l_0\frac{c^2-v^2}{c^2+v^2}
% \end{gather*}
\end{document}